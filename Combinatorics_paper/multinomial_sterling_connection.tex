\documentclass{article}
\usepackage{maa-monthly}
\usepackage{todonotes}

\usepackage[unicode=true, pdfusetitle, bookmarks=true,
  bookmarksnumbered=false, bookmarksopen=false, breaklinks=true, 
  pdfborder={0 0 0}, backref=false, colorlinks=true, linkcolor=blue,
  citecolor=blue, urlcolor=blue]{hyperref}
\hypersetup{final}

%% IF YOU HAVE FONTS INSTALLED
%\usepackage{mtpro2}
%\usepackage{mathtime}

\theoremstyle{theorem}
\newtheorem{theorem}{Theorem}
 
\theoremstyle{definition}
\newtheorem*{definition}{Definition}
\newtheorem*{remark}{Remark}

\begin{document}

\title{The Connection Between Multinomial Coefficients and Sterling Numbers of the Second Kind}
\markright{Multinomials and Sterling Numbers}
\author{Markus Nöth}

\maketitle

\begin{abstract}
The connection between Sterling numbers of the second kind and the multinomial coefficients is highlighted in this paper. 
This is motivated by the combinatorial interpretation of these objects and also proven by explicit calculation.
\end{abstract}


\noindent

\section{Introduction and notation.}

I would like to discuss the relationship between the multinomial coefficients and the Sterling numbers of the second kind.
For some reason this relationship is very rarely discussed in textbooks about combinatorics. In fact the only reference I
am aware of is the handbook of mathematical functions \cite{abramowitz1964handbook}[p 823], where the equation in 
question is marked corrected but is incorrect. 

\subsection{Binomial coefficients.}

The multinomial coefficients are usually introduced in terms of binomial coefficients, which can be defined in different ways,
the reader and I follow the convention of the book ''Concrete Mathematics``\cite{graham1994concrete}.

\begin{definition}[binomial coefficient]
For \(a\in\mathbb{C}, b\in\mathbb{Z}\) we define
\begin{equation}
\begin{pmatrix}a\\b\end{pmatrix} := \left\{\begin{matrix}
\frac{1}{b!}\prod_{l=0}^{b-1} (a-l) \quad \text{for } b\ge 0\\
0 \quad\quad\quad~~~~~  \text{else.}
\end{matrix} \right.
\end{equation}
\end{definition}

By defining the coefficients for negative lower index to be zero, we do not have to worry about boundary conditions in
many formulas. The combinatorial interpretation of the binomial coefficient \(\begin{pmatrix}a \\ b\end{pmatrix}\) for
\(a,b\in\mathbb{N}\) is \textbf{the number of ways to choose \(b\) elements of a set of \(a\) elements}. 

\subsection{Multinomial coefficients}

The multinomial coefficients are a product of binomial coefficients.
\begin{definition}[multinomial coefficient]
For \(g\in\mathbb{N}\) and \(a\in\mathbb{N},\vec{b}\in\mathbb{N}^g\) with \(\sum_{k=1}^g b_k=a\)
we define
\begin{equation}
\begin{pmatrix}a\\\vec{b}\end{pmatrix}:= \prod_{k=1}^{g-1} \begin{pmatrix}a-\sum_{l=1}^{k-1} b_k\\b_k\end{pmatrix},
\end{equation}
where \(\sum_{l=1}^0 f(l):=0\) holds for any summand \(f\) by convention. 
\end{definition}

The combinatorial interpretation of \(\begin{pmatrix}a\\\vec{b}\end{pmatrix}\) for \(g\in\mathbb{N}\) and
 \(a\in\mathbb{N},\vec{b}\in\mathbb{N}^g\) is the \textbf{number of ways to partition a set of \(a\) elements into 
 \(g\) distinct sets, where the \(j\)-th set has \(b_j\) elements}. 
 
The multinomial coefficients are applied e.g. in the well known multinomial theorem\cite{abramowitz1964handbook}[p 823]:

\begin{theorem}[multinomial theorem]
For any \(g,n\in\mathbb{N}\) and \(x_1,\dots x_n\in\mathbb{C}\) the following equality holds true
\begin{equation}
\left(\sum_{k=1}^g x_k\right)^n= \sum_{\stackrel{\vec{b}\in\mathbb{N}_0^g}{|\vec{b}|=n}}\begin{pmatrix}n\\\vec{b}\end{pmatrix} \prod_{k=1}^g x_k^{b_k},
\end{equation}
where one defines \(|\vec{b}|:=\sum_{k=1}^g b_k\). 
\end{theorem} 



\subsection{Sterling numbers of the second kind}
 
The Sterling numbers of the second kind \(\left\{ \begin{matrix} n\\k\end{matrix}\right\}\), for \(n,k\in\mathbb{N}\) are usually introduced by their combinatorial interpretation, which is \textbf{the number of ways to partition a set of \(n\) elements into \(k\)
nonempty sets, \todo{find better formulation}}where the order of the resulting sets is neglected. One can find an explicit formula for the sterling numbers of the second kind \cite{stanley2011enumerative}[p82f]:

\begin{equation}
\left\{\begin{matrix}n\\k\end{matrix}\right\} = \frac{1}{k!} \sum_{j=0}^k (-1)^{k-j} \begin{pmatrix}k\\j\end{pmatrix}j^n.
\end{equation}


\section{Connecting the Concepts.}

\subsection{Combinatorial Justification.} By the interpretation just introduced for the multinomial coefficients we see that for \(g,k\in\mathbb{N}\) the term

\begin{equation*}
\frac{1}{g!}\sum_{\stackrel{\vec{v}\in\mathbb{N}^g}{\left|\vec{v}\right|=k}} \begin{pmatrix} k \\ \vec{v} \end{pmatrix}
\end{equation*}

is the number of ways to partition a set of \(k\) elements into \(g\) sets. The division by \(g!\) is necessary, since for counting
partitions one treats the sets to be partitioned into as indistinguishable, where as in the sum they occur in every order.
The Sterling number of the second kind also have exactly this interpretation, so we suspect the two objects to always agree.

\subsection{Analytic Proof}

One can also see this by mathematical induction.

\begin{theorem}
For any \(g\in\mathbb{N},k\in\mathbb{N}\)
\begin{equation}
\sum_{\stackrel{\vec{v}\in\mathbb{N}^g}{|\vec{v}|=k}} \binom{k}{\vec{v}}=\sum_{l=0}^g (-1)^l (g-l)^k \binom{g}{l}
\end{equation}
holds.
\end{theorem}
\begin{proof}
We would like to apply the multinomial theorem but there are all the summands missing where at least
one of the entries of \(\vec{v}\) is zero, so we add an appropriate expression of zero. We also give the expression in
question a name, since we will later on arrive at a recursive expression.
\begin{multline}
F(g,k):=\sum_{\stackrel{\vec{v}\in\mathbb{N}^g}{|\vec{v}|=k}} \binom{k}{\vec{v}}
= \sum_{\stackrel{\vec{v}\in\mathbb{N}_0^g}{|\vec{v}|=k}} \binom{k}{\vec{v}}
- \sum_{\stackrel{\stackrel{\vec{v}\in\mathbb{N}_0^g}{|\vec{v}|=k}}{\exists l: v_l=0}} \binom{k}{\vec{v}}\\
= g^k 
 - \sum_{\stackrel{\stackrel{\vec{v}\in\mathbb{N}_0^g}{|\vec{v}|=k}}{\exists l: v_l=0}} \binom{k}{\vec{v}}
=g^k 
- \sum_{n=1}^{g-1} \sum_{\stackrel{\vec{v}\in\mathbb{N}_0^g}{|\vec{v}|=k}}
 \binom{k}{\vec{v}} 1_{\exists! i_1\dots i_n : \left(\forall i\neq k: i_l\neq i_k\right) \wedge \forall l :v_l=0}
\end{multline}
where in the last line the indicator function is to enforce there being exactly n different indices \(l\) for which \(v_l=0\)
holds. Now now since it does not matter which entries of the vector vanish because the multinomial coefficient 
is symmetric and its value identical to the corresponding multinomial coefficient where the vanishing entries
are omitted, we can further simplify the sum:

\begin{equation*}
F(g,k)= g^k -  \sum_{n=1}^{g-1} \binom{g}{n} \sum_{\stackrel{\vec{v}\in\mathbb{N}^n}{|\vec{v}|=k}}
 \binom{k}{\vec{v}}
\end{equation*}

The inner sum turns out to be \(F(g-n,k)\), so we found the recursive relation for \(F\):
\begin{equation}\label{combinatorics solution recursive}
F(g,k)= g^k -  \sum_{n=1}^{g-1} \binom{g}{n} F(g-n,k).
\end{equation}

By iteratively applying this equation, we find the following formula, which we will now prove by induction

\begin{multline}\label{combinatorics induction}
\forall d\in\mathbb{N}_0: F(g,k)=\sum_{l=0}^d (-1)^l (g-l)^k \binom{g}{l}\\
+(-1)^{d+1} \sum_{n=1}^{g-d-1} \binom{n+d-1}{d} \binom{g}{n+d} F(g-d-n,k).
\end{multline}

We already showed the start of the induction, so what's left is the induction step. Before we do so the
following remark is in order: We are only interested in the case \(d=g\) and the formula seems meaningless
for \(d>g\); however, the additional summands in the left sum vanish, where as the the right sum is empty
for these values of \(d\) since the  upper bound of the summation index is lower than its lower bound.

For the induction step, pick \(d\in\mathbb{N}_0\), we pull the first summand out of the second sum 
of \eqref{combinatorics induction},
on this summand we apply the recursive relation \eqref{combinatorics solution recursive} resulting in

\begin{multline}
F(g,k)=\sum_{l=0}^{d+1} (-1)^l (g-l)^k \binom{g}{l}\\
+(-1)^{d+1} \sum_{n=2}^{g-d-1} \binom{n+d-1}{d} \binom{g}{n+d} F(g-d-n,k)\\
-(-1)^{d+1} \binom{g}{d+1} \sum_{n=1}^{g-d-2} \binom{g-d-1}{n} F(g-d-1-n,k)\\
=\sum_{l=0}^{d+1} (-1)^l (g-l)^k \binom{g}{l}\\
+(-1)^{d+1} \sum_{n=1}^{g-d-2} \binom{n+d}{d} \binom{g}{n+d+1} F(g-d-1-n,k)\\
-(-1)^{d+1} \binom{g}{d+1} \sum_{n=1}^{g-d-2} \binom{g-d-1}{n} F(g-d-1-n,k).
\end{multline}
After the index shift we can combine the last two sums. 

\begin{multline}
F(g,k)= \sum_{l=0}^{d+1} (-1)^l (g-l)^k \binom{g}{l}\\
+ \sum_{n=1}^{g-d-2}\left[\binom{g}{d+1} \binom{g-d-1}{n} - \binom{n+d}{d} \binom{g}{n+d+1} \right] 
\\(-1)^{d+2} F(g-d-1-n,k).
\end{multline}


In order to combine the two binomials we disassemble \(\binom{g}{d+1}\) into a product and use
the absorbtion identity\cite{graham1994concrete}[p. 157] \(d+1\) times.

\begin{equation}\tag{absorption identity}
\forall a \in \mathbb{C}\ \forall b \in \mathbb{Z}: b \binom{a}{b} = a \binom{a-1}{b-1} 
\end{equation}

This results in
\begin{multline}
F(g,k)= \sum_{l=0}^{d+1} (-1)^l (g-l)^k \binom{g}{l}\\
+ \sum_{n=1}^{g-d-2}\left[\binom{n+d+1}{d+1} - \binom{n+d}{d}\right] \binom{g}{n+d+1} 
\\(-1)^{d+2} F(g-d-1-n,k)\\
=\sum_{l=0}^{d+1} (-1)^l (g-l)^k \binom{g}{l}\\
+(-1)^{d+2}  \sum_{n=1}^{g-d-2} \binom{n+d}{d+1} \binom{g}{n+d+1}  F(g-d-1-n,k),
\end{multline}
where we used the addition formula for binomials:

\begin{equation}
\forall n\in \mathbb{C} \forall k \in \mathbb{Z}: \binom{n}{k} = \binom{n-1}{k} + \binom{n-1}{k-1}.
\end{equation}
This concludes the proof by induction. By setting \(d=g\) in equation \eqref{combinatorics induction} 
we arrive at the desired result.
\end{proof}



\begin{acknowledgment}{Acknowledgment.}
The authors wish to thank the Greek polymath Anonymous, whose prolific works are an endless source of inspiration.
\end{acknowledgment}

\bibliographystyle{amsplain}


\begin{thebibliography}{1}
\bibitem{graham1994concrete} Graham, Ronald L and Knuth, Donald E and Patashnik, Oren, Concrete mathematics: a foundation for computer science \textit{Addison-Wesley, Reading} (1994).

\bibitem{abramowitz1964handbook} Abramowitz, Milton and Stegun, Irene A \textit{Handbook of mathematical functions: with formulas, graphs, and mathematical tables}, Courier Corporation, 1964.

\bibitem{stanley2011enumerative}Richard P. Stanley,Enumerative combinatorics 2nd edition, Springer 2011.

\end{thebibliography}

%
%\begin{biog}
%\item[Woodrow Wilson] received his Ph.D. in history and political science from Johns Hopkins University. He held visiting positions at Cornell and Wesleyan before joining the faculty at Princeton, where he was eventually appointed president of the university.  Among his proudest accomplishments was the abolition of eating clubs at Princeton on the grounds that they were elitist.
%\begin{affil}
%Office of the President, Princeton University, Princeton NJ 08544\\
%twoodwilson@princeton.edu
%\end{affil}

%\item[Herbert Hoover] entered Stanford University in 1891, after failing all of the entrance exams except mathematics.  He received his B.S. degree in geology in 1895, spent time as a mining engineer, then was appointed by his co-author to the U.S. Food Administration and the Supreme Economic Council, where he orchestrated the greatest famine relief efforts of all time.
%\begin{affil}
%Hoover Institution, Stanford University, Stanford CA 94305\\
%herbhoover@stanford.edu
%\end{affil}
%\end{biog}
\vfill\eject

\end{document}