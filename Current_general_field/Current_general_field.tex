\documentclass[oneside,reqno,12pt]{amsart}

%\usepackage{fontspec}

\usepackage[a4paper, top=2.7cm, bottom=2.7cm]{geometry}
%\usepackage[T1]{fontenc}
%\usepackage[utf8]{inputenc}
\usepackage{fontspec}
%\setmainfont{YuMincho}
%Hiragino Maru Gothic ProN
\usepackage{bbm}
\usepackage{graphicx}
\usepackage{slashed}
\usepackage{eurosym}
\usepackage{amsmath}
\usepackage{enumitem}
\usepackage{amsfonts}
\usepackage{longtable}
\usepackage[mathscr]{eucal}
\usepackage{mathabx}
\usepackage{mathtools}
\usepackage{dsfont}


\setcounter{secnumdepth}{5}

%commutative diagram
\usepackage{amsmath,amscd}
%picture
\usepackage{wrapfig}

\usepackage[unicode=true, pdfusetitle, bookmarks=true,
  bookmarksnumbered=false, bookmarksopen=false, breaklinks=true, 
  pdfborder={0 0 0}, backref=false, colorlinks=true, linkcolor=blue,
  citecolor=blue, urlcolor=blue]{hyperref}



% \numberwithin{equation}{section}
\allowdisplaybreaks[1]

\newtheorem{axiom}{Axiom}
\newtheorem{Def}{Definition}[section]
\newtheorem{DefLem}{Definition and Lemma}[section]
\newtheorem{Conj}[Def]{Conjecture}
\newtheorem{Thm}[Def]{Theorem}
\newtheorem{Prp}[Def]{Proposition}
\newtheorem{Lemma}[Def]{Lemma}
\newtheorem{lemma}{Lemma}
\newtheorem{Remark}[Def]{Remark}
\newtheorem{Corollary}[Def]{Corollary}
\newtheorem{Example}[Def]{Example}
\newtheorem{Assumption}[Def]{Assumption}

  
\DeclareMathOperator{\tr}{tr}
\DeclareMathOperator{\supp}{supp}


\newcommand{\Z}[2]{Z_{\stackrel{1}{#1}}\left(#2\right)}
\newcommand{\id}{{\mathbbm 1}}
\newcommand{\equaltext}[1]{\ensuremath{\stackrel{\text{#1}}{=}}}
\newcommand{\letext}[1]{\ensuremath{\stackrel{\text{#1}}{\le}}}
\newcommand{\Conv}{\mathop{\scalebox{1.7}{\raisebox{-0.2ex}{\(\ast\)}}}}
\newcommand{\CONV}{\mathop{\scalebox{3.0}{\raisebox{-0.2ex}{\(\ast\)}}}}
% Annotations
%\usepackage[normalem]{ulem}
% \usepackage{refcheck}
\usepackage[colorinlistoftodos,shadow,textsize=scriptsize,textwidth=2.75cm]{todonotes}
\newcommand{\Dirk}[1]{ \todo[color=orange!60]{Dirk: #1} }
\newcommand{\DirkBox}[1]{ \mbox{}\todo[inline,caption={},color=red!60]{Dirk: #1} }
\newcommand{\Markus}[1]{ \todo[color=green!20]{Markus: #1} }
\newcommand{\dirk}{ \color{orange} }
\newcommand{\markus}{ \color{green} }
\newcommand{\noch}[1]{ \todo[color=blue!20]{Todo: #1} }
\newcommand{\black}{ \color{black} }

\makeatletter



\renewcommand\section{\@startsection {section}{1}{\z@}%
                                   {-2.0ex \@plus -1ex \@minus -.2ex}%
                                   {2.3ex \@plus.2ex}%
                                   {\normalfont\Large\bfseries}}
\renewcommand\subsection{\@startsection {subsection}{1}{\z@}%
                                   {-0.5ex \@plus -0.5ex \@minus -.2ex}%
                                   {0.5em}%
                                   {\normalfont\bfseries}}
\renewcommand\subsubsection{\@startsection {subsubsection}{1}{\z@}%
                                   {-0.3ex \@plus -0.4ex \@minus -.2ex}%
                                   {0.1 em}%
                                   {\normalfont\sc}}  
\renewcommand\paragraph{\@startsection {paragraph}{1}{\z@}%
                                   {-0.2ex \@plus -1ex \@minus -.2ex}%
                                   {0.1 em}%
                                   {\normalfont\it}}                                   
\makeatother

\parindent 0cm


\begin{document}


I begin with a collection of definitions and formulas useful in this setting.

\section{basic definitions}
\begin{Def}
Throuought this document the letters \(A,G\) and \(F\) with or without indices represent four-potentials,  elements of \(C_c^\infty(\mathbb{R}^4, \mathbb{R}^4)\). Furthermore we assume that for some fixed compact \(K\subseteq \mathbb{R}^4\) all the appearing four-potentials are supported in \(K\).
\end{Def}
\begin{Def}
Likewise the greek letter \(\Sigma\) with or without indices and with or without (multiple) \('\)s attached to it represent spacelike hypersurfaces of Minkowski spacetime.
\end{Def}

\begin{Def}
Furthermore

\begin{equation}
\forall \Sigma: \mathcal{H}_\Sigma:= L^2(\Sigma, \mathbb{C}^4, i_\gamma(d^4x)).
\end{equation}
\end{Def}

\begin{Def}
We denote the one-particle Dirac time evolution operator by
\begin{equation}
U^A_{\Sigma, \Sigma'} : \mathcal{H}_\Sigma\rightarrow \mathcal{H}_{\Sigma'}.
\end{equation}
\end{Def}

\begin{Def}
For initial polarization \(V_{\Sigma_0}\subseteq \mathcal{H}_{\Sigma_0}\) with \(\Sigma_0\) before \(K\) and \(\Phi_{\Sigma_0}: \ell_2\rightarrow \mathcal{H}_{\Sigma_0}\) with \(\text{range}\Phi_{\Sigma_0}=V_{\Sigma_0}\), we define
\begin{equation}
\forall \Sigma, \forall A: C_{\Sigma}(A):=\{V_\Sigma^A\mid V_\Sigma^A \approx U^A_{\Sigma, \Sigma_0} V_{\Sigma_0}\}.
\end{equation}
It turns out (ref) that \(C_{\Sigma}(A)\) only depends on the projection \(\Sigma\ni x\mapsto x_\alpha A^\alpha(x)\), furthermore we define
\begin{equation}
\mathcal{F}_{\Sigma}^A:=\mathcal{F}( U_{\Sigma, \Sigma_0}^A\Phi_{\Sigma_0}).
\end{equation}
\end{Def}




\begin{Def}\label{lift}
We call \(\tilde{U}_{\Sigma',\Sigma}^A:\mathcal{F}_{\Sigma}^A\rightarrow\mathcal{F}_{\Sigma'}^{A'}\) unitary such that
\(\forall A, A', \Sigma, \Sigma', \Sigma''\):
\begin{align*}\label{func}\tag{func}
&\tilde{U}^A_{\Sigma'',\Sigma'}\tilde{U}^A_{\Sigma',\Sigma} = \tilde{U}^A_{\Sigma'',\Sigma}\\\label{loc}\tag{loc}
& \text{germ}_{\text{Vol}\Sigma',\Sigma}(A)=\text{germ}_{\text{Vol}\Sigma',\Sigma}(A')\Rightarrow \tilde{U}^A_{\Sigma',\Sigma}=\tilde{U}^{A'}_{\Sigma',\Sigma}\\\label{reg} \tag{reg}
&
\end{align*}
a lift of \(U^A_{\Sigma',\Sigma}\). Here the germ of two functions \(A\) and \(A'\) are equal iff
\begin{equation}
\exists U\supseteq \Sigma, U open: \left.A\right|_{U}=\left. A'\right|_{U}.
\end{equation}
\end{Def}

\begin{Def}
For \(\Sigma_{\text{in}}\) before \(K\) and \(\Sigma_{\text{out}}\) after \(K\) we define
\begin{equation}
\prescript{}{A}{\tilde{S}}_{A+F}:= \tilde{U}^A_{\Sigma_{\text{in}},\Sigma_{\text{out}}} \tilde{U}^{A+F}_{\Sigma_{\text{out}},\Sigma_{\text{in}}}:\mathcal{F}^A_{\Sigma_{\text{in}}}\righttoleftarrow
\end{equation}
\end{Def}

\begin{Def}
We denote for a fixed \( U_{\Sigma_{\text{in}},\Sigma_0}^A \Phi_{\Sigma_0}\sim \Psi_{\text{in}}:\ell^2 \rightarrow \mathcal{H}_{\text{in}}\) by
\begin{equation}
\mathds{1}_{\text{in}}:\mathcal{H}_{\text{in}}\righttoleftarrow,
\end{equation}
the identity on that space and denote by
\begin{equation}
\overline{\mathds{1}}_{\text{in}}:\mathcal{F}(\Psi_{\text{in}})\rightarrow\mathcal{F}(U_{\Sigma_{\text{in}},\Sigma_0}^A \Phi_{\Sigma_0})
\end{equation}
its lift.
\end{Def}


\begin{Def}
We also introduce the projector notation:
\begin{equation}
\prescript{}{A}{S}_{F --}:= \Psi_{\text{in}}^* ~\prescript{}{A}{S}_{F}\Psi_{\text{in}}: \ell^2 \righttoleftarrow.
\end{equation}
\end{Def}

\begin{Def}
and the only (partial) lift which can naturally be written down
\begin{equation}\label{manualLift}
\prescript{}{A}{\overline{S}}_{F}:=\mathcal{L}_{\prescript{}{A}{S}_{F}}\mathcal{R}_{\prescript{}{F}{S}_{A}} 
\frac{1}{\sqrt{\det \prescript{}{A}{S}_{F --} ~\prescript{}{F}{S}_{A --}}}:\mathcal{F}(\Psi_{\text{in}})\righttoleftarrow,
\end{equation}
where the last factor is just for normalisation i.e. to make this operator unitary. Please note that this is not a lift in the sense of Definition \ref{lift} since it may not fulfill any of the conditions.
\end{Def}

\begin{Def}
We can make a connection between the proper lift and \eqref{manualLift} by
\begin{equation}
\prescript{}{A}{\hat{S}}_F:= \overline{\mathds{1}}_{\text{in}}^* \prescript{}{A}{\tilde{S}}_{F} \overline{\mathds{1}}_{\text{in}}:\mathcal{F}(\Psi_{\text{in}})\righttoleftarrow.
\end{equation}
Now \(\overline{S}\) and \(\hat{S}\) agree up to a phase (ref), so we define
\begin{equation}\label{def:z}
\prescript{}{A}{z}_{F} \circ \prescript{}{A}{\overline{S}}_{F} := \prescript{}{A}{\hat{S}}_{F}.
\end{equation} 
\end{Def}
There is yet another phase which characterizes the deficiency of \eqref{manualLift}, namely
\begin{Def}
\begin{equation}\label{def:gamma}
\forall A,B,C: \Gamma^{-1}_{A,B,C} \circ \mathds{1} = \prescript{}{A}{\overline{S}}_B ~ \prescript{}{B}{\overline{S}}_C ~ \prescript{}{C}{\overline{S}}_A.
\end{equation}
\end{Def}

We also introduce a notation for  general complex numbers
\begin{Def}
\begin{equation}
\forall z\in\mathbb{C}\backslash \{0\}: \text{arg}(z) := \frac{z}{|z|}.
\end{equation}
\end{Def}

We introduce the Greens functions of the Dirac equation.
\begin{Def}
For \(x\in\mathbb{R}^4\) we define
\begin{align}
\Delta^\pm (x):&= \frac{-1}{(2\pi)^4} \int_{\mathbb{R}^4 \pm i \varepsilon e_0} \frac{\slashed{p}+m}{p^2-m^2} e^{-ip x} d^4 p \\
&=\pm \frac{i \slashed{\partial} +m}{2\pi} \Theta(\pm x^0) \left[ \delta(x^2) - \Theta (x^2) \frac{m}{2 \sqrt{x^2}} J_1 (m \sqrt{x^2})\right].
\end{align}
This function is the retarded(advanced) Greens function of the Dirac equation. (for a proof see e.g. Scharf)

The difference between the two Greens functions is denoted by \(\Delta^{0}=\Delta^+-\Delta^-\).
\end{Def}


\begin{DefLem}\label{Def:LOperator}
Furthermore we define components of the single particle time evolution operator. For \(\Psi \in C^\infty\left(\mathbb{R}^4, \mathbb{C}^4\right)\), define
\begin{align}
&L^{\pm,0}_A \Psi:= \Delta^{\pm, 0} \ast (\slashed{A} \Psi),\\
&\Omega^{\pm}_A:= \sum_{k=0}^\infty {L^{\pm}_A}^k=\left( 1 - L_A^{\pm}\right)^{-1}\\
&S_A := \left( \Omega_A^-\right)^{-1} \Omega_A^+ = (1-L_A^{-}) \sum_{k=0}^\infty {L^{+}_A}^k = 1+ L_A^0 \Omega^+_A\\
&S_A^{-1} = 1- L^0_A \Omega^-_A.
\end{align}
\end{DefLem}



\section{useful formulas}



\begin{lemma}\label{lemma:temp_sep}
It is true that
\begin{equation}\label{temp_sep}\tag{temporal separation}
\forall F\prec G: \prescript{}{A}{\tilde{S}}_{A+F+G}=\prescript{}{A}{\tilde{S}}_{A+G}\prescript{}{A}{\tilde{S}}_{A+F}
\end{equation}
holds.
\end{lemma}

\begin{lemma}\label{lemma:z_temp_sep}
It is true that
\begin{equation}\label{z_temp_sep}
\forall F\prec G: \prescript{}{A+G}{z}_{A+F+G}=\prescript{}{A}{z}_{A+F}
\end{equation}
holds.
\end{lemma}

\begin{lemma}\label{lemma:gamma}
There are more ways to conveniently express \(\Gamma_{A,B,C}\) for all \(A,B\) and \(C\), namely
\begin{equation}\label{gamma_to_z}
\Gamma_{A,B,C}=\prescript{}{A}{z}_{B} ~\prescript{}{B}{z}_{C}~\prescript{}{C}{z}_A,
\end{equation}
and
\begin{equation}\label{gamma_by_argdet}
\Gamma^{-1}_{A,B,C}= \text{argdet}(\prescript{}{A}{S}_{B --} ~ \prescript{}{B}{S}_{C --} ~\prescript{}{C}{S}_{A --}).
\end{equation}
Furthermore it is true that
\begin{equation}\label{gamma_cycle}
\Gamma_{A,B,C}=\Gamma_{B,C,A}=\Gamma^{-1}_{C,B,A}
\end{equation}
holds. Furthermore, the tetrahedron rule holds for all \(A,B,C\) and \(D\)
\begin{equation}\label{gamma_tetrahedron}
\Gamma_{B,C,D}=\Gamma_{A,C,D}\Gamma_{B,A,D} \Gamma_{B,C,A}.
\end{equation}

\end{lemma}

\begin{lemma}\label{lemma:z_gamma_causal}
For the case \(F\prec G\) one can find a relation between \(z\) and \(\Gamma\) involving just one instance of each object:
\begin{equation}
\partial_F \partial_G \ln \prescript{}{A}{z}_{A+F+G} = - \partial_F \partial_G \ln \Gamma_{A,A+G,A+F+G}
\end{equation}

\end{lemma}

Finally we present a compact formula that connects derivatives of the current to \(\Gamma\):
\begin{lemma}\label{lemma:current_causal}
In the case \(F\prec G\),
\begin{equation}
\partial_F j_{A+F}(G)= i \partial_F \partial_G \ln \frac{\Gamma_{A,A+F,A+F+G}}{\Gamma_{A,A+G,A+F+G}}
\end{equation}
holds.
\end{lemma}
Lemma \ref{lemma:current_causal} makes it easy to see that the derivative of the current is antisymmetric with respect to \(F\) and \(G\).

There is another simplification of the derivative of the current that follows from the symmetries of \(\Gamma\).
\begin{lemma}\label{lemma:current_compact}
For \(F\prec G\) we can simplify the result of lemma \ref{lemma:current_causal} to
\begin{equation}
\partial_F j_{A+F}(G)=- 2 i \partial_F \partial_G \ln \Gamma_{A,A+G,A+F} = 2 i \partial_F \partial_G \ln \prescript{}{A}{z}_{A+F+G}.
\end{equation}
\end{lemma}

\begin{Thm}\label{thm:current_explicit}
For \(F\prec G\), the derivative of the current can more explicitly be expressed as
\begin{equation}
\partial_F j_{A+F}(G)= -2  \Im \tr\left[ \left( \partial_G \prescript{}{A}{S}_{A+G}\right)_{-+} \left( \partial_F \prescript{}{A}{S}_{A+F}\right)_{+-}\right].
\end{equation}
\Markus{todo: understand better why the off-diagonal parts plus all of their derivatives are Hilbert-Schmidt operators}
\end{Thm}

%Using theorem \ref{thm:current_explicit} and definition/lemma \ref{Def:LOperator} we can rewrite the current once more
%\begin{lemma} For \(F\prec G\) we rewrite \(\partial j\) as
%\begin{equation}
%\partial_F j_{A+F}(G):= - 2 \Im \tr \left[ \left(S_A^{-1} \partial_G \left( L^0_{A+G} \Omega^+_{A+G}\right)\right)_{-+}
%\left(S_A^{-1} \partial_G \left( L^0_{A+F} \Omega^+_{A+F}\right)\right)_{+-}\right].
%\end{equation}
%\end{lemma}

Next we want an explicit formula for the nth derivative of the current, since this will be handy in a perturbative expansion of the scattering operator. For this we need a formula for the nth derivative of \(S_A\). 
\begin{lemma}\label{lemma:nDerivativesSMatrix}
Using definition/lemma \ref{Def:LOperator} we find for \(n\in\mathbb{N}\)
\begin{alignat}{2}\tag*{}
&\left(\prod_{k=1}^n \partial_{F_k} \right) S_{A+\sum_{k=1}^n F_k} 
= \sum_{l=1}^n &&L^0_{F_l}\Omega^+_A  \sum_{\sigma \in S(\{1,\dots,n\}\backslash\{l\})} \prod_{\stackrel{k=1}{ k\neq l}}^n \left( L^+_{F_{\sigma(k)}} \Omega^+_{A}\right)\\
& &&+ L^0_{A} \Omega^+_A  \sum_{\sigma \in S(\{1,\dots,n\})} \prod_{k=1}^n \left(L^+_{F_{\sigma(k)}}\Omega^+_{A}\right).
\\
\tag*{}
&\left(\prod_{k=1}^n \partial_{F_k} \right) S^{-1}_{A+\sum_{k=1}^n F_k} 
= -\sum_{l=1}^n &&L^0_{F_l}\Omega^-_A  \sum_{\sigma \in S(\{1,\dots,n\}\backslash\{l\})} \prod_{\stackrel{k=1}{ k\neq l}}^n \left( L^-_{F_{\sigma(k)}} \Omega^-_{A}\right)\\
& &&- L^0_{A} \Omega^-_A  \sum_{\sigma \in S(\{1,\dots,n\})} \prod_{k=1}^n \left(L^-_{F_{\sigma(k)}}\Omega^-_{A}\right).
\end{alignat}
\end{lemma}

Using this lemma and lemma \ref{thm:current_explicit} we find the nth derivative of the current.
\begin{Thm}\label{thm:GeneralDerivative}
Using the abbreviations \(X_a:= X\backslash \{a\}\) and \(X^a:= X\cup \{a\}\) for any set \(X\) and element \(a\), we find for the current
\begin{align*}
&\left(\prod_{k=1}^{n+1} \partial_{F_k}\right) j_{A+\sum_{k=1}^{n+1}F_k}(F_0) 
= -2 \sum_{\stackrel{B,C,D,E\subseteq \{1,\dots, n\}}{B\dot{\cup}C\dot{\cup}D\dot{\cup}E=\{1,\dots, n\}}}
\Im \tr \\
&\left[ 
\left\{\left(\delta_{B,\emptyset} - \sum_{l\in B} L^0_{F_l} \Omega^-_A \sum_{\sigma \in S(B_l)} \prod_{b\in B_l} \left( L^-_{F_{\sigma(b)}} \Omega^-_A\right) - L^0_A \Omega^-_A \sum_{\sigma \in S(B)} \prod_{b\in B} \left( L^-_{F_{\sigma(b)}}\Omega^-_A\right)
\right) \right.\right. \\
&\left. \left. \left(  \sum_{l\in C^0} L^0_{F_l} \Omega^+_A \sum_{\sigma \in S(C^0_l)} \prod_{b\in C^0_l} \left( L^+_{F_{\sigma(b)}} \Omega^+_A\right) 
+ L^0_A \Omega^+_A \sum_{\sigma \in S(C^0)} \prod_{b\in C^0} \left( L^+_{F_{\sigma(b)}}\Omega^+_A\right)
\right) \right\}_{-+}\right. \\
&\left.  \left\{\left(\delta_{D,\emptyset} - \sum_{l\in D} L^0_{F_l} \Omega^-_A \sum_{\sigma \in S(D_l)} \prod_{b\in D_l} \left( L^-_{F_{\sigma(b)}} \Omega^-_A\right) - L^0_A \Omega^-_A \sum_{\sigma \in S(D)} \prod_{b\in D} \left( L^-_{F_{\sigma(b)}}\Omega^-_A\right)\right)\right.\right.\\
&\left. \left. \left( \sum_{l\in E^{n+1}} L^0_{F_l} \Omega^+_A \sum_{\sigma \in S\left(E^{n+1}_l\right)} \prod_{b\in E^{n+1}_l} \left( L^+_{F_{\sigma(b)}} \Omega^+_A\right) 
 + L^0_A \Omega^+_A \sum_{\sigma \in S\left(E^{n+1}\right)} \prod_{b\in E^{n+1}} \left( L^+_{F_{\sigma(b)}}\Omega^+_A\right)
\right) \right\}_{+-}
\right]
\end{align*}
\end{Thm}

Setting \(A=0\) this expression simplifies somewhat
\begin{Corollary}
Using the same abbreviations as in theorem \ref{thm:GeneralDerivative} for \(A=0\) the derivatives of the current are given by
\begin{align*}
&\left(\prod_{k=1}^{n+1} \partial_{F_k}\right) j_{\sum_{k=1}^{n+1}F_k }(F_0) 
= -2 \sum_{\stackrel{B,C,D,E\subseteq \{1,\dots, n\}}{B\dot{\cup}C\dot{\cup}D\dot{\cup}E=\{1,\dots, n\}}}
\Im \tr \\
&\left[ 
\left\{\left(\delta_{B,\emptyset} - \sum_{l\in B} L^0_{F_l} \sum_{\sigma \in S(B_l)} \prod_{b\in B_l}  L^-_{F_{\sigma(b)}}
\right) 
 \left(  \sum_{l\in C^0} L^0_{F_l} \sum_{\sigma \in S(C^0_l)} \prod_{b\in C^0_l}  L^+_{F_{\sigma(b)}}
\right) \right\}_{-+}\right. \\
&\left.  \left\{\left(\delta_{D,\emptyset} - \sum_{l\in D} L^0_{F_l} \sum_{\sigma \in S(D_l)} \prod_{b\in D_l}  L^-_{F_{\sigma(b)}} \right)
 \left( \sum_{l\in E^{n+1}} L^0_{F_l} \sum_{\sigma \in S\left(E^{n+1}_l\right)} \prod_{b\in E^{n+1}_l}  L^+_{F_{\sigma(b)}} 
\right) \right\}_{+-}
\right]
\end{align*}
\end{Corollary}


\section{proofs of useful formulas}
Proof of lemma \ref{lemma:temp_sep}: Let \(F\) and \(G\) be such that \(F\prec G\) holds. Then choose \(\Sigma\) such that \(\Sigma\) is before \(\supp G\) but after \(\supp F\). Then it follows that
\begin{align*}
&\prescript{}{A}{\tilde{S}}_{A+F+G}=\prescript{}{A}{\tilde{S}}_{A+G}\prescript{}{A}{\tilde{S}}_{A+F}\\
&\iff  \tilde{U}^A_{\Sigma_{\text{in}},\Sigma_{\text{out}}} \tilde{U}^{A+F+G}_{\Sigma_{\text{out}},\Sigma_{\text{in}}}
= \tilde{U}^A_{\Sigma_{\text{in}},\Sigma_{\text{out}}} \tilde{U}^{A+G}_{\Sigma_{\text{out}},\Sigma_{\text{in}}}
 \tilde{U}^A_{\Sigma_{\text{in}},\Sigma_{\text{out}}} \tilde{U}^{A+F}_{\Sigma_{\text{out}},\Sigma_{\text{in}}}\\
 &\iff  \tilde{U}^{A+F+G}_{\Sigma_{\text{out}},\Sigma_{\text{in}}}
=  \tilde{U}^{A+G}_{\Sigma_{\text{out}},\Sigma_{\text{in}}}
 \tilde{U}^A_{\Sigma_{\text{in}},\Sigma_{\text{out}}} 
 \tilde{U}^{A+F}_{\Sigma_{\text{out}},\Sigma_{\text{in}}}\\
 &\stackrel{\eqref{func}}{\iff}
 \tilde{U}^{A+F+G}_{\Sigma_{\text{out}},\Sigma}
 \tilde{U}^{A+F+G}_{\Sigma,\Sigma_{\text{in}}}=
 \tilde{U}^{A+G}_{\Sigma_{\text{out}},\Sigma} \tilde{U}^{A+G}_{\Sigma,\Sigma_{\text{in}}}
 \tilde{U}^A_{\Sigma_{\text{in}},\Sigma} \tilde{U}^A_{\Sigma ,\Sigma_{\text{out}}} 
 \tilde{U}^{A+F}_{\Sigma_{\text{out}},\Sigma} \tilde{U}^{A+F}_{\Sigma,\Sigma_{\text{in}}}\\
 &\stackrel{\eqref{loc}}{\iff}
 \tilde{U}^{A+G}_{\Sigma_{\text{out}},\Sigma}
 \tilde{U}^{A+F}_{\Sigma,\Sigma_{\text{in}}}= 
 \tilde{U}^{A+G}_{\Sigma_{\text{out}},\Sigma} \tilde{U}^{A}_{\Sigma,\Sigma_{\text{in}}} 
  \tilde{U}^A_{\Sigma_{\text{in}},\Sigma} \tilde{U}^A_{\Sigma ,\Sigma_{\text{out}}} 
   \tilde{U}^{A}_{\Sigma_{\text{out}},\Sigma} \tilde{U}^{A+F}_{\Sigma,\Sigma_{\text{in}}}\\
 &\iff 
 \tilde{U}^{A+G}_{\Sigma_{\text{out}},\Sigma}
 \tilde{U}^{A+F}_{\Sigma,\Sigma_{\text{in}}}= 
 \tilde{U}^{A+G}_{\Sigma_{\text{out}},\Sigma}\tilde{U}^{A+F}_{\Sigma,\Sigma_{\text{in}}}.
\end{align*}
\qed

Now we prove lemma \ref{lemma:z_temp_sep}. Let \(F \prec G\). Using definition \eqref{def:z}, as well as lemma \ref{lemma:temp_sep} which holds for \(\hat{S}\) as well (simply by inserting the proper identities) we compute
\begin{align*}
&\prescript{}{A+G}{z}_{A+F+G} \circ \prescript{}{A+G}{\overline{S}}_{A+F+G}
 = \prescript{}{A+G}{\hat{S}}_{A+F+G} 
 =\prescript{}{A+G}{\hat{S}}_A \prescript{}{A}{\hat{S}}_{A+F+G}\\
&\stackrel{ F\prec G}{=} \prescript{}{A+G}{\hat{S}}_{A}
 ~\prescript{}{A}{\hat{S}}_{A+G} ~\prescript{}{A}{\hat{S}}_{A+F}
 = \prescript{}{A}{\hat{S}}_{A+F}
 = \prescript{}{A}{z}_{A+F}\circ \prescript{}{A}{\overline{S}}_{A+F}.
\end{align*}
Now by evaluation \(\prescript{}{A+F+G}{\overline{S}}_{A+G} ~\prescript{}{A}{\overline{S}}_{A+F}\)  we find the relation between the appearing phases. By a computation analogous to the one we just did for the one-particle scattering operator we see that the left-operation part of this operator is just the identity. The right-operation may still contribute a determinant; however, since \(\overline{S}\) is unitary the determinant may only be a phase. Therefore we see that
\begin{align*}
&\prescript{}{A+G}{z}_{A+F+G} ~\prescript{}{A+F}{z}_{A} 
= \text{argdet}((\prescript{}{A+F}{S}_A)_{--} (\prescript{}{A+G}{S}_{A+F+G})_{--})\\
&= \text{argdet}((\prescript{}{A+F}{S}_A)_{--} (\prescript{}{A+G}{S}_A ~ \prescript{}{A}{S}_{A+F+G})_{--})
= \text{argdet}((\prescript{}{A+F}{S}_A)_{--} ( \prescript{}{A}{S}_{A+F})_{--})\\
&= \text{argdet}((\prescript{}{A+F}{S}_A)_{--} ( \prescript{}{A+F}{S}_{A})^*_{--})=1
\end{align*}
holds.
\qed

Now for lemma \ref{lemma:gamma}. Formula \eqref{gamma_by_argdet} can be seen from the definition of \(\Gamma\) by taking the vacuum expectation value. Formula \(\eqref{gamma_cycle}\) can directly be seen form the definition of \(\Gamma\). We prove \eqref{gamma_to_z}, by observing that
\begin{equation}
\prescript{}{A}{\tilde{S}}_{C}= \prescript{}{A}{\tilde{S}}_{B} ~\prescript{}{B}{\tilde{S}}_{C}
\end{equation}
holds, therefore it also holds for \(\hat{S}\). Inserting definitions \eqref{def:z} and \eqref{def:gamma} yields
\begin{equation}
\prescript{}{A}{z}_{C} \circ \prescript{}{A}{\overline{S}}_{C}= 
\prescript{}{A}{z}_{B}~ \prescript{}{B}{z}_{C} \circ \prescript{}{A}{\overline{S}}_{B}~\prescript{}{B}{\overline{S}}_{C}
\end{equation}
and
\begin{equation}
\prescript{}{A}{z}_{C} ~\prescript{}{B}{z}_{A}~\prescript{}{C}{z}_{B} \circ \mathds{1}= \prescript{}{A}{\overline{S}}_B ~ \prescript{}{B}{\overline{S}}_C ~\prescript{}{C}{\overline{S}}_A
=\Gamma^{-1}_{A,B,C}. 
\end{equation}
Rearranging yields \eqref{gamma_to_z}. For the tetrahedron rule we simply insert \eqref{gamma_to_z} into the right hand side and get
\begin{align*}
&\Gamma_{A,C,D} \Gamma_{B,A,D} \Gamma_{B,C,A}
= \prescript{}{A}{z}_{C} ~\prescript{}{C}{z}_{D}~\prescript{}{D}{z}_{A} ~
\prescript{}{B}{z}_{A} ~\prescript{}{A}{z}_{D}~\prescript{}{D}{z}_{B}~
\prescript{}{B}{z}_{C} ~\prescript{}{C}{z}_{A}~\prescript{}{A}{z}_{B}\\
&= \prescript{}{C}{z}_{D}~
\prescript{}{D}{z}_{B}~
\prescript{}{B}{z}_{C} =\Gamma_{B,C,D}.
\end{align*}

\qed

Now to prove lemma \ref{lemma:z_gamma_causal}. Let again \(F\prec G\) be true. By adding terms which vanish after splitting products into sums in the logarithm and application of derivatives we obtain
\begin{equation*}
\partial_F \partial_G \ln \prescript{}{A}{z}_{A+F+G} = - \partial_F \partial_G \ln \prescript{}{A+F+G}{z}_A ~\prescript{}{A}{z}_{A+G} ~\prescript{}{A}{z}_{A+F}.
\end{equation*}
Modifying the last factor by \eqref{z_temp_sep} yields
\begin{align*}
\partial_F \partial_G \ln \prescript{}{A}{z}_{A+F+G} = -\partial_F \partial_G \ln \prescript{}{A+F+G}{z}_A ~\prescript{}{A}{z}_{A+G} ~ \prescript{}{A+G}{z}_{A+F+G} = - \partial_F \partial_G \ln \Gamma_{A,A+G,A+F+G}.
\end{align*}
\qed

We come to the proof of lemma \ref{lemma:current_causal}. Let \(F\prec G\) be true. we start with the definition of the current:
\begin{align*}
j_A(G)= i \partial_G \left\langle  \bigwedge \Phi, \prescript{}{A}{\tilde{S}}_{A+G} \bigwedge \Phi\right\rangle
\end{align*}
Now we take the derivative of this expression and insert the definition of \(z\)
\begin{align*}
\partial_F j_{A+F}(G)=i \partial_F \partial_G \prescript{}{A+F}{z}_{A+F+G} \left\langle  \bigwedge \Phi, \prescript{}{A+F}{\overline{S}}_{A+F+G} \bigwedge \Phi\right\rangle.
\end{align*}
As the expression which we take derivatives of is equal to \(1\) at \(G=0\) and the linearisation of the logarithm around \(1\) is the identity we can safely insert a logarithm, yielding
\begin{align*}
\partial_F j_{A+F}(G)=i \partial_F \partial_G \ln \prescript{}{A+F}{z}_{A+F+G} \left\langle  \bigwedge \Phi, \prescript{}{A+F}{\overline{S}}_{A+F+G} \bigwedge \Phi\right\rangle.
\end{align*}
This will greatly simplify the upcoming calculations. Next we insert the relation between \(\Gamma_{A+F,A+F+G,A}\) and \(z\) using \(\prescript{}{A}{z}_{A+F}\) with respect to \(G\) vanishes, \eqref{gamma_to_z} giving
\begin{align*}
\partial_F j_{A+F}(G)=i \partial_F \partial_G \ln  \prescript{}{A}{z}_{A+F+G} \Gamma_{A+F,A+F+G,A} \left\langle  \bigwedge \Phi, \prescript{}{A+F}{\overline{S}}_{A+F+G} \bigwedge \Phi\right\rangle.
\end{align*}
Now we insert the identity twice inside the scalar product
\begin{align}\tag*{}
&\partial_F j_{A+F}(G)=i \partial_F \partial_G \ln  \prescript{}{A}{z}_{A+F+G} \Gamma_{A+F,A+F+G,A} \\
&\left\langle  \bigwedge \Phi, \prescript{}{A+F}{\overline{S}}_A~ \prescript{}{A}{\overline{S}}_{A+F}~ \prescript{}{A+F}{\overline{S}}_{A+F+G}~ \prescript{}{A+F+G}{\overline{S}}_{A} ~\prescript{}{A}{\overline{S}}_{A+F+G}  \bigwedge \Phi\right\rangle.
\end{align}
The central three occurrence of \(\overline{S}\) give \(\Gamma^{-1}_{A,A+F,A+F+G}\) cancelling exactly the gamma factor in front after cyclic permutation. As a next step we evaluate the scalar product. Since the operators \(\overline{S}\) are unitary this yields the argument of a determinant:
\begin{align*}
&\left \langle  \bigwedge \Phi, \prescript{}{A+F}{\overline{S}}_A ~\prescript{}{A}{\overline{S}}_{A+F+G}  \bigwedge \Phi\right\rangle\\
&=\left\langle  \bigwedge \Phi, \mathcal{L}_{\prescript{}{A+F}{S}_A} \mathcal{R}_{(\prescript{}{A}{S}_{A+F})_{--}} \mathcal{L}_{\prescript{}{A}{S}_{A+F+G}} \mathcal{R}_{(\prescript{}{A+F+G}{S}_A)_{--}} \bigwedge \Phi \right\rangle \frac{1}{N}\\
&=\left\langle  \bigwedge \Phi,  \mathcal{L}_{\prescript{}{A+F}{S}_A~\prescript{}{A}{S}_{A+F+G}} \mathcal{R}_{(\prescript{}{A+F+G}{S}_A)_{--}~(\prescript{}{A}{S}_{A+F})_{--}} \bigwedge \Phi \right\rangle \frac{1}{N}\\
&=\left\langle  \bigwedge \Phi,  \mathcal{L}_{\prescript{}{A+F}{S}_{A+F+G}} \bigwedge \Phi (\prescript{}{A+F+G}{S}_A)_{--}~(\prescript{}{A}{S}_{A+F})_{--} \right\rangle \frac{1}{N}\\
&=\text{argdet}((\prescript{}{A+F}{S}_{A+F+G})_{--}(\prescript{}{A+F+G}{S}_A)_{--}~(\prescript{}{A}{S}_{A+F})_{--}),
\end{align*}
which is, by \eqref{gamma_by_argdet}, given by
\begin{equation}
\left \langle  \bigwedge \Phi, \prescript{}{A+F}{\overline{S}}_A ~\prescript{}{A}{\overline{S}}_{A+F+G}  \bigwedge \Phi\right\rangle = \Gamma_{A,A+F,A+F+G}.
\end{equation}
Taking all of this together yields
\begin{align*}
&\partial_F j_{A+F}(G)=i \partial_F \partial_G \ln  \prescript{}{A}{z}_{A+F+G} \Gamma_{A,A+F,A+F+G}.
\end{align*}
Now we replace the appearance of \(z\) using lemma \ref{lemma:z_gamma_causal}, giving
\begin{equation}
\partial_F j_{A+F}(G)=i \partial_F \partial_G \ln \frac{\Gamma_{A,A+F,A+F+G}}{\Gamma_{A,A+G,A+F+G}}.
\end{equation}
\qed

Now for lemma \ref{lemma:current_compact}. We will show the first equality, the second follows by lemma \ref{lemma:z_gamma_causal}. For this proof we abbreviate, for variables \(a,b,c,\overline{a},\overline{b},\overline{b}\in \mathbb{R}\), \(x:=((a,\overline{a}),(b,\overline{b}),(c,\overline{c}))\)
\begin{equation}
f(x):=f((a,\overline{a}),(b,\overline{b}),(c,\overline{c})):=\ln \Gamma_{A+a\cdot F+\overline{a}\cdot G,A+b\cdot F+\overline{b}\cdot G,A+c\cdot F+\overline{c} \cdot G}.
\end{equation}
Now we are interested in 
\begin{align}\tag*{}
&\partial_F\partial_G\ln \Gamma_{A,A+F,A+F+G} 
= \partial_\varepsilon \partial_\delta f((0,0),(\varepsilon,0),(\varepsilon,\delta))\\ \label{current_compact_first_half}
& = (\partial_b + \partial_c)\partial_{\overline{c}} \left.f(x)\right|_{x=0}
 =\partial_b \partial_{\overline{c}} \left. f(x)\right|_{x=0}.
\end{align}
The last equality holds due to \(f((a,\overline{a}),(b,\overline{b}),(c,\overline{c}))=-f((b,\overline{b}),(a,\overline{a}),(c,\overline{c}))\), which implies
\begin{equation}
\partial_c \partial_{\overline{c}} f(x)|_{x=0}=-\partial_c \partial_{\overline{c}} f(x)|_{x=0}=0.
\end{equation}
For the very same reason we conclude
\begin{align*}
&-\partial_F \partial_G \ln \Gamma_{A,A+G,A+F+G} 
= - \partial_\varepsilon \partial_\delta f((0,0),(0,\delta),(\varepsilon,\delta))\\
&= - \partial_c (\partial_{\overline{b}} + \partial_{\overline{c}}) f(x)|_{x=0}
= - \partial_c \partial_{\overline{b}} f(x)|_{x=0}
= \partial_b \partial_{\overline{c}} f(x)|_{x=0},
\end{align*}
where the last equality follows again from the antisymmetry of \(f\). We conclude
\begin{equation}
-\partial_F \partial_G \ln \Gamma_{A,A+G,A+F+G} =\partial_F\partial_G\ln \Gamma_{A,A+F,A+F+G} .
\end{equation}
and thereby the lemma.\qed

We will now continue with the proof of theorem \ref{thm:current_explicit}. The major part of this proof is contained in the following auxiliary

\begin{lemma}\label{lemma:Aux}
For operators 
\begin{align*}
&A: \mathbb{R}\rightarrow \left( \mathcal{H}\rightarrow\mathcal{H}\right)\\
&B: \mathbb{R}^2 \rightarrow \left( \mathcal{H}\rightarrow\mathcal{H}\right),
\end{align*}
such that \(A(\varepsilon),B(\varepsilon,\delta)\in I_1\), \(A^*=A\),\(B(\varepsilon,0)=B^*(\varepsilon,0)\) and \(A(0)=0=B(\varepsilon,0)\) for all \(\varepsilon,\delta\in\mathbb{R}\) hold, 
\begin{equation}
\partial_\delta \partial_\varepsilon \ln \text{argdet}\left[ 1 + A(\varepsilon) + B(\varepsilon,\delta))\right] 
= i \Im \tr D_1 D_2 B(x)|_{x=0}
\end{equation}
is true.
\end{lemma}
We mention a corollary which we will not prove.
Since most of the time we work with the \(\text{arg}\) of complex numbers it is worth noting that
\begin{Corollary}\label{corollary:phase'}
\begin{equation}
\forall z:\mathbb{R}\rightarrow \mathbb{C}\backslash \{0\}: \left( \frac{z}{|z|}\right)'=  i \frac{z}{|z|} \Im \frac{ z'}{z}
\end{equation}
\end{Corollary}
holds. 

Proof of the lemma \ref{lemma:Aux} We use corollary \ref{corollary:phase'} to find
\begin{align*}
&\partial_\varepsilon \ln \text{argdet}\left[ 1 + A(\varepsilon) + B(\varepsilon,\delta))\right] 
= \frac{\partial_\varepsilon \text{argdet}[1+A(\varepsilon) + B(\varepsilon,\delta)]}{\text{argdet}[1+A(0)+B(0,\delta)}\\
&=i \frac{\text{argdet}[1+B(0,\delta)]}{\text{argdet}[1+B(0,\delta)]} \Im \left[  \frac{\partial_\varepsilon \det[1+A(\varepsilon)+B(\varepsilon,\delta)]}{\det[1+B(0,\delta)]}\right]\\
&=i \Im \left[ \frac{\partial_\varepsilon \det[1+A(\varepsilon)+B(\varepsilon,\delta)]}{\det[1+B(0,\delta)]}\right].
\end{align*}
Now we use that the linearisation of the determinant around the identity equal to the trace is. This yields
\begin{equation}
\partial_\varepsilon \ln \text{argdet}\left[ 1 + A(\varepsilon) + B(\varepsilon,\delta))\right] 
=i \Im \left[ \frac{ \tr [\partial_\varepsilon A(\varepsilon)+\partial_\varepsilon B(\varepsilon,\delta)]}{\det[1+B(0,\delta)]}\right].
\end{equation}
Inserting the second derivative simplifies the expression after one recognizes that the second summand inside the imaginary part is real, since its a product of derivatives of selfadjoint traceclass operators. The corresponding calculation is
\begin{align*}
&\partial_{\delta} \partial_{\varepsilon} \ln \text{argdet}\left[ 1 + A(\varepsilon) + B(\varepsilon,\delta))\right] \\
&=i \Im \left[ \tr \partial_{\delta}\partial_{\varepsilon} B(\varepsilon,\delta) - \frac{\tr[\partial_{\varepsilon}A(\varepsilon) + \partial_{\varepsilon} B(\varepsilon,0)]}{\det(1+B(0,0))^2} \tr[\partial_{\delta} B(0,\delta)]\right]\\
&=i \Im \left[ \tr \partial_{\delta}\partial_{\varepsilon} B(\varepsilon,\delta) - \tr[\partial_{\varepsilon}A(\varepsilon)] \tr[\partial_{\delta} B(0,\delta)]\right]\\
&=i \Im \left[ \tr \partial_{\delta}\partial_{\varepsilon} B(\varepsilon,\delta) \right],
\end{align*}
where we used that \(B(0,\delta)\) is selfadjoint and \(B(\varepsilon,0)=0\). This concludes the proof of the lemma.

So we start out with the most recent result about \(\partial_F j_{A+F}(G)\), use lemma \ref{lemma:gamma} and manipulate the appearing projections to bring it in a form that more explicitly has a determinant:
\begin{align*}
&\partial_F \partial_G \ln \Gamma_{A,A+G,A+F+G} =
\partial_F \partial_G \ln \text{argdet} \left( \prescript{}{A}{S}_{A+G}\right)_{--} \left( \prescript{}{A+G}{S}_{A+F}\right)_{--} \left( \prescript{}{A+F}{S}_{A}\right)_{--}\\
=\partial_F& \partial_G \ln \text{argdet} \left[ \left( \prescript{}{A}{S}_{A+F}\right)_{--}\left(\prescript{}{A+F}{S}_A\right)_{--} - \left( \prescript{}{A}{S}_{A+G}\right)_{-+} \left( \prescript{}{A+G}{S}_{A+F}\right)_{+-} \left( \prescript{}{A+F}{S}_{A}\right)_{--}\right]\\
=\partial_F \partial_G & \ln \text{argdet} \left[\mathds{1}_{--}- \left( \prescript{}{A}{S}_{A+F}\right)_{-+}\left(\prescript{}{A+F}{S}_A\right)_{+-} - \left( \prescript{}{A}{S}_{A+G}\right)_{-+} \left( \prescript{}{A+G}{S}_{A+F}\right)_{+-} \left( \prescript{}{A+F}{S}_{A}\right)_{--}\right].
\end{align*}
As we now take the derivative of a trace-class perturbation of the identity we can see that 1. this expression is well-defined, since the off 
diagonal components of the scattering matrix and its derivatives are Hilbert-Schmidt,  and 2. we can use the lemma we just proved. This results in 

\begin{align*}
&\partial_F \partial_G \ln \Gamma_{A,A+G,A+F+G} =
- i \partial_F \partial_G \Im \tr  \left( \prescript{}{A}{S}_{A+G}\right)_{-+} \left( \prescript{}{A+G}{S}_{A+F}\right)_{+-} \left( \prescript{}{A+F}{S}_{A}\right)_{--}\\
&=- i \Im \tr \left( \partial_G \prescript{}{A}{S}_{A+G}\right)_{-+} \left(\partial_F \prescript{}{A}{S}_{A+F}\right)_{+-},
\end{align*}
where the last equality follows by acknowledging that terms vanish if they contain a factor of \(\mathds{1}_{+-}\).
The theorem follows by inserting lemma \ref{lemma:current_compact}.\qed 

\vspace{1cm}
Proof of lemma \ref{lemma:nDerivativesSMatrix}. First we plug in the definition of \(S\) and use the product rule, resulting in
\begin{align}
\left(\prod_{k=1}^n \partial_{F_k} \right) S_{A+\sum_{k=1}^n F_k} 
= &\sum_{l=1}^n L^0_{F_l} \left( \prod_{\stackrel{k=1}{ k\neq l}}^n \partial_{F_k}\right) \Omega^+_{A+\sum_{\stackrel{k=1}{ k\neq l}}^n F_k}\\
&+ L^0_{A} \left( \prod_{k=1}^n \partial_{F_k}\right) \Omega^+_{A+\sum_{k=1}^n F_k},
\end{align}
Now since \(\Omega_A^\pm=(1-L_A^\pm)^{-1}\) holds we have \(\partial_F \Omega_{A+F}^\pm = \Omega^\pm_{A} L^\pm_F \Omega^\pm_A\). 
Applying this \(n\) times results in a sum over all permutations of the pertubative fields we are taking derivatives with respect to. Yielding
\begin{align}\tag*{}
\left(\prod_{k=1}^n \partial_{F_k} \right) S_{A+\sum_{k=1}^n F_k} 
= \sum_{l=1}^n &L^0_{F_l}\Omega^+_A  \sum_{\sigma \in S(\{1,\dots,n\}\backslash\{l\})} \prod_{\stackrel{k=1}{ k\neq l}}^n \left( L^+_{F_{\sigma(k)}} \Omega^+_{A}\right)\\
&+ L^0_{A} \Omega^+_A  \sum_{\sigma \in S(\{1,\dots,n\})} \prod_{k=1}^n \left(L^+_{F_{\sigma(k)}}\Omega^+_{A}\right).
\end{align}
Analogously for \(S^{-1}\) we find
\begin{align}\tag*{}
\left(\prod_{k=1}^n \partial_{F_k} \right) S^{-1}_{A+\sum_{k=1}^n F_k} 
= -\sum_{l=1}^n &L^0_{F_l}\Omega^-_A  \sum_{\sigma \in S(\{1,\dots,n\}\backslash\{l\})} \prod_{\stackrel{k=1}{ k\neq l}}^n \left( L^-_{F_{\sigma(k)}} \Omega^-_{A}\right)\\
&- L^0_{A} \Omega^-_A  \sum_{\sigma \in S(\{1,\dots,n\})} \prod_{k=1}^n \left(L^-_{F_{\sigma(k)}}\Omega^-_{A}\right).
\end{align}
\qed

Proof of theorem \ref{thm:GeneralDerivative} 
: Lemma \ref{lemma:nDerivativesSMatrix} and theorem 
\ref{thm:current_explicit} already provides most of the proof of this theorem. We use the abbreviations
\begin{align*}
\tilde{A}:= A+\sum_{k=1}^{n}F_k\\
\tilde{A}^0:= A+\sum_{k=0}^{n}F_k\\
\tilde{A}^{n+1}:= A+\sum_{k=1}^{n+1}F_k.
\end{align*}
 Starting with this theorem we have for any \(n\in\mathbb{N}\)
\begin{equation}
\left(\prod_{k=1}^{n+1} \partial_{F_k}\right) j_{\tilde{A}^{n+1}}(F_0) 
=-2\left(\prod_{k=1}^{n} \partial_{F_k}\right)  \Im \tr\left[ \left( \partial_{F_0} \prescript{}{\tilde{A}}{S}_{\tilde{A}^{0}}\right)_{-+} 
\left( \partial_{F_{n+1}} \prescript{}{\tilde{A}}{S}_{\tilde{A}^{n+1}}\right)_{+-}\right].
\end{equation}
Using definition/lemma \ref{Def:LOperator} we can expand the right hand side a bit further to 
\begin{align*}
=-2\left(\prod_{k=1}^{n} \partial_{F_k}\right)  \Im \tr&\left[ 
\left( \left(1-L^0_{\tilde{A}}\Omega^-_{\tilde{A}} \right)  \partial_{F_0}\left(1+ L^0_{\tilde{A}^{n+1}} \Omega^+_{\tilde{A}^{n+1}}\right)\right)_{-+}\right. \\
&\left.\left( \left(1-L^0_{\tilde{A}}\Omega^-_{\tilde{A}} \right)  \partial_{F_{n+1}}\left(1+ L^0_{\tilde{A}^{0}} \Omega^+_{\tilde{A}^0}\right)\right)_{+-}
\right].
\end{align*}
Now we need to distribute \(n\) derivatives over four terms. The possibilities of doing so are found by the analogous problem of distributing \(n\) 
labelled balls over four labelled boxes. The product rule for non commuting products tells us that we have to sum over these possibilities. 
We realise this sum by a sum over all possible distributions of the elements of \(\{1,\dots, n\}\) into four disjoint sets \(B,C,D,E\) whose union is again \(\{1,\dots, n\}\). This yields for the right hand side
\begin{align*}
=-2\sum_{\stackrel{B,C,D,E\subseteq \{1,\dots, n\}}{B\dot{\cup}C\dot{\cup}D\dot{\cup}E=\{1,\dots, n\}}}  \Im \tr&\left[ 
\left( \left(\prod_{l\in B} \partial_{F_l}\right)\left(1-L^0_{\tilde{A}}\Omega^-_{\tilde{A}} \right) 
\left(\prod_{l\in C} \partial_{F_l}\right) \partial_{F_0}\left(1+ L^0_{\tilde{A}^{n+1}} \Omega^+_{\tilde{A}^{n+1}}\right)\right)_{-+}\right. \\
&\left.\left( \left(\prod_{l\in D} \partial_{F_l}\right)\left(1-L^0_{\tilde{A}}\Omega^-_{\tilde{A}} \right) 
\left(\prod_{l\in E} \partial_{F_l}\right) \partial_{F_{n+1}}\left(1+ L^0_{\tilde{A}^{0}} \Omega^+_{\tilde{A}^0}\right)\right)_{+-}
\right].
\end{align*}
Applying the derivatives to the brackets and using once more the product rule yields the theorem. In doing so one should not forget that the \(L\) operators are linear, therefore yield zero when linearised twice with respect to different fields. \qed






\end{document}