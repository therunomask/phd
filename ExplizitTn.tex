\documentclass[oneside,reqno,12pt]{amsart}

%\usepackage{fontspec}

\usepackage[a4paper, top=2.7cm, bottom=2.7cm]{geometry}
%\usepackage[T1]{fontenc}
%\usepackage[utf8]{inputenc}
\usepackage{fontspec}
%\setmainfont{YuMincho}
%Hiragino Maru Gothic ProN
\usepackage{bbm}
\usepackage{graphicx}
\usepackage{slashed}
\usepackage{eurosym}
\usepackage{amsmath}
\usepackage{enumitem}
\usepackage{amsfonts}
\usepackage{longtable}
\usepackage[mathscr]{eucal}

\setcounter{secnumdepth}{5}

%commutative diagram
\usepackage{amsmath,amscd}
%picture
\usepackage{wrapfig}

\usepackage[unicode=true, pdfusetitle, bookmarks=true,
  bookmarksnumbered=false, bookmarksopen=false, breaklinks=true, 
  pdfborder={0 0 0}, backref=false, colorlinks=true, linkcolor=blue,
  citecolor=blue, urlcolor=blue]{hyperref}



% \numberwithin{equation}{section}
\allowdisplaybreaks[1]

\newtheorem{axiom}{Axiom}
\newtheorem{Def}{Definition}[section]
\newtheorem{Conj}[Def]{Conjecture}
\newtheorem{Thm}[Def]{Theorem}
\newtheorem{Prp}[Def]{Proposition}
\newtheorem{Lemma}[Def]{Lemma}
\newtheorem{lemma}{Lemma}
\newtheorem{Remark}[Def]{Remark}
\newtheorem{Corollary}[Def]{Corollary}
\newtheorem{Example}[Def]{Example}
\newtheorem{Assumption}[Def]{Assumption}

\newenvironment{mueq}
  {\equation\aligned}
  {\endaligned\endequation}
  
\DeclareMathOperator{\tr}{tr}
\DeclareMathOperator{\supp}{supp}


\newcommand{\Z}[2]{Z_{\stackrel{1}{#1}}\left(#2\right)}
\newcommand{\id}{{\mathbbm 1}}
\newcommand{\equaltext}[1]{\ensuremath{\stackrel{\text{#1}}{=}}}
\newcommand{\letext}[1]{\ensuremath{\stackrel{\text{#1}}{\le}}}
\newcommand{\Conv}{\mathop{\scalebox{1.7}{\raisebox{-0.2ex}{\(\ast\)}}}}
\newcommand{\CONV}{\mathop{\scalebox{3.0}{\raisebox{-0.2ex}{\(\ast\)}}}}
% Annotations
%\usepackage[normalem]{ulem}
% \usepackage{refcheck}
\usepackage[colorinlistoftodos,shadow,textsize=scriptsize,textwidth=2.75cm]{todonotes}
\newcommand{\Dirk}[1]{ \todo[color=orange!60]{Dirk: #1} }
\newcommand{\DirkBox}[1]{ \mbox{}\todo[inline,caption={},color=red!60]{Dirk: #1} }
\newcommand{\Markus}[1]{ \todo[color=green!20]{Markus: #1} }
\newcommand{\dirk}{ \color{orange} }
\newcommand{\markus}{ \color{green} }
\newcommand{\noch}[1]{ \todo[color=blue!20]{Todo: #1} }
\newcommand{\black}{ \color{black} }

\makeatletter



\renewcommand\section{\@startsection {section}{1}{\z@}%
                                   {-2.0ex \@plus -1ex \@minus -.2ex}%
                                   {2.3ex \@plus.2ex}%
                                   {\normalfont\Large\bfseries}}
\renewcommand\subsection{\@startsection {subsection}{1}{\z@}%
                                   {-0.5ex \@plus -0.5ex \@minus -.2ex}%
                                   {0.5em}%
                                   {\normalfont\bfseries}}
\renewcommand\subsubsection{\@startsection {subsubsection}{1}{\z@}%
                                   {-0.3ex \@plus -0.4ex \@minus -.2ex}%
                                   {0.1 em}%
                                   {\normalfont\sc}}  
\renewcommand\paragraph{\@startsection {paragraph}{1}{\z@}%
                                   {-0.2ex \@plus -1ex \@minus -.2ex}%
                                   {0.1 em}%
                                   {\normalfont\it}}                                   
\makeatother

\parindent 0cm

\begin{document}
Let \((\varphi_l)_{l\in\mathbb{Z}/\{0\}}=:B \subset \mathcal{H}\) , \((\varphi_l)_{l\in\mathbb{N}}=:B_+ \subset \mathcal{H}^+\) 
and \((\varphi_k)_{k\in -\mathbb{N}}=:B_-\subset \mathcal{H}^-\) be an ONB of all of, respectively positive respectively negative part of the Hilbert space. 

We investigate linear operators on Fockspace which fulfill the following commutation relations
\begin{align}
\left[T_m(A) , a(\phi)\right]= a\left(Z_m (A) \phi \right) + 1_{[2,\infty[}(m)\sum_{j=1}^{m-1} \begin{pmatrix} m \\ j \end{pmatrix} a\left(Z_j (A) \phi \right)\circ T_{m-j}(A), \label{Tk_commutation_relation_annihilator}\\
\left[T_m(A) , a^*(\phi)\right]= a^*\left(Z_m (A) \phi \right) + 1_{[2,\infty[}(m)\sum_{j=1}^{m-1} \begin{pmatrix} m \\ j \end{pmatrix} a^*\left(Z_j (A) \phi \right)\circ T_{m-j}(A),\label{Tk_commutation_relation_creator}
.\end{align}
We also know that the following equation is fulfilled for all \(n\in\mathbb{N}\), due to the unitarity of the scattering operator.
\begin{equation}\label{unitarity}
\sum_{k=0}^n \begin{pmatrix}n \\ k\end{pmatrix} T_{k}^* T_{n-k}=0=\sum_{k=0}^n \begin{pmatrix}n \\ k\end{pmatrix} T_k T_{n-k}^*
.\end{equation}

In this document\todo{edit this sentence when copying} we guess the explicit form of the first few of these operators directly and normalize such that they have the correct vacuum expectation value. The correct vacuum expectation value is inferred by other arguments not presented here.

The following construction will be occurring at every order, so we introduce it now. Denote by \(Q\) the following set \(Q:=\{f: \mathcal{H}\rightarrow \mathcal{H} \text{ linear} \mid i \cdot f \text{ is selfadjoint}\}\).
\begin{Def}
 Let \(G\) be the following function
\begin{align*}\tag{Def G} \label{Def G}
G: \quad & Q\rightarrow \left( \mathcal{F}\rightarrow \mathcal{F} \right)\\
& f\mapsto \sum_{n\in\mathbb{N}}a^*(f \varphi_n) a(\varphi) - \sum_{n\in -\mathbb{N}} a(\varphi_n) a^*(f \varphi_n).
\end{align*}
\end{Def}

\begin{Lemma}
Every image of the map \(G\) has vanishing vacuum expectation value and fulflls the following commutation relation. 
\begin{equation}\label{Commutation G}
\forall \alpha \in \mathcal{H},\forall f\in Q: \left[ G(f), a^{\#} (\alpha)\right]=a^{\#} \left( G(f)\alpha\right)
,\end{equation}
where \(a^{\#}\) is either \(a\) or \(a^*\).
\end{Lemma}
{\bf Proof:} The vanishing expectation value of \(G(f)\) can be read off directly from \eqref{Def G}. The rest of the proof consists of short calculations, which uses the anticommutation relations of the creation and annihilation operators. Let \(f\in Q\),\(\alpha\in \mathcal{H}\) be given. We divide the proof into two cases.
\begin{enumerate}
\item[Case \(a^\#=a\):] In this case we get
\begin{align*}
&\left[ G(f), a^{\#} (\alpha)\right]= \left[ \sum_{n\in\mathbb{N}}a^*(f \varphi_n) a(\varphi) - \sum_{n\in -\mathbb{N}} a(\varphi_n) a^*(f \varphi_n), a(\alpha)\right]\\
&=\left[ \sum_{n\in\mathbb{N}}a^*(f \varphi_n) a(\varphi), a(\alpha)\right] - \left[\sum_{n\in -\mathbb{N}} a(\varphi_n) a^*(f \varphi_n), a(\alpha)\right]\\
&= -\sum_{n\in\mathbb{N}} \left\{ a^*(f \varphi_n) , a(\alpha)\right\}a(\varphi)
-\sum_{n\in -\mathbb{N}}  a(\varphi_n) \left\{  a^*(f \varphi_n), a(\alpha)\right\}\\
&= -\sum_{n\in\mathbb{Z}} \left< \alpha, f \varphi_n\right> a(\varphi_n)
\stackrel{f=-f^*}{=}\sum_{n\in\mathbb{Z}} \left< f \alpha, \varphi_n\right> a(\varphi_n)= a\left( \sum_{n\in\mathbb{Z}} \varphi_n \left<\varphi_n,f \alpha \right> \right)\\
&= a(f\alpha).
\end{align*}

\item[Case \(a^\#=a^*\):] In this case we get 
\begin{align*}
&\left[ G(f), a^{\#} (\alpha)\right]
= \left[ \sum_{n\in\mathbb{N}}a^*(f \varphi_n) a(\varphi) - \sum_{n\in -\mathbb{N}} a(\varphi_n) a^*(f \varphi_n), a^*(\alpha)\right]\\
&=\left[ \sum_{n\in\mathbb{N}}a^*(f \varphi_n) a(\varphi), a^*(\alpha)\right] - \left[\sum_{n\in -\mathbb{N}} a(\varphi_n) a^*(f \varphi_n), a^*(\alpha)\right]\\
&= \sum_{n\in\mathbb{N}} a^*(f \varphi_n) \left\{ a( \varphi_n) , a^*(\alpha)\right\}
+\sum_{n\in -\mathbb{N}}   \left\{  a( \varphi_n), a^*(\alpha)\right\}a^*(f \varphi_n)\\
&= \sum_{n\in\mathbb{Z}} \left< f\alpha,  \varphi_n\right> a^*(\varphi_n)
= a^*\left( \sum_{n\in\mathbb{Z}} \varphi_n \left<\varphi_n,f \alpha \right> \right)\\
&= a^*(f\alpha),
\end{align*}
which ends the proof.
\end{enumerate}

\section{\(T_1\)}

The linearisation of the scattering operator \(T_1\) is supposed to fulfil \eqref{Tk_commutation_relation_annihilator} and \eqref{Tk_commutation_relation_creator} for \(n=1\), which coincides with \eqref{Commutation G} for \(f=Z_1\). This operator is antiselfadjoint, so it is in the domain of definition of \(G\).

\section{\(T_2\)}

The commutation relations for the second order can be put into the form 
\begin{equation}
\forall \alpha \in \mathcal{H} \left[ T_2, a^{\#}(\alpha)\right]= a^{\#}\left( Z_2 \alpha \right) + 2 a^{\#} \left( Z_1 \alpha \right) T_1,
\end{equation}
where \(a^{\#}\) is either \(a\) or \(a^*\). We compute the corresponding commutation relation for \(G(Z_2-Z_1Z_1)+T_1T_1\), let  \(a^{\#}\) be either \(a\) or \(a^*\) and \(\alpha\in\mathcal{H}\) be given. The anti selfadjointness of \(Z_2-Z_1Z_1\) can be directly checked using \eqref{unitarity}.  We get
\begin{align*}
&\left[G(Z_2-Z_1Z_1)+T_1T_1, a^{\#}(\alpha)\right] = \left[G(Z_2-Z_1Z_1), a^{\#}(\alpha)\right] +T_1\left[T_1, a^{\#}(\alpha)\right] +\left[T_1, a^{\#}(\alpha)\right] T_1\\
&= a^{\#}\left( (Z_2-Z_1Z_1)\alpha\right) + T_1 a^{\#}\left(Z_1 \alpha\right) + a^{\#}\left( Z_1 \alpha \right) T_1\\
&=a^{\#}\left( Z_2\alpha\right)-a^{\#}\left( Z_1Z_1\alpha\right) + \left[ T_1,a^{\#}\left(Z_1 \alpha\right)\right] + a^{\#}\left(Z_1 \alpha\right)T_1+a^{\#}\left(Z_1 \alpha\right)T_1\\
&=a^{\#}\left( Z_2\alpha\right)+2a^{\#}\left(Z_1 \alpha\right)T_1
.\end{align*}
So we found the operator \(T_2\) up to a multiple of the identity. This ambiguity is fixed by the condition 
\begin{equation}
\left< \Omega, T_2 \Omega \right>=\langle T_2 \rangle,
\end{equation}
where \(\langle T_2\rangle\) is fixed by other conditions. For our candidate we get the following vacuum expectation value
\begin{align*}
&\langle \Omega, G(Z_2-Z_1Z_1)+T_1T_1\Omega\rangle 
= \langle \Omega, T_1T_1 \Omega\rangle \\
&=-\langle \Omega, \sum_{n\in \mathbb{N}} \sum_{k \in -\mathbb{N}} a^*\left(Z_{\stackrel{1}{-+}}\varphi_n\right) a(\varphi_n) a(\varphi_k) a^* \left( Z_{\stackrel{1}{+-}} \varphi_k\right)\Omega \rangle\\
&=\langle \Omega, \sum_{k\in-\mathbb{N}}a^*\left(Z_{\stackrel{1}{-+}}Z_{\stackrel{1}{+-}}  \varphi_k\right)=\tr \left(Z_{\stackrel{1}{-+}}Z_{\stackrel{1}{+-}} \right).
\end{align*}

This leads us to the correct representation of \(T_2\)
\begin{equation}
T_2=G(Z_2-Z_1Z_1)+T_1T_1 -\tr \left(Z_{\stackrel{1}{-+}}Z_{\stackrel{1}{+-}}\right) + \langle T_2\rangle.
\end{equation}

\section{\(T_3\)}
The commutation relations for the third order can be put into the form 
\begin{equation}\label{T3 commutation}
\forall \alpha \in \mathcal{H} \left[ T_3, a^{\#}(\alpha)\right]= a^{\#}\left( Z_3 \alpha \right) + 3 a^{\#} \left( Z_1 \alpha \right) T_2+3 a^{\#} \left( Z_2 \alpha \right) T_1,
\end{equation}
We will divide the calculation into parts, since it gets quite lengthy otherwise. Our initial guess for this operator is 
\begin{equation}
T_3'=G\left( Z_3 - \frac{3}{2}Z_2Z_1 - \frac{3}{2}Z_1Z_2 + 2 Z_1Z_1Z_1\right)+ \frac{3}{2}T_2T_1 + \frac{3}{2} T_1T_2 - 2 T_1T_1T_1 
.\end{equation}
We begin by checking that the argument of \(G\) is anti selfadjoint. 
\begin{align*}
&\left( Z_3 - \frac{3}{2}Z_2Z_1 - \frac{3}{2}Z_1Z_2 + 2 Z_1Z_1Z_1\right)^*
=Z_3^*-  \frac{3}{2}Z^*_2 Z^*_1 - \frac{3}{2}Z^*_1 Z^*_2 + 2 Z^*_1 Z^*_1 Z^*_1\\
&\stackrel{\eqref{unitarity}}{=}
(-Z_3 -3Z_2^*Z_1 - 3 Z_1^* Z_2 )+ \frac{3}{2}(-Z_2 + 2Z_1Z_1)Z_1 + \frac{3}{2}Z_1(-Z_2+2Z_1Z_1) - 2 Z_1Z_1Z_1\\
&\stackrel{\eqref{unitarity}}{=}
-Z_3 -3(-Z_2+2Z_1Z_1)Z_1 + 3 Z_1 Z_2 - \frac{3}{2}Z_2 Z_1 - \frac{3}{2}Z_1Z_2 + 4 Z_1Z_1Z_1\\
&=-Z_3  + \frac{3}{2}Z_2 Z_1 +\frac{3}{2}Z_1Z_2 -2 Z_1Z_1Z_1
=-\left( Z_3 - \frac{3}{2}Z_2Z_1 - \frac{3}{2}Z_1Z_2 + 2 Z_1Z_1Z_1\right)
\end{align*}
holds. Since we already know the commutation relation for \(G\) the next step is computing the commutation relations for the terms containing \(T_2\). Let \(\alpha\in\mathcal{H}\) be given and \(a^{\#}\) be \(a\) or \(a^*\), the commutation relation is
\begin{align*}
&\left[\frac{3}{2} T_2 T_1 +\frac{3}{2} T_1 T_2, a^{\#}(\alpha)\right]
=\left[\frac{3}{2} T_2 T_1, a^{\#}(\alpha)\right]+\left[\frac{3}{2} T_1 T_2, a^{\#}(\alpha)\right]\\
&=\frac{3}{2} T_2 \left[ T_1, a^{\#}(\alpha)\right]+\frac{3}{2}  \left[ T_2, a^{\#}(\alpha)\right]T_1+\frac{3}{2} T_1\left[ T_2, a^{\#}(\alpha)\right]+\frac{3}{2} \left[ T_1, a^{\#}(\alpha)\right]T_2\\
&=\frac{3}{2} T_2 a^{\#}\left(Z_1\alpha\right)+\frac{3}{2}  \left(a^{\#}\left(Z_2\alpha\right)+2a^{\#}\left(Z_1\alpha\right)T_1\right)T_1+\\ 
&\quad \frac{3}{2} T_1\left( a^{\#}\left(Z_2 \alpha\right)+2a^{\#}\left(Z_1 \alpha\right)T_1\right)+\frac{3}{2}  a^{\#}\left(Z_1 \alpha\right)T_2\\
&=\frac{3}{2}  a^{\#}\left(Z_1\alpha\right)T_2 
+ \frac{3}{2} a^{\#}\left(Z_2 Z_1 \alpha \right) 
+3 a^{\#}\left(Z_1Z_1 \alpha \right) T_1
+\frac{3}{2}a^{\#}\left(Z_2\alpha\right)T_1
+3a^{\#}\left(Z_1\alpha\right)T_1T_1+\\ 
& \frac{3}{2}  a^{\#}\left(Z_2 \alpha\right)T_1
+\frac{3}{2}  a^{\#}\left(Z_1 Z_2 \alpha\right)
+3a^{\#}\left(Z_1 \alpha\right)T_1 T_1
+3a^{\#}\left(Z_1 Z_1 \alpha\right) T_1
+\frac{3}{2}  a^{\#}\left(Z_1 \alpha\right)T_2\\
&=3  a^{\#}\left(Z_1\alpha\right)T_2 \tag{\(T_2T_1\)}\label{T_2T_1}
+3a^{\#}\left(Z_2\alpha\right)T_1
+\frac{3}{2}  a^{\#}\left(Z_1 Z_2 \alpha\right)
+ \frac{3}{2} a^{\#}\left(Z_2 Z_1 \alpha \right) \\
&+6 a^{\#}\left(Z_1Z_1 \alpha \right) T_1
+6a^{\#}\left(Z_1\alpha\right)T_1T_1
.\end{align*}
The terms in the last line do not correspond to \(G\), nor to the original commutator of \(T_3\), so we hope they cancel with terms of the \(T_1T_1T_1\) summand of \(T_3'\). Let \(\alpha\in\mathcal{H}\) be given and \(a^{\#}\) be \(a\) or \(a^*\), indeed
\begin{align*}
&-2\left[ T_1T_1T_1,a^{\#}(\alpha)\right]
=-2T_1T_1\left[ T_1,a^{\#}(\alpha)\right]
-2T_1\left[ T_1,a^{\#}(\alpha)\right]T_1
-2\left[ T_1,a^{\#}(\alpha)\right]T_1T_1\\
&=-2T_1T_1a^{\#}(Z_1\alpha)
-2T_1a^{\#}(Z_1\alpha)T_1
-2a^{\#}(T_1\alpha)T_1T_1\\
&=-2T_1a^{\#}(Z_1\alpha)T_1
-2T_1a^{\#}(Z_1 Z_1\alpha)
-2a^{\#}(Z_1\alpha)T_1T_1
-2a^{\#}(Z_1Z_1\alpha)T_1
-2a^{\#}(T_1\alpha)T_1T_1\\
&=-2a^{\#}(Z_1\alpha)T_1T_1
-2a^{\#}(Z_1Z_1\alpha)T_1
-2a^{\#}(Z_1 Z_1\alpha)T_1
-2a^{\#}(Z_1Z_1 Z_1\alpha)\\
&-2a^{\#}(Z_1\alpha)T_1T_1
-2a^{\#}(Z_1Z_1\alpha)T_1
-2a^{\#}(T_1\alpha)T_1T_1\\
&=-6a^{\#}(Z_1\alpha)T_1T_1\label{T_1T_1T_1}\tag{\(T_1T_1T_1\)}
-6a^{\#}(Z_1Z_1\alpha)T_1
-2a^{\#}(Z_1Z_1 Z_1\alpha)
.\end{align*}
holds. Adding up equations \eqref{T_2T_1}, \eqref{T_1T_1T_1} and the commutator of \\
 \(G\left( Z_3 - \frac{3}{2}Z_2Z_1 - \frac{3}{2}Z_1Z_2 + 2 Z_1Z_1Z_1\right)\) results exactly in the right hand side of \eqref{T3 commutation}. 
 
 What about the vacuum expectation value? The abstract argument should also hold for \(T_1^3\), but \(\langle \Omega, T_1^3 \Omega\rangle\neq 0\) holds! To obtain \(T_3\) we need to adjust the vacuum expectation value. In order to do so, we first compute the vacuum expectation value of \(T_3'\):
 \begin{align*}
 &\langle \Omega, T_3' \Omega\rangle= \frac{3}{2} \langle \Omega, T_2T_1 \Omega \rangle + \frac{3}{2} \langle \Omega T_1 T_2 \Omega \rangle - 2 \langle \Omega. T_1 T_1 T_1 \Omega \rangle\\
 &=\frac{3}{2} \langle \Omega, G\left( Z_2-Z_1Z_1\right)T_1\Omega \rangle +\frac{3}{2} \langle \Omega, T_1 G \left( Z_2-Z_1Z_1\right) \Omega \rangle + \langle \Omega, T_1 T_1 T_1 \Omega \rangle\\
 &=- \frac{3}{2} \sum_{k\in \mathbb{N}} \sum_{l\in -\mathbb{N}} \langle \Omega, a^*\left( Z_2-Z_1Z_1 \varphi_k\right)a(\varphi_k)a(\varphi_l) a^*\left(Z_1\varphi_l\right) \Omega \rangle \\
 &-\frac{3}{2} \sum_{k\in \mathbb{N}} \sum_{l\in -\mathbb{N}} \langle \Omega, a^*\left( Z_1 \varphi_k\right)a(\varphi_k)a(\varphi_l) a^*\left(Z_2-Z_1Z_1\varphi_l\right) \Omega \rangle \\
 &- \sum_{k\in\mathbb{N}}\sum_{l\in-\mathbb{N}} \sum_{c\in\mathbb{N}} \langle \Omega, a^*\left( Z_1 \varphi_k\right) a(\varphi_k) \left( a^*\left( Z_1 \varphi_c\right) a(\varphi_c) - a(\varphi_{-c}) a^*\left( Z_1 \varphi_{-c}\right) \right) a(\varphi_l) a^*\left( Z_1 \varphi_l\right) \Omega \rangle \\
 &= -\frac{3}{2} \tr \left(Z_1Z_1 Z_{\stackrel{1}{+-}} \right)
 -\frac{3}{2} \tr \left( Z_{\stackrel{1}{-+}}Z_1 Z_1 \right) 
 + \tr \left( Z_{\stackrel{2}{-+}} Z_{\stackrel{1}{+-}}\right) 
 +\tr \left(  Z_{\stackrel{1}{-+}}Z_{\stackrel{2}{+-}}\right) \\
  &- \sum_{k\in\mathbb{N}}\sum_{l\in-\mathbb{N}} \langle \Omega,  a(\varphi_k) a^*\left( Z_1 Z_{\stackrel{1}{-+}}\varphi_{k}\right) a(\varphi_l) a^*\left( Z_1 \varphi_l\right) \Omega \rangle \\
   &+ \sum_{k\in\mathbb{N}}\sum_{l\in-\mathbb{N}} \langle \Omega, a^*\left( Z_1 \varphi_k\right) a(\varphi_k) a^*\left( Z_1 Z_{\stackrel{1}{+-}} \varphi_l\right)   a(\varphi_l) \Omega \rangle \\
&= -\frac{3}{2} \tr \left(Z_1Z_1 Z_{\stackrel{1}{+-}} \right)
 -\frac{3}{2} \tr \left( Z_{\stackrel{1}{-+}}Z_1 Z_1 \right) 
 + \tr \left( Z_{\stackrel{2}{-+}} Z_{\stackrel{1}{+-}}\right) 
 +\tr \left(  Z_{\stackrel{1}{-+}}Z_{\stackrel{2}{+-}}\right) \\
&- \tr\left( Z_{\stackrel{1}{+-}} Z_{\stackrel{1}{--}} Z_{\stackrel{1}{-+}}\right)  
+ \tr\left(Z_{\stackrel{1}{-+}} Z_{\stackrel{1}{++}} Z_{\stackrel{1}{+-}}  \right)  \\
&=-4 \tr\left( Z_{\stackrel{1}{+-}} Z_{\stackrel{1}{--}} Z_{\stackrel{1}{-+}}\right)  
-2 \tr\left(Z_{\stackrel{1}{-+}} Z_{\stackrel{1}{++}} Z_{\stackrel{1}{+-}}  \right)  
 + \tr \left( Z_{\stackrel{2}{-+}} Z_{\stackrel{1}{+-}}\right) 
 +\tr \left(  Z_{\stackrel{1}{-+}}Z_{\stackrel{2}{+-}}\right) \\
 \end{align*}
 
 \section{\(T_4\)}
Let \(b \in\mathbb{R}\) be arbitrary we find that there is a \(C\in\mathbb{C}\) such that \(T_4\) is given by
\begin{align}\tag*{}
&T_4:= 2 T_1T_3 + 2 T_3 T_1 + 3 T_2 T_2 - b T_1 T_1 T_2 - b T_2 T_1 T_1 - 2 ( 6-b) T_1 T_2 T_1 \\\label{explicitT4}
&+ 6 T_1 T_1 T_1 T_1+G\left(Z_4-2Z_1Z_3-2Z_3Z_1-3Z_2Z_2\right.\\\tag*{}
&\left.+bZ_1^2Z_2+2(6-b)Z_1Z_2Z_1+bZ_2Z_1^2-6Z_1^4 \right)  + C
.\end{align}
This can be seen as follows. We know the commutator of \(T_4\) to be
\begin{equation}
\left[ T_4 , a^\# (\alpha) \right]
=a^\# \left( Z_4 \alpha\right) + 4 a^\# \left( Z_1\alpha\right) T_3 + 6 a^\#\left( Z_2\alpha\right) T_2 + 4 a^\# \left( Z_3 \alpha\right) T_1.
\end{equation}
We would like to subtract from \(T_4\) linear combinations of  \(T_3T_1\), \(T_1T_3\), \(T_2T_2\), \(T_2T_1^2\), \(T_1T_2T_1\), \(T_1^2T_2\), \(T_1^4\) such that the remainder on the right hand only involves a sum of creation/annihilation operators. We group the first three summands together with binomial coefficients, since those are needed to produce an anti selfadjoint one-particle operator inside the creation/annihilation operator on the right hand side of the commutation relation. So we compute the commutation relation of these summands. These can be found to be
\begin{align}\tag*{}
&\left[ 2 T_1 T_3 + 2 T_3 T_1 + 3 T_2 T_2 ,a^\#(\alpha)\right] = 6 a^\#\left( Z_2 \alpha\right)T_2 + 4 a^\#\left(Z_3 \alpha\right) T_1 + 4 a^\#\left( Z_1\alpha\right) T_3\\\tag*{}
&+2 a^\#\left(Z_1 Z_3 \alpha \right)  + 2 a^\#\left(Z_3 Z_1 \alpha \right) +3 a^\#\left(Z_2 Z_2 \alpha \right) \\\tag*{}
&+12 a^\#\left(Z_1 \alpha\right) T_2 T_1+12 a^\#\left(Z_1 \alpha\right) T_1 T_2
+12 a^\#\left(Z_1Z_1 \alpha\right) T_2+12 a^\#\left(Z_2 \alpha\right) T_1T_1\\
&+12 a^\#\left(Z_1Z_2 \alpha\right) T_1+12 a^\#\left(Z_2Z_1 \alpha\right) T_1
+12 a^\#\left(Z_1Z_1 \alpha\right) T_1T_1
,\end{align}
\begin{align}\tag*{}
&\left[T_1^2 T_2, a^\#(\alpha)\right]= a^\#\left(Z_1^2 Z_2 \alpha\right) 
+a^\#\left(Z_1^2 \alpha\right)T_2+2 a^\#\left(Z_1 Z_2 \alpha\right)T_1
+2 a^\#\left(Z_1^3 \alpha\right)T_1\\
&+4 a^\#\left(Z_1^2 \alpha\right)T_1^2 
+2 a^\#\left(Z_1  \alpha\right)T_1T_2+ a^\#\left(Z_2 \alpha\right)T_1^2
+ 2 a^\#\left(Z_1 \alpha\right)T_1^3
,\end{align}
\begin{align}\tag*{}
&\left[T_1 T_2 T_1, a^\#(\alpha)\right]= a^\#\left(Z_1 Z_2 Z_1 \alpha\right) \\
&+a^\#\left(Z_1^2 \alpha\right)T_2+ a^\#\left(Z_1 Z_2 \alpha\right)T_1
+ a^\#\left(Z_2 Z_1 \alpha\right)T_1
+2 a^\#\left(Z_1^3 \alpha\right)T_1\\\tag*{}
&+ a^\#\left(Z_1  \alpha\right)T_1T_2+ a^\#\left(Z_1  \alpha\right)T_2T_1
+ a^\#\left(Z_2 \alpha\right)T_1^2
+4 a^\#\left(Z_1^2 \alpha\right)T_1^2+ 2 a^\#\left(Z_1 \alpha\right)T_1^3
,\end{align}
\begin{align}\tag*{}
&\left[T_2T_1^2 , a^\#(\alpha)\right]= a^\#\left(Z_2Z_1^2  \alpha\right) 
+a^\#\left(Z_1^2 \alpha\right)T_2+2 a^\#\left(Z_2 Z_1 \alpha\right)T_1
+2 a^\#\left(Z_1^3 \alpha\right)T_1\\
&+2 a^\#\left(Z_1  \alpha\right)T_2T_1+ a^\#\left(Z_2 \alpha\right)T_1^2
+4 a^\#\left(Z_1^2 \alpha\right)T_1^2
+ 2 a^\#\left(Z_1 \alpha\right)T_1^3
,\end{align}
\begin{align}
&\left[T_1^4 , a^\#(\alpha)\right]= a^\#\left(Z_1^4  \alpha\right) 
+4 a^\#\left(Z_1^3 \alpha\right)T_1+6 a^\#\left( Z_1^2 \alpha\right)T_1^2
+4 a^\#\left(Z_1 \alpha\right)T_1^3
.\end{align}
Demanding the equation
\begin{align}\tag*{}
&\left[ T_4 + 2 T_1 T_3 + 2 T_3 T_1 + 3 T_2 T_2+ c_1 T_1^2 T_2 \right.\\
 &\left. + c_2 T_1T_2T_1+ c_3 T_2 T_1^2 + c_4 T_1^4,a^\#(\alpha)\right]\stackrel{!}{=} a^\#\left( F \alpha\right),
\end{align}
where \(c_1,c_2,c_3,c_4\in\mathbb{R}\) and \(F\) is an anti selfadjoint one-particle operator, we arrive at \eqref{explicitT4}.

\end{document}

