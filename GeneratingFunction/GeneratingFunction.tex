\documentclass[a4paper,11pt]{article}

%\usepackage{german}

\usepackage[dvipsnames]{xcolor}
\usepackage{graphicx}

\usepackage{amssymb}

\usepackage{amsfonts}

\usepackage{amsmath}

\usepackage{amsthm}
\usepackage{mathabx}


\usepackage[unicode=true, pdfusetitle, bookmarks=true,
  bookmarksnumbered=false, bookmarksopen=false, breaklinks=true, 
  pdfborder={0 0 0}, backref=false, colorlinks=true, linkcolor=blue,
  citecolor=blue, urlcolor=blue]{hyperref}
\usepackage{slashed}
\usepackage{authblk}
%identity sign

\usepackage{cancel}
\usepackage{dsfont}
\usepackage{todonotes}

\setlength{\marginparwidth}{2.6cm}

%commutative diagrams
\usepackage{amsmath,amscd}
\usepackage{enumitem}
\usepackage{moreenum}

\addtolength{\textwidth}{2.2cm} \addtolength{\hoffset}{-1.0cm}

\addtolength{\textheight}{3.0cm} \addtolength{\voffset}{-2cm} 

\parindent 0cm

\pagestyle{empty}

\begin{document}
\title{Calculation for Generating Function}
\author{ }
\date{\today}
\maketitle


Yesterday we arrived at the following expression for 
the generating function of 
the relevant part of \(c^+\)
\begin{align}
G=\int_{\big(\mathbb{R}^+\big)^{2n}}dy ~i \bigg( \frac{\pi}{\sum_{j=1}^{2n}y_j}\bigg)^{d/2}
\exp\Bigg(\sum_{j=1}^{2n}y_j(q_j^2-m^2)-\frac{\big(\sum_{j=1}^{2n}y_j q_j + i/2 \xi\big)^2}{\sum_{j=1}^{2n}y_j}\Bigg).
\end{align}
For this expression we can further treat \(\sum_{j=1}^{2n}y_j=t\) 
as an independent variable and perform its integral. For this (and the next few steps) 
we introduce the following abbreviations
\begin{align}
  t&:=\sum_{j=1}^{2n}y_j\\
  z_j&:=y_j/t\\
  \overline{q}&:=\sum_{j=1}^{2n} z_j q_j\\
  \overline{q^2}&:=\sum_{j=1}^{2n} z_j q_j^2\\
  \lambda&:=\sqrt{m^2-\overline{q^2}+\overline{q}^2}.
\end{align}
For the following calculations we need 
\(\text{Re}(\lambda),-\text{Re}(\xi^2)\ge 0\). 
We then arrive at 
\begin{align}
  G=\frac{i}{\Gamma(2n)}E_z\Bigg[\int_{\mathbb{R}^+}dt ~t^{2n-1}\bigg(\frac{\pi}{t}\bigg)^{d/2} 
  \exp\bigg(-t(m^2-\overline{q^2}+\overline{q}^2)-i\xi \overline{q}+\frac{\xi^2}{4 t}\bigg)\Bigg].
\end{align}
This can be brought into a  
\href{https://dlmf.nist.gov/10.32#E10}{known integral expression}
for the 
 modified Bessel function.
\begin{align}
  G=\frac{i \pi^{d/2}}{\Gamma(2n)}E_z\Bigg[e^{-i \xi \overline{q}}\lambda^{d-4n}\int_{\mathbb{R}^+}\frac{d\tau}{\tau^{d/2-2n +1}}e^{-\tau - \frac{-\xi^2 \lambda^2}{4 \tau}} \Bigg]\\
  =\frac{i \pi^{d/2}}{\Gamma(2n)}E_z\bigg[e^{-i\xi \overline{q}} \lambda^{d-4n}2 (\lambda \sqrt{-\xi^2}/2)^{2n-d/2}K_{d/2-2n}(\lambda \sqrt{-\xi^2}) \bigg]\\
  \overset{K_\nu(z)=K_{-\nu}(z)}{=}\frac{i \pi^{d/2}}{\Gamma(2n)}2^{1-2n+d/2}E_z\Bigg[e^{-i\xi\overline{q}}\bigg(\frac{\sqrt{-\xi^2}}{\lambda}\bigg)^{2n-d/2}K_{2n-d/2}(\lambda\sqrt{-\xi^2}) \Bigg].
\end{align}

In the limit \(\xi^2\rightarrow 0\) one should 
use the asymptotic behavior of \(K\) according to
\begin{equation}
K_\nu(z)\approx \frac{\Gamma(\nu)}{2} (z/2)^{-\nu} \quad \text{for Re}(\nu)>0, \text{ and } z\rightarrow 0 
\end{equation}
to recover the case we discussed the last time.

More precisly, we use the series expression for \(\nu\in\mathbb{N}_0\):
\begin{align}
  K_\nu(z)=\frac{1}{2}(z/2)^{-\nu}\sum_{k=0}^{\nu-1}\frac{(\nu-k-1)!}{k!}(-z^2/4)^k
  +(-1)^{\nu+1}\ln(z/2)I_{\nu}(z)\\
  +(-1)^\nu \frac{1}{2}(z/2)^\nu \sum_{k=0}^\infty (\psi(k+1)+\psi(\nu+k+1))\frac{(z^2/4)^k}{k!(\nu+k)!}\\
  =\frac{1}{2}(z/2)^{-\nu}\sum_{k=0}^{\nu-1}\frac{(\nu-k-1)!}{k!}(-z^2/4)^k\\
  +(-1)^\nu \frac{1}{2}(z/2)^\nu \sum_{k=0}^\infty (\psi(k+1)+\psi(\nu+k+1)-\ln(z^2/4))\frac{(z^2/4)^k}{k!(\nu+k)!}.
\end{align}
Plugging this into the expression for \(G\)
we find
\begin{align}
  G=\frac{i \pi^{d/2}}{\Gamma(2n)} E_z\Bigg[e^{-i\xi \overline{q}}2^{d-4n}\lambda^{d-4n}\sum_{k=0}^{2n-d/2-1}\frac{(2n-d/2-k-1)!}{k!}(\lambda^2\xi^2/4)^k\\
   +e^{-i\xi\overline{q}}(\xi^2)^{2n-d/2} \sum_{k=0}^\infty (\psi(k+1)+\psi(2n-d/2+k+1)-\ln(-\lambda^2 \xi^2/4))\frac{(-\lambda^2\xi^2/4)^k}{k!(2n-d/2+k)!} \Bigg].
\end{align}
After taking at most \(2n\) derivatives with
respect to \(\xi\) at \(\xi=0\) we find that
for \(n=2,d=4\) the second term diverges logarithmically.


\end{document}





