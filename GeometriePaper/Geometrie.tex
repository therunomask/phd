\documentclass[a4paper,11pt]{article}

%\usepackage{german}

\usepackage[dvipsnames]{xcolor}
\usepackage{graphicx}

\usepackage{amssymb}

\usepackage{amsfonts}

\usepackage{amsmath}

\usepackage{amsthm}
\usepackage{mathabx}


\usepackage[unicode=true, pdfusetitle, bookmarks=true,
  bookmarksnumbered=false, bookmarksopen=false, breaklinks=true, 
  pdfborder={0 0 0}, backref=false, colorlinks=true, linkcolor=blue,
  citecolor=blue, urlcolor=blue]{hyperref}
\usepackage{slashed}
\usepackage{authblk}
%identity sign

\usepackage{cancel}
\usepackage{dsfont}
\usepackage{todonotes}

\setlength{\marginparwidth}{2.6cm}

%commutative diagrams
\usepackage{amsmath,amscd}
\usepackage{enumitem}
\usepackage{moreenum}

\newtheorem{de}{Definition}
\newtheorem{thm}{Theorem}
\newtheorem{cor}{Corollary}
\newtheorem{rmk}{Remark}
\newtheorem{lem}{Lemma}
\newtheorem{claim}{Claim}

\newcommand{\supp}{\operatorname{supp}}
\newcommand{\AG}{\operatorname{AG}}
\newcommand{\ag}{\operatorname{ag}}
\DeclareMathOperator{\tr}{tr}
\DeclareMathOperator{\dom}{dom}

\addtolength{\textwidth}{2.2cm} \addtolength{\hoffset}{-1.0cm}

\addtolength{\textheight}{3.0cm} \addtolength{\voffset}{-2cm} 

\parindent 0cm

\pagestyle{empty}

\begin{document}
\title{The Phase of the Second Quantised Time Evolution Operator}

\author{
D.-A. Deckert\thanks{deckert@math.lmu.de \\ \tiny{Mathematisches Institut der Ludwig-Maximilians-Universit\"at M\"unchen,}
    \tiny{Theresienstr. 39, 80333 M\"unchen, Germany}},
Franz Merkl\thanks{merkl@math.lmu.de \\     \tiny{Mathematisches Institut der Ludwig-Maximilians-Universit\"at M\"unchen,}
    \tiny{Theresienstr. 39, 80333 M\"unchen, Germany}}
	 ~and
Markus Nöth\thanks{noeth@math.lmu.de \\\tiny{Mathematisches Institut der Ludwig-Maximilians-Universit\"at M\"unchen,}
    \tiny{Theresienstr. 39, 80333 M\"unchen, Germany}}

}
\date{\today}



\maketitle

\begin{abstract}
abstract to be written
\end{abstract}


\todo{Kapitel über Analytizität von Gamma}
\section{Introduction}


We follow the necessary definitions in \cite{ivp2}.
\begin{de}
For a Cauchy surface \(\Sigma\), we define \(\mathcal{H}_\Sigma\) to be the Hilbert space of \(\mathcal{C}^4\) valued, square integrable functions on \(\Sigma\).
Furthermore, let \(\mathrm{Pol}(\mathcal{H}_\Sigma)\) denote the set of all closed, linear subspaces \(V\subset \mathcal{H}_\Sigma\) such that both
\(V\) and \(V^\perp\) are infinite dimensional. Any \(V\in \mathrm{Pol}(\mathcal{H}_\Sigma)\) is called a polarisation of \(\mathcal{H}\). For \(V\in \mathrm{Pol}\),
let \(P^V_\Sigma:\mathcal{H}_\Sigma\rightarrow V\) denote the orthogonal projection of \(\mathcal{H}_\Sigma\) onto \(V\).

The Fock space corresponding to polarisation \(V\) on Cauchy surface \(\Sigma\) is then defined by 

\begin{equation}
\mathcal{F}(V,\mathcal{H}_\Sigma):=\bigoplus_{c\in\mathbb{Z}}\mathcal{F}_c(V,\mathcal{H}_\Sigma), \quad 
\mathcal{F}_c(V,\mathcal{H}_\Sigma):=\bigoplus_{\overset{n,m\in\mathbb{N}_0}{c=m-n}} \big( V^\perp\big)^{\wedge n} \otimes {\overline{V}}^{\wedge m},
\end{equation}
where \(\bigoplus\) denotes the Hilbert space direct sum, \(\wedge\) the antisymmetric tensor product of Hilbert spaces, and \(\overline{V}\) the conjugate complex vector
space of \(V\), which coincides with \(V\) as a set and has the same vector space operations as \(V\) with the exception of the scalar multiplication, which is 
replaced by \((z,\psi)\mapsto z^*\psi\) for \(z\in\mathbb{C}),\psi\in V\).
\end{de}

Each polarisation \(V\) splits the Hilbert space \(\mathcal{H}_\Sigma\) into a direct sum, i.e., \(\mathcal{H}_\Sigma=V^\perp \oplus V\). The ''standard`` polarisation 
\(\mathcal{H}^+_\Sigma\) and \(\mathcal{H}^-_\Sigma\) are determined by the orthogonal projectors \(P^+_\Sigma\) and \(P^-_\Sigma\) onto the free positive and 
negative energy Dirac solutions, respectively, restricted to \(\Sigma\):
\begin{equation}
\mathcal{H}_\Sigma^+:=P^+_\Sigma \mathcal{H}=(1-P_\Sigma^-)\mathcal{H}_\Sigma, \quad \mathcal{H}^-_\Sigma:=P^-_\Sigma \mathcal{H}_\Sigma.
\end{equation}

Given two Cauchy surfaces \(\Sigma, \Sigma'\) and two polarisations \(V\in \mathrm{Pol}(\mathcal{H}_\Sigma)\) and \(W\in \mathrm{Pol}(\Sigma_{\Sigma'})\)
a sensible lift of the one particle Dirac evolution \(U_{\Sigma'\Sigma}^A:\mathcal{H}\rightarrow \mathcal{H}_{\Sigma}\) should be given by a unitary operator
\(\tilde{U}_{\Sigma',\Sigma}^A:\mathcal{F}(V,\mathcal{H}_\Sigma)\rightarrow \mathcal{F}(W,\mathcal{H}_{\Sigma'})\) that fulfils 
\todo{Don't cite so much of ivp0, just the bare essential stuff for what is needed here. }
\begin{equation}\label{liftcondition}
S_{A}\psi_{V,\Sigma}(f)(\tilde{U}^A_{\Sigma',\Sigma})^{-1}=\psi_{W,\Sigma'}(U^A_{\Sigma',\Sigma}f),\quad \forall f\in \mathcal{H}_\Sigma.
\end{equation}

Here, \(\psi_{V,\Sigma}\) denotes the Dirac field operator corresponding to Fock space \(\mathcal{F}(V,\Sigma)\), i.e.,
\begin{equation}
\psi_{V,\Sigma}(f):=b_{\Sigma}\big(P^{V^\perp}_\Sigma f\big) + d^*_\Sigma\big(P^V_\Sigma f\big), \quad \forall f \in \mathcal{H}_\Sigma,
\end{equation}

where \(b_\Sigma\), \(d^*_\Sigma\) denote the annihilation and creation operators on the \(V^\perp\) and \(\overline{V}\) sectors of 
\(\mathcal{F}_c(V,\mathcal{H}_\Sigma)\), respectively. Note that \(P^{V^\perp}_\Sigma:\mathcal{H}_\Sigma\rightarrow \overline{V}\) is anti-linear; 
thus, \(\psi_{V,\Sigma}(f)\) is anti-linear in its argument \(f\). The condition under which such a lift \(\tilde{U}_{\Sigma',\Sigma}^A\) exists can be inferred from
a straight-forward application of Shale and Stinespring's well-known theorem \cite{shale stinespring?}

\begin{thm}[Shale-Stinespring]
The following statements are equivalent:
\begin{enumerate}[label=(\greek*)]
\item There is a unitary operator \(\tilde{U}_{\Sigma \Sigma'}^A:\mathcal{F}(V,\mathcal{H}_\Sigma)\rightarrow \mathcal{F}(W,\mathcal{H}_{\Sigma'})\) which 
fulfils \eqref{liftcondition}.
\item The off-diagonals \(P_{\Sigma'}^{W^\perp} U_{\Sigma' \Sigma}^A P^V_\Sigma\) and \(P^W_{\Sigma'}U^A_{\Sigma' \Sigma}\) are Hilbert-Schmidt operators.\label{enu 2}
\end{enumerate}
\end{thm}

Please note that condition \ref{enu 2} for fixed polarisations \(V\), \(W\) and general external field \(A\) is not always satisfied; see e.g. \cite{18 of ivp2}. 
However, when carefully adapting the choices of polarisation \(V\) to \(A |_{\Sigma}\) and \(W\) to \(A|_{\Sigma'}\) one can always fulfil condition \ref{enu 2}
and therefore construct a lift \(\tilde{U}_{\Sigma' \Sigma}^A\), see \cite{ivp0,ivp1,ivp2}. 

Furthermore condition \eqref{liftcondition} does not fix the phase of the lift \(\tilde{U}_{\Sigma' \Sigma}^A\). Considering Bogolyubov's formula

\begin{align}
j^\mu(x)=i \tilde{U}^A_{\Sigma_{\mathrm{in},\Sigma_{\mathrm{out}}}}\frac{\delta \tilde{U}^A_{\Sigma_{\mathrm{out},\Sigma_{\mathrm{in}}}}}{\delta A_\mu (x)},
\end{align}
where \(\Sigma_{\mathrm{in}}, \Sigma_{\mathrm{out}}\) are Cauchy-surfaces in the remote past and future, respectively.
The current operator thus depends in a rather sensitive way on the phase of \(\tilde{U}^A\). Since the current is experimentally accessible 
we would like to fix the phase by additional physical constraints. This paper is a step this direction. 

Next we introduce the set of four potentials we work with, as well as the argument of a complex number and an 
invertible bounded operator. For complex numbers the convention we chose here differs slightly from the standard in the literature,
which is why we also use a slightly non standard name for this function.
\begin{de}[vector potentials, polar decomposition and spectral projections]
We define the set \(\mathcal{V}\) of four potentials
\begin{equation}
\mathcal{V}:= C_c^\infty(\mathbb{R}^4,\mathbb{R}^4).
\end{equation}
We denote by \(\mathcal{H}=L^2(\mathbb{R}^3,\mathbb{C})\).
For \(X:\mathcal{H}\rightarrow \mathcal{H}\) bounded
\begin{equation}
\AG(X):=X |X|^{-1}.
\end{equation}
Furthermore, we define for any complex number \(z\in \mathbb{C}\backslash \{0\}\)
\begin{equation}
\ag(z):=\frac{z}{|z|} .
\end{equation}
\end{de}

We introduce the standard polarisation for the free Dirac equation\todo{think of some more intuitive notation for the projector.}
\begin{align}
P^-:=1_{\mathrm{spec}(H^0)<0}, \quad P^+=1-P^-.
\end{align}



\begin{de}[scattering operator and phases]\label{def: S bar, gamma, cA}
We define for all \(A,B\in\mathcal{V} \)
\begin{align}
S_{A,B}:=U^{A}_{\Sigma_{\mathrm{in}},\Sigma_{\mathrm{out}}}U^{B}_{\Sigma_{\mathrm{out}},\Sigma_{\mathrm{in}}},
\end{align}
\todo{vllt direkt mit \(S_A\) definieren, warte ab bis abschätzung der ableitungen geschehen ist}
where \(\Sigma_{\mathrm{out}}\) and \(\Sigma_{\mathrm{int}}\)  are Cauchy-surfaces of Minkowski spacetime such that 
\begin{align}
&\forall (x,y)\in \supp A\cup\supp B\times \Sigma_{\mathrm{in}}: (x-y)^2 \ge 0\Rightarrow x^0>y^0,\\
&\forall (x,y)\in \supp A\cup\supp B\times \Sigma_{\mathrm{out}}: (x-y)^2 \ge 0\Rightarrow x^0<y^0
\end{align}
holds.
Let 
\begin{equation}
\mathrm{dm}:=\{A,B\in\mathcal{V}\mid P^- S_{A,B}P^-\text{ and } P^- S_{B,A}P^-:\mathcal{H}^- \righttoleftarrow  \text{ are invertible}\},
\end{equation}
we define
\begin{equation}\label{dom s bar}
\dom\overline{S}:=\{A,B\in \mathrm{dm}\mid \{A\}\times \overline{A~B} \cup \overline{A~B}\times \{B\}\subseteq \mathrm{dm} \},
\end{equation}
where \(\overline{A~B}\) is the line segment connecting \(A\) and \(B\) in \(\mathcal{V}\).
Furthermore, we choose for all \(A,B\in \dom \overline{S}\) the lift
\begin{align}
\overline{S}_{A,B}=\mathcal{R}_{\AG((P^- S_{A,B}P^-)^{-1})} \mathcal{L}_{S_{A,B}}.
\end{align}

For \((A,B),(B,C), (C,A)\in\dom\overline{S}\), we define 
 the complex numbers 
\begin{align}\label{def: gamma}
\gamma_{A,B,C}&:=\ag(\det_{\mathcal{H}^-} (P^- S_{A,B} P^- S_{B,C} P^- S_{C,A}P^-)),\\
\Gamma_{A,B,C}&:=\ag(\gamma_{A,B,C}).
\end{align}
We will see in lemma \ref{gamma attri} that \(\gamma_{A,B,C}\neq 0\) and
\( P^-S_{A,B}P^-S_{B,C}P^-S_{C,A}-1\) is traceclass, so that \(\Gamma_{A,B,C}\) is well-defined.
Lastly we introduce the partial derivative in the direction of any four-potential \(F\) by
\begin{equation}
\partial_F T(F):=\partial_{\varepsilon}T(\varepsilon F)|_{\varepsilon =0}
\end{equation}
and for  \(A,B,C\in\mathcal{V}\) the function
\begin{equation}
c_A(F,G):=-i \partial_F \partial_G  \Im \tr [P^- S_{A,A+F} P^+ S_{A,A+G} P^-] .
\end{equation}
\end{de}

\begin{lem}
The set \(\dom \overline{S}\) has the following properties:
\begin{enumerate}
\item contains the diagonal: \(\{(A,A)\mid A\in\mathcal{V}\}\subseteq \dom\overline{S}\).
\item openness: \(\forall n \in \mathbb{N}: \left\{s\in\mathbb{R}^{2n}\mid \left(\sum_{k=1}^n s_k A_k,\sum_{k=n+1}^{2n} s_{k} A_{k} \right)\in \dom\overline{S}\right\}\) is an open subset of \(\mathbb{R}^{2n}\) for all \(A_1,\dots A_{2n}\in\mathcal{V}\).
\item symmetry:  \((A,A')\in\dom\overline{S}\iff (A',A)\in\dom\overline{S}\)
\item star-shaped: \((A,t A)\in \dom\overline{S}\Rightarrow \forall s\in \overline{1~~t} : (A,s A)\in\dom\overline{S}\)
\item well defined-ness of \(\overline{S}\): \(\dom\overline{S}\subseteq \{A,B\in\mathcal{V}\mid P^- S_{A,B} P^-:\mathcal{H}^- \righttoleftarrow \text{is invertible} \}\).
\end{enumerate}
We will only prove openness, as the other properties follow directly from the definition \eqref{dom s bar}. So pick \(n\in\mathbb{N}\), 
\(A_i\in\mathcal{V}\) for \(i\in \mathbb{N}, i\le 2 n\) and \(s\in\mathbb{R}^{2n}\)
such that \(\left(\sum_{k=1}^n s_k A_k,\sum_{k=n+1}^{2n} s_{k} A_{k} \right)\in\dom\overline{S}\). 
We have to find a neighbourhood \(U\subseteq \mathbb{R}^{2n}\) of \(s\) such that 
\(\{ \left(\sum_{k=1}^n s_k' A_k,\sum_{k=n+1}^{2n} s_{k}' A_{k} \right) \mid s'\in U\}\subseteq \dom\overline{S}\) holds. 
In doing so we have to ensure that the lines 

\begin{align}
\left\{\sum_{k=1}^n s_k A_k\right\} \times \overline{\sum_{k=1}^n s_k A_k~~~\sum_{k=n+1}^{2n} s_{k} A_{k} }\\
\overline{\sum_{k=1}^n s_k A_k~~~\sum_{k=n+1}^{2n} s_{k} A_{k}} \times \left\{\sum_{k=n+1}^{2n} s_{k} A_{k}\right\} 
\end{align}
stay subsets of \(\mathrm{dm}\) for all \(s'\in U\). Now pick a metric \(d\) on \(\mathbb{R}^n\) and define 
\begin{align}\nonumber
r:=\inf \left\{d(s,s')\mid  \left\{\sum_{k=1}^n s_k' A_k\right\}\times \overline{\sum_{k=1}^n s_k' A_k~~\sum_{k=n+1}^{2n} s_k' A_k } ~~\cap \mathrm{dm}^c\neq \emptyset \vee\right.\\
 \left. \overline{\sum_{k=1}^n s_k' A_k~~\sum_{k=n+1}^{2n} s_k' A_k }\times \left\{\sum_{k=n_1}^{2n} s_k' A_k \right\}~~\cap \mathrm{dm}^c\neq \emptyset  \right\}.
\end{align}
It cannot be the case that \(r=0\), because the metric is continuous, the line segment compact in \(\mathbb{R}^{2n}\) 
and the set of invertible bounded operators (defining \(\mathrm{dm}\)) is open in the topology generated by the operator norm.
 \todo{reicht das, oder soll ich das ausführlicher schreiben?}
If \(r=\infty\) then \(U=\mathbb{R}^{2n}\) will suffice. If \(r\in\mathbb{R}^+\) then \(U=B_r(s,t)\) the open ball of radius \(r\) around \(s\) works.
\end{lem}

\section{Main Result}


\begin{de}[causal splitting]
We define a causal splitting as a function 
\begin{align}
&c^+:\mathcal{V}^3\rightarrow \mathbb{C}, \\
&(A,F,G)\mapsto c_A^+(F,G),
\end{align}
such that \(c^+\) restricted to any finite dimensional subspace is smooth in the 
first argument and linear in the second and third argument.
Furthermore \(c^+\) should satisfy
\begin{align}\label{c+ 1}
c_A(F,G)=c_A^+(F,G)-c_A^+(G,F),\\\label{c+ 2}
\partial_H c^+_{A+H}(F,G)=\partial_G c^+_{A+G}(F,H),\\\label{c+ 3}
\forall F \prec G: c_A^+(F,G)=0.
\end{align}
\end{de}

\begin{de}[current]
Given a lift \(\hat{S}_{A,B}\) of the one-particle scattering operator \(S_{A,B}\) for which the derivative in the following expression exists,
 we define the associated current by Bogolyubov's formula:
\begin{equation}
j_A^{\hat{S}}(F):=i\partial_F \left\langle \Omega, \hat{S}_{A,A+F} \Omega\right\rangle.
\end{equation}
\end{de}

\begin{thm}[existence of causal lift]\label{thm: geometry}
Given a causal splitting \(c^+\), there is a second quantised scattering operator \(\tilde{S}\), lift of the one-particle scattering operator \(S\)
with the following properties
\begin{align}
&\forall A,B,C\in \mathcal{V}:\tilde{S}_{A,B}\tilde{S}_{B,C}=\tilde{S}_{A,C}\\
&\forall F\prec G: \tilde{S}_{A,A+F}=\tilde{S}_{A+G,A+F+G}
\end{align}
and the associated current satisfies
\begin{equation}
\partial_G j_{A+G}^{\tilde{S}}(F)=\left\{\begin{matrix} -2i c_A(F,G)  &\text{ for } G\prec F\\ 0 &\text{  otherwise.}  \end{matrix} \right.
\end{equation}
\end{thm}

\section{Proofs}


Since the phase of a lift relative to any other lift is fixed by a single matrix element, we may use the vacuum expectation values to characterise the
phase of a lift. The function \(c\) captures the dependence of this object on variation of the external fields, the connection between vacuum expectation
values and \(c\) becomes clearer with the next lemma.

\begin{lem}[properties of \(\Gamma\)]\label{gamma attri}
The function \(\Gamma\) has the following properties for all  \(A,B,C,D\in\mathcal{V}\) such that the expressions occurring in each equation are well defined:
\begin{align}\label{gamma attri0}
&\gamma_{A,B,C}\neq 0\\\label{gamma attri1}
&\Gamma_{A,B,C}=\det_{\mathcal{H}^-}(P^--P^-S_{A,C}P^+S_{C,A}P^- - P^- S_{A,B} P^+ S_{B,C} P^- S_{C,A} P^-)\\\label{gamma attri2}
&\Gamma_{A,B,C}^{-1}=\ag(\langle \Omega, \overline{S}_{A,B} \overline{S}_{B,C} \overline{S}_{C,A}\Omega\rangle )\\\label{gamma attri3}
&\Gamma_{A,B,C}=\Gamma_{B,C,A}=\frac{1}{\Gamma_{B,A,C}}\\\label{gamma attri4}
&\Gamma_{A,A,B}=1\\\label{gamma attri5}
&\Gamma_{A,B,C}\Gamma_{B,A,D}\Gamma_{A,C,D}\Gamma_{C,B,D}=1 \\\label{gamma attri6}
&\overline{S}_{A,C}=\Gamma_{A,B,C}\overline{S}_{A,B}\overline{S}_{B,C}\\\label{gamma attri7}
&c_A(B,C)=\partial_B \partial_C \ln \Gamma_{A,A+B,A+C}.
\end{align}
\end{lem}
\begin{proof}
Pick  \(A,B,C\in\mathcal{V}\) such that \(\|1-S_{X,Y}\|<1\) for \(X,Y\in\{A,B,C\}\). 
By definition \(\gamma\) is
\begin{equation}
\gamma_{A,B,C}=\det_{\mathcal{H}^-}(P^- S_{A,B} P^- S_{B,C} P^- S_{C,A} P^-).
\end{equation}
The operator whose determinant we take in the last line is a product
\begin{equation}
P^- S_{A,B} P^- S_{B,C} P^- S_{C,A} P^-=P^- S_{A,B}P^-~~ P^- S_{B,C}P^-~~ P^- S_{C,A} P^-.
\end{equation}
The three factors appearing in this product are all invertible, hence the product is also invertible as operators of type \(\mathcal{H}^-\rightarrow \mathcal{H}^-\)
 because of the conditions of \(\{A,B,C\}\) imply that \(\|P^--P^-S_{X,Y}P^-\|<1\) which means that the Von Neumann series of the inverse converges, 
therefore if the determinant exists we have \(\gamma_{A,B,C}\neq 0\).
To see that it does exist, we reformulate
\begin{align}
&\gamma_{A,B,C}=\det_{\mathcal{H}^-}(P^- S_{A,B} P^- S_{B,C} P^- S_{C,A} P^-)\\
&=\det_{\mathcal{H}^-}(P^-S_{A,C}P^-S_{C,A}P^- - P^- S_{A,B} P^+ S_{B,C} P^- S_{C,A} P^-)\\\label{eq: gamma proof of first attri}
&=\det_{\mathcal{H}^-}(P^--P^-S_{A,C}P^+S_{C,A}P^- - P^- S_{A,B} P^+ S_{B,C} P^- S_{C,A} P^-),
\end{align}
now we know by a classic result of Ruisnaars \cite{ruisnaar&ivp0} that \(P^+S_{X,Y}P^-\) is a Hilbert-Schmidt operator for our setting,
hence \(\gamma\) and also \(\Gamma\) are well defined.

 Equation 
\eqref{eq: gamma proof of first attri} also proves \eqref{gamma attri1}. Next we show \eqref{gamma attri2}. 
Borrowing notation from \cite[section 2]{ivp0} to identify \(\Omega=\bigwedge \Phi\) with the injection \(\Phi: \mathcal{H}^-\hookrightarrow\mathcal{H}\)
and \(\bigwedge\) is used to construct the infinite wedge spaces that are the perspective of Fock space introduced in \cite{ivp0}. 
We begin by reformulating the right hand side of  \eqref{gamma attri2}
\begin{align}
 \langle \Omega, \overline{S}_{A,B} \overline{S}_{B,C} \overline{S}_{C,A}\Omega\rangle \\\notag
=\langle \bigwedge \Phi, \bigwedge \left(S_{A,B}S_{B,C}S_{C,A}\Phi \AG(P^-S_{C,A} P^-)^{-1}\right.\\
\left. \AG(P^-S_{B,C} P^-)^{-1} \AG(P^-S_{A,B} P^-)^{-1}  \right)\rangle \\
=\langle \bigwedge \Phi, \bigwedge \left(\Phi \AG(P^-S_{C,A} P^-)^{-1} \AG(P^-S_{B,C} P^-)^{-1} \AG(P^-S_{A,B} P^-)^{-1}  \right)\rangle \\
=\det_{\mathcal{H}^-}\left( (\Phi)^*  \left[\Phi \AG(P^- S_{C,A} P^-)^{-1} \AG(P^-S_{B,C} P^-)^{-1} \AG(P^-S_{A,B} P^-)^{-1} \right] \right)\\
=\det_{\mathcal{H}^-}\left(  \AG(P^- S_{C,A} P^-)^{-1} \AG(P^- S_{B,C} P^-)^{-1} \AG(P^-S_{A,B} P^-)^{-1}  \right)\\\label{eq:1/arg}
=\frac{1}{\det_{\mathcal{H}^-}\left(  \AG(P^- S_{A,B} P^-) \AG(P^- S_{B,C} P^-) \AG(P^-S_{C,A} P^-)  \right)}.
\end{align}
We first note that \(\det_{\mathcal{H}^-}|P^- S_{X,Y}P^-|\in\mathbb{R}^+\) for \(X,Y\in \{A,B,C\}\). This is well defined because 
\begin{align}
&\langle \Omega, \overline{S}_{X,Y}\Omega\rangle
=\langle \bigwedge \Phi, \bigwedge(S_{X,Y}\Phi \AG(P^- S_{X,Y}P^-)^{-1})\rangle\\
&=\det_{\mathcal{H}^-} \left( \Phi^* S_{X,Y}\Phi \AG(P^- S_{X,Y}P^-)^{-1}\right)
=\det_{\mathcal{H}^-} \left( P^- S_{X,Y} P^- \AG(P^- S_{X,Y}P^-)^{-1}\right)\\
&=\det_{\mathcal{H}^-} \left(\AG(P^- S_{X,Y}P^-)^{-1} P^- S_{X,Y} P^- \right)\\
&=\det_{\mathcal{H}^-} \left(\AG(P^- S_{X,Y}P^-)^{-1} \AG(P^- S_{X,Y}P^-) |P^- S_{X,Y} P^-| \right)
=\det_{\mathcal{H}^-} |P^- S_{X,Y}P^-|
\end{align}
holds. Moreover this determinant does not vanish, since the \(P^- S_{X,Y}P^-\) is invertible. Also clearly the eigenvalues are positive
since \(|P^- S_{X,Y}P^-|\) is an absolute value. We continue with the result of \eqref{eq:1/arg}. Thus, we find
\begin{align}\label{eq: arg representation}
&\langle \Omega, \overline{S}_{A,B} \overline{S}_{B,C} \overline{S}_{C,A}\Omega\rangle^{-1}
=\det_{\mathcal{H}^-}\left(  \AG(P^- S_{A,B} P^-) \AG(P^- S_{B,C} P^-) \AG(P^-S_{C,A} P^-)  \right)\\
&=\det_{\mathcal{H}^-}\left(  \AG(P^- S_{A,B} P^-) \AG(P^- S_{B,C} P^-) P^-S_{C,A} P^- |P^-S_{C,A} P^-|^{-1}  \right)\\
&=\det_{\mathcal{H}^-}\left(  \AG(P^- S_{A,B} P^-) \AG(P^- S_{B,C} P^-) P^-S_{C,A} P^-  \right) \det_{\mathcal{H}^-}|P^-S_{C,A} P^-|^{-1} \\
&=\det_{\mathcal{H}^-}\left( P^-S_{C,A} P^- \AG(P^- S_{A,B} P^-) \AG(P^- S_{B,C} P^-)  \right) \det_{\mathcal{H}^-}|P^-S_{C,A} P^-|^{-1} \\
&=\frac{\det_{\mathcal{H}^-}\left(P^- S_{A,B} P^- P^- S_{B,C} P^- P^-S_{C,A} P^-  \right)}
{\det_{\mathcal{H}^-}|P^- S_{A,B} P^-| \cdot \det_{\mathcal{H}^-}|P^-S_{B,C} P^-| \cdot  \det_{\mathcal{H}^-}|P^-S_{C,A} P^-|}.
\end{align}
Now since the denominator of this fraction is real we can use \eqref{gamma attri1} to identity
\begin{equation}
\ag(\langle \Omega, \overline{S}_{A,B} \overline{S}_{B,C} \overline{S}_{C,A}\Omega\rangle)=\Gamma_{A,B,C}^{-1},
\end{equation}
which proves \eqref{gamma attri2}.

For the first equality in \eqref{gamma attri3} we use \(\det X (1+Y) X^{-1}=\det (1+Y)\) for any \(Y\) trace-class and 
 \(X\) bounded and invertible. So we can cyclicly permute the factors \(P^- S_{X,Y}P^-\) in the determinant and find
\begin{align*}
&\Gamma_{A,B,C}=\ag(\det_{\mathcal{H}^-} P^- S_{A,B}P^-S_{B,C}P^-S_{C,A}P^- )\\
&\quad=\ag(\det_{\mathcal{H}^-} P^-S_{C,A}P^- S_{A,B}P^-S_{B,C}P^- )
=\Gamma_{C,A,B}.
\end{align*}
For the second equality of \eqref{gamma attri3} we use \eqref{gamma attri1}
to represent both \(\Gamma_{A,B,C}\) and \(\Gamma_{B,A,C}\). Using this and the manipulations of the determinant 
we already employed, we arrive at
\begin{align}
&\Gamma_{A,B,C}\Gamma_{B,A,C}\\
&=\ag(\det_{\mathcal{H}^-} (P^-S_{A,B}P^-S_{B,C}P^-S_{C,A}P^-))\\
&\quad\times\ag(\det_{\mathcal{H}^-} (P^-S_{B,A}P^-S_{A,C}P^-S_{C,B}P^-))\\
&=\ag(\det_{\mathcal{H}^-} (P^-S_{A,B}P^-S_{B,C}P^-S_{C,A}P^-))\\
&\quad\times(\ag(\det_{\mathcal{H}^-} (P^-S_{B,C}P^-S_{C,A}P^-S_{A,B}P^-)))^*\\
&=\ag(\det_{\mathcal{H}^-} (P^-S_{A,B}P^-S_{B,C}P^-S_{C,A}P^-))\\
&\quad\times(\ag(\det_{\mathcal{H}^-} (P^-S_{A,B}P^-S_{B,C}P^-S_{C,A}P^-)))^*\\
&=|\ag(\det_{\mathcal{H}^-} (P^-S_{A,B}P^-S_{B,C}P^-S_{C,A}P^-))|^2=1,
\end{align}
which proves \eqref{gamma attri3}. 

Next, using \eqref{def: gamma} inserting twice the same argument yields
\begin{equation}
\gamma_{A,A,C}=\det_{\mathcal{H}^-} P^-S_{A,C}P^-S_{C,A}P^-
=\det_{\mathcal{H}^-} (P^- S_{C,A} P^-)^*P^-S_{C,A}P^-\in \mathbb{R}^+,
\end{equation}
hence \eqref{gamma attri4} follows.


For proving \eqref{gamma attri5} we will use the definition of \(\Gamma\) directly and repeatedly use that we can cyclicly permute operator groups of the form
\(P^- S_{X,Y}P^-\) for \(X,Y\in \{A,B,C,D\}\) in the determinant, i.e.
\begin{equation}\label{tetraeder 1}
\det P^- S_{X,Y}P^- O = \det O P^- S_{X,Y}P^-.\tag{\(\circlearrowleft\)}
\end{equation}
This is possible, because \(P^- S_{X,Y}P^-\) is bounded and invertible. Furthermore we will use that 
\begin{equation}\label{tetraeder 2}
\det O_1 O_2 = \det O_1 \det O_2\tag{\(\leftrightarrow\)}
\end{equation}
holds whenever 
both \(O_1\) and \(O_2\) have a determinant. Moreover for any  \((P^- S_{X,Y}P^-)^*P^- S_{X,Y}P^-\) is the modulus squared of an invertible operator and hence
its determinant is positive which means that 
\begin{equation}\label{tetraeder 3}
\ag \det (P^- S_{X,Y}P^-)^*P^- S_{X,Y}P^-=1.\tag{\(\ag\!|\,\, |\)}
\end{equation}
These three rules will be repeatedly used. We calculate
\begin{align}
&\Gamma_{A,B,C}\Gamma_{B,A,D}\Gamma_{A,C,D}\Gamma_{C,B,D}\\
&=\ag \det_{\mathcal{H}^-}P^- S_{A,B}P^- S_{B,C}P^- S_{C,A}P^- \ag \det_{\mathcal{H}^-}P^- S_{B,A}P^- S_{A,D}P^- S_{D,B}P^-~~ \Gamma_{A,C,D}\Gamma_{C,B,D}\\
&\overset{\eqref{tetraeder 1}}{=}
\ag \det_{\mathcal{H}^-}P^-S_{A,D}P^- S_{D,B}P^- S_{B,A}P^- \ag \det_{\mathcal{H}^-}P^- S_{A,B}P^- S_{B,C}P^- S_{C,A}P^- ~~  \Gamma_{A,C,D}\Gamma_{C,B,D}\\
&\overset{\eqref{tetraeder 2}}{=}
\ag \det_{\mathcal{H}^-}P^-S_{A,D}P^- S_{D,B} \big[P^- S_{B,A} P^- S_{A,B}P^-\big] S_{B,C}P^- S_{C,A}P^- ~~ \Gamma_{A,C,D}\Gamma_{C,B,D}\\
&\overset{\eqref{tetraeder 1}}{=}
\ag \det_{\mathcal{H}^-}P^-S_{B,C}P^- S_{C,A}P^-S_{A,D}P^- S_{D,B} \big[P^- S_{B,A} P^- S_{A,B}P^-\big]  ~~ \Gamma_{A,C,D}\Gamma_{C,B,D}\\
&\overset{\eqref{tetraeder 2}}{=}
\ag \det_{\mathcal{H}^-}P^-S_{B,C}P^- S_{C,A}P^-S_{A,D}P^- S_{D,B}P^-  \ag \det_{\mathcal{H}^-} P^- S_{B,A} P^- S_{A,B}P^-  ~~ \Gamma_{A,C,D}\Gamma_{C,B,D}\\
&\overset{\eqref{tetraeder 3}}{=}
\ag \det_{\mathcal{H}^-}P^-S_{B,C}P^- S_{C,A}P^-S_{A,D}P^- S_{D,B}P^-   ~~ \Gamma_{A,C,D}\Gamma_{C,B,D}\\
&\overset{\eqref{tetraeder 1}}{=}
\ag \det_{\mathcal{H}^-}P^-S_{A,D}P^- S_{D,B}P^-S_{B,C}P^- S_{C,A}P^-   ~~ \Gamma_{A,C,D} \Gamma_{C,B,D}\\
&=\ag \det_{\mathcal{H}^-}P^- S_{A,C}P^- S_{C,D}P^- S_{D,A}P^-
\ag \det_{\mathcal{H}^-}P^-S_{A,D}P^- S_{D,B}P^-S_{B,C}P^- S_{C,A}P^-   ~~ \Gamma_{C,B,D}\\
&\overset{\eqref{tetraeder 2}}{=}
\ag \det_{\mathcal{H}^-}P^- S_{A,C}P^- S_{C,D}P^-\big[ P^- S_{D,A}P^-P^-S_{A,D}P^- \big] S_{D,B}P^-S_{B,C}P^- S_{C,A}P^-   ~~ \Gamma_{C,B,D}\\
&\overset{\eqref{tetraeder 1}}{=}
\ag \det_{\mathcal{H}^-}P^- S_{D,B}P^-S_{B,C}P^- S_{C,A}P^- S_{A,C}P^- S_{C,D}P^-\big[P^- S_{D,A}P^-P^-S_{A,D}P^- \big]    ~~ \Gamma_{C,B,D}\\
&\overset{\eqref{tetraeder 2}}{=}
\ag \det_{\mathcal{H}^-}P^- S_{D,B}P^-S_{B,C}P^- S_{C,A}P^- S_{A,C}P^- S_{C,D}P^- \ag \det_{\mathcal{H}^-} P^-S_{D,A}P^-P^-S_{A,D}P^-    ~~ \Gamma_{C,B,D}\\
&\overset{\eqref{tetraeder 3}}{=}
\ag \det_{\mathcal{H}^-}P^- S_{D,B}P^-S_{B,C}P^-\big[P^- S_{C,A}P^- S_{A,C}P^-\big] P^- S_{C,D}P^-    ~~ \Gamma_{C,B,D}\\
&\overset{\eqref{tetraeder 1}}{=}
\ag \det_{\mathcal{H}^-}P^- S_{C,D}P^- S_{D,B}P^-S_{B,C}P^- \big[P^- S_{C,A}P^- S_{A,C}P^-\big]     ~~ \Gamma_{C,B,D}\\
&\overset{\eqref{tetraeder 2}}{=}
\ag \det_{\mathcal{H}^-}P^- S_{C,D}P^- S_{D,B}P^-S_{B,C}P^- \ag \det P^- S_{C,A}P^- S_{A,C}P^-     ~~ \Gamma_{C,B,D}\\
&\overset{\eqref{tetraeder 3}}{=}
\ag \det_{\mathcal{H}^-}P^- S_{C,D}P^- S_{D,B}P^-S_{B,C}P^-  ~~ \Gamma_{C,B,D}\\
&=
\ag \det_{\mathcal{H}^-}P^- S_{C,D}P^- S_{D,B}P^-S_{B,C}P^-  \ag \det_{\mathcal{H}^-} P^- S_{C,B}P^- S_{B,D}P^- S_{D,C}P^- \\
&=
|\ag \det_{\mathcal{H}^-}P^- S_{C,D}P^- S_{D,B}P^-S_{B,C}P^-|^2=1. 
\end{align}

For \eqref{gamma attri6} we realise that according to \cite{ivp0} that two lifts can only differ by a phase,
that is
\begin{equation}
\overline{S}_{A,C}=\alpha \overline{S}_{A,B} \overline{S}_{B,C}
\end{equation}
for some \(\alpha\in \mathbb{C}, |\alpha|=1\). 

In order to identify \(\alpha\) we recognise that \(\overline{S}_{X,Y}=\overline{S}_{Y,X}^{-1}\) for four potentials \(X,Y\) 
and find
\begin{equation}
\mathds{1}\alpha^{-1}= \overline{S}_{A,B} \overline{S}_{B,C}\overline{S}_{C,A}.
\end{equation}
Now we take the vacuum expectation value on both sides of this equation and use \eqref{gamma attri2} to find
\begin{equation}
\alpha^{-1}=\langle\Omega,\overline{S}_{A,B} \overline{S}_{B,C}\overline{S}_{C,A}\Omega\rangle = \Gamma_{A,B,C}^{-1}.
\end{equation}

Finally we prove \eqref{gamma attri7}. We start from the right hand side of this equation 
and work our way towards the left hand side of it. In the following calculation we will repeatedly make use of the fact that 
\((P^-S_{A,A+B}P^-S_{A+B,A}P^- )\) is the absolute value squared of an invertible operator and has a determinant, which is therefore positive. 
For the marked equality we will use that for a differentiable function \(z:\mathbb{R}\rightarrow \mathbb{C}\) at points \(t\) where \(z(t)\in\mathbb{R}^+\)
holds, we have
\begin{align}
(z/|z|)'(t)=\frac{z'}{|z|}(t)+\frac{-z}{|z|^2}\frac{z'z^*+ {z^*}'z}{2|z|}(t) =\frac{z'}{2|z|}(t)-\frac{z^2{z^*}'}{2 |z|^3}(t)=i (\Im (z'))/z(t).
\end{align}
Furthermore, we will use the following  expressions for the derivative of the determinant which holds for all functions 
\(M:\mathbb{R}\rightarrow (\mathcal{H}\rightarrow \mathcal{H})\) such that \(M(t)-1\) is traceclass and \(M\) is invertible 
for all \(t\in\mathbb{R}\)
\begin{align}\label{diff det}
\partial_\varepsilon \det M(\varepsilon)|_{\varepsilon=0}=\det M(0) \tr (M^{-1}(0)\partial_\varepsilon M(\varepsilon)|_{\varepsilon}),
\end{align}
likewise we need the following expression for the derivative of \(M^{-1}\) for 
\(M:\mathbb{R}\rightarrow (\mathcal{H}\rightarrow \mathcal{H})\) such that \(M(t)\) is invertible and bounded for every \(t\in\mathbb{R}\)
\begin{equation}
\partial_{\varepsilon}M^{-1}(\varepsilon)|_{\varepsilon=0}=-M^{-1}(0) \partial_{\varepsilon}M(\varepsilon)|_{\varepsilon=0}M^{-1}(0).
\end{equation}
We compute
\begin{align}
\partial_B \partial_C \ln \Gamma_{A,A+B,A+C}
\overset{\eqref{gamma attri1}}{=} 
\partial_B \partial_C \ln \ag(\det_{\mathcal{H}^-} (P^-S_{A,A+B}P^-S_{A+B,A+C}P^- S_{A+C,A}P^-))\\
=\partial_B \frac{\partial_C\ag(\det_{\mathcal{H}^-}(P^-S_{A,A+B}P^-S_{A+B,A+C}P^- S_{A+C,A}P^-))}{\ag(\det_{\mathcal{H}^-} (P^-S_{A,A+B}P^-S_{A+B,A}P^- ))}\\
=\partial_B \partial_C\ag(\det_{\mathcal{H}^-}(P^-S_{A,A+B}P^-S_{A+B,A+C}P^- S_{A+C,A}P^-))\\
\overset{*}{=}
i\partial_B \frac{\Im \partial_C \det_{\mathcal{H}^-}(P^-S_{A,A+B}P^-S_{A+B,A+C}P^- S_{A+C,A}P^-)}{\det_{\mathcal{H}^-}(P^-S_{A,A+B}P^-S_{A+B,A}P^-)}\\\notag
=i\partial_B \Big[\frac{\det_{\mathcal{H}^-}(P^-S_{A,A+B}P^-S_{A+B,A}P^-)}{\det_{\mathcal{H}^-}(P^-S_{A,A+B}P^-S_{A+B,A}P^-)}\\
\times \Im  \tr ((P^-S_{A,A+B}P^-S_{A+B,A}P^-)^{-1}\partial_C P^-S_{A,A+B}P^-S_{A+B,A+C}P^- S_{A+C,A}P^-)\Big]\label{eq ca 1}
\end{align}
The fraction in front of the trace equals \(1\). As a next step we replace the second but last projector \(P^-=1-P^+\), the resulting first summand vanishes,
because the dependence on \(C\) cancels. This results in 
\begin{align}
\eqref{eq ca 1}=-i\partial_B \Im \tr((P^-S_{A,A+B}P^-S_{A+B,A}P^-)^{-1}\partial_C P^-S_{A,A+B}P^-S_{A+B,A+C}P^+ S_{A+C,A}P^-).\label{eq ca 2}
\end{align}
Now, because \(P^+P^-=0\) only one summand of the product rule survives:
\begin{align}
\eqref{eq ca 2}=-i\partial_B \Im \tr((P^-S_{A,A+B}P^-S_{A+B,A}P^-)^{-1}\partial_C P^-S_{A,A+B}P^-S_{A+B,A}P^+ S_{A+C,A}P^-).\label{eq ca 3}
\end{align}
Next we use 
\((M N )^{-1}= N^{-1} M^{-1}\) for invertible operators \(M\) and \(N\) for the first factor in the trace and cancel as much as possible of
the second factor:
\begin{align}
\eqref{eq ca 3}=-i\partial_B \Im \tr((P^-S_{A+B,A}P^-)^{-1}  P^-S_{A+B,A}P^+ \partial_C S_{A+C,A}P^-)\\
=-i \Im \tr(\partial_B[(P^-S_{A+B,A}P^-)^{-1}  P^-S_{A+B,A}P^+ \partial_C S_{A+C,A}P^-])\\
=-i \Im \tr(\partial_B  P^-S_{A+B,A}P^+ \partial_C S_{A+C,A}P^-)\\
=-i \Im \tr(\partial_B  P^-S_{A,A+B}P^+ \partial_C S_{A,A+C}P^-)\\
=-i \partial_B\partial_C\Im \tr(  P^-S_{A,A+B}P^+  S_{A,A+C}P^-)
\end{align}
which proves the claim.

\end{proof}



In order to construct the lift announced in theorem \ref{thm: geometry}, we first construct a reference lift \(\hat{S}\), that is well defined on all of \(\mathcal{V}\). 
Afterwards we will study the dependence of the relative phase between 
this global lift \(\hat{S}_{0,A}\) and a local lift given by \(\hat{S}_{0,B}\overline{S}_{B,A}\) for \(B-A\) small. 
By exploiting properties of this phase and the causal splitting \(c^+\) we will construct a global lift that has the desired properties.

Since \(\mathcal{V}\) is star shaped, we may reach any four-potential \(A\) from \(0\) through the straight line
\(\{t A\mid t \in [0,1]\}\). 

\begin{de}[ratio of lifts]\label{def:ratio}
For any  \(A,B\in\mathcal{V}\) and any two lifts \(S_{A,B}', S_{A,B}''\) of the one particle scattering operator \(S_{A,B}\)
we define the ratio
\begin{equation}
\frac{S_{A,B}'}{S_{A,B}''}\in S^1
\end{equation}
to be the unique complex number \(z\in S^1\) such that 
\begin{equation}
z ~S_{A,B}'' = S_{A,B}'
\end{equation}
holds.
\end{de}

\begin{thm}[existence of global lift]\label{thm: ex s hat}
There is a unique operator \(\hat{S}\) which for any  \(A\in\mathcal{V}\) is a lift of \(S_{0,A}\) and solvesthe differential equation
\begin{equation}\label{def s hat}
A,B\in\mathcal{V}\text{ linearly dependent}\Rightarrow \partial_B \frac{\hat{S}_{0,A+B}}{\hat{S}_{0,A}\overline{S}_{A,A+B}}=0,
\end{equation}
subject to the initial condition \(\hat{S}_{0,0}=\mathds{1}\).
\end{thm}

The proof of theorem \ref{thm: ex s hat} is divided into two lemmas due to its length. We will introduce the integral flow \(\phi_A\) associated 
with the differential equation \eqref{def s hat} for some \(A\in\mathcal{V}\). We will then study the properties of \(\phi_A\)
in the two lemmas and finally construct \(\hat{S}_{0,A}=1 \phi_A(0,1)\). In the first lemma we will establish the existence of a 
local solution. The solution will be constructed along the line \(\overline{0 ~~ A}\). In the second lemma we patch local solutions together
to a global one.



\begin{lem}[\(\phi\) local existence and uniqueness]\label{lem phi local}
There is a unique \(\phi_A:\{(t,s)\in\mathbb{R}^2\mid (t A, s A)\in \dom\overline{S}\} \rightarrow U(\mathcal{F})\) for every \(A\in\mathcal{V}\)
satisfying
\begin{align}\label{phi prop1}
\forall (t,s)\in \dom \phi_A: \phi_A(t,s) \text{ is a lift of } S_{tA, sA}\\\label{phi prop2}
\forall (t,s),(s,l),(l,t)\in \dom\phi_A: \phi_A(t,s)\phi_A(s,l)=\phi_A(t,l)\\\label{phi prop3}
\forall t\in\mathbb{R}: \phi_A(t,t)=1\\\label{phi prop4}
\forall s\in\mathbb{R}: \partial_{t} \left.\frac{\phi_A(s,t)}{\overline{S}_{sA,tA}}\right|_{t=s}=0.
\end{align}
\end{lem}
\begin{proof}
We first define the phase
\begin{align}
z:\{(A,B)\in\dom\overline{S}\mid A, B \text{ linear dependent}\} \rightarrow S^1
\end{align}
by the differential equation
\begin{equation}
\frac{d}{d x} \ln z(t A, x A) = - \left.\left( \frac{d}{dy} \ln \Gamma_{tA,x A, yA}\right)\right|_{y=x}
\end{equation}
and the initial condition 
\begin{equation}\label{z initial}
z(A,A)=1
\end{equation}
 for any \(A\in \mathcal{V}\). The phase \(z\) takes the form
\begin{equation}
z(tA,xA)=\exp\left(-\int_{t}^x dx' \left.\left( \frac{d}{dx'} \ln \Gamma_{tA,y A, x' A}\right)\right|_{y=x'}\right).
\end{equation}
Please note that both differential equation and initial condition are invariant under rescaling of the potential \(A\), so \(z\) is well defined. 
We will now construct a local solution to \eqref{def s hat} and define \(\phi_A\) using this solution.
Pick \(A\in \mathcal{V}\) the expression
\begin{equation}\label{loc s hat}
\hat{S}_{0,s A} = \hat{S}_{0, A} \overline{S}_{ A, s A} z( A, s A)
\end{equation}
solves \eqref{def s hat} locally. Local here means that \(s\) is close enough to \(1\) such that \(( A, s A)\in\dom\overline{S}\).
Calculating the argument of the derivative of \eqref{def s hat} we find:
\begin{align}
0= \frac{\hat{S}_{0,(s+\varepsilon) A}}{\hat{S}_{0,s A}\overline{S}_{s A, (s+\varepsilon)A}}
= \frac{\hat{S}_{0,A} \overline{S}_{A, (s+\varepsilon)A} z(A, (s+\varepsilon)A)}
{\hat{S}_{0, A} \overline{S}_{ A, s A} \overline{S}_{s A,(s+\varepsilon)}z( A, s A)}\\
\overset{\eqref{gamma attri6}}{=}
 \frac{\hat{S}_{0,A}\overline{S}_{ A, s A} \overline{S}_{s A,(s+\varepsilon)} \Gamma_{A, sA, (s+\varepsilon)A} z(A, (s+\varepsilon)A)}
{\hat{S}_{0, A} \overline{S}_{t A, s A} \overline{S}_{s A,(s+\varepsilon)}z( A, s A)}\\
= \frac{\Gamma_{tA,sA,(s+\varepsilon)A}z(tA, (s+\varepsilon)A)}{z( A, sA)}
\end{align}
Now we take the derivative with respect to \(\varepsilon\) at \(\varepsilon=0\), 
cancel the factor that does not depend on \(\varepsilon\) and relabel \(s=x\) to obtain
\begin{align}
0=\left.\left(\frac{d}{dy}(\Gamma_{ A, x A, y A} ~z( A, y A))\right)\right|_{y=x}\\
\iff \frac{d}{dx}\ln z(tA, xA)=\left.\left(-\frac{d}{dy}\ln \Gamma_{t A, xA, yA}\right)\right|_{y=x},
\end{align}
which is exactly the defining differential equation of \(z\). The initial condition of \(z\) equation \eqref{z initial} 
is necessary to match the initial condition in \eqref{loc s hat} for \(s=1\).
The connection to \(\phi\) from the statement of the lemma can now be made.
We define 
\begin{equation}\label{def phi}
\phi_A(t,s):=z(t A, s A) \overline{S}_{t A, s A},
\end{equation}
for \((tA,sA)\in\dom\overline{S}\). Having introduced \(\phi_A\) we realise that \eqref{phi prop1} and \eqref{phi prop3} 
are direct consequences of the way \(\phi_A\) was constructed. Equation \eqref{phi prop4} follows by plugging in 
\eqref{def phi} and using the differential equation for \(z\). It remains to see that \eqref{phi prop2},  

\begin{align}\label{consistency phi}
\phi_A(t,s)\phi_A(s,l)=\phi_A(t,l)
\end{align}
holds for \((tA,sA),(sA,lA),(tA,lA)\in\dom\overline{S}\). In order to do so we plug in the definition of \(\phi\) and obtain
\begin{align}
\phi_A(t,s)\phi_A(s,l)=\phi_A(t,l)\\
\iff z(tA,sA) z(sA,lA)\overline{S}_{tA,sA}\overline{S}_{sA,lA}=z(tA,lA)\overline{S}_{tA,lA}\\
\iff z(tA,sA) z(sA,lA)\overline{S}_{tA,sA}\overline{S}_{sA,lA}=z(tA,lA)\overline{S}_{tA,s A} \overline{S}_{sA,lA}\Gamma_{tA,sA,lA}\\
\iff z(tA,sA) z(sA,lA)z(tA,lA)^{-1}=\Gamma_{tA,sA,lA}.
\end{align}
In order to check the validity of this equality we plug in the integral formula for \(z\), we also abbreviate \(\frac{d}{d x}=\partial_x\)

\begin{align}
z(tA,sA) z(sA,lA)z(tA,lA)^{-1}\\
=e^{-\int_t^s d x' \left.(\partial_{x'} \ln \Gamma_{tA,yA,x'A} )\right|_{y=x'}  -\int_s^l d x' \left.(\partial_{x'} \ln \Gamma_{sA,yA,x'A} )\right|_{y=x'}  +\int_t^l d x' \left.(\partial_{x'} \ln \Gamma_{tA,yA,x'A} )\right|_{y=x'}  }\\
=e^{-\int_l^s d x' \left.(\partial_{x'} \ln \Gamma_{tA,yA,x'A} )\right|_{y=x'}  -\int_s^l d x' \left.(\partial_{x'} \ln \Gamma_{sA,yA,x'A} )\right|_{y=x'} }\\
\overset{\eqref{gamma attri5}}{=}
e^{-\int_l^s d x' \left.(\partial_{x'} \ln \Gamma_{sA,yA,x'A} )\right|_{y=x'} 
-\int_l^s d x' \left.(\partial_{x'} \ln \Gamma_{tA,sA,x'A} )\right|_{y=x'}}\\ 
e^{-\int_l^s d x' \left.(\partial_{x'} \ln \Gamma_{tA,yA,sA} )\right|_{y=x'} 
 -\int_s^l d x' \left.(\partial_{x'} \ln \Gamma_{sA,yA,x'A} )\right|_{y=x'} }\\
 =e^{-\int_l^s d x' \left.(\partial_{x'} \ln \Gamma_{tA,sA,x'A} )\right|_{y=x'}}\\
  =e^{-\int_l^s d x'\partial_{x'} \ln \Gamma_{tA,sA,x'A} }=e^{-\ln \Gamma_{t A, s A, s A} + \ln\Gamma_{t A, s A, l A}}\\
  \overset{\eqref{gamma attri4}}{=}\Gamma_{t A, s A, l A},
\end{align}
which proves the validity of the consistency relation \eqref{consistency phi}.

In order to prove uniqueness we pick \(A\in \mathcal{V}\) and assume there is \(phi'\) also defined on \(\dom \phi_A\) 
satisfying \eqref{phi prop1}-\eqref{phi prop4}. Then we may use \eqref{phi prop1} to conclude that for any \((t,s)\in\dom\phi_A\) there is
\(\gamma(t,s)\in S^1\) such that
\begin{equation}
\phi_A(t,s)=\phi'(t,s) \gamma(t,s)
\end{equation}
holds true. Picking \(l\) such that \((t,s),(s,l),(t,l)\in \dom\phi_A\) and  using \eqref{phi prop2} we find

\begin{align}
\phi'(t,s) \gamma(t,s)=\phi_A(t,s)=\phi_A(t,l)\phi_A(l,s)\\
=\gamma(t,l)\phi'(t,l) \gamma(l,s)\phi'(l,s)=\gamma(t,l) \gamma(l,s)\phi'(t,s) ,
\end{align}
hence we have
\begin{equation}
 \gamma(t,s)=\gamma(t,l) \gamma(l,s).
\end{equation}
From property \eqref{phi prop3} we find
\begin{equation}
\gamma(t,t)=1,
\end{equation}
for any \(t\).
Furthermore equation \eqref{phi prop4} helps us to conclude that 

\begin{align}
0=\partial_t \left.\frac{\phi'(s,t)}{\overline{S}_{sA,tA}}\right|_{t=s}
=\partial_t \left.\frac{\phi_A(s,t) \gamma(s,t)}{\overline{S}_{sA,tA}}\right|_{t=s}\\
=\partial_t \left.\gamma(s,t) \frac{\phi_A(s,t)}{\overline{S}_{sA,tA}}\right|_{t=s}
=\partial_t \left.\gamma(s,t)\right|_{t=s} + \partial_t  \left.\frac{\phi_A(s,t)}{\overline{S}_{sA,tA}}\right|_{t=s}\\
=\partial_t \left.\gamma(s,t)\right|_{t=s}.
\end{align}
Finally we find for general \((s,t)\in \dom\phi_A\):
\begin{equation}
\partial_x \gamma(s,x)|_{x=t}=\partial_x ( \gamma(s,t) \gamma(t,x))|_{x=t}=\gamma(s,t) \partial_x \gamma(t,x)|_{x=t}=0.
\end{equation}
So \(\gamma(t,s)=1\) everywhere.

\end{proof}



\begin{lem}[\(\phi\) global existence and uniqueness]\label{lem: phi global}
The unitary operator \(\phi\) constructed in lemma \ref{lem phi local} can for any \(A\in\mathcal{V}\) be uniquely extended to all of \(\mathbb{R}^2\), 
keeping its defining properties 
\begin{align}\label{global phi prop1}
\forall (t,s)\in \mathbb{R}^2: \phi_A(t,s) \text{ is a lift of } S_{tA, sA}\\\label{global phi prop2}
\forall (t,s),(s,l),(l,t)\in \mathbb{R}^2: \phi_A(t,s)\phi_A(s,l)=\phi_A(t,l)\\\label{global phi prop3}
\forall t\in\mathbb{R}: \phi_A(t,t)=1\\\label{global phi prop4}
\forall s\in\mathbb{R}: \partial_{t} \left.\frac{\phi_A(s,t)}{\overline{S}_{sA,tA}}\right|_{t=s}=0.
\end{align}
\end{lem}
\begin{proof}

 
Because of the structure of \(\dom\overline{S}=\bigcup_{x\in\mathbb{R}} \{x\} \cup U_x\), where \(U_x\) is open for any \(x\) 
we have defined \(\phi_A\) defined for arguments that are
close enough to each other. Since any extension is going to fulfil properties \eqref{global phi prop4} and \eqref{global phi prop3} 
we don't prove them in the induction step.





The proof is by induction over the domain of definition of \(\phi_A\). 
The induction hypothesis is that \(\psi_n\) defined
 on  \(\dom \psi_n=\dom_n \times\dom_n\), is the unique function satisfying the conditions
 \begin{align}\label{induction psi1}
 \forall (t,s)\in \dom_n\times \dom_n: \phi_A(t,s) \text{ is a lift of } S_{tA, sA}\\\label{induction psi2}
\forall s,k,l\in\dom_n: \psi_n(k,s)\psi_n(s,l)=\psi_n(k,l),\\\label{induction psi3}
\forall (x,y)\in\dom_n: (xA, yA)\in\dom\overline{S}\Rightarrow  \psi_n(x,y)=\phi_A(x,y).
\end{align}

We start with \(\psi_0=\phi_A\)  
restricted to \(U_0 \times U_0\) chosen such that for any \(t,s\in U_0\), \((t A, s A)\in\dom\overline{S}\) holds. 
 This function is an extension of \(\phi_A\) and because of lemma \ref{lem phi local} it 
 fulfils all of the required properties directly. 
 
For the induction step we pick \(z\in \partial \dom_n\) and \(U_z\ni z\) open such that \(U_z\times U_z \subseteq \dom\overline{S}\) holds.
Fix \(t\in U_z\cap \dom_n\).
 We define \(\psi_{n+1}\) on the domain \((\dom_n\cup U_z)\times (\dom_n\cup U_z)\) by
 \begin{equation}\label{psi induction def}
 \psi_{n+1}(x,y):=\left\{\begin{matrix}
 \psi_n(x,y) \quad &\text{for }x,y\in \dom_n\\
 \phi_A(x,y) \quad &\text{for } x,y\in U_z\\
 \psi_n(x,t)\phi_A(t,y)\quad &\text{for } x\in \dom_n, y\in U_z\\
 \phi_A(x,t)\psi_n(t,y)\quad &\text{for } y\in \dom_n, x\in U_z.
 \end{matrix} \right.
 \end{equation} 
  In order to complete the induction step we have to show that \(\psi_{n+1}\) is well defined and fulfils \eqref{induction psi1}-\eqref{induction psi3} 
  with \(\psi_n\) replaced by \(\psi_{n+1}\)  and is the unique function to do so.
  
  To see that \(\psi_{n+1}\) is well defined we have to check that the cases in the definition agree when they overlap. 
  In each item we will implicitly assume that \(x\) and \(y\) are such that none of the previous items applies.
  \begin{enumerate}
  \item If we have \(x,y \in \dom_n\cap U_z\) all four cases overlap; however, the cases all equal \(\phi_A(x,y)\) because of the
  induction hypothesis  \eqref{induction psi3} and \eqref{consistency phi}:
  \begin{equation}
  \psi_n(x,y)\overset{\eqref{induction psi3}}{=}\phi_A(x,y)\overset{\eqref{consistency phi}}{=}\phi_A(x,t) \phi_A(t,y)
  \overset{\eqref{induction psi3}}{=}\left\{\begin{matrix}\psi_A(x,t) \phi_n(t,y)\\ \phi_A(x,t) \psi_n(t,y)\end{matrix}\right..
  \end{equation}
  \item If we have \(x\in \dom_n\), \(y\in \dom_n\cap U_z\) case one and three overlap. Here both cases are equal to
  \(\psi_n(x,y)\), since \(x,y\in \dom_n\) and we may apply \eqref{induction psi2} and \eqref{induction psi3}:
  \begin{equation}
  \psi_n(x,y)\overset{\eqref{induction psi2}}{=}\psi_n(x,t)\psi_n(t,y)\overset{\eqref{induction psi3}}{=}\psi_n(x,t)\phi_A(t,y).
  \end{equation}
  \item If we have \(y\in \dom_n\), \(x\in \dom_n\cap U_z\) case one and four overlap. Here both cases are equal to
  \(\psi_n(x,y)\), since \(x,y\in \dom_n\) and we may apply \eqref{induction psi2} and \eqref{induction psi3}:
  \begin{equation}
  \psi_n(x,y)\overset{\eqref{induction psi2}}{=}\psi_n(x,t)\psi_n(t,y)\overset{\eqref{induction psi3}}{=}\psi_A(x,t)\psi_n(t,y).
  \end{equation}
  \item If we have \(y\in U_z\), \(x\in \dom_n\cap U_z\) case two and three overlap. Here both cases are equal to
  \(\phi_A(x,y)\), since \(x,t\in U_z\) we may first apply equation \eqref{induction psi3}, then equation \eqref{consistency phi}:
  \begin{equation}
  \phi_A(x,y)\overset{\eqref{consistency phi}}{=}\phi_A(x,t)\phi_A(t,y)\overset{\eqref{induction psi3}}{=}\psi_n(x,t)\phi_A(t,y).
  \end{equation}
  \item If we have \(x\in U_z\), \(y\in \dom_n\cap U_z\) case two and four overlap. Here both cases are equal to
  \(\phi_A(x,y)\), since \(y,t\in U_z\) we may first apply equation \eqref{induction psi3}, then equation \eqref{consistency phi}:
    \begin{equation}
  \phi_A(x,y)\overset{\eqref{consistency phi}}{=}\phi_A(x,t)\phi_A(t,y)\overset{\eqref{induction psi3}}{=}\phi_A(x,t)\psi_n(t,y).
  \end{equation}
  \end{enumerate}
We proceed with showing the induction hypothesis for \(\psi_{n+1}\), 
starting with \eqref{induction psi1} for \(\psi_{n+1}\). By the induction hypothesis we know that \(\psi_n(x,y)\) as well as 
\(\phi_A(x,y)\) are lifts of \(S_{xA,yA}\) for any \((x,y)\) in their domain of definition. Therefore we have for \(x,y\in \dom_n \cup U_z\)
 \begin{equation}
 \psi_{n+1}(x,y):=\left\{\begin{matrix}
 \psi_n(x,y) \quad &\text{for }x,y\in \dom_n\\
 \phi_A(x,y) \quad &\text{for } x,y\in U_z\\
 \psi_n(x,t)\phi_A(t,y)\quad &\text{for } x\in \dom_n, y\in U_z\\
 \phi_A(x,t)\psi_n(t,y)\quad &\text{for } y\in \dom_n, x\in U_z.
 \end{matrix} \right.,
 \end{equation} 
where each of the lines is a lift of \(S_{xA,y A}\) whenever the expression is defined.


Equation \eqref{induction psi2} we will again show in a case by case manner depending on where the arguments \(s,k\) and \(l\) are.
\begin{enumerate}
\item Let \(s,k,l\in \dom_{n}\): then equation  \eqref{induction psi2} follows directly from the induction hypothesis.
\item Let \(s,k\in \dom_{n}\) and \(l\in U_z\): then we have
\begin{equation*}
\psi_{n+1}(s,k)\psi_{n+1}(k,l)=\psi_n(s,k)\psi_n(k,t)\phi_A(t,l)\overset{\eqref{induction psi2}}{=} \psi_n(s,t)\phi_A(t,l)=\psi_{n+1}(s,l)
\end{equation*}
\item Let \(s,l\in \dom_n\) and  \(k\in U_z\): then we have
\begin{align*}
\psi_{n+1}(s,k)\psi_{n+1}(k,l)=\psi_{n}(s,t)\phi_A(t,k)\phi_A(t,k)\psi_n(t,l)\overset{\eqref{phi prop3},\eqref{phi prop2}}{=}\\
\psi_{n}(s,t)\psi_n(t,l)\overset{\eqref{induction psi2}}{=}\psi_n(s,l)=\psi_{n+1}(s,l)
\end{align*}
\item Let \(s\in \dom_n\) and  \(k,l\in U_z\): then we have
\begin{align*}
\psi_{n+1}(s,k)\psi_{n+1}(k,l)=\psi_n(s,t)\phi_A(t,k)\phi_A(k,l)\overset{\eqref{phi prop2}}{=}\psi_n(s,t)\phi_A(t,l)
=\psi_{n+1}(s,l).
\end{align*}
\item Let \(k,l\in \dom_n\) and  \(s\in U_z\): then we have
\begin{align*}
\psi_{n+1}(s,k)\psi_{n+1}(k,l)=\phi_A(s,t)\psi_n(t,k)\psi_n(k,l)\overset{\eqref{induction psi2}}{=}\phi_A(s,t)\psi_n(t,l)=\psi_{n+1}(s,l).
\end{align*}
\item Let \(k\in \dom_n\) and  \(s,l\in U_z\): then we have
\begin{align*}
\psi_{n+1}(s,k)\psi_{n+1}(k,l)=\phi_A(s,t)\psi_n(t,k)\psi_n(k,t)\phi_A(t,l)\\
\overset{\eqref{induction psi2}}{=}\phi_A(s,t)\psi(t,t)\phi_A(t,l)
\overset{\eqref{induction psi3},\eqref{phi prop3}}{=}\phi_A(s,t)\phi_A(t,l)\overset{\eqref{phi prop2}}{=} \phi_A(s,l)=\psi_{n+1}(s,l).
\end{align*}
\item Let \(l\in \dom_n\) and  \(s,k\in U_z\): then we have
\begin{align*}
\psi_{n+1}(s,k)\psi_{n+1}(k,l)=\phi_A(s,k)\phi_A(k,t)\psi_n(t,l)\overset{\eqref{phi prop2}}{=}\phi_A(s,t)\psi_n(t,l)=\psi_{n+1}(s,l).
\end{align*}
\item Let \(s,k,l\in U_z\): then we have
\begin{align*}
\psi_{n+1}(s,k)\psi_{n+1}(k,l)=\phi_A(s,k)\phi_A(k,l)\overset{\eqref{phi prop2}}{=}\phi_A(s,l)=\psi_{n+1}(s,l).
\end{align*}
\end{enumerate}
To see \eqref{induction psi3}, that  \(\psi_{n+1}\) is coincides with \(\phi_A\) where both functions are defined pick \(x,y\in \dom_{n+1}\cap\dom\overline{S}\). 
Now in the definition of \(\psi_{n+1}\),  \eqref{psi induction def} we may use the first line to see that \(\psi_{n+1}(x,y)=\phi_A(x,y)\)

Now only uniqueness is left to show. So let \(\tilde{\psi}\) defined on \((\dom_n\cup U_z)\times(\dom_n\cup U_z)\) fulfil 
 \begin{align}\label{uniqueness0}
  \forall (t,s)\in \dom_n\times \dom_n: \tilde{\psi}(t,s) \text{ is a lift of } S_{tA, sA}\\\label{uniqueness psi1}
\forall s,k,l\in\mathbb{R}: \tilde{\psi}(k,s)\tilde{\psi}(s,l)=\tilde{\psi}(k,l),\\\label{uniqueness psi2}
\forall (x,y)\in\dom_n: (xA, yA)\in\dom\overline{S}\Rightarrow  \tilde{\psi}(x,y)=\phi_A(x,y).
\end{align}
Now pick \(x,y \in (\dom_n\cup U_z)\). Let without loss of generality \(y>x\). 
Pick for each \(z \in [x,y]\) an open set \(U_z \subseteq \dom\overline{S}\) then 
\begin{equation}
 \bigcup_{z\in [x,y]} U_z \supseteq [x,y]
\end{equation}
is an open cover of \([x,y]\). Since \([x,y]\) is compact there is a finite selection of \(z_i\in [x,y], 1\le i\le n\in\mathbb{N}\) such that
also 
\begin{equation}
 \bigcup_{i=1}^n U_{z_i}\supseteq [x,y]
\end{equation}
is an open cover of \([x,y]\). Without loss of generality let \(z_1=x, z_n=y\). We next reduce \(\tilde{\psi}(x,y)\) to the following product
\begin{equation}
\tilde{\psi}(x,y)=\prod_{i=1}^{n=1}\tilde{\psi}(z_i,z_{i+1})=\prod_{i=1}^{n=1}\phi_A(z_i,z_{i+1}),
\end{equation}
by repeated application of equation \eqref{uniqueness psi1} and application of \eqref{uniqueness psi2}. By the very same procedure we can reduce \(\psi_{n+1}(x,y)\) to the same 
product, proving \(\psi_{n+1}(x,y)=\tilde{\psi}(x,y)\). This ends the induction and proves the lemma.

\end{proof}


\begin{de}[global lift]\label{de: s hat}
Lemma \ref{lem: phi global} enables us to define a global lift.
For any \(A\in\mathcal{V}\) we define 
\begin{equation}
\hat{S}_{0,A}:=\phi_A(1,0) 1.
\end{equation}
The operator \(\hat{S}\) fulfils the required differential equation and its uniqueness is inherited from the 
uniqueness of \(\phi_A\) for \(A\in\mathcal{V}\).
\end{de}


%\begin{rmk}
%The lift \(\hat{S}_{0,A}\) can also be calculated differently: pick \(N\in\mathbb{N}\) and a series \((\delta_k)_{k\in\mathbb{N}}\subset \mathbb{R}^+\)
% such that \(\|\mathds{1}-S_{\sum_{k=1}^{n-1}\delta_k A,\sum_{k=1}^{n}\delta_k A}\|<1\)  holds true for all \(k\) and \(\sum_{k=1}^{N}\delta_k=1\). 
%Then 

%\textcolor{red}{is there some kind of direct correspondence between \(\|1-S_{0,\delta_n A}\|\) and \(\|1-S_{\sum_{k=1}^{n-1}\delta_k A,\sum_{k=1}^{n}\delta_k A}\|\)??}
%\begin{equation}
%P^-S_{\sum_{k=1}^{n-1}\delta_k A,\sum_{k=1}^{n}\delta_k A}P^-
%\end{equation}
%is invertible. Now, 
%\begin{equation}\label{hat finite spacing}
%\hat{S}_{0,A}=\prod_{n=0}^{N} \overline{S}_{\sum_{k=1}^{n-1}\delta_k A,\sum_{k=1}^{n}\delta_kA}.
%\end{equation}
%Before proving this claim, we notice the following property of \(\overline{S}\): For any  \(A\in\mathcal{V}\) and \(\alpha, \beta, \gamma>0\)
% such that all factors exist the following identity holds
%\begin{align}
%\overline{S}_{\alpha A, \gamma A} = \overline{S}_{\alpha A, \beta A} \overline{S}_{\beta A, \gamma A}.
%\end{align}
%This property is by \eqref{gamma attri6} equivalent to 
%\begin{equation}\label{gamma equal 1}
%\Gamma_{\alpha A, \beta A, \gamma A}=1.
%\end{equation}
%This claim can be reduced to the one where \(\alpha=1\) by the following renaming scheme
%\begin{align}
%\alpha A&=: \tilde{A}\\
%\beta/\alpha &=: \tilde{\beta}\\
%\gamma/\alpha&=:\tilde{\gamma}.
%\end{align}
%In fact, the claim holds, as
%the following calculation shows:
%\begin{align}
%\ln &\Gamma_{A,\beta A, \gamma A} = \int_1^\beta d \beta' \partial_{\beta'}\ln \Gamma_{A,\beta' A, \gamma A} 
%+ \overbrace{\ln \Gamma_{A,A,\gamma A}}^{\overset{\eqref{gamma attri4}}{=}0}\\
%&= \int_1^\beta d\beta' \left(\int_1^\gamma d\gamma' \partial_{\gamma'}\partial_{\beta'} \ln \Gamma_{A,\beta' A, \gamma' A} 
%+ \partial_{\beta'} \overbrace{\ln \Gamma_{A,\beta' A, A}}^{\overset{\eqref{gamma attri4}}{=}0} \right)\\
%&=\int_1^\beta d\beta' \int_1^\gamma d\gamma' c_A((\beta'-1) A, (\gamma'-1) A)\\
%&=\int_1^\beta d\beta' \int_1^\gamma d\gamma'  (\beta'-1) (\gamma'-1) \overbrace{c_A( A,  A) }^{\overset{\eqref{gamma attri3},\eqref{gamma attri7}}{=}0}=0,
%\end{align}
%where we have used various properties of lemma \ref{gamma attri}.
%
%\textcolor{blue}{generalisation:}
%claim: for all \(A,B\in \mathcal{A}\) and all \(\alpha,\beta\in\mathbb{R}\), \(|\alpha|,|\beta|\) small enough, such that the expression appearing is well defined
%\begin{equation}
%\Gamma_{A,A+\alpha B, A+ \beta B} =1.
%\end{equation}
%proof:
%\begin{align}
%\ln \Gamma_{A,A+\alpha B, A+ \beta B} = \overbrace{\ln \Gamma_{A,A+\alpha B, A}}^{=0}+ \int_0^{\beta} d\beta' \partial_{\beta'} \ln \Gamma_{A,A+\alpha B, A+ \beta' B}\\
%=\int_0^{\beta} d\beta' \partial_{\beta'} \left(\overbrace{\ln \Gamma_{A,A, A+ \beta' B}}^{=0}+ \int_0^{\alpha}d\alpha' \partial_{\alpha'} \ln \Gamma_{A,A+\alpha' B, A+ \beta' B}\right)\\
%=\int_0^{\beta} d\beta'  \int_0^{\alpha}d\alpha' \partial_{\beta'} \partial_{\alpha'} \ln \Gamma_{A,A+\alpha' B, A+ \beta' B}\\
%=\int_0^{\beta} d\beta'  \int_0^{\alpha}d\alpha' c_A(\alpha' B, \beta' B) 
%=\int_0^{\beta} d\beta'  \int_0^{\alpha}d\alpha' \alpha' \beta' \overbrace{ c_A( B,  B)}^{=0}.
%\end{align}


%Because of \eqref{gamma equal 1} we see that \eqref{hat finite spacing} is actually independent of the choice of sequence \((\delta_k)_{k}\).
%The right side of equation \eqref{hat finite spacing} satisfies the initial condition of the ordinary differential equation, so we only need to check the differential equation itself.
%Next we reformulate the ordinary differential equation \eqref{def s hat}, pick \(B=t A\) then we have
%\begin{equation}
%0=\partial_B \frac{\hat{S}_{0,A+B}}{\hat{S}_{0,A}\overline{S}_{A,A+B}}
%=\partial_\varepsilon \frac{\hat{S}_{0,A(1+\varepsilon t)}}{\hat{S}_{0,A}\overline{S}_{A,A(1+\varepsilon t)}}
%=t\partial_\varepsilon \frac{\hat{S}_{0,A(1+\varepsilon)}}{\hat{S}_{0,A}\overline{S}_{A,A(1+\varepsilon)}}
%\end{equation}


%What is more, we can pick \((\delta_k)_k\) such that \(\sum_{k=1}^N \delta_k=1\) and \(\delta_{N+1}=\varepsilon\) then we have

%\begin{align}
%\hat{S}_{0,A(1+\varepsilon)}=\prod_{n=0}^{N+1} \overline{S}_{\sum_{k=1}^{n-1}\delta_k A,\sum_{k=1}^{n}\delta_kA}
%=\prod_{n=0}^{N} \overline{S}_{\sum_{k=1}^{n-1}\delta_k A,\sum_{k=1}^{n}\delta_kA}~ \overline{S}_{A,A(1+\varepsilon)}\\
%=\hat{S}_{0,A}\overline{S}_{A,A+B},
%\end{align}

%or in other words

%\begin{equation}
%\frac{\hat{S}_{0,A+B}}{\hat{S}_{0,A}\overline{S}_{A,A+B}}=1.
%\end{equation}
%If \(A=0\) holds, we see directly that 

%\begin{equation}
%\frac{\hat{S}_{0,B}}{\hat{S}_{0,0}\overline{S}_{0,B}}= 1,
%\end{equation}

%so in both cases the ordinary differential equation is satisfied. 
%\end{rmk}

\begin{de}[relative phase]
Let \(A,B \in\dom\overline{S}\), we define \(z(A,B)\in S^1\) by
\begin{equation}\label{def z}
z(A,B):=\frac{\hat{S}_{0,B}}{\hat{S}_{0,A}\overline{S}_{A,B}}.
\end{equation}
Please note that for such \(A,B\) the lift \(\bar{S}_{A,B}\) is well defined. This means that the product in the denominator is a lift of \(S_{0,B}\)
and according to def \ref{de: s hat} the ratio is well defined.
\end{de}


\begin{lem}[properties of the relative phase]
For all \((A,F),(F,G),(G,A)\in\dom\overline{S}\), as well as for all \(H,K\in \mathcal{V}\), we have
\begin{align}\label{z antisym}
z(A,F)&=z^{-1}(F,A)\\\label{z gamma}
z(F,A)z(A,G)z(G,F)&=\Gamma_{F,A,G}\\\label{z c}
\partial_{\varepsilon_1}\partial_{\varepsilon_2}\ln  z_(A+\varepsilon_1 H,A+\varepsilon_2 K)&=c_A(H,K).
\end{align}
\end{lem}
\begin{proof}
Pick \(A,F,G\in\mathcal{V}\) as in the lemma. We start off by analysing
\begin{align}
&\hat{S}_{0,F}\overline{S}_{F,G}\overset{\eqref{def z}}{=}z(A,F)\hat{S}_{0,A}\overline{S}_{A,F}\overline{S}_{F,G}\\\label{phase comparison}
&\overset{\eqref{gamma attri6}}{=}z(A,F) \Gamma_{A,F,G}^{-1} \hat{S}_{0,A} \overline{S}_{A,G}.
\end{align}
Exchanging  \(A\) and \(F\) in this equation yields
\begin{equation}
\hat{S}_{0,A}\overline{S}_{A,G}=z(F,A) \Gamma_{F,A,G}^{-1} \hat{S}_{0,F} \overline{S}_{F,G}.
\end{equation}
This is equivalent to
\begin{equation}
\hat{S}_{0,F} \overline{S}_{F,G}=z(F,A)^{-1} \Gamma_{F,A,G} \hat{S}_{0,A}\overline{S}_{A,G}~,
\end{equation}
taking \eqref{gamma attri3} into account this means that 
\begin{equation}
z(A,F)=z^{-1}(F,A)
\end{equation}
holds true. Equation \eqref{phase comparison} solved for \(\hat{S}_{0,A}\overline{S}_{A,G}\) also gives us

\begin{align}
\hat{S}_{0,G}\overset{\eqref{def z}}{=}z(A,G)\hat{S}_{0,A}\overline{S}_{A,G}\overset{\eqref{phase comparison}}{=}z(A,G)z^{-1}(A,F)\Gamma_{A,F,G}\hat{S}_{0,F}\overline{S}_{F,G}.
\end{align}
The latter equation compared with 
\begin{equation}
\hat{S}_{0,G}\overset{\eqref{def z}}{=}z(F,G)\hat{S}_{0,F}\overline{S}_{F,G},
\end{equation}
yields  a direct connection between \(\Gamma\) and \(z\):
\begin{equation}
\frac{z(A,G)}{z(A,F)}\Gamma_{A,F,G}=z(F,G),
\end{equation}
or by \eqref{z antisym}
\begin{equation}
\Gamma_{A,F,G}=z(F,G)z(A,F)z(G,A).
\end{equation}
Finally, in this equation we replace \(F=A+\varepsilon_1 H\) as well as  \(G=A+\varepsilon_2 K\), where \(\varepsilon_1,\varepsilon_2\) is small enough so that 
\(z\) and \(\Gamma\) are still well defined. Then we take the logarithm and derivatives to find
\begin{equation}
 \partial_{\varepsilon_1}\partial_{\varepsilon_2}\ln  z(A+\varepsilon_1 H,A+\varepsilon_2 K)=\partial_{\varepsilon_1}\partial_{\varepsilon_2}\ln\Gamma_{A,A+\varepsilon_1 H,A+\varepsilon_2 K}\overset{\eqref{gamma attri7}}{=}c_A(H,K). 
\end{equation}

\end{proof}

So we find that \(c\) is the mixed derivative of \(z\). In the following we will characterise \(z\) more thoroughly by \(c\) and \(c^+\).
\begin{de}[p-forms of four potentials, phase integral]
We introduce for \(p\in\mathbb{N}\)
\begin{align}\notag
&\Omega^p:=\{\omega :\mathcal{V}^{p+1}\rightarrow \mathbb{C}\mid \forall \sigma \in S^p, \alpha\in\mathbb{R}, A_1,\dots, A_{p+1}\in\mathcal{V}: \\
&\hspace{2cm}\omega(A_1,\alpha A_{\sigma(1)},\dots, A_{\sigma(p)})=\mathrm{sgn}(\sigma) \alpha \omega(A_1,A_2,\dots, A_2)\},
\end{align}
where \(\mathrm{sgn}(\sigma)\) is the sign of the permutation \(\sigma\).
We define the one form \(\chi\in \Omega^1(\mathcal{V})\) by
\begin{equation}\label{de chi}
\chi_A(B):=\partial_B\ln z(A,A+B)
\end{equation}
for all \(A,B\in\mathcal{V}\).
Furthermore for a differential form \(\omega\in \Omega^p(\mathcal{V})\) for some \(p\in\mathbb{N}\) we define the exterior derivative of 
\(\omega\), \(d \omega\in\Omega^{p+1}(\mathcal{V})\) by
\begin{equation}
(d\omega)_A(B_1,\dots, B_{p+1}):=\sum_{k=1}^{p+1} (-1)^{k+1} \partial_{B_k}\omega_{A+B_k}(B_1,\dots , \cancel{B_k},\dots, B_{p+1}),
\end{equation}
for general \(A,B_1,\dots, B_{p+1}\in\mathcal{V}\), where the notation \(\cancel{B_k}\) denotes that \(B_k\) is not to be inserted as an argument.

\end{de}

\begin{lem}[connection between \(c\) and the relative phase]\label{connection between c and the relative phase}
The differential form \(\chi\) fulfils 
\begin{equation}
(d\chi)_A(F,G)=2 c_A(F,G)
\end{equation}
for all \(A,F,G\in\mathcal{V}\).
\end{lem}
\begin{proof}
Pick \(A,F,G\in \mathcal{V}\), we calculate
\begin{align}
&(d\chi)_A(F,G)=\partial_F\partial_G \ln z(A+F,A+F+G)-\partial_F \partial_G \ln z(A+G,A+F+G)\\
&=\partial_F\partial_G (\ln  z(A,A+F+G)+\ln z(A+F,A+G))\\
 &\quad - \partial_F\partial_G(  \ln z(A,A+F+G)+\ln z(A+G,A+F))\\
&\overset{\eqref{z antisym}}{=} 2 \partial_F\partial_G \ln z(A+F,A+G)\overset{\eqref{z c}}{=}2 c_A(F,G).
\end{align}
\end{proof}

Now since \(d c=0\), we might use Poincaré's lemma as a method independent of \(z\) to construct a differential form \(\omega\) such that \(d\omega=c\). 
In order to execute this plan, we first need to prove Poincaré's lemma for our setting:

\begin{lem}[Poincaré]\label{lem poincare}
Let \(\omega\in \Omega^p(\mathcal{V})\) for \(p\in\mathbb{R}\) be closed, i.e. \(d \omega =0\). Then \(\omega\) is also exact, more precisely we have
\begin{equation}
\omega=d \int_{0}^1 \iota^*_t i_X f^* \omega dt,
\end{equation}
where \(X\), \(\iota_t\) for \(t\in\mathbb{R}\) and \(f\) are given by
 \begin{align}
 &X: \mathbb{R}\times\mathcal{V}\rightarrow \mathbb{R}\times\mathcal{V},\\
 &\hspace{2cm} (t,B)\mapsto (1,0) \\
&\forall t \in \mathbb{R}: \iota_t: \mathcal{V}\rightarrow \mathbb{R}\times\mathcal{V},\\
&\hspace{2cm} B\mapsto (t,B)\\
&f:\mathbb{R}\times \mathcal{V}\mapsto \mathcal{V},\\
&\hspace{2cm} (t,B) \mapsto t B\\
 &\forall t \in \mathbb{R}: f_t:=f(t,\cdot).
 \end{align}
\end{lem}
\begin{proof}
Pick some \(\omega \in \Omega^p(\mathcal{V})\).
We will first show the more general formula 
\begin{equation}\label{poincare more general}
f^*_b\omega-f^*_a \omega=d \int_{a}^b \iota_t^* i_X f^* \omega ~dt+ \int_{a}^b \iota_t^* i_X f^* d \omega dt.
\end{equation}
The lemma follows then by \(b=1, a= 0\), \(f^*_1\omega=\omega, f^*_0 \omega=0\) and \(d \omega=0\) for a closed \(\omega\). 
We begin by rewriting the right hand side of \eqref{poincare more general}:
\begin{equation}\label{poincare 1 manipulation}
d \int_{a}^b \iota_t^* i_X f^* \omega ~dt+ \int_{a}^b \iota_t^* i_X f^* d \omega dt=
\int_a^b (d\iota_t^* i_X f^* \omega+ \iota_t^* i_X f^* d \omega )dt.
\end{equation}
Next we look at both of these terms separately. Let therefore \(p\in \mathbb{N}\), \(t, s_k\in \mathbb{R}\) and \(A,B_k\in \mathcal{V}\) for each \(p+1\ge k\in\mathbb{N}\).
First, we calculate \(d \iota^*_t i_X f^* \omega\)
\begin{align}
&(f^*\omega)_{(t,A)}((s_1,B_1),\dots, (s_p,B_p))=\omega_{tA}(s_1A+tB_1,\dots, s_p A+t B_p)\\[0.3cm]
&(i_X f^* \omega)_{(t,A)}((s_1,B_1),\dots, (s_{p-1},B_{p-1}))=\omega_{tA}(A,s_1A+tB_1,\dots, s_{p-1}A + t B_{p-1}\\[0.3cm]
&(\iota^*_t i_X f^* \omega)_{A}(B_1,\dots, B_{p-1})= t^{p-1} \omega_{t A}(A,B_1,\dots, B_{p-1})\\[0.3cm]\notag
&(d\iota^*_t i_X f^* \omega)_{A}(B_1,\dots, B_{p})=\partial_\varepsilon |_{\varepsilon=0} \sum_{k=1}^p (-1)^{k+1} t^{p-1} \omega_{t A + \varepsilon t B_k}(A,B_1,\dots, \cancel{B_k},\dots, B_p)\\
&+ \partial_\varepsilon|_{\varepsilon=0} \sum_{k=1}^p (-1)^{k+1} t^{p-1} \omega_{tA}(A+\varepsilon B_k,B_1,\dots, \cancel{B_k},\dots, B_p)\\\label{li derivative 1}
&=\partial_{\varepsilon}|_{\varepsilon=0}\sum_{k=1}^p t^p (-1)^{k+1} \omega_{tA+\varepsilon B_k}(A,B_1,\dots, \cancel{B_k},\dots, B_p)+p t^{p-1}\omega_{tA}(B_1,\dots, B_p).
\end{align}


Now, we calculate \(\iota^*_t i_X f^* d \omega\):
\begin{align}
&(d\omega)_A(B_1,\cdots, B_{p+1})=\partial_\varepsilon |_{\varepsilon=0} \sum_{k=1}^{p+1} (-1)^{k+1} \omega_{A+\varepsilon B_k}(B_1,\dots , \cancel{B_k}, \dots, B_{p+1})\\[0.3cm]
&(f^* d \omega){(t,A)}((s_1,B_1),\dots , (s_{p+1},B_{p+1}))= (d\omega)_{tA}(s_1A + t B_1, \dots, s_{p+1}A+t B_{p+1})\\
&=\partial_{\varepsilon}|_{\varepsilon =0} \sum_{k=1}^{p+1}(-1)^{k+1} \omega_{tA + \varepsilon(s_kA + t B_k)}(s_1A+tB_1, \dots,  \cancel{s_k A + t B_k},\dots , s_p A  + t B_p )\\[0.4cm]
&(i_X f^* d \omega)_{(t,A)}((s_1,B_1),\dots , (s_p, B_p))= \partial_{\varepsilon} |_{\varepsilon =0} \omega_{(t+\varepsilon)A}(s_1A+tB_1, \dots, s_pA+t B_p)\\
&+\partial_{\varepsilon}|_{\varepsilon=0} \sum_{k=1}^p (-1)^k \omega_{tA + \varepsilon(s_k A + t B_k)} (A,s_1A+t B_1, \dots, \cancel{s_k A + t B_k},\dots, s_p A + t B_p)\\
&= t^p \partial_{\varepsilon}|_{\varepsilon=0}\omega_{(t+\varepsilon)A }(B_1,\dots, ,\cancel{B_k},\dots, B_p))\\
&\hspace{3cm}+ \sum_{k=1}^p s_k t^{p-1} (-1)^{k+1} \partial_{\varepsilon}|_{\varepsilon=0}\omega_{(t+\varepsilon)A}(A,B_1,\dots, B_p) \\
&+\partial_{\varepsilon}|_{\varepsilon=0}\sum_{k=1}^p(-1)^k t^{p-1}(\omega_{(t+s_k\varepsilon)A}(A,B_1,\dots, \cancel{B_k},\dots, B_p) \\
&\hspace{3cm} + \omega_{tA + \varepsilon t B_k}(A,B_1, \dots, \cancel{B_k},\dots, B_p))\\
&=t^p\partial_{\varepsilon}|_{\varepsilon=0} \left(\omega_{(t+\varepsilon)A}(B_1,\dots, B_p) +\sum_{k=1}^p (-1)^k \omega_{tA+\varepsilon B_k}(A,B_1,\dots, \cancel{B_k},\dots, B_p)\right)\\
&(\iota_t^* i_X f^* d \omega)_{A}(B_1,\dots ,  B_p)=t^p\partial_{\varepsilon}|_{\varepsilon=0} \Big(\omega_{(t+\varepsilon)A}(B_1,\dots, B_p)\\\label{li derivative 2}
&\hspace{5cm}+\sum_{k=1}^p (-1)^k \omega_{tA+\varepsilon B_k}(A,B_1,\dots, \cancel{B_k},\dots, B_p)\Big)
\end{align}

Adding \eqref{li derivative 1} and \eqref{li derivative 2} we find for \eqref{poincare 1 manipulation}:

\begin{align}
\int_a^b (d\iota_t^* i_X f^* \omega+ \iota_t^* i_X f^* d \omega )dt=\\
\int_a^b  \Big( t^p\partial_{\varepsilon}|_{\varepsilon=0} \omega_{(t+\varepsilon)A}(B_1,\dots, B_p)
+p t^{p-1}\omega_{tA}(B_1,\dots, B_p)\Big) dt\\
=\int_a^b  \frac{d}{dt} (t^p \omega_{tA}(B_1,\dots, B_p))dt =\int_a^b  \frac{d}{dt} (f^*_t \omega)_{A}(B_1,\dots, B_p)dt\\
=(f^*_b\omega)_A(B_1,\dots,B_p)-(f^*_a\omega)_A(B_1,\dots,B_p).
\end{align}

\end{proof}

\begin{de}[integral of a closed p form]
For a closed exterior form \(\omega\in\Omega^{p}(\mathcal{V})\) we define the form \(\prod [\omega]\)
\begin{equation}
\prod\![\omega]:=\int_{0}^1 \iota^*_t i_X f^* \omega dt.
\end{equation}
For \(A,B_1,\dots , B_{p-1}\in\mathcal{V}\) it takes the form 
\begin{equation}
\prod\![\omega]_A(B_1,\dots, B_p)=\int_0^1 t^{p-1} \omega_{tA}(A,B_1,\dots, B_{p-1})dt.
\end{equation}
By lemma \ref{lem poincare} we know \(d\prod [\omega]=\omega\) if \(d\omega=0\).
\end{de}

Now we found two one forms each produces \(c\) when the exterior derivative is taken. The next lemma informs us about their relationship.

\begin{lem}[inversion of lemma \ref{connection between c and the relative phase}]
The following equality holds
\begin{equation}
\chi=2 \prod\![c].
\end{equation}
\end{lem}
\begin{proof}
We have \(d(\chi-2 \prod\![c])=0\) so by lemma \ref{lem poincare} we know that there is \(v:\mathcal{V}\rightarrow \mathbb{R}\) such that
\begin{equation}
dv=\chi-2 \prod\![c]
\end{equation}
holds. Now \eqref{def s hat} translates into the following ODE for \(z\):
\begin{equation}
\partial_B \ln z(0,B)=0, \quad \partial_\varepsilon \ln z(A,(1+\varepsilon)A)|_{\varepsilon =0}=0
\end{equation}
for all \(A,B\in\mathcal{V}\). This means that
\begin{equation}
\chi_0(B)=0=\prod\![c]_0(B), \quad \chi_{A,A}=0=\prod\![c]_A(A)
\end{equation}
hold. This implies
\begin{equation}
\partial_\varepsilon v_{\varepsilon A}=0, \quad \partial_\varepsilon v_{A+\varepsilon A}=0,
\end{equation}
which means that \(v\) is constant.
\end{proof}


From this point on we will assume the existence of  a function \(c^+\) fulfilling \eqref{c+ 1},\eqref{c+ 2} and \eqref{c+ 3}.
Recall equation \eqref{c+ 2}: 
\begin{equation}
\forall A,F,G,H: \partial_H c_{A+H}^+(F,G)=\partial_G c^+_{A+G}(F,H).
\end{equation}

For a fixed \(F\in\mathcal{V}\), this condition can be read as \(d( c^+_{\cdot} (F,\cdot))=0\). As a consequence we can apply lemma \ref{lem poincare} to define a one form.

\begin{de}[integral of the causal splitting]
For any \(F\in\mathcal{V}\), we define

\begin{equation}
\beta_A(F):=2 \prod\![c^+_{\cdot}(F,\cdot)]_A.
\end{equation}
\end{de}

\begin{lem}[relation between the integral of the causal splitting and the phase integral]
The following two equations hold:
\begin{align}\label{beta c}
&d \beta=-2 c\\
&d(\beta +\chi)=0.
\end{align}
\end{lem}
\begin{proof}
We start with the exterior derivative of \(\beta\). Pick \(A,F,G\in\mathcal{V}\):
\begin{align}
d\beta_A(F,G)=\partial_F \beta_{A+F}(G)-\partial_G \beta_{A+G}(F)\\
=d\Big(\prod\![c^+_\cdot(G,\cdot)]\Big)_A(F)-d\Big(\prod\![c^+_\cdot(F,\cdot)]\Big)_A(G)\\
=2 c^+_A(G,F)-2 c_A^+(F,G)\overset{\eqref{c+ 1}}{=}-2 c_A(F,G).
\end{align}
This proves the first equality. The second equality follows directly by \(d \chi=2 c\).
\end{proof}

\begin{de}[corrected lift]
Since \(\beta+\chi\) is closed, we may use lem \ref{lem poincare} again to define the phase
\begin{equation}\label{def alpha}
\alpha:=\prod\![\beta+\chi].
\end{equation}
Furthermore, for all \(A,B\in\mathcal{V}\) we define the corrected second quantised scattering operator 
\begin{align}
&\tilde{S}_{0,A}:=e^{-\alpha_A} \hat{S}_{0,A}\\
&\tilde{S}_{A,B}:=\tilde{S}^{-1}_{0,A}\tilde{S}_{0,B}.
\end{align}
\end{de}

\begin{cor}[group structure of the corrected lift]
We have \(\tilde{S}_{A,B} \tilde{S}_{B,C}=\tilde{S}_{A,C}\) for all \(A,B,C\in\mathcal{V}\).
\end{cor}

\begin{thm}[causality of the corrected lift]
The corrected second quantised scattering operator fulfils the following causality condition for all \(A,F,G\in \mathcal{V}\) such that \(F\prec G\):
\begin{equation}
\tilde{S}_{A,A+F}=\tilde{S}_{A+G,A+G+F}.
\end{equation}
\end{thm}
\begin{proof}
Let \(A,F,G\in\mathcal{V}\) such that \(F\prec G\) We note that for the first quantised scattering operator we have
\begin{equation}
S_{A+G,A+G+F}=S_{A,A+F},
\end{equation}
so by definition of \(\overline{S}\) we also have
\begin{equation}\label{s bar causal}
\overline{S}_{A+G,A+G+F}=\overline{S}_{A,A+F}.
\end{equation}
So for any lift this equality is true up to a phase, meaning that 

\begin{equation}\label{f causal}
f(A,F,G):=\frac{\tilde{S}_{A+G,A+G+F}}{\tilde{S}_{A,A+F}}
\end{equation}
is well defined. We see immediately
\begin{equation}\label{vanish at axis}
f(A,0,G)=1=F(A,F,0).
\end{equation}

Pick \(F_1,F_2\prec G_1,G_2\). We abbreviate \(F=F_1+F_2, G=G_1+G_2\) and we calculate
\begin{align}
&f(A,F,G)=\frac{\tilde{S}_{A+G,A+F+G}}{\tilde{S}_{A,A+F}}\\
&=\frac{\tilde{S}_{A+G,A+F+G}}{\tilde{S}_{A+G_1,A+G_1+F}}\frac{\tilde{S}_{A+G_1,A+G_1+F}}{\tilde{S}_{A,A+F}}\\
&=\frac{\tilde{S}_{A+G,A+G+F_1} \tilde{S}_{A+G+F_1,A+F+G}}{\tilde{S}_{ A+G_1,A+F_1+G_1} \tilde{S}_{ A+G_1+F_1,A+G_1+F}}  \frac{\tilde{S}_{A+G_1,A+G_1+F}}{\tilde{S}_{A,A+F}}\\
&=\frac{\tilde{S}_{A+G,A+G+F_1}}{\tilde{S}_{A+G_1,A+F_1+G_1}} \frac{\tilde{S}_{A+G+F_1,A+F+G}}{\tilde{S}_{A+G_1+F_1,A+G_1+F}}   f(A,G_1,F_1+F_2)\\
&=f(A+G_1,F_1,G_2)f(A+G_1+F_1,G_2,F_2)f(A,G_1,F_1+F_2).
\end{align}
Taking the logarithm and differentiating we find:
\begin{equation}\label{shift to small G,F}
\partial_{F_2}\partial_{G_2}\ln f(A,F_1+F_2,G_1+G_2)=\partial_{F_2}\partial_{G_2}\ln f(A+F_1+G_1,F_2,G_2).
\end{equation}
Next we pick \(F_2=\alpha_1 F_1\) and \(G_2=\alpha_2 G_1\) for \(\alpha_1,\alpha_2\in\mathbb{R}^+\) small enough so that

 \begin{align}
 \|1-S_{A+F+G,A+F_1+G_1}\|<1\\
 \|1-S_{A+F+G,A+F_1+G}\|<1\\
 \|1-S_{A+F+G,A+F+G_1}\|<1
 \end{align} 
 hold. We abbreviate \(A'=A+G_1+F_1\), use \eqref{def z} and compute
 
 \begin{align}\notag
 f(A',F_2,G_2)=\frac{e^{-\alpha_{A'+F_2+G_2}+\alpha_{A'+G_2}}z(A',A'+F_2+G_2)z^{-1}(A',A'+G_2)}{e^{-\alpha_{A'+F_2}+\alpha_{A'}}z(A',A'+F_2)z^{-1}(A',A')} \\
 \frac{\overline{S}_{A'+G_2,A'}\overline{S}_{A',A'+F_2+G_2}}{\overline{S}_{A',A'}\overline{S}_{A',A'+F_2}}.
 \end{align}

The second factor in this product can be simplified significantly:
\begin{align}
 \frac{\overline{S}_{A'+G_2,A'}\overline{S}_{A',A'+F_2+G_2}}{\overline{S}_{A',A'}\overline{S}_{A',A'+F_2}}
 = \frac{\overline{S}_{A'+G_2,A'}\overline{S}_{A',A'+F_2+G_2}}{\overline{S}_{A',A'+F_2}}\\
 \overset{\eqref{gamma attri6}}{=}\Gamma^{-1}_{A'+G_2,A',A'+F_2+G_2}\frac{\overline{S}_{A'+G_2,A'+F_2+G_2}}{\overline{S}_{A',A'+F_2}}\\
 \overset{\eqref{s bar causal}}{=}\Gamma_{A',A'+G_2,A'+F_2+G_2}\\
\overset{ \eqref{z gamma}}{=}z(A',A'+G_2)z(A'+G_2,A'+G_2+F_2)z(A'+F_2+G_2,A').
\end{align}
So in total we find

\begin{align}
f(A',F_2,G_2)=\frac{e^{-\alpha_{A'+F_2+G_2}+\alpha_{A'+G_2}}z(A',A'+F_2+G_2)z^{-1}(A',A'+G_2)}{e^{-\alpha_{A'+F_2}+\alpha_{A'}}z(A',A'+F_2)z^{-1}(A',A')}  \\\
\times z(A',A'+G_2)z(A'+G_2,A'+G_2+F_2)z(A'+F_2+G_2,A')\\
=\exp(-\alpha_{A'+F_2+G_2}+\alpha_{A'+G_2}+\alpha_{A'+F_2}-\alpha_{A'})\\
\times z(A'+G_2,A'+G_2+F_2)z^{-1}(A',A'+F_2).
\end{align}
Most of the factors do not depend on \(F_2\) and \(G_2\), so taking the mixed logarithmic derivative things simplify:
\begin{align}
\partial_{G_2}\partial_{F_2} \ln f(A',F_2,G_2)= \partial_{G_2}\partial_{F_2} ( -\alpha_{A'+F_2+G_2} + \ln z(A'+G_2,A'+G_2+F_2))\\
\overset{\eqref{def alpha}, \eqref{de chi}}{=}  \partial_{G_2} (-\beta_{A'+G_2}(F_2)-\chi_{A'+G_2}(F_2) + \chi_{A'+G_2}(F_2))\\
\overset{\eqref{beta c}}{=}2 c^+_{A'}(F_2,G_2)\overset{F_2\prec G_2, \eqref{c+ 3}}{=}0.
\end{align}
So by \eqref{shift to small G,F} we also have
\begin{equation}
\partial_{F_2}\partial_{G_2}\ln f(A,F_1+F_2,G_1+G_2)=0=\partial_{\alpha_1}\partial_{\alpha_2}\ln f(A,F_1(1+\alpha_1),G_1(1+\alpha_2))
\end{equation}
But then we can integrate and obtain
\begin{align}
&0=\int_{-1}^0d \alpha_1 \int_{-1}^0d \alpha_2  \partial_{\alpha_1}\partial_{\alpha_2}\ln f(A,F_1(1+\alpha_1),G_1(1+\alpha_2))\\
&=\ln f(A,F_1,G_1)-\ln f(A,0,G_1)-\ln f(A,F_1,0) + \ln f(A,0,0)\\
&\overset{\eqref{vanish at axis}}{=}\ln f(A,F_1,G_1).
\end{align}
remembering equation \eqref{f causal}, the definition of \(f\),  this ends our proof.
\end{proof}

Using \(\tilde{S}\) we introduce the current associated to it.

%\begin{de}
%Let \(A,F\in\mathcal{V}\), define
%\begin{equation}
%j_A(F):=i\partial_F \left\langle\Omega, \tilde{S}_{A,A+F}\Omega\right\rangle = i \partial_F \ln \left\langle\Omega, \tilde{S}_{A,A+F}\Omega\right\rangle.
%\end{equation}
%\end{de}

\begin{thm}[evaluation of the current of the corrected lift]
For general \(A,F\in\mathcal{V}\) we have
\begin{equation}
j_A(F)=-i\beta_A(F).
\end{equation}
So in particular for \(G\in\mathcal{V}\)
\begin{equation}
\partial_G j_{A+G}(F)=-2i c^+_A(F,G).
\end{equation}
holds.
\end{thm}
\begin{proof}
Pick \(A,F\in\mathcal{V}\) as in the theorem. We calculate
\begin{align}
i\partial_F \ln \left\langle\Omega, \tilde{S}_{A,A+F}\Omega\right\rangle\\
=i\partial_F\left( -\alpha_{A+F}-\alpha_A + \ln \left\langle\Omega, \hat{S}_{0,A}^{-1} \hat{S}_{0,A+F}\Omega\right\rangle\right)\\
=i\partial_F\left( -\alpha_{A+F}+\ln z(A,A+F) + \ln \left\langle\Omega, \overline{S}_{A,A+F}\Omega\right\rangle\right)
\end{align}
The last summand vanishes, as can be seen by the following calculation
\begin{align}
\partial_{F} \ln \left\langle \Omega, \overline{S}_{A,A+F}\Omega\right\rangle
=i\partial_F\ln \det_{\mathcal{H}^-} (P^-S_{A,A+F}P^-\AG(P^-S_{A,A+F}P^-)^{-1})\\
=i\partial_F\ln \det_{\mathcal{H}^-} |P^-S_{A,A+F}P^-|=\frac{i}{2} \partial_F \ln \det_{\mathcal{H}^-} ((P^- S_{A,A+F}P^-)^*P^- S_{A,A+F}P^-)\\
=\frac{i}{2} \partial_F \det (P^- S_{A+F,A}P^-S_{A,A+F}P^-)=\frac{i}{2}\tr(\partial_F P^-S_{A+F,A}P^-S_{A,A+F}P^-)\\
=\frac{i}{2}\tr(\partial_F P^-S_{A,A+F}P^-+\partial_F P^-S_{A+F,A}P^-)=0\end{align}
where we made use of \eqref{diff det}.
So we are left with
\begin{equation}
j_A(F)=i\partial_F (-\alpha_{A+F}+\ln z(A,A+F))=i(-\beta_A(F)-\chi_A(F)+\chi_A(F))=-i\beta_A(F).
\end{equation}

Finally by taking the derivative with respect to \(G\in\mathcal{V}\) and using the definition of \(\beta\) we find
\begin{equation}
\partial_G j_{A+G}(F)=-2i c_A^+(F,G).
\end{equation}


\end{proof}


\bibliographystyle{plain}
\bibliography{ref}

\end{document}





