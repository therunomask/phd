\documentclass[a4paper,12pt]{article}

\usepackage{german}

\usepackage{graphicx}

\usepackage{amssymb}

\usepackage{amsfonts}

\usepackage{amsmath}

\usepackage{amsthm}

\usepackage{slashed}

%identity sign
\usepackage{dsfont}

%commutative diagrams
\usepackage{amsmath,amscd}



\addtolength{\textwidth}{2.2cm} \addtolength{\hoffset}{-1.0cm}

\addtolength{\textheight}{3.0cm} \addtolength{\voffset}{-2cm} 

\parindent 0cm

\pagestyle{empty}



\begin{document}

Gesucht waren für \(n \in \mathbb{N}\), \(z_1,\dots z_n\in ]0,1[ \), 
\(k_1, \dots, k_{n-1}\in \mathbb{R}^4+i \text{Causal}^+\) sodass

\begin{equation}
\sum_{\stackrel{l,j=0}{j>l}}^{n-1} z_j z_l \left( \sum_{c=l+1}^j k_c \right)^2\in \mathbb{R}^+
\end{equation}
gilt.

Ein Beispiel hierfür ist \(n=4, \forall j: z_j=\frac{1}{4}\), \(k_1=\left( -\frac{7}{3} + i \frac{2}{\sqrt{15}}\right) e_0, k_2=k_3=\left(1+i \frac{2}{\sqrt{15}}\right) e_0\), hierfür gilt nämlich

\begin{align}
&\sum_{\stackrel{l,j=0}{j>l}}^{3} z_j z_l \left( \sum_{c=l+1}^j k_c \right)^2\\
&=\frac{1}{16} \left(k_1^2 + k_2^2 + k_3^2 + (k_1+k_2)^2 + (k_2+k_3)^2 + (k_1+k_2+k_3)^2 \right)= \frac{1}{2}.
\end{align}

\end{document}

