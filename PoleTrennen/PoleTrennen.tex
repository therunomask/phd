\documentclass[a4paper,12pt]{article}

\usepackage{german}

\usepackage{graphicx}

\usepackage{amssymb}

\usepackage{amsfonts}

\usepackage{amsmath}

\usepackage{amsthm}

\usepackage{slashed}

\usepackage{todonotes}

%identity sign
\usepackage{bbm}

%commutative diagrams
\usepackage{amsmath,amscd}

\newcommand{\equaltext}[1]{\ensuremath{\stackrel{\text{#1}}{=}}}
\newcommand{\equalmath}[1]{\ensuremath{\stackrel{#1}{=}}}

\newtheorem{lemma}{Lemma}

\newcommand{\id}{{\mathbbm 1}}

\addtolength{\textwidth}{2.2cm} \addtolength{\hoffset}{-1.0cm}

\addtolength{\textheight}{3.0cm} \addtolength{\voffset}{-2cm} 

\parindent 0cm

\pagestyle{empty}



\begin{document}
Über das Trennen von Polen unseres standard Integraden.


Ich zeige hier, dass zwei beliebige Pole durch Koordinatenwahl getrennt werden können. Nehmen wir an es gibt \(k_1,\dots, k_n \in \mathbb{R}^4+i \text{Causal}^+\backslash \partial\text{Causal}^+\). Dann suchen wir uns zwei Indices \(l,r\in [n]\) mit \(l<r\). Ich erinnere nochmal an die Notation

\begin{equation}
\forall m\in [n]: k(m):=\sum_{c=1}^m k_c.
\end{equation}

Nun wählen wir unsere Koordinaten so, dass \(\Im (k(l)) \in i e_0 \mathbb{R}\) gilt. Die Pole in der komplexen Ebene von \(p^0\) sind dann gegeben durch

\begin{equation}
E_1^\pm = k(l)^0 \pm \sqrt{m^2 + (\vec{p}-\vec{k}(l))^2}, \quad E_2^\pm =k^0(r) \pm \sqrt{m^2 + (\vec{p}-\vec{k}(r))^2},
\end{equation}

wobei wir in in diesen Koordinaten \(\vec{p}\) reell wählen können.

\begin{lemma}
In der eben beschriebenen Situation gilt \(\Im (E_1^\pm)<\Im (E_2^\pm) \) für jede Vorzeichenwahl.
\end{lemma}
\textbf{ Beweis:} Zuerst bemerken wir, dass der Wurzelterm in \(E_1\) vergessen werden kann, da er reell ist. Wir wollen also die Ungleichung

\begin{equation}\label{Ung}
\Im (k^0(r)-k^0(l))> \left| \Im\left(\sqrt{m^2+(\vec{p}-\vec{k}(r))^2} \right) \right|
\end{equation}
zeigen. Zuerst bemerken wir, dass wir für eine wir für jede komplexe Zahl \(x=\alpha + i \beta\not \in  \mathbb{R}^-_0\) den Hauptzweig der Wurzel wie folgt schreiben können (unter Verwendung der lokalen Stammfunktion der Wirbelform aus Franz' Analysis 2 Skript)

\begin{equation}
\sqrt{x}= \sqrt{\sqrt{\alpha^2+\beta^2} e^{i 2 \arctan \frac{\beta}{\alpha+\sqrt{\alpha^2+\beta^2}}}}=\sqrt[4]{\alpha^2+\beta^2} e^{i  \arctan \frac{\beta}{\alpha+\sqrt{\alpha^2+\beta^2}}}.
\end{equation}

Für den Imaginärteil gilt also
\begin{equation}\label{imroot}
\Im \sqrt{x} = \sqrt[4]{\alpha^2+\beta^2} \sin \arctan \frac{\beta}{\alpha+\sqrt{\alpha^2+\beta^2}}=\frac{\beta}{\sqrt{2}\sqrt{\alpha + \sqrt{\alpha^2+\beta^2}}}.
\end{equation}

Um die folgende Rechnung lesbar zu halten führe ich noch ein paar Abkürzungen ein:
\begin{subequations}
\begin{align}
a&:= | \vec{p}-\Re (\vec{k}(r))|\\
b&:=|\Im (\vec{k}(r))|\\
a b \cos\varphi&:=(\vec{p}-\Re (\vec{k}(r))) \cdot \Im (\vec{k}(r)).
\end{align}
\end{subequations}

Damit lässt sich die gewünschte Ungleichung \eqref{Ung} wie folgt umschreiben

\begin{equation}
\Im (k^0(r)-k^0(l))> \frac{\sqrt{2} a b \cos \varphi}{\sqrt{m^2 + a^2 - b^2 + \sqrt{(m^2 + a^2 - b^2)+4 a^2 b ^2 \cos^2\varphi}}},
\end{equation}
wir wissen aber, dass \(\Im (k(r)-k(l)\) zeitartig ist und \(\Im (\vec{k}(r)) = \Im(\vec{k}(r)-\vec{k}(l)) \) gilt,
also ist diese Ungleichung gleichbedeutend mit der folgenden

\begin{equation}\label{reduzierteU}
\frac{\sqrt{2} a \cos \varphi}{\sqrt{m^2 + a^2 - b^2 + \sqrt{(m^2 + a^2 - b^2)+4 a^2 b ^2 \cos^2\varphi}}}\le 1.
\end{equation}

Letztere Ungleichung gilt aber, wie sich wie folgt einsehen lässt: Es gilt
\begin{align}
\eqref{reduzierteU}\iff 2 a^2 \cos^2\varphi \le m^2 + a^2 - b^2 + \sqrt{(m^2 + a^2 - b^2)^2 + 4 a^2 b^2 \cos^2 \varphi},
\end{align}

Wenn nun\( (2\cos^2\varphi - 1)a^2 + b^2 - m^2 \le 0\) gilt sind wir fertig, andernfalls ist noch
\begin{align}
( (2\cos^2\varphi -1)a^2 + b^2 -m^2)^2 \le (m^2 + a^2 - b^2)^2 + 4a^2 b^2 \cos^2 \varphi
\end{align}
zu zeigen. Nach umordnen der Terme ergibt sich jedoch folgende Ungleichung

\begin{align}
a^4 (2\cos^2\varphi - 1)^2 + 2 a^2 (2\cos^2\varphi-1) b^2 - 2 m^2 a^2 (2\cos^2\varphi^2-1) \le \\
a^4 + 2 m^2a^2 -2a^2 b^2  + 4 a^2 b^2 \cos^2\varphi\\
\iff 
-2m^2a^2\cos^2\varphi \le a^4(1-(2\cos^2-1)^2) = a^4 (1-\cos^2(2\varphi))=a^4\sin^2(2\varphi),
\end{align}

welche immer erfüllt ist. Die Ursprüngliche Ungleichung folgt also wenn \( (2\cos^2\varphi - 1)a^2 + b^2 - m^2 \le 0\)
gilt direkt und andernfalls folgt sie aus einer wahren Ungleichung. \qed

\vspace{2cm}
Dieses Resultat gilt allerdings nur zwischen zwei Polen in speziellen Koordinaten, wie das folgende 
Beispiel zeigt: Wir wählen wie zuvor \(k(r),k(l) \in \mathbb{R}^4+i {\text{Causal}^+}^0\) nur diesmal lassen wir keinen
der Imaginärteile in Zeitrichtung schauen, stattdessen aber deren Differenz. Wir wählen 
\begin{equation}
k(r)-k(l)=i e_0 ~m + q e_0, ~~~ q \in \mathbb{R}
\end{equation}
und
\begin{equation}
k(l)=r~m e_1 + i 0.99 ~m~ e_1 + i m e_0,
\end{equation}
wobei \(r\) gegeben ist durch
\begin{equation}
r=2*0.99*\sqrt{\frac{1+4(1-0.99^2)}{(8*0.99^2-1)^2-1}}.
\end{equation}

Für diese spezielle Wahl gilt \(\Im (E^-(r) + E^+(l))=0\) für \(\vec{p}=0\). Der Beweis dieser Aussage erfolgt durch direkte Rechnung unter Verwendung von \eqref{imroot}. Für die Wahl der Vorzeichen und Vektoren sind die beiden Terme die aus der Wurzel kommen identisch. 
einzige Skala, m, lässt sich kürzen und die resultierende Gleichung kann durch einsetzen überprüft werden.

Man kann auch ein bisschen allgemeiner \(\vec{p}\) in den reellen Teil von \(\vec{k}(l)\) absorbieren dann gilt \(\Im (E^+(r) - E^-(l))=0\) für solche \(\vec{k}(l),\vec{p}\), die auch \(\vec{k}(l)-\vec{p}= r m e_1 + i 0.99 m e_1\) erfüllen.  Nachdem hier von \(q\) nirgends Gebrauch gemacht wird, kann durch spezielle Wahl von \(q\) insgesamt \(E^+(r)=E^-(l)\) gesetzt werden. 




\end{document}

