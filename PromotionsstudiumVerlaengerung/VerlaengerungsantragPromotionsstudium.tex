\documentclass[11pt,a4paper]{article}

\usepackage[a4paper,text={150mm,240mm},centering,headsep=10mm,footskip=15mm]{geometry}

\usepackage[ngerman]{babel}
\usepackage[utf8]{inputenc}
%\usepackage{graphicx}
\usepackage[T1]{fontenc}
%\usepackage{epsfig, float}

%\usepackage{mathcomp}
%\usepackage{dsfont}
%\usepackage{lmodern}
%\usepackage{scrextend}
%\usepackage{sectsty} % Allows customizing section commands
%\usepackage{color}
%\allsectionsfont{\centering \normalfont\scshape} % Make all sections centered, the default font and small caps
%Lustige Packages von Simone damit es schoener aussieht

\usepackage[nottoc]{tocbibind}

\usepackage{amsmath,amssymb}
\usepackage{amsfonts}
\usepackage{url}
\usepackage{bbm}
\usepackage{mathrsfs} 

% Dirk:
\usepackage[normalem]{ulem}
\usepackage{todonotes}

%\usepackage{hyperref}

\newcommand{\R}{\mathbb{R}}
\newcommand{\C}{\mathbb{C}}
\newcommand{\N}{\mathbb{N}}
\newcommand{\Z}{\mathbb{Z}}
\newcommand{\id}{\mathbbm{1}}
\newcommand{\supp}{{\rm supp} \, }
\newcommand{\diag}{{\rm diag} }

% theorem environments
\newtheorem{theorem}{Theorem}[section]
\newtheorem{lemma}[theorem]{Lemma}
\newtheorem{frage}[theorem]{Frage}
\newtheorem{vermutung}[theorem]{Vermutung}

%Nummeriert Formeln kapitelweise
%\def\theequation{\thesection.\arabic{equation}}
%\numberwithin{equation}{section} 

\newenvironment{definitionmo}[1][Definition des Modells:]{\begin{trivlist}
\item[\hskip \labelsep {\bfseries #1}]}{\end{trivlist}}
\newenvironment{definition}[1][Definition:]{\begin{trivlist}
\item[\hskip \labelsep {\bfseries #1}]}{\end{trivlist}}
\newenvironment{remark}[1][Bemerkung:]{\begin{trivlist}
\item[\hskip \labelsep {\bfseries #1}]}{\end{trivlist}}

\newcommand{\qed}{\hfill\ensuremath{\square}}


\setlength{\parindent}{0pt} 
%Entfernt die haesslichen Einrueckungen im ganzen Dokument, man spart sich das \noindent

%\bibliographystyle{unsrt} % nach Erscheinen im Text sortiert
\bibliographystyle{abbrv} % alphabetisch sortiert


\begin{document}
\title{Verlängerung der Immatrikulation von Markus Nöth im Promotionsstudium }

\author{An die Studentenkanzlei, Sachgebiet 5}

\date{\today}
\maketitle


Sehr geehrte Damen und Herren,

hiermit bestätige ich, dass ich die Promotion von Herrn Markus Nöth seit April 2016 betreue.

Herr Nöth forscht in einem Bereich der mathematischen Physik, in dem noch bedeutende Grundlagenarbeit zu leisten ist. Seine wissenschaftliche Arbeit im Themenbereich der Quantenelektrodynamik erfordert schwierige funktionalanalytische Methoden, um die Restfreiheit in der Phase des Zeitentwicklungsoperators eindeutig festlegen zu können. Wir erwarten die Publikation von einem wissenschaftlichen Artikeln in den kommenden Monaten, was zu einer Verlängerung der Promotionszeit über drei Jahre hinaus führen wird.

Das Cusanuswerk fördert das Projekt von Herrn Nöth mit einem Promotionsstipendium, das voraussichtlich bis zum Ende des Sommersemesters 2020 fortlaufen wird.

Aus den zuvor genannten Gründen unterstütze ich die Verlängerung der Studienzeit von Herrn Markus Nöth um ein weiteres Semester bis zum Sommersemester 2020.

\vspace*{2cm}
Dr. Dirk-Andr\'e Deckert, Mathematisches Institut

\maketitle




\end{document}
