\documentclass[a4paper,11pt]{article}

%\usepackage{german}

\usepackage[dvipsnames]{xcolor}
\usepackage{graphicx}

\usepackage{amssymb}

\usepackage{amsfonts}

\usepackage{amsmath}

\usepackage{amsthm}

\usepackage[unicode=true, pdfusetitle, bookmarks=true,
  bookmarksnumbered=false, bookmarksopen=false, breaklinks=true, 
  pdfborder={0 0 0}, backref=false, colorlinks=true, linkcolor=blue,
  citecolor=blue, urlcolor=blue]{hyperref}
\usepackage{slashed}
\usepackage{authblk}
%identity sign
\usepackage{dsfont}
\usepackage{todonotes}

\setlength{\marginparwidth}{2.6cm}

%commutative diagrams
\usepackage{amsmath,amscd}
\usepackage{enumitem}

\newtheorem{de}{Definition}
\newtheorem{thm}{Theorem}
\newtheorem{rmk}{Remark}
\newtheorem{lem}{Lemma}

\newcommand{\supp}{\operatorname{supp}}
\DeclareMathOperator{\term}{Term}

\addtolength{\textwidth}{2.2cm} \addtolength{\hoffset}{-1.0cm}

\addtolength{\textheight}{3.0cm} \addtolength{\voffset}{-2cm} 

\parindent 0cm

\pagestyle{empty}

\begin{document}

Wir haben eine Funktion \(c: (C_c^\infty(\mathbb{R}^4,\mathbb{R}^4))^3\rightarrow \mathbb{C}, (A,F,G)\mapsto c_A(F,G)\) mit folgenden Eigenschaften:

\begin{align}
&\forall A,F,G \in C_c^\infty(\mathbb{R}^4,\mathbb{R}^4): c_A(F,G)=-c_A(G,F)\\
&\forall A,F,G, H \in C_c^\infty(\mathbb{R}^4,\mathbb{R}^4), \forall \lambda \in \mathbb{C}: c_A(F,G+\lambda H)= c_A(F,G)+ \lambda c_A(F,H).
\end{align} 

Man kann \(c\) also als 2-Form über \( C_c^\infty(\mathbb{R}^4,\mathbb{R}^4)\) verstehen. Zusätzlich erfüllt \(c\) die Eigenschaft, dass für alle 
\( A,F,G, H \in C_c^\infty(\mathbb{R}^4,\mathbb{R}^4)\): 
\begin{align}
\partial_\varepsilon c_{A+\varepsilon F}(G,H)|_{\varepsilon =0}+\partial_\varepsilon c_{A+\varepsilon G}(H,F)|_{\varepsilon =0} 
+ \partial_{\varepsilon} c_{A+\varepsilon H}(F,G)|_{\varepsilon =0}=: (d c)_A (F,G,H)=0
\end{align}
gilt.
Man kann also sagen, dass \(c\) geschlossen ist. Weil \(C_c^\infty(\mathbb{R}^4,\mathbb{R}^4)\) sternförmig ist, stellt sich die Frage ob \(c\) auch exakt ist, 
so wie das für alle 2-Formen auf endlich dimensionalen sternförmigen Gebieten der Fall ist. 



\newpage

\begin{align}
\int_{(\mathbb{R}^+\times \mathbb{R}^3)^2} (\square_x + m^2)(\square_y + m^2) \varphi(x,y) (G_1\otimes G_2) * (\delta((\cdot_1-\cdot_2)^2) \psi) (x,y) d^4 x d^4 y\\
=\int_{(\mathbb{R}^+\times \mathbb{R}^3)^2} \varphi(x,y) \delta ( (\cdot_1 - \cdot_2)^2) \psi(x,y) d^4 x d^4y\\
:=\int\varphi(x,y) \delta ( (\cdot_1 - \cdot_2)^2) \psi(x,y) 1_{x,y\in \mathbb{R}^+\times \mathbb{R}^3} d^4 x d^4y
\end{align}

\end{document}