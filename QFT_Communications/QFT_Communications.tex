\documentclass[a4paper,11pt]{article}

%\usepackage{german}

\usepackage[dvipsnames]{xcolor}
\usepackage{graphicx}

\usepackage{amssymb}

\usepackage{amsfonts}

\usepackage{amsmath}

\usepackage{amsthm}

\usepackage[unicode=true, pdfusetitle, bookmarks=true,
  bookmarksnumbered=false, bookmarksopen=false, breaklinks=true, 
  pdfborder={0 0 0}, backref=false, colorlinks=true, linkcolor=blue,
  citecolor=blue, urlcolor=blue]{hyperref}
\usepackage{slashed}
\usepackage{authblk}
%identity sign
\usepackage{dsfont}
\usepackage{todonotes}

\setlength{\marginparwidth}{2.6cm}

%commutative diagrams
\usepackage{amsmath,amscd}
\usepackage{enumitem}

\newtheorem{de}{Definition}
\newtheorem{thm}{Theorem}
\newtheorem{rmk}{Remark}
\newtheorem{lem}{Lemma}

\newcommand{\supp}{\operatorname{supp}}
\DeclareMathOperator{\term}{Term}

\addtolength{\textwidth}{2.2cm} \addtolength{\hoffset}{-1.0cm}

\addtolength{\textheight}{3.0cm} \addtolength{\voffset}{-2cm} 

\parindent 0cm

\pagestyle{empty}

\begin{document}
\title{The Relationship Between Hadamard States, the Fermionic Projector, and Admissible Polarisation Classes}

\author{
D.-A. Deckert\thanks{deckert@math.lmu.de \\ \tiny{Mathematisches Institut der Ludwig-Maximilians-Universit\"at M\"unchen,}
    \tiny{Theresienstr. 39, 80333 M\"unchen, Germany}},
Felix Finster\thanks{felix.finster@mathematik.uni-regensburg.de \\     \tiny{Universität Regensburg, Universitätsstraße 31, 93053 Regensburg, Germany}}
	 ~and
Markus Nöth\thanks{noeth@math.lmu.de \\\tiny{Mathematisches Institut der Ludwig-Maximilians-Universit\"at M\"unchen,}
    \tiny{Theresienstr. 39, 80333 M\"unchen, Germany}}

}
\date{\today}



\maketitle

\begin{abstract}
This paper compares central objects of three different types of approaches to quantum field theory (QFT): The study of Hadamard states, the
Fermionic Projector, and admissible polarisation classes.
These approaches have different scopes and pursue different motivations, which makes a direct comparison difficult.
Nevertheless, the basic protagonists in all three approaches share much more similarity than is apparent at first glance.
It is the purpose of this work to highlight these common features. 
Since they all study different aspects of QFT; however, one can make out objects in each of these theories which closely resemble
one another. 
\end{abstract}

\section{Introduction}


The first class of objects of interest are Hadamard states which appear in the algebraic approach to 
QFT \cite{fulling1978singularity}. 

The third class of objects are the continuum limit of the Fermionic Projectors \cite{finster2006principle}. 

The second class of objects are the (\(I_2\)-almost) projectors \(P^\lambda_{\Sigma}\) which are 
closely linked to polarisation classes of the vacuum of external field quantum electrodynamics (QED) \cite{ivp0, ivp1, ivp2}. 


In section \ref{sec:hadamard} we give the definition of a Hadamard state, briefly motivate it's usage 
and give its explicit form in the case of flat spacetime subject to an external field as computed in the physics literature \cite{schlemmer2015current}.

In section \ref{sec:ferm proj} ...

In section \ref{sec:pol classes} we briefly describe why in the approach of admissible polarisation classes one only keeps track of the time
evolution of the projector up to an error that is a Hilbert-Schmidt operator. Furthermore we will find a class of candidate \(I_2\)-almost projectors
that have a simple time evolution.

In section \ref{sec:comparison hadamard fermion}...\textcolor{green}{comparison fermionic projector and hadamard state}

In section \ref{sec:comparison hadamard pol} we find that each Hadamard state corresponds to an \(I_2\)-almost projector in a natural way.\todo{is that specific enough?}

Throughout the paper
\(\Sigma, \Sigma', \Sigma''\) denote arbitrary Cauchy surfaces, while for the sake of simplicity we choose \(A\in C_c^\infty(\mathbb{R}^4,\mathbb{R}^4)\) 
to be a four-potential and \(\Sigma_{\text{in}}\) denotes a Cauchy surface earlier than the support of \((A)\).

%some introductory math needs to be written also we should mention that we are concerned with the Dirac field
In this paper we will focus on a system of Dirac fields subject to an external electromagnetic four-potential \(A\) in flat Minkowski spacetime.
We choose the sign convention \(\eta=\text{diag}(+1,-1,-1,-1)\). We denote the minimally coupled differential by \(\nabla_{\alpha} = \partial_{\alpha} + i A_{\alpha}\)
and make use of the Feynman slash notation \(\slashed{\nabla}=:\nabla_{\alpha}\gamma^\alpha\) with \(\gamma^\alpha\) fulfilling 
\(\{\gamma^\alpha , \gamma^\beta\}:=\gamma^\alpha  \gamma^\beta+\gamma^\beta  \gamma^\alpha  = 2 \eta^{\alpha,\beta}\)  each field solving 
Dirac's equation
\begin{equation}\label{dirac}
(i\slashed{\nabla}-m)\psi =: D\psi= 0
\end{equation}
and collectively constituting the free vacuum prior to the support of said external field.\todo{macht das überhaupt Sinn so wenn ich von Dirac Felder spreche? Bei operatoren ist ja der Zustand nicht der Zustand des Feldes, sondern der Zustand ist extra.}
For a 4-spinor \(\psi\in\mathbb{C}^4\) (viewed as a column vector), \(\overline{\psi}\) stands for the row vector \(\psi^*\gamma^0\), where \({}^*\) denotes hermitian conjugation.



\section{Hadamard States}\label{sec:hadamard}
In the algebraic approach to QFT one puts less emphasise on the Hilbert space than is commonly done in non relativistic physics because 
it is not a relativistically invariant object.
Instead one focuses on the algebra of operators that are chosen to do the bookkeeping of statistical outcomes of measurements. 

To infer predictions, some part of the necessary computation can be conducted on the level of this algebra. However,
eventually, expectation values are to be computed in a certain representation, usually found by the GNS construction
with respect to a certain state.
This choice has to be made on physical grounds. 
Hadamard states are often thought to be physically sensible states because they have positive energy in a certain sense.

In order to introduce Hadamard states we have to first define the notion of wavefront set, which itself needs some preliminaries. 
For the introduction of these concepts we follow Hömander \cite[Chapter 8]{hormander2003analysis}.

We begin by introducing the singular support of a distribution

\begin{de}
Let for \(n,m\in\mathbb{N}\),  \(v\in (C^\infty(\mathbb{R}^n,\mathbb{C}^m))'\) the singular support of \(v\) is defined to be the subset of points \(x\in \mathbb{R}^n\) such
that there is no neighbourhood \(U\) of \(x\) such that there is a smooth function \(\phi_{x,v}\in C^\infty (\mathbb{R}^n,\mathbb{C}^m)\) such that \(v\) acts on 
test functions \(\varphi \in C^\infty_c (U,\mathbb{C}^m)\) as
\begin{align}
v(\varphi)=\int \phi^\dagger_{x,v} (x) \varphi(x) dx.
\end{align}
\end{de}

The singular support contains all the points of a distribution such that the distribution does not act like a smooth function at that point. The wavefront set which we are
about to introduce gives an additional directional information of where these singularities propagate. We incorporate this information by the Fourier transform 
in the following definition.

\begin{de}
Let for \(n,m\in\mathbb{N}\),  \(v\in (C^\infty(\mathbb{R}^n,\mathbb{C}^m))'\), we denote by \(\Xi(v)\subset \mathbb{R}^n\backslash\{0\}\) the set of all \(\eta\) such that there is no
cone \(V\subset \mathbb{R}^n\), neighbourhood of \(\eta\), such that for all \(a\in\mathbb{N}\) there are \(C_a>0\) such that for all \(\xi\in V\) we have
\begin{equation}
|\hat{v}(\xi)|\le \frac{C_a}{1+|\xi|^a}.
\end{equation}

Furthermore for each \(x\in \mathbb{R}^n\) we define 
\begin{equation}
\Xi_x (v) := \bigcap_{\overset{\phi \in C^\infty_c (\mathbb{R}^n)}{x\in \mathrm{supp}(\phi)}} \Xi(v \phi),
\end{equation}
Where \(v\phi\) is the pointwise multiplication of a distribution and a scalar test function, which acts as \(v\phi: C^\infty(\mathbb{R}^n,\mathbb{C}^m)\ni \psi\mapsto v(\psi \phi)\). 
\end{de}

We have collected the tools to introduce the wavefront set and the notion of Hadamard states.

\begin{de}
Let for \(n,m\in\mathbb{N}\), \(v\in (\mathcal{S}(\mathbb{R}^n,\mathbb{C}^m))'\) be a tempered distribution. The wavefront set \(\mathrm{WF}(v)\) of 
the distribution \(v\) is defined as \todo{soll ich das schon alles direkt für Distributionen die zwei Funktionen schlucken hinschreiben?}
\begin{equation}
\mathrm{WF}(v):= \left\{(x,\xi)\in \mathbb{R}^n\times \mathbb{R}^n\mid \xi \in \Xi_x(v) \right\}.
\end{equation}
\end{de}

\begin{de}\label{def: hadamard}
A map \(H: C_c^\infty(\mathbb{R}^4,\mathbb{C}^4)\times C_c^\infty(\mathbb{R}^4,\mathbb{C}^4) \rightarrow \mathbb{C}\), is called Hadamard state if it fulfils
for all \( f,g\in C_c^\infty(\mathbb{R}^4,\mathbb{C}^4)\):

\begin{align}
H(Df,g)=0\\
H(f,g)+H(g,f)=i S(f,g)\\
\overline{H(f,g)}=H(\overline{f},\overline{g})\\
\text{WF}(H)\subset C_+,
\end{align}
where \(S(f,g)\) is the propagator of the Dirac equation, 
 and  \(C_+:=\{(x,y;k_1,-k_2)\in\mathbb{R}^{16}\mid (x;k_1)\approx (y;k_2) , k_1^2\ge0, k_1^0>0\}\)
and \((x;k_1)\approx(y;k_2)\) holds whenever \((x-y)^2=0\) and \((y-x)\parallel k_1 = k_2\).
\end{de}
It is in the sense of the fourth condition that Hadamard states are of positive energy.

In the scenario of Minkowski spacetime in an external field Dirac \cite{Dirac34} already studied the Hadamard states, although that name was not established at the time.
More recently the subject has attracted considerable attention \cite{zahn2014renormalized, schlemmer2015current}, which computed the Hadamard states. 
They are given in terms of the Klein-Gordon operator corresponding to Dirac's equation:


\begin{de}
The Klein-Gordon operator corresponding to the Dirac equation \eqref{dirac} reads
\begin{align}
&P: C^\infty(\mathbb{R}^4,\mathbb{C}^4)\rightarrow C^\infty(\mathbb{R}^4,\mathbb{C}^4)\\
&P=(i\slashed{\nabla}-m)(-i\slashed{\nabla}-m)=\nabla_\alpha \nabla^\alpha + \frac{i}{2} \gamma^\alpha \gamma^\beta F_{\alpha,\beta} +m^2,
\end{align}
where \(F_{\alpha,\beta}=\partial_\alpha A_\beta - \partial_\beta A_\alpha\) is the field strength tensor of the electromagnetic field. Furthermore we 
define for \(f\in \mathcal{C}_c^\infty(\mathbb{R}^4\times \mathbb{R}^4,\mathbb{C}^{16})\) the differential operator
\begin{equation}
\slashed{\nabla}^* f(x,x')=\left(\frac{\partial}{\partial y^\alpha} - i A_\alpha(y)\right)f(x,y)\gamma^\alpha.
\end{equation}
\end{de}


For the special case of a Dirac field in Minkowski space-time Zahn \cite{schlemmer2015current} gave a more explicit form of
the Hadamard states \(H \in (C_c^\infty(\mathcal{M})\times C_c^\infty(\mathcal{M}))'\)  on which we base our analysis below.
According to this, \(H\) 
acts for \(f_1,f_2\in C_c^\infty(\mathcal{M})\otimes \mathbb{C}^4\) as 
\begin{equation}\label{eq:hadamard1}
H(f_1,f_2)=\lim_{\varepsilon\searrow 0} \int_{\mathbb{R}^4}d^4 x \overline{f_1}(x) \int_{\mathbb{R}^4} d^4y ~h_\varepsilon(x,y) f_2(y),
\end{equation}
where \(h_\varepsilon\) is of the form

\begin{align}\nonumber
h_\varepsilon (x,y)&=\\\label{eq:hadamardexp1}
&\frac{-1}{2(2\pi)^2}(-i \slashed{\nabla} + i \slashed{\nabla}^\ast -2m)\Big[\frac{e^{-i (x-y)^\alpha\int_0^1 ds A_\alpha (x s + (1-s)y)}}{ (y-x-i \varepsilon e_0)^2}\Big]\\\label{eq:hadamardexp2}
+&\frac{-1}{2(2\pi)^2}(-i \slashed{\nabla} + i \slashed{\nabla}^\ast -2m)\Big[ V(x,y) \ln (-(y-x-i \varepsilon e_0)^2)\Big]\\\label{eq:hadamardexp3}
&+B(x,y),
\end{align}
%
%für das Vorzeichen, beachte, dass H^- aus dem Zahn paper genommen wird, was das gleiche ist wie x und y zu tauschen
%
where \(V,B:\mathbb{R}^4\rightarrow \mathbb{C}^4\) are smooth functions, \(B\) is completely arbitrary, 
whereas \(V\) is fixed by the external potential. The expansion
\begin{equation}
V^N(x,y):=\sum_{k=1}^N \frac{1}{ 4^{k} k!(k-1)!} V_k(x,y) (x-y)^{2(k-1)},
\end{equation}
is an asymptotic expansion for \(V\) for \(k\rightarrow \infty\), in the sense that
\((V-V^N)(x,y)\ln(-(x-y)^2)\) as a function of \(x\) and \(y\) is in \(C^{N-2}(\mathbb{R}^{4+4})\)
and \(V-V^N=\mathcal{O}\left(\left((x-y)^{2}\right)^{N-2}\right)\).
The functions \(V_k\) fulfil a recursive set of partial differential differential equations
\begin{align}\label{Hadamard recursive equ.}
(x-y)^\alpha (\partial_{x,\alpha}+i A_\alpha(x)) V_{n}(x,y) + n V_{n}(x,y)=-n P V_{n-1}(x,y),
\end{align}
where \(V_{0}(x,y)=e^{-i (x-y)^\alpha\int_0^1 ds A_\alpha (x s + (1-s)y)}\). 

For the rest of this paper, we assume that for any \(A\in C_c^\infty(\mathbb{R}^4)\) there are \(H\), \((h_\varepsilon)_{\varepsilon>0}\) and \(V\)
fulfilling all of the conditions described in this paragraph.




%note to self: should I mention that fields are operator valued distributions? 

%write that one needs to restrict to Hadamard states for the perturbative expansion of interacting QFT (cite this wald overview paper, p.15)

%give some reply to the standard argument that notions of particles are non unique in QFT in curved spacetimes. 

\section{The Fermionic Projector}\label{sec:ferm proj}

\section{Projectors for Polarisation Classes}\label{sec:pol classes}

The concept of polarisation classes arises naturally in the study of QED in external electromagnetic fields. It does need some
machinery to be introduced and related to more familiar objects which we are going to introduce first. In doing so we follow \cite{ivp2}

\begin{de}
We define a Cauchy surface \(\Sigma\) in \(\mathbb{R}^4\) to be a smooth, 3-dimensional submanifold of \(\mathbb{R}^4\) that fulfills
the following three conditions:
\begin{enumerate}[label=\alph*)]
\item Every inextensible, two-sided, time- or light-like, continuous path in \(\mathbb{R}^4\) intersects \(\Sigma\) in a unique point.
\item For every \(x\in\Sigma\), the tangential space \(T_x\Sigma\) is space-like.
\item The tangential spaces to \(\Sigma\) are bounded away from light-like directions in the following sense: The only light-like accumulation point of 
\(\bigcup_{x\in \Sigma} T_x \Sigma\) is zero.
\end{enumerate}
\end{de}

In coordinates, every Cauchy surface \(\Sigma\) can be parametrised as
\begin{equation}
\Sigma= \{ \pi_\Sigma(\vec{x}):=(t_\Sigma(\vec{x}),\vec{x})\mid \vec{x}\in\mathbb{R}^3\}
\end{equation}
with a smooth function \(t_\Sigma: \mathbb{R}^3\rightarrow \mathbb{R}\). For convenience and without restricting generality of our results we keep a global constant 
\begin{equation}
0<V_{\max}<1
\end{equation}
fixed and work only with Cauchy surfaces \(\Sigma\) such that 
\begin{equation}\label{gamma max}
\sup_{\vec{x}\in\mathbb{R}^3} |\mathrm{grad}~ t_\Sigma( \vec{x})|<V_{\max}.
\end{equation}

The standard volume form over \(\mathbb{R}^4\) is denoted by \(d^4x=dx^0dx^1dx^2dx^3\); the product of forms is understood as wedge product. The symbol \(d^3x\) 
mean the 3-form \(d^3x=dx^1dx^2dx^3\) on \(\mathbb{R}^4\) and on \(\mathbb{R}^3\) respectively. Contraction of a form \(\omega\) with a vector \(v\) is
denoted by \(i_v(\omega)\). The notation \(i_v(\omega)\) is also used for the spinor matrix valued 
vector \(\gamma= (\gamma^0,\gamma^1,\gamma^2,\gamma^3)=\gamma^\mu e_\mu\):

\begin{equation}
i_\gamma(d^4x)=\gamma^\mu i_{e_\mu}(d^4x).
\end{equation}

For \(x\in \Sigma\) the restriction of the spinor matrix valued 3-form \(i_\gamma(d^4x)\) to the tangential space \(T_x\Sigma\) is given by
\begin{equation}
i_\gamma(d^4x)=\slashed{n}(x)i_n(d^4x)= \left(\gamma^0-\sum_{\mu=1}^3 \gamma^\mu \frac{\partial t_\Sigma (\vec{x})}{\partial x^\mu}\right) d^3x=:\Gamma(\vec{x})d^3x
\end{equation}

As a consequence of \eqref{gamma max}, there is a positive constant \(\Gamma_{\max}=\Gamma_{\max}(V_{\max})\) such that
\begin{equation}
\|\Gamma(\vec{x})\le \Gamma_{\max}, \quad \forall \vec{x}\in\mathbb{R}^3.
\end{equation}



As is well known equation \eqref{dirac} gives us a 
one-particle time evolution operator for each pair of Cauchy surfaces \(\Sigma, \Sigma'\)
\begin{equation}
U_{\Sigma',\Sigma}:\mathcal{H}_\Sigma \rightarrow \mathcal{H}_{\Sigma'},
\end{equation}
where \(\mathcal{H}_\Sigma\) is the space of square integrable functions on the Cauchy surface \(\Sigma\). The scalar product on \(\mathcal{H}_\Sigma\)
is given by
\begin{equation}
\phi,\psi \mapsto \int_{\Sigma} \overline{\phi}(x) i_{\gamma}(d^4\gamma) \psi(x).
\end{equation}


We start out by characterising the polarisation classes for free motion.
\begin{de}
The projector \(P^{\mathcal{H}^-}_{\Sigma} \) has the well known representation as the weak limit of the integral operator 
with the kernel\cite{ivp2}

\begin{equation}
p^-_\varepsilon(x,y)= -\frac{m^2}{4\pi^2}(i\slashed{\partial}_x+m) \frac{K_1(m \sqrt{-(y-x-i \varepsilon e_0)^2})}{m \sqrt{-(y-x-i \varepsilon e_0)^2}},
\end{equation}
where the square is a Minkowski square and the square root denotes its principle value. By weak limit we mean

\begin{align}
\langle \phi, P^{\mathcal{H}^-}_{\Sigma} \psi\rangle = \lim_{\varepsilon \searrow 0} \int_{\Sigma\times \Sigma} \overline{\phi}(x) i_\gamma(d^4x) p^-_\varepsilon(x,y ) i_\gamma(d^4y) \psi(y),
\end{align}

for general \(\phi, \psi \in \mathcal{H}_\Sigma\).
\end{de}

\begin{rmk}
By inserting the expansion of \(K_1\) in terms of a Laurent series and a logarithm, \cite{abramowitz1965handbook} one obtains:

\begin{align}\label{K1series}
K_1(\xi) = \frac{1}{\xi}- \frac{\xi}{4} \sum_{k=0}^\infty \left(2 \psi(k+1)+\frac{1}{k+1}+2\ln 2 -2 \ln \xi \right) \frac{\left(\frac{\xi^2}{4}\right)^k}{k!^2 (k+1)}\\\label{def:Q}
:=\frac{1}{\xi} + \xi Q_1(\xi^2) \ln \xi +\xi Q_2(\xi^2)=:\frac{1}{\xi} + \xi Q_3(\xi).
\end{align}

It is not obvious from the equation \eqref{Hadamard recursive equ.} but well known that the
 vacuum of Minkowski spacetime does indeed
correspond to  a Hadamard state subject to vanishing four potential. In fact, the Hadamard states were constructed to 
agree with the Minkowski vacuum up to smooth terms.
\end{rmk}




\begin{de}
Let \(\text{Pol} (\mathcal{H}_\Sigma)\) denote the set of all closed, linear subspaces \(V\subset \mathcal{H}\)
such that both \(V\) and \(V^\perp\) are infinite dimensional. Any \(V\in \text{Pol}(\mathcal{H}_\Sigma)\) is called 
\emph{polarisation} of \(\mathcal{H}_\Sigma\). For \(V\in \text{Pol}(\mathcal{H}_\Sigma)\), let \(P_\Sigma^V:\mathcal{H}_{\Sigma}\rightarrow V\) 
denote the orthogonal projection of \(\mathcal{H}_\Sigma\) onto \(V\).
The Fock space corresponding to \(V\) on the Cauchy surface \(\Sigma\) is defined to be
\begin{equation}
\mathcal{F}(V,\mathcal{H}_\Sigma) := \bigoplus_{c\in\mathbb{Z}} \mathcal{F}_c (V,\mathcal{H}_\Sigma), \quad 
\mathcal{F}_c(V,\mathcal{H}_\Sigma):= \bigoplus_{\overset{n,m\in\mathbb{N}_0}{c=m-n}}(V^\perp)^{\wedge n} \otimes \overline{V}^{\wedge m},
\end{equation}

where \(\bigoplus\) is the Hilbert space direct sum, \(\wedge\) the antisymmetric tensor product of Hilbert spaces and 
\(\overline{V}\) is the conjugate complex vector space of \(V\), which coincides with \(V\) as a set, has the same vector 
space operations as \(V\) except for scalar multiplication, which is defined by \((z,\psi)\mapsto z^* \psi\) for 
\(z\in\mathbb{C},\psi\in V\).
\end{de}

\begin{rmk}\label{WhyUseApprox}
Given two polarisations \(V,W\in \text{Pol}(\mathcal{H}_\Sigma)\), for two Fockspaces \(\mathcal{F}(V,\mathcal{H}_\Sigma)\) and 
\(\mathcal{F}(W,\mathcal{H}_\Sigma)\) there is a unitary operator \(U: \mathcal{F}(V,\mathcal{H}_\Sigma) \rightarrow \mathcal{F}(W,\mathcal{H}_\Sigma\)
if and only if \(P^V_\Sigma - P^W_\Sigma \in I_2(\mathcal{H}_\Sigma)\) by the theorem of Shale and Stinespring 
\cite{Shale Stinespring 1965}.
\end{rmk}
Remark \ref{WhyUseApprox} gives us a natural limit with 
respect to which it is useful
to analyse the regularity of Projectors \(P_\Sigma^A\). We will therefore be content with the following equivalence class.
\begin{de}
For \(V,W\in \text{Pol}(\mathcal{H}_\Sigma)\) we write
\begin{equation}
V\approx W \iff P^V_\Sigma - P^W_\Sigma \in I_2(\mathcal{H}_\Sigma),
\end{equation}
\begin{equation}
C_\Sigma (A):= [ U_{\Sigma \Sigma_{\text{in}}}^A \mathcal{H}_{\Sigma_{\text{in}}}^-]_\approx.
\end{equation}
\end{de}


The equivalence class \(C_\Sigma(A)\) transforms naturally with respect to gauge and Lorentz transforms\cite{ivp2}. 
Now for hyperplanes \(\Sigma\cap \text{supp}(A)\neq \emptyset\) the operator \(P^{\mathcal{H}^-}_{\Sigma} \) does not represent
\(C_\Sigma(A)\). Because all we are interested in is representations of equivalence classes, we are content with finding
objects that differ from a Projector onto \(U_{\Sigma,\Sigma_{\text{in}}}^A \mathcal{H}_{\Sigma_in}^-\) by a Hilbert-Schmidt operator.
Therefore we need not keep track of the exact evolution of the projection operators, but define a whole class of admissible ones.

\begin{de}\label{def:lambda}
The set \(\mathcal{G}^A\) denotes the set of all functions \(\lambda^A \in C^\infty (\mathbb{R}^4\times \mathbb{R}^4, \mathbb{R})\) that satisfy
\begin{enumerate}[label= \roman*) ]
\item There is a compact set \(K\subset \mathbb{R}^4\) such that \(\supp \lambda \subseteq K\times \mathbb{R}^4 \cup \mathbb{R}^4 \times K\).
\item \(\lambda\) satisfies \(\forall x \in \mathbb{R}^4: \lambda(x,x)=0\).
\item On the diagonal the first derivatives fulfil 
\begin{equation}
\forall x,y\in\mathbb{R}^4: \partial_x \lambda(x,y)|_{y=x}=-\partial_y \lambda(x,y)|_{y=x} = A(x).
\end{equation}
\end{enumerate}
We futhermore define a corresponding (quasi) projector \(P^\lambda\):
\begin{align}
\langle \phi, P^\lambda_\Sigma \psi\rangle =& \lim_{\varepsilon \searrow 0} \langle \phi, P^{A,\varepsilon}_\Sigma \psi\rangle ,\\\label{def:p lambda}
\langle \phi, P^{\lambda,\varepsilon}_\Sigma \psi\rangle :=&
\int_{\Sigma\times\Sigma}\overline{\phi}(x)i_\gamma(d^4x) 
\overbrace{e^{-i \lambda(x,y)} p^-_\varepsilon(y-x)}^{=:p^\lambda_\varepsilon(x,y)}i_\gamma(d^4y) \psi(y),
\end{align}
for general \(\phi, \psi \in \mathcal{H}_\Sigma\).
\end{de}

\begin{rmk}\label{main results of ivp2}
\(P_{\Sigma}^\lambda\) and \( P_{\Sigma'}^\lambda\) are equivalent if transported appropriately by time evolution operators\cite[theorem 2.8]{ivp2}:

\begin{equation}
P_{\Sigma}^\lambda-U_{\Sigma,\Sigma'}^A P_{\Sigma'}^\lambda U_{\Sigma',\Sigma}\in I_2(\mathcal{H}_\Sigma).
\end{equation}
Also for four-potentials \(A,B\in C_c^\infty(\mathbb{R}^4)\) 
the corresponding projectors are equivalent if and only if the four potentials projected onto the hypersurface agree\cite[theorem 1.5]{ivp2}:

\begin{equation}\label{equiv:pLambda}
P_{\Sigma}^{\lambda^A} - P_{\Sigma}^{\lambda^B} \in I_2(\mathcal{H}_\Sigma) \iff \forall x\in \Sigma~ \forall z\in T_x \Sigma: z^\alpha(A_\alpha(x)-B_\alpha(x))=0.
\end{equation}
\end{rmk}



Taking into account the freedom within each classification the notions Hadamard state and  projectors of polarisation classes are extremely close. 
This is the topic of the next section.

\section{Comparison Between Hadamard States and the Fermionic Projector}\label{sec:comparison hadamard fermion}

\section{Comparison Between Hadamard States and \(P^\lambda_\Sigma\)}\label{sec:comparison hadamard pol}

The following theorem is the basis of our comparison between Hadamard states and \(I_2-\) almost Projectors. We first
discuss its consequences and postpone the proof until the appendix 


\begin{thm}\label{thm:hadamard=>Pol}
Given as four-potential \(A\in C_c^\infty (\mathbb{R}^4)\), and a Hadamard state \(H\) of the form 
\eqref{eq:hadamard1}-\eqref{eq:hadamardexp3}, there is a family of smooth functions 
\((w_\varepsilon)_{\varepsilon>0}\) in \( C^\infty(\mathbb{R}^4\times \mathbb{R}^4,\mathbb{C}^{4\times 4})\), such that for any Cauchy surface 
\(\Sigma\) and \(\tilde{P}\) acting as
\begin{equation}
\mathcal{H}_{\Sigma} \ni \psi \mapsto \tilde{P}\psi= \lim_{\varepsilon \rightarrow 0} \int_{\Sigma} (h_\varepsilon - w_\varepsilon)(\cdot,y)i_{\gamma}(d^4y) \psi(y),
\end{equation}
is a bounded operator on \(\mathcal{H}_\Sigma\) that fulfils \(P^{\lambda^A}-\tilde{P}\in I_2(\mathcal{H}_\Sigma)\) for any \(\lambda^A\in \mathcal{G}^A\) . 
Additionally, the pointwise limit of \((w_\varepsilon)_{\varepsilon}\) for \(\varepsilon \rightarrow 0\) is smooth.
\end{thm}

\begin{thm}\label{thm:Pol=>hadamard}
Given a four-potential \(A\in C_c^\infty (\mathbb{R}^4)\) and a \(\lambda^A\in\mathcal{G}^A\), there is a family of functions
\((w_\varepsilon)_\varepsilon\) in \( C^\infty(\mathbb{R}^4 \times \mathbb{R}^4 ,\mathbb{C}^{4\times 4})\) such that for all Cauchy surfaces \(\Sigma\)
the \(L^2(\Sigma\times\Sigma)+C^\infty(\mathbb{R}^{4+4})\)\todo{is being in \(L^2(A)+C^\infty(B)\) even a sensible statement?} %\(L^2_{\text{loc}}(\Sigma\times \Sigma)\) 
limit for \(\varepsilon \rightarrow0\) exists. 
Furthermore, the \(\tilde{H}\) acting on test functions \(f_1,f_2\in C^\infty(\mathbb{R}^{4}, \mathbb{C}^4)\) as

\begin{equation}
\tilde{H}(f_1,f_2)=\lim_{\varepsilon \rightarrow 0} \int_{\mathbb{R}^{4}\times\mathbb{R}^4} \overline{f_1}(x)
\left(p_\varepsilon^{\lambda^A}(x,y)-w_\varepsilon(x,y)\right)f_2(y) d^4x d^4y
\end{equation}

is a Hadamard state of the form \eqref{eq:hadamard1}-\eqref{eq:hadamardexp3}.
\end{thm}

\begin{rmk}
In the approach of polarisation classes, two projectors represent the same polarisation class if and only if their difference is a Hilbert-Schmidt operator. 
Because of this fact, it is enough to keep track of the time evolution of a projector only up to changes by a Hilbert-Schmidt operator, i.e. it is enough
to find the time evolution of an \(I_2\)-almost projector representing the correct polarisation class. 

Keeping this in mind,
theorem \ref{thm:hadamard=>Pol} states that, up to a \(C^\infty\) correction, the integral kernel of the Hadamard state is the integral kernel of a \(I_2\)-almost projector.
Theorem \ref{thm:Pol=>hadamard} states that, up to \(L^2+C^\infty\) corrections, the integral kernel of a \(I_2\)-almost projector is a Hadamard state.

The \(C^\infty\) freedom is due to the definition of a Hadamard state. The \(L^2\) freedom originates from the definition of \(I_2\)-almost projector.

In this sense, the relevant singularity structure of a Hadamard state to become a \(I_2\)-almost projector is given by terms \eqref{wurm A} - \eqref{wurm E} only
\end{rmk}


\label{wurm A}
%So on the one hand each Hadamard state \(h^A\) corresponds to a \(I_2\)- almost Projector \(h^a-w^A\), 
%with \(w^A\in C^\infty\), representing the same polarisation class as \(P^{\lambda^A}\).
%On the other hand each \(P^{\lambda^A}\) corresponds to a Hadamard state \(p^{\lambda^A}-w^A\), with \(w^A \in I_2\) 
%\todo{typen chaos? in welchem operatorraum ist \(w^A\)?}




%strategy:
%1. Hadamard form of sate <=> Hadamard state fulfils microlocal condition,(cite SchlemmerZahn, as well as the original things such as Radzikowski etc.)
% i.e. the only sets of points (x,y) where it is not smooth are those with (x-y)^2=0
%2. subtract from the Hadamard state a smooth function such that it vanishes except in a ball of radius \epsilon (in relative coordinates). This is possible due to 1.
% 3. compare  reducedHadamard state with P^\lambda for \lambda=-ie \int_x^y A(...) => Most divergent term is reduced to something bounded, remaining 1/(x-y) and log(x-y)
% terms may not cancel each other, however since they are I_2 anyway, we don't care. 
%4. Since P^\lambda and the reduced Hadamard state agree up to square integrable terms the reduced Hadamard state can also be viewed as a map H_\Sigma -> H_\Sigma
%5. By 3. and 4. the theorem follows.



\section{Appendix: Proof of theorem \ref{thm:hadamard=>Pol} and \ref{thm:Pol=>hadamard}}

We first break down the problem of the proof of theorem \ref{thm:hadamard=>Pol} to a problem on a bounded domain. Then we go on 
to show that the relevant functions are locally bounded so that an application of Lebesgue dominated convergence yields our result.
The same estimates we need for theorem \ref{thm:hadamard=>Pol} will be used again for the proof of theorem \ref{thm:Pol=>hadamard}.

\begin{de} Let \(A\in C_c^\infty(\mathbb{R}^4)\) and \(\Sigma\) be a Cauchy surface, we introduce the sets
\begin{align}
J_1&:=\{(x,y)\in\mathbb{R}^{4+4}\mid (x-y)^2\ge -\delta^2/4 \wedge \overline{x~y} \cap \mathrm{supp}(A)\neq \emptyset\}\\
J_2&:=\{(x,y)\in\mathbb{R}^{4+4}\mid (x-y)^2\ge -\delta^2 \wedge \overline{x~y} \cap \left(B_{\delta/2}(0)+\mathrm{supp}(A)\right)\neq \emptyset\}\\
J_3&:=\{(x,y)\in \Sigma \times \Sigma \mid (x-y)^2 \ge-\delta^2\wedge \overline{x ~ y} \cap \left(B_{\delta/2}(0)+\mathrm{supp}(A) \right)\neq \emptyset \},
\end{align}
where the sum of two sets is defined as \(S_1+S_2:=\{s_1+s_2\mid s_1\in S_1, s_2\in S_2\}\) and we denote the line segment between \(x\) and \(y\) by \(\overline{x~y}\). 
\end{de}

\begin{de}
For ease of comparison, we define for a four-potential \(A\in C_c^\infty(\mathbb{R}^4)\)
\(\tilde{\lambda}:\mathbb{R}^{4+4}\rightarrow \mathbb{C}\):
\begin{equation}
\tilde{\lambda}(x,y)= (x-y)^\alpha\int_0^1 ds A_\alpha (x s + (1-s)y),
\end{equation}
and furthermore introduce the abbreviation \(G(x,y):=e^{-i \tilde{\lambda}(x,y)} \).
\end{de}

\begin{rmk}
The function \(\tilde{\lambda}\not\in \mathcal{G}^A\), because it does not satisfy one of the technical conditions in definition
\ref{def:lambda}, there is not compact \(K\subset\mathbb{R}^4\) such that 
\(\mathrm{supp}(\tilde{\lambda})\subset K\times \mathbb{R}^4 \cup \mathbb{R}^4\times K\) holds. However, the other conditions are all fulfilled. 
We need to introduce it nonetheless, because it is present in the representation for the Hadamard states.
\end{rmk}


\begin{lem}\label{lem:pointwise}
For any \(A\in C_c^\infty(\mathbb{R}^4)\) and any Hadamard state \(H\) of the form \eqref{eq:hadamard1} and \eqref{eq:hadamardexp1}-\eqref{eq:hadamardexp3}
and any \(\lambda^A \in \mathcal{G}^A\),  there is a smooth family of
functions \(w_\varepsilon\in C^\infty(\mathbb{R}^4\times \mathbb{R}^4,\mathbb{C}^{4\times4})\) with \(\varepsilon \in [0,1]\) and a smooth function 
 \(w\in C^\infty(\mathbb{R}^4\times \mathbb{R}^4,\mathbb{C}^{4\times 4})\)  obeying
 
\begin{align}
&\int_{\mathbb{R}^4\times\mathbb{R}^4} \overline{f_1}(x) w_\varepsilon(x,y) f_2(y)dxdy =
\int_{\mathbb{R}^4\times\mathbb{R}^4} \overline{f_1}(x) (h_\varepsilon - p_\varepsilon^\lambda)(x,y) f_2(y)dxdy\\
&\int_{\mathbb{R}^4\times\mathbb{R}^4} \overline{f_1}(x) w(x,y)f_2(y)dxdy =
\lim_{\epsilon\rightarrow 0}\int_{\mathbb{R}^4\times\mathbb{R}^4} \overline{f_1}(x)(h_\epsilon (x,y)-p^\lambda_\epsilon(x,y)) f_2(y)dxdy,
\end{align}

for all test functions \(f_1,f_2\) such that \(\mathrm{supp}(f_1)\times\mathrm{supp}(f_2)\subset J_2^c\)
 
Additionally \(\lim_{\varepsilon \rightarrow 0}w_\varepsilon=w\) pointwise and in addition 
is continuous as a function of type \(\mathbb{R}^{4+4} \times[0,1]\rightarrow \mathbb{C}^{4\times 4}\).
\end{lem}
\begin{proof}
Let \(A\in C_c^\infty(\mathbb{R}^4), \varepsilon>0\).
 Let \(H\) be a Hadamard state acting as 
\begin{equation}
H(f_1,f_2)=\lim_{\varepsilon\searrow 0} \int_{\mathbb{R}^4}d^4 x \overline{f_1}(x) \int_{\mathbb{R}^4} d^4y ~h_\varepsilon(x,y) f_2(y).
\end{equation}

be of the form of equation \eqref{eq:hadamardexp1}-\eqref{eq:hadamardexp3}. 
Pick a function \(\lambda\in \mathcal{G}^A\). 

We may choose \(w'\in C^\infty(\mathbb{R}^4\times\mathbb{R}^4,\mathbb{C}^{4\times 4})\)
such that the following conditions is fulfilled:

For all test functions \(f_1,f_2\in C_c^\infty(\mathbb{R}^4,\mathbb{C}^4)\) such that \( \mathrm{supp}(f_1)\times \mathrm{supp}(f_2) \subset J_1^c \)
we have

\begin{equation}\label{eq:conditionfull}
\int_{\mathbb{R}^4\times\mathbb{R}^4} \overline{f_1}(x) w'(x,y)f_2(y)dxdy =
\lim_{\epsilon\rightarrow 0}\int_{\mathbb{R}^4\times\mathbb{R}^4} \overline{f_1}(x)(h_\epsilon (x,y)-p^\lambda_\epsilon(x,y)) f_2(y)dxdy.
\end{equation}

To understand why, we consider two cases regarding the support of the testfunctions \(f_1,f_2\), 
\begin{enumerate}
\item \((x,y)\in\mathrm{supp}(f_1)\times\mathrm{supp}(f_2)\)
implies \((x-y)^2<-\delta^2/4\).
\item \((x,y)\in\mathrm{supp}(f_1)\times\mathrm{supp}(f_2)\)
implies \(\overline{x_y}\cup \mathrm{supp}(A)=\emptyset\).
\end{enumerate}

Regarding 1: In the representation of the Hadamard state \(H\), equation \eqref{Hadamard recursive equ.}, can be explicitly 
solved recursively (See \cite{bar2007wave}[lemma 2.2.2]. The factor \(V_0\) was already given, for \(k\ge 1\) the recursion is given by

\begin{equation}\label{eq:line integral}
V_k(x,y)=-k G(x,y) \int_0^1 ds s^{k-1} G(x+ s(y-x),x) P V_{k-1}(x,x+s(y-x)).
\end{equation}

Recall that \(P\) depends on the four potential in a local manner.
Here we observe  that, if the support of the external field \(A\) does not intersect the straight line connecting \(x\) and \(y\), the function
\(V(x,y)\) in the expression for \(h_\varepsilon(x,y)\) can be calculated to be

\begin{equation}
V(x,y)=\sum_{k=0}^\infty \frac{( (y-x)^2m^2/4)^k}{k! (k+1)!},
\end{equation}

which equals the logarithmic part of \(p^-_\varepsilon\), corresponding to \(Q_1(\xi)\) in \eqref{K1series}. 

This shows that in this case
\(p_\varepsilon^-(x,y)\) agrees with \(h_\varepsilon(x,y)\) up to smooth terms in case 1.
 
Regarding 2: we notice that the only points \((x,y)\) in the singular support  of both \(h_\varepsilon\) and \(p_\varepsilon^{\lambda}\) need to fulfil
\((y-x)^2=0\), if we further demand that both \(x\) and \(y\) belong to the same spacelike hypersurface \(x=y\) follows. 
Now pick some \(\delta>0\).
This implies that for functions \(f_1,f_2\in C_c^\infty(\mathbb{R}^4,\mathbb{C}^4)\) such that \((x,y)\in \mathrm{supp}(f_1)\times\mathrm{supp}(f_2)\) 
implies \((x-y)^2\ge -\delta^2\) these operators act as integral operators with smooth kernel.


In fact, as \(w'\) is smooth, condition \eqref{eq:conditionfull} specifies the values of \(w'\)
uniquely for arguments \((x,y)\in\mathbb{R}^{4+4}\) in the complement of \(J_1\).
Analogously, we define a smooth function \(w_\varepsilon' \in C^\infty(\mathbb{R}^4\times\mathbb{R}^4,\mathbb{C}^{4\times 4})\) for
every \(\varepsilon>0\) fulfilling 

\begin{align}\label{eq:wepsilon_condition}
\int_{\mathbb{R}^4\times\mathbb{R}^4} \overline{f_1}(x) w'_\varepsilon(x,y) f_2(y)dxdy =
\int_{\mathbb{R}^4\times\mathbb{R}^4} \overline{f_1}(x) (h_\varepsilon - p_\varepsilon^\lambda)(x,y) f_2(y)dxdy
\end{align}

for test functions \(f_1,f_2\in C^\infty_c(\mathbb{R}^4,\mathbb{C}^4)\) such that \(\mathrm{supp}(f_1)\times \mathrm{supp}(f_2)\subset J_1\).
Next, we pick a function \(\chi\in C^\infty(\mathbb{R}^{4+4})\) such that 
\begin{align}
\left.\chi \right|_{J_2^c}=1, \quad \left. \chi \right|_{J_1}=0
\end{align}
hold. Now, we define

\begin{align}
w:=\chi w',\quad w_\varepsilon:= \chi w_\varepsilon',
\end{align}

where \(w',w_{\varepsilon}'\) are extended by the zero function inside \(J_1\).

The functions \(w, w_\varepsilon\) are now uniquely fixed in all of \(\mathbb{R}^{4+4}\) and fulfil  \eqref{eq:conditionfull} and \eqref{eq:wepsilon_condition}, respectively
for test functions \(f_1,f_2\) such that \(\mathrm{supp}(f_1)\times \mathrm{supp}(f_2) \subset J_2^c\) holds. Observing the exact form of 
\(h_\varepsilon\), \eqref{eq:hadamardexp1}-\eqref{eq:hadamardexp3} and \(p_\varepsilon^{\lambda}\), \eqref{def:p lambda}, we notice that in fact 
\(w_{\varepsilon}(x,y) \xrightarrow{\varepsilon\rightarrow 0} w(x,y)\) for all \(x,y\in J_1^c\), because of the cutoff function \(\chi\) this also holds for \((x,y)\in J_1\). 
Moreover, we notice from the explicit form of \(h_\varepsilon\) and \(p_\varepsilon^{\lambda}\) that,
\(w_{\cdot_3}(\cdot_1,\cdot2)\) is even continuous as a function of type \(\mathbb{R}^4\times\mathbb{R}^4\times[0,1]\rightarrow \mathbb{C}^{4\cdot 4}\).
\end{proof}


\begin{lem}\label{lem:bound,worstterm}
we define
\begin{align}
&g_1: J_3\times ]0,1[\rightarrow \mathbb{C}^{4\times 4}\\
&(x,y,\varepsilon)\mapsto e^{-i\tilde{\lambda}(x,y)} \frac{m^2}{4\pi^2} (i\slashed{\partial}_x+m)\frac{1}{m^2(y-x-i\varepsilon e_0)^2} - \term_\eqref{eq:hadamardexp1}(x,y).
\end{align}
There is a constant \(M_1\in\mathbb{R}\), such that
\begin{equation}
\forall \varepsilon, x,y |g_1(x,y,\varepsilon)|\le M_1.
\end{equation}
Also define
\begin{align}
&g_1:\mathbb{R}^{4+4}\times [0,1]\ni (x,y)\mapsto \frac{1}{(y-x-i\varepsilon e_0)^2} \left(\slashed{A}(x) + \slashed{A}(y)
-  2 \int_0^1 \mathrm{d}s \slashed{A}(s x + (1-s)y)  \right.\\
&\left.+(y-x)^\alpha \int_0^1 \mathrm{d}s (1-2s) (\slashed{\partial}A_\alpha)(sx+(1-s)y)   \right).
\end{align}
There is a function \(g_3 \in C(\mathbb{R}^{4+4},\mathbb{R}^+)\) such that
\begin{equation}
\forall \varepsilon \in [0,1]: |g_2(x,y,\varepsilon)|\le g_3(x,y)
\end{equation}
holds.
\end{lem}
\begin{proof}
Pick a Cauchy surface \(\Sigma\), a four-potential \(A \in C_c^\infty(\mathbb{R}^4\).
A direct calculation yields
\begin{align}
&e^{-i\tilde{\lambda}(x,y)} \frac{m^2}{4\pi^2} (i\slashed{\partial}_x+m)\frac{1}{m^2(y-x-i\varepsilon e_0)^2} -\term_\eqref{eq:hadamardexp1}(x,y)\\\nonumber
=&\frac{G(x,y)}{4\pi^2} \left( (i \slashed{\partial}_x +m) \frac{1}{(y-x-i \varepsilon e_0)^2} -\frac{1}{G(x,y)}(i \slashed{\nabla}/2 - i \slashed{\nabla}^*/2 +m) \frac{G(x,y)}{(y-x-i\varepsilon e_0)^2}\right)\\\nonumber
=&\frac{iG(x,y)}{4\pi^2}  \left( \slashed{\partial}_x \frac{1}{(y-x-i \varepsilon e_0)^2} -\frac{1}{2G(x,y)}(\slashed{\partial}_x + i \slashed{A}(x) - \slashed{\partial}_y + i \slashed{A}(y) ) \frac{G(x,y)}{(y-x-i\varepsilon e_0)^2}\right)\\\nonumber
=&\frac{iG(x,y)}{4\pi^2} \left(-i\frac{\slashed{A}(x) + \slashed{A}(y)}{2 (y-x-i\varepsilon e_0)^2} + \slashed{\partial}_x \frac{1}{(y-x-i \varepsilon e_0)^2} -\frac{1}{2}(\slashed{\partial}_x- \slashed{\partial}_y ) \frac{1}{(y-x-i\varepsilon e_0)^2}\right.\\\nonumber
&\left.-\frac{1}{2G(x,y)} \frac{(\slashed{\partial}_x- \slashed{\partial}_y )G(x,y)}{(y-x-i\varepsilon e_0)^2}\right)\\\nonumber
=&\frac{iG(x,y)}{4\pi^2}  \left(-i\frac{\slashed{A}(x) + \slashed{A}(y)}{2 (y-x-i\varepsilon e_0)^2} +\frac{1}{2}(\slashed{\partial}_x+ \slashed{\partial}_y ) \frac{1}{(y-x-i\varepsilon e_0)^2}
+\frac{1}{2G(x,y)} \frac{(-\slashed{\partial}_x+ \slashed{\partial}_y )G(x,y)}{(y-x-i\varepsilon e_0)^2}\right)\\\nonumber
=&\frac{-iG(x,y)}{8\pi^2}  \frac{1}{(y-x-i\varepsilon e_0)^2} \left(i\slashed{A}(x) + i\slashed{A}(y)
+G(x,y)^{-1} \left( \slashed{\partial}_x- \slashed{\partial}_y\right) G(x,y) \right)\\\nonumber
=&\frac{G(x,y)}{8\pi^2} \frac{1}{(y-x-i\varepsilon e_0)^2} \left(\slashed{A}(x) + \slashed{A}(y)
-  2 \int_0^1 \mathrm{d}s \slashed{A}(s x + (1-s)y)  \right.\\
&\left.+(y-x)^\alpha \int_0^1 \mathrm{d}s (1-2s) (\slashed{\partial}A_\alpha)(sx+(1-s)y)   \right).
\end{align}

Now using Taylor's series for \(A\) around \(x,y\) as well as \((x+y)/2\) reveals 

\begin{align}\nonumber
\slashed{A}(x) + \slashed{A}(y)- 2 \int_0^1 \mathrm{d}s &\slashed{A}(s x + (1-s)y) =  \\\label{estimate1}
&\frac{(x-y)^\alpha}{12} (x-y)^\beta (\partial_\alpha \partial_\beta \slashed{A})((x+y)/2)+ \mathcal{O}(\|x-y\|^3)\\\nonumber
(x-y)^\alpha \int_0^1 \mathrm{d}s (1-2s) &(\slashed{\partial}A_\alpha)(sx+(1-s)y)=
 \\\label{estimate2}
& -\frac{(x-y)^\alpha}{2}(x-y)^\beta  (\slashed{\partial}\partial_\beta A_\alpha)((x+y)/2) + \mathcal{O}(\|x-y\|^3).
\end{align}

Because \(A\) is smooth and \(J_3\) is a compact region the terms \eqref{estimate1} and \eqref{estimate2} are 
bounded in this region. Because of their behaviour close to \(y=x\)
the function
is a bounded and point-wise an upper bound of the absolute value of 
\begin{align}
\left|e^{-i\tilde{\lambda}(x,y)} \frac{m^2}{4\pi^2} (i\slashed{\partial}_x+m)\frac{1}{m^2(y-x-i\varepsilon e_0)^2} -\term_\eqref{eq:hadamardexp1}(x,y)\right|\\
\label{upper bound h first term}
\le\frac{\|\term_\eqref{estimate1}\|(x,y)+\|\term_\eqref{estimate2}\|(x,y)}{8\pi^2}\frac{1}{|(y-x)^2|}.
\end{align}

Also from the Taylor's expansion \eqref{estimate1} and \eqref{estimate2} directly follows that the right hand side of \eqref{upper bound h first term} is continuous.
\end{proof}


\begin{lem}\label{lem:bound,ivp2}
For any Cauchy surface \(\Sigma\), four-potential \(A\in C_c^\infty(\mathbb{R}^4)\) and \(\lambda^A\in \mathcal{G}^A\),  the function 
\begin{align}
&\Sigma\times \Sigma \times ]0,1[\rightarrow \mathbb{C}^{4\times 4}\\\label{midterm}
&(x,y,\varepsilon)\mapsto \left(e^{-i\tilde{\lambda}(x,y)}-e^{-i\lambda^A(x,y)}\right) \frac{1}{4\pi^2}(i\slashed{\partial}_x+m)\frac{1}{(y-x-i\varepsilon e_0)^2}
\end{align}
is bounded by
\begin{align}
&\left|\left(e^{-i\tilde{\lambda}(x,y)}-e^{-i\lambda^A(x,y)}\right) \frac{1}{4\pi^2}(i\slashed{\partial}_x+m)\frac{1}{(y-x-i\varepsilon e_0)^2}\right|  \\
&\le \frac{\left| e^{-i\tilde{\lambda}(x,y)}-e^{-i\lambda^A(x,y)} \right|}{4\pi^2}\left(\eta(y-x)+\frac{\|y-x\|}{((y-x)^2)^2} + \frac{m}{|(y-x)^2|}\right):=M_2(x,y),
\end{align}
where \(\eta\) is given by
\begin{equation}
\eta(y-x):=\frac{1}{(-(y-x)^2 {\varepsilon^*}^{-0.5}+{\varepsilon^*}^{1.5})^2+\varepsilon^*(y^0-x^0)^2}
\end{equation}
and \(\varepsilon^*\) is given by
\begin{equation}
\varepsilon^* :=\frac{1}{\sqrt{6}} \sqrt{-\beta-2\alpha + \sqrt{(2\alpha + \beta)^2+12 \alpha^2}}.
\end{equation}
Furthermore \(M_2\in L^2_{\mathrm{loc}}(\Sigma\times\Sigma)\).
\end{lem}

\begin{proof}
Pick \(A\in C_c^\infty(\mathbb{R}^4)\), Cauchy surface \(\Sigma\) and \(\lambda^A\in \mathcal{G}^A\).
By expanding \(\tilde{\lambda}(x,y)-\lambda^A(x,y)\) in a Taylor series around \(\frac{y+x}{2}\) we see directly that
\begin{equation}
e^{-i\tilde{\lambda}(x,y)}-e^{-i\lambda^A(x,y)}=\mathcal{O}(\|x-y\|^2).
\end{equation}

In order to show that \eqref{midterm} has a square integrable upper bound, we consider each term separately.  First, we look at the mass term. 
It obeys

\begin{align}
\left\|\frac{m}{(y-x-i\varepsilon e_0)^2}\right\|= \frac{m}{\sqrt{((y-x)^2-\varepsilon^2)^2+\varepsilon^2(y^0-x^0)^2}}<\frac{m}{|(y-x)^2|},
\end{align}
since \((y-x)^2<0\) for \(x,y\in \Sigma\). The term with the derivative will be split into two:

\begin{align}
\left\| i\slashed{\partial}_x \frac{1}{(y-x-i\varepsilon e_0)^2} \right\|\le \frac{\| \slashed{y}-\slashed{x}-i\varepsilon \slashed{e_0}\|}{| (y-x-i\varepsilon e_0)^2|^2}
\le \frac{\varepsilon + \|x-y\|}{((y-x)^2-\varepsilon^2)^2+\varepsilon^2 (y^0-x^0)^2}\\\label{eq:uglyterm}
<\frac{1}{(-(y-x)^2 \varepsilon^{-0.5}+\varepsilon^{1.5})^2+\varepsilon(y^0-x^0)^2} +\frac{\|y-x\|}{((y-x)^2)^2},
\end{align}
we notice at this point, that the second term becomes locally square integrable in \(\Sigma\times\Sigma\) 
once multiplied with a function of type \(\mathcal{O}(\|x-y\|^2)\) close to \(x=y\). Indeed, the first term also has this property, in order to
deduce this more readily, we will maximise this term now.  Considering the limits \(\varepsilon \rightarrow 0\) and \(\varepsilon \rightarrow \infty\)
we see that this term has for arbitrary \((x,y\in\Sigma)\) a maximum in the interval \(]0,\infty[\). 
Abbreviating \(-(y-x)^2:=\alpha, (y^0-x^0)^2:=\beta\), this value of \(\varepsilon\) fulfills

\begin{align}
\partial_\varepsilon ( \varepsilon^{-0.5}\alpha + \varepsilon^{1.5})^2+\beta=0\\
\iff 3 \varepsilon^4 + \varepsilon^2(2 \alpha+\beta)-\alpha^2=0\\
\iff \varepsilon^2= \frac{-\beta-2\alpha + \sqrt{(2\alpha + \beta)^2+12 \alpha^2}}{6}\\
\iff \varepsilon=\varepsilon^* :=\frac{1}{\sqrt{6}} \sqrt{-\beta-2\alpha + \sqrt{(2\alpha + \beta)^2+12 \alpha^2}}.
\end{align}

So we can find an upper bound on the first summand of \eqref{eq:uglyterm} by replacing \(\varepsilon\) by what we just found. 
Now because all the terms in the resulting expression

\begin{equation}
\frac{1}{(-(y-x)^2 {\varepsilon^*}^{-0.5}+{\varepsilon^*}^{1.5})^2+\varepsilon^*(y^0-x^0)^2}=:\eta(y-x)
\end{equation}

are positive, we did not introduce any new singularities. Also the expression is homogenous of degree \(-3\):
\begin{align}
\eta(\delta(y-x))=\delta^{-3} \eta(y-x),
\end{align}
so we can conclude that also this term as well as the second term in \eqref{eq:uglyterm}, are of type \(\mathcal{O}(\|x-y\|^{-3})\) and therefore
locally square integrable once multiplied by the difference of exponentials in \eqref{midterm}. So summarising
\eqref{eq:uglyterm} can be bounded by 

\begin{align}\nonumber
\| \term_\eqref{eq:uglyterm}(x,y)\|\le&\\ \label{bound ivp2}
& \frac{\left| e^{-i\lambda^A(x,y)}-e^{-i\tilde{\lambda}(x,y)} \right|}{4\pi^2}\left(\eta(y-x)+\frac{\|y-x\|}{((y-x)^2)^2} + \frac{m}{|(y-x)^2|}\right),
\end{align}
which is locally square integrable in on \(\Sigma\times \Sigma\).
\end{proof}

\begin{lem}\label{lem:log term}
For any Cauchy surface \(\Sigma\), four-potential \(A\in C_c^\infty(\mathbb{R}^4)\), the function 
\begin{align}
&\Sigma\times \Sigma \times ]0,1[\rightarrow \mathbb{C}^{4\times 4}\\\label{midterm}
&(x,y,\varepsilon)\mapsto (i\slashed{\partial}_x+m) \ln(-m^2(y-x-i\varepsilon e_0)^2)
\end{align}
has the locally in \(\Sigma\times\Sigma\) square integrable bound
\begin{align}
&\left| (i\slashed{\partial}_x+m)\ln(-(y-x-i\varepsilon e_0)^2)\right|\le\\
& ~~~~~~~~~~~~\frac{2\|y-x\|}{|(y-x)^2|} + \frac{2}{\sqrt{4|(y-x)^2| + (y^0-x^0)^2}}:=M_3(x,y),
\end{align}
while the function
\begin{align}
&\Sigma\times \Sigma \times ]0,1[\rightarrow \mathbb{C}^{4\times 4}\\\label{midterm}
&(x,y,\varepsilon)\mapsto \ln(-m^2(y-x-i\varepsilon e_0)^2)
\end{align}
has the \(L^2_{\mathrm{loc}}\) bound
\begin{align}
& \| \ln(-m^2(y-x-i\varepsilon e_0)^2)\| \le  |\ln(-(y-x)^2)| +\pi/2:=M_4(x,y).
\end{align}
\end{lem}


\begin{proof}
Pick \(A\in C_c^\infty(\mathbb{R}^4)\), a Cauchy surface \(\Sigma\) and \(f\in C(\Sigma\times\Sigma,\mathbb{C}^{4\times 4})\).
The terms containing \(\slashed{\partial}\ln(-(y-x-i\varepsilon e_0)^2)\) are bounded as follows

\begin{align}
\|i\partial_\alpha\ln(-(y-x-i\varepsilon e_0)^2)\|=2\left\| \frac{y^\alpha-x^\alpha-i\varepsilon e_0^\alpha}{-(y-x-i\varepsilon e_0)^2}\right|\\
\le \frac{2\|y-x\|}{|(y-x)^2|} + \frac{2\varepsilon}{|(y-x-i\varepsilon e_0)^2|}
=\frac{2\|y-x\|}{|(y-x)^2|} + \frac{2}{\sqrt{(-(y-x)^2/\varepsilon + \varepsilon)^2 +(y^0-x^0)^2}}\\\label{bound d log}
\le \frac{2\|y-x\|}{|(y-x)^2|} + \frac{2}{\sqrt{4|(y-x)^2| + (y^0-x^0)^2}}, 
\end{align}
where inequality  in \eqref{bound d log} we maximised the expression with respect to \(\varepsilon \in ]0,\infty[\). 
The terms containing the logarithm without
any derivative are directly bounded by

\begin{align}\nonumber
|\ln(-(y-x-i\varepsilon e_0)^2)|&\\\label{bound log}\le |\ln(-(y-x)^2)| +\pi/2:=\term_{\eqref{bound log}}.
\end{align}
\end{proof}



\begin{lem}\label{lem:Hadamard=>Pol}
For every four-potential \(A\in C_c^\infty (\mathbb{R}^4)\) and every Hadamard
state \(H\) of the form \eqref{eq:hadamard1}-\eqref{eq:hadamardexp3} the  
smooth function \(w\in C^\infty(\mathbb{R}^4\times \mathbb{R}^4,\mathbb{C}^{4\times 4})\) and family 
\((w_\varepsilon)_{\varepsilon \in ]0,1[ }\subset C^\infty(\mathbb{R}^4\times \mathbb{R}^4,\mathbb{C}^{4\times4})\) of lemma \ref{lem:pointwise} also satisfy for any
 Cauchy surface \(\Sigma\) and any \(\lambda^A\in\mathcal{G}^A\):

\begin{equation}
\left.h_\varepsilon^A-w_\varepsilon-p^{\lambda^A}_\varepsilon \right|_{\Sigma\times\Sigma} \in L^2(\Sigma\times\Sigma)
\end{equation}

and the \(L^2(\Sigma\times\Sigma)\) limit \(\lim_{\varepsilon\rightarrow 0} \left.h_\varepsilon^A-w_\varepsilon-p^{\lambda^A}_\varepsilon \right|_{\Sigma\times\Sigma}\) exists.
\end{lem}

 

\begin{proof}[Proof of lemma \ref{lem:Hadamard=>Pol}:]


Pick a Cauchy surface \(\Sigma\), a four-potential \(A \in C_c^\infty(\mathbb{R}^4)\), a Hadamard state of the form 
 \eqref{eq:hadamard1} -\eqref{eq:hadamardexp3}, \(\lambda^A\in\mathcal{G}^A\)
 and for \(\varepsilon \in]0,1[\) smooth functions \(w_\varepsilon \in C^\infty(\mathbb{R}^{4+4},\mathbb{C}^{4\times 4})\)
 according to lemma \ref{lem:pointwise}. 
Our aim is to show that 
\(\left.h_\varepsilon - w_\varepsilon -p_\varepsilon^{\lambda^A} \right|_{\Sigma\times\Sigma}\) converges in the 
sense of \(L^2(\Sigma\times\Sigma)\) in the limit \(\varepsilon\rightarrow 0\).
According to lemma \ref{lem:pointwise} we have that
 \(h_\varepsilon-w_\varepsilon-p^{\lambda^A}_{\varepsilon}|_{\Sigma\times \Sigma}\) vanishes 
 outside the set \(J_3\) which is bounded, independent of \(\varepsilon\) and of finite measure. 
 Taking the exact form of \(h_\varepsilon\) and \(p^{\lambda^A}_\varepsilon\) into account,
 one notices that the function \(h_\varepsilon-w_\varepsilon-p^{\lambda^A}_{\varepsilon}|_{\Sigma\times \Sigma} (x,y)\) converges point-wise to a function defined
 on \(\Sigma\times\Sigma \backslash \{(x,x)\mid x\in \Sigma\}\). In order to show that the convergence also holds in the sense of \(L^2(\Sigma\times\Sigma)\) we
 would like to use dominated convergence. Hence we should find a function \(M\in L^2(\Sigma\times\Sigma)\) such that 
 
 
 \begin{equation}\label{eq: def bound}
\left|  (h_\varepsilon-w_\varepsilon-p^{\lambda^A}_{\varepsilon})(x,y)\right|\le |M|(x,y)
 \end{equation}
 
 holds almost for almost all \(x,y\in\Sigma\) and all \(\varepsilon\) small enough. We may pick \(\left. M\right|_{J_3^c}=0\), of course. 
So we only have to pick \(M\) for arguments inside \(J_3\). Now because \(w_\varepsilon\) is continuous as a function from 
\(\mathbb{R}^{4+4}\times [0,1]\rightarrow \mathbb{C}^{4\times 4}\) and hence also when restricted to \(J_3\times[0,1]\). The 
set \(J_3\times[0,1]\) is compact which implies that there is a constant \(M_5\) such that

\begin{equation}
\forall (x,y,\varepsilon)\in J_3\times[0,1]: |w_\varepsilon(x,y)|\le M_5
\end{equation}
holds. So by the triangular inequality we only need to bound \(h_\varepsilon-p_\varepsilon^{\lambda^A}\)

We now dissect \(h_\varepsilon\) as well as \(p_\varepsilon^{\lambda^A}\) each into three pieces

\begin{align}\label{eq:h1}
h_\varepsilon(x,y)= \frac{-1}{2(2\pi)^2} (-i\slashed{\nabla} + i \slashed{\nabla}^*-2m) \frac{G(x,y)}{(y-x-i\varepsilon e_0)^2}\\\label{eq:h2}
+ \frac{-1}{2(2\pi)^2} (-i\slashed{\nabla} + i \slashed{\nabla}^*-2m)  V(x,y) \ln (-(y-x-i\varepsilon e_0)^2)\\\label{eq:h3}
+ B(x,y),
\end{align}
 
 
 \begin{align}\label{eq:p1}
 p_\varepsilon^{\lambda^A}(x,y)=e^{-i\lambda^A(x,y)}\frac{m^2}{4\pi^2}(i\slashed{\partial}+m) \frac{1}{m^2(y-x-i\varepsilon e_0)^2}\\\label{eq:p2}
 -\frac{m^2}{4\pi^2}(i\slashed{\partial}+m) Q_1(-m^2(y-x-i\varepsilon e_0)^2) \ln (-m^2(y-x-i\varepsilon e_0)^2)\\\label{eq:p3}
 -\frac{m^2}{4\pi^2}(i\slashed{\partial}+m) Q_2(-m^2(y-x-i\varepsilon e_0)^2).
 \end{align}
 
Let us first consider the most singular terms of  \(h_\varepsilon-w_\varepsilon-p^{\lambda^A}_{\varepsilon}|_{\Sigma\times \Sigma}\), 
namely \(\term_\eqref{eq:h1}-\term_\eqref{eq:p1}\). In order to better compare these terms, we will add \(0\) in the form of \(\term_\eqref{eq:p1}-\term_\eqref{eq:p1}\) where we
replace \(\lambda^A\) by \(\tilde{\lambda}\).
By lemma \ref{lem:bound,worstterm} the term 
\begin{equation}
\term_\eqref{eq:h1}- G(x,y)\frac{m^2}{4\pi^2}(i\slashed{\partial}+m) \frac{1}{m^2(y-x-i\varepsilon e_0)^2}
\end{equation}
is uniformly bounded in \(J_3\).

%The first term then yields
%\begin{align}\nonumber
%&e^{-i\tilde{\lambda}(x,y)} \frac{m^2}{4\pi^2} (i\slashed{\partial}_x+m)\frac{1}{m^2(y-x-i\varepsilon e_0)^2} - \eqref{eq:h1}(x,y)\\
%=&\frac{G(x,y)}{4\pi^2} \left( (i \slashed{\partial}_x +m) \frac{1}{(y-x-i \varepsilon e_0)^2} -\frac{1}{G(x,y)}(i \slashed{\nabla}/2 - i \slashed{\nabla}^*/2 +m) \frac{G(x,y)}{(y-x-i\varepsilon e_0)^2}\right)\\\nonumber
%&=\frac{iG(x,y)}{4\pi^2}  \left( \slashed{\partial}_x \frac{1}{(y-x-i \varepsilon e_0)^2} -\frac{1}{2G(x,y)}(\slashed{\partial}_x + i \slashed{A}(x) - \slashed{\partial}_y + i \slashed{A}(y) ) \frac{G(x,y)}{(y-x-i\varepsilon e_0)^2}\right)\\\nonumber
%&=\frac{iG(x,y)}{4\pi^2} \left(-i\frac{\slashed{A}(x) + \slashed{A}(y)}{2 (y-x-i\varepsilon e_0)^2} + \slashed{\partial}_x \frac{1}{(y-x-i \varepsilon e_0)^2} -\frac{1}{2}(\slashed{\partial}_x- \slashed{\partial}_y ) \frac{1}{(y-x-i\varepsilon e_0)^2}\right.\\\nonumber
%&\left.-\frac{1}{2G(x,y)} \frac{(\slashed{\partial}_x- \slashed{\partial}_y )G(x,y)}{(y-x-i\varepsilon e_0)^2}\right)\\\nonumber
%&=\frac{iG(x,y)}{4\pi^2}  \left(-i\frac{\slashed{A}(x) + \slashed{A}(y)}{2 (y-x-i\varepsilon e_0)^2} +\frac{1}{2}(\slashed{\partial}_x+ \slashed{\partial}_y ) \frac{1}{(y-x-i\varepsilon e_0)^2}
%+\frac{1}{2G(x,y)} \frac{(-\slashed{\partial}_x+ \slashed{\partial}_y )G(x,y)}{(y-x-i\varepsilon e_0)^2}\right)\\\nonumber
%&=\frac{-iG(x,y)}{8\pi^2}  \frac{1}{(y-x-i\varepsilon e_0)^2} \left(i\slashed{A}(x) + i\slashed{A}(y)
%+G(x,y)^{-1} \left( \slashed{\partial}_x- \slashed{\partial}_y\right) G(x,y) \right)\\\nonumber
%&=\frac{G(x,y)}{8\pi^2} \frac{1}{(y-x-i\varepsilon e_0)^2} \left(\slashed{A}(x) + \slashed{A}(y)
%-  2 \int_0^1 \mathrm{d}s \slashed{A}(s x + (1-s)y)  \right.\\\nonumber
%&\left.+(y-x)^\alpha \int_0^1 \mathrm{d}s (1-2s) (\slashed{\partial}A_\alpha)(sx+(1-s)y)   \right).
%\end{align}

%Now using Taylor's series for \(A\) around \(x,y\) as well as \((x+y)/2\) reveals 

%\begin{align}\nonumber
%\slashed{A}(x) + \slashed{A}(y)- 2 \int_0^1 \mathrm{d}s &\slashed{A}(s x + (1-s)y) =  \\\label{estimate1}
%&\frac{(x-y)^\alpha}{12} (x-y)^\beta (\partial_\alpha \partial_\beta \slashed{A})((x+y)/2)+ \mathcal{O}(\|x-y\|^3)\\\nonumber
%(x-y)^\alpha \int_0^1 \mathrm{d}s (1-2s) &(\slashed{\partial}A_\alpha)(sx+(1-s)y)=
 %\\\label{estimate2}
%& -\frac{(x-y)^\alpha}{2}(x-y)^\beta  (\slashed{\partial}\partial_\beta A_\alpha)((x+y)/2) + \mathcal{O}(\|x-y\|^3).
%\end{align}

%Because \(A\) is smooth and \(J_3\) is a compact region the terms \eqref{estimate1} and \eqref{estimate2} are bounded in this region. Because of their behaviour close to \(y=x\)
%the function
%\begin{equation}\label{upper bound h first term}
%\frac{|\eqref{estimate1}|(x,y)+|\eqref{estimate2}|(x,y)}{8\pi^2}\frac{1}{|(y-x)^2|}
%\end{equation}
%is square integrable and point-wise an upper bound of the absolute value of 
%\(e^{-i\tilde{\lambda}(x,y)} \frac{m^2}{4\pi^2} (i\slashed{\partial}_x+m)\frac{1}{m^2(y-x-i\varepsilon e_0)^2} - \eqref{eq:h1}(x,y)\).

Next we find a square integrable upper bound on 
\begin{equation}
e^{-i\tilde{\lambda}(x,y)} \frac{m^2}{4\pi^2} (i\slashed{\partial}_x+m)\frac{1}{m^2(y-x-i\varepsilon e_0)^2}-\term_\eqref{eq:p1}(x,y).
\end{equation}
We rewrite this term as

\begin{align}
e^{-i\tilde{\lambda}(x,y)} \frac{m^2}{4\pi^2} (i\slashed{\partial}_x+m)\frac{1}{m^2(y-x-i\varepsilon e_0)^2}-\term_\eqref{eq:p1}(x,y)=\\\label{eq:ivp2estimate}
\left(e^{-i\tilde{\lambda}(x,y)}-e^{-i\lambda(x,y)}\right) \frac{1}{4\pi^2}(i\slashed{\partial}_x+m)\frac{1}{(y-x-i\varepsilon e_0)^2},
\end{align}

according to lemma \ref{lem:bound,ivp2} this has the upper bound \(M_2\in L^2_{\mathrm{loc}}(\Sigma\times\Sigma\), which is independend of \(\varepsilon\). 
The term \(\term_\eqref{eq:p2} -\term_\eqref{eq:h2}\) are bounded by lemma \ref{lem:log term}.
Thus we obtain for \(M\) for \((x,y)\in J_3\):

%\begin{align}
%M(x,y)=\frac{|\eqref{estimate1}|(x,y)+|\eqref{estimate2}|(x,y)}{8\pi^2}\frac{1}{|(y-x)^2|}\\
%+\frac{\left| e^{-i\lambda(x,y)}-e^{-i\tilde{\lambda}(x,y)} \right|}{4\pi^2}\left(\eta(y-x)+\frac{\|y-x\|}{((y-x)^2)^2} + \frac{m}{|(y-x)^2|}\right)\\
%+\frac{1}{2\pi^2}(4 | V(x,y)| + m^2 \sup_{\varepsilon \in [0,1]}|Q_1(-m^2(y-x-i\varepsilon e_0)^2)|) \left( \frac{\|y-x\|}{|(y-x)^2|}+\frac{1}{\sqrt{4|(y-x)^2|+(y^0-x^0)^2}} \right)\\
%+ \frac{1}{4\pi^2}(|(i\slashed{\nabla}-i\slashed{\nabla}^*-2m)V(x,y)|/2+m^2 \sup_{\varepsilon \in [0,1]} |(i\slashed{\partial}+m)Q_1(-m^2(y-x-i\varepsilon e_0))| ) (|\ln(-(y-x)^2)| +\pi/2)\\
%+|B(x,y)|+\frac{m^2}{4\pi^2}\sup_{\varepsilon\in[0,1]} |(i\slashed{\partial}_x+m)Q_2(-m^2(y-x-i\varepsilon e_0)^2)|
%\end{align}


\begin{align*}
&M(x,y):=M_5+M_1
+M_2(x,y)\\
&+\frac{1}{4\pi^2}\left(4 | V(x,y)| + m^2 \sup_{\varepsilon \in [0,1]}|Q_1(-m^2(y-x-i\varepsilon e_0)^2)|\right)M_3(x,y)\\
&+ \frac{1}{4\pi^2}\left(\!|(i\slashed{\nabla}\!-i\slashed{\nabla}^*\!-2m)V(x,y)|/2+m^2\!\! \sup_{\varepsilon \in [0,1]}
 \!|(i\slashed{\partial}+m)Q_1(-m^2(y-x-i\varepsilon e_0))| \right)M_4(x,y)\\
&+|B(x,y)|+\frac{m^2}{4\pi^2}\sup_{\varepsilon\in[0,1]} |(i\slashed{\partial}_x+m)Q_2(-m^2(y-x-i\varepsilon e_0)^2)| + \sup_{\varepsilon \in ]0,1]}|w_\varepsilon|(x,y)
\end{align*}

The argument of \(h_\varepsilon-w_\varepsilon-p^\lambda_{\varepsilon}|_{\Sigma\times \Sigma}(x,y)\) 
converging to \(h_0-w-p^\lambda_{0}|_{\Sigma\times \Sigma}(x,y)\) in the sense of \(L^2(\Sigma\times\Sigma)\)
is completed by applying dominated convergence theorem. 

As a final point: the construction seems to depend on the function \(\lambda^A\) we chose, however for any different \({\lambda'}^A\in \mathcal{G}^A\) 
we have due to remark \ref{main results of ivp2}

\begin{equation}
P^{\lambda}-P^{\lambda'}\in I_2(\mathcal{H}_\Sigma ),
\end{equation}

therefore also the limit \(\lim_{\varepsilon\rightarrow 0} h_\varepsilon^A - w_\varepsilon - p_\varepsilon^{{\lambda'}^A}\) exists in the sense of \(L^2\).



%\begin{align}
%&\|H^\delta-w^\delta-P^\lambda\|_{I_2}^2\\
%&=\lim_{\varepsilon \rightarrow 0}\int_{\{x,y\in J_\delta\}} | h_\varepsilon -w - p_\varepsilon^\lambda |^2(x,y)
%+\int_{\{x,y\in J_\delta\}^c} | p^\lambda_0 |^2(x,y)\\
%&\le \int_{\{x,y\in J_\delta\}^c} | p^\lambda_0 |^2(x,y) +\lim_{\varepsilon \rightarrow 0}\int_{\{x,y\in J_\delta \}} 3(C+ \\
%&| V(x,y)\ln(-(y-x-i\varepsilon e_0)^2) -Q(m \sqrt{-(y-x-i\varepsilon e_0)}) -w(x,y))  |^2) \\
%&+ 3\|P^\lambda - P^{\tilde{\lambda}}\|_{L^2(J_\delta,\mathbb{C}^4)}^2
%<C_2<\infty ,
%\end{align}
%where \(C\) is the upper bound on \(\frac{\eqref{estimate1}+\eqref{estimate2}}{8 \pi^2 (y-x-i\varepsilon e_0)^2}\) and \(Q\) was defined in \eqref{def:Q}. 
%The factor of \(3\) is due to \((a+b+c)^2\le 3(a^2+b^2+c^2)\) after applying the triangular inequality to the parts of \(\|H^\delta-w^\delta-P^\lambda\|\).
% The Integral over \(J_\delta\) is over a domain of finite measure of locally square integrable functions. 

\end{proof}


\begin{proof}[Proof of theorem \ref{thm:hadamard=>Pol}:]
We pick for a four-potential \(A\in C_c^\infty(\mathbb{R}^4)\) a Hadamard state \(H\) of the form \eqref{eq:hadamard1} -\eqref{eq:hadamardexp3} 
and a \(\lambda^A\in\mathcal{G}^A\). Then we pick 
\(w_\varepsilon,w \in C_c^\infty(\mathbb{R}^{4+4},\mathbb{C}^{4\times 4})\) for all \(\varepsilon\in [0,1]\) according to lemma \ref{lem:pointwise}.
Because of lemma \ref{lem:Hadamard=>Pol}, we have that

\begin{equation}
h_\varepsilon^A-w_\varepsilon-p^{\lambda^A}_\varepsilon \in L^2(\Sigma\times\Sigma)
\end{equation}

holds. We can define the operator \(Z \in I_2(\mathcal{H}_\Sigma)\) for any Cauchy surface \(\Sigma\) as

\begin{equation}
\mathcal{H}_\Sigma\ni \psi\mapsto Z \psi := \lim_{\varepsilon \rightarrow 0} 
\int_{\Sigma} (h_\varepsilon^A-w_\varepsilon-p^{\lambda^A}_\varepsilon)(\cdot,y) i_\gamma(d^4y) \psi(y) 
\end{equation}

and

\begin{equation}
\tilde{P}_\Sigma:=Z + P^{\lambda^A}_\Sigma.
\end{equation}

Because of \eqref{equiv:pLambda} we have for any other \(\lambda' \in\mathcal{G}^A\) that \(\tilde{P}_\Sigma-P^{\lambda'}\in I_2(\mathcal{H}_\Sigma)\).


\end{proof}

\begin{proof}[Proof of theorem \ref{thm:Pol=>hadamard}]
Pick a four-potential \(A\in C_c^\infty(\mathbb{R}^4)\)
and a \(\lambda^A\in\mathcal{G}^A\).
Also pick some Hadamard state of the form \eqref{eq:hadamard1}-\eqref{eq:hadamardexp3}.
Next, we pick \(w_\varepsilon\) according to lemma \ref{lem:pointwise}; however, due to notational reasons we will denote it by \(R_\varepsilon\).

Define for each \(\varepsilon \in ]0,1]\) the function \(w_\varepsilon\) as
\begin{align}\label{wurm A}
w_\varepsilon &(x,y):= -\frac{e^{-i (x-y)^\alpha\int_0^1 ds A_\alpha (x s + (1-s)y)}-e^{-i\lambda^A(x,y)}}{(2\pi)^2}(i\slashed{\partial}_x +m) \frac{1}{(y-x-i\varepsilon e_0)^2}\\
&+\frac{e^{-i (x-y)^\alpha\int_0^1 ds A_\alpha (x s + (1-s)y)}}{2(2\pi)^2 (y-x-i\varepsilon e_0)^2}\left( \slashed{A}(x)+\slashed{A}(y) -2 \int_0^1 ds \slashed{A}(sx +(1-s)y) \right.\\
&\left.+ (x-y)^\alpha \int_0^1 ds (1-2s)(\slashed{\partial}A_\alpha)(sx + (1-s)y) \right)\\
&+ \frac{1}{2(2\pi)^2} (-i\slashed{\nabla}+i\slashed{\nabla}^* -2m) V(x,y)\ln (-(y-x-i\varepsilon e_0)^2)\\\label{wurm E}
&-\frac{m^2 e^{-i \lambda^A(x,y)}}{4\pi^2} (i\slashed{\partial}_x+m)\left(Q_1(m\sqrt{-(y-x-i\varepsilon e_0)^2}\ln(-m^2(y-x-i\varepsilon e_0)^2)\right.\\
&\left.+Q_2(m\sqrt{-(y-x-i\varepsilon e_0)^2}\right) + R_\varepsilon(x,y).
\end{align}

We further define the distribution \(H^{\lambda^A}\) by its action on test functions 
\(f_1,f_2\in C_c^\infty(\mathbb{R}^4,\mathbb{C}^4)\)

\begin{equation}
\tilde{H}(f_1,f_2):=\lim_{\varepsilon \rightarrow 0} \int_{\mathbb{R}^4}\int_{\mathbb{R}^4} \overline{f}_1(x) (p_\varepsilon^{\lambda^A}(x,y)-w_\varepsilon(x,y)) f_2(y) d^4x d^4y,
\end{equation}

with \(w_\varepsilon\) given by the terms from \eqref{wurm A} until \eqref{wurm E}. First, we verify that \(H^{\lambda^A}\) is indeed a Hadamard state. Pick \(\varepsilon>0\), we find

\begin{align*}
&p_\varepsilon^{\lambda^A}(x,y)-w_\varepsilon(x,y)=-e^{-i\lambda^A(x,y)}\frac{m^2}{4\pi^2} (i\slashed{\partial}_x+m)
\frac{K_1(m\sqrt{-(y-x-i\varepsilon e_0)^2})}{m\sqrt{-(y-x-i\varepsilon e_0)^2}} - w_\varepsilon(x,y)\\
&=-e^{-i\lambda^A(x,y)}\frac{m^2}{4\pi^2} (i\slashed{\partial}_x+m) \left( \frac{-1}{m^2(y-x-i\varepsilon e_0)^2} \right.\\
&\left.+ Q_1(-m^2 (y-x-i\varepsilon e_0)^2) \ln (m\sqrt{-(y-x-i\varepsilon e_0)^2})
+ Q_2(-m^2 (y-x-i\varepsilon e_0)^2)\right) -w_\varepsilon(x,y).
\end{align*}

Inserting \(w_\varepsilon(x,y)\) we find that the most divergent terms proportional to \(e^{-i\lambda^A}\), as well as the terms involving \(Q_1\) and \(Q_2\) cancel.
This results in 

\begin{align*}
&p_\varepsilon^{\lambda^A}(x,y)-w_\varepsilon(x,y)\\
&=\frac{e^{-i(x-y)^\alpha \int_0^1 ds A_\alpha (xs + (1-s)y)}}{(2\pi)^2}(i\slashed{\partial}_x+m)\frac{1}{(y-x-i\varepsilon e_0)^2}\\
&+\frac{e^{-i (x-y)^\alpha\int_0^1 ds A_\alpha (x s + (1-s)y)}}{2(2\pi)^2 (y-x-i\varepsilon e_0)^2}\left( \slashed{A}(x)+\slashed{A}(y) -2 \int_0^1 ds \slashed{A}(sx +(1-s)y) \right.\\
&\left.+ (x-y)^\alpha \int_0^1 ds (1-2s)(\slashed{\partial}A)(sx + (1-s)y) \right)\\
&- \frac{1}{2(2\pi)^2} (-i\slashed{\nabla}+i\slashed{\nabla}^* -2m) V(x,y)\ln (-(y-x-i\varepsilon e_0)^2)\\
&-R_\varepsilon(x,y).
\end{align*}

Now the terms involving \(e^{-i(x-y)^2\int_0^1ds A_\alpha(xs+(1-s)y)}\) can be summarised

\begin{align*}
&\frac{e^{-i(x-y)^\alpha \int_0^1 ds A_\alpha (xs + (1-s)y)}}{(2\pi)^2}(i\slashed{\partial}_x+m)\frac{1}{(y-x-i\varepsilon e_0)^2}\\
&+\frac{e^{-i (x-y)^\alpha\int_0^1 ds A_\alpha (x s + (1-s)y)}}{2(2\pi)^2 (y-x-i\varepsilon e_0)^2}\left( \slashed{A}(x)+\slashed{A}(y) -2 \int_0^1 ds \slashed{A}(sx +(1-s)y) \right.\\
&\left.+ (x-y)^\alpha \int_0^1 ds (1-2s)(\slashed{\partial}A)(sx + (1-s)y) \right)\\
&=\frac{1}{2(2\pi)^2 }(i\slashed{\nabla}-i\slashed{\nabla}^*+2m) \frac{e^{-i (x-y)^\alpha\int_0^1 ds A_\alpha (x s + (1-s)y)}}{(y-x-i\varepsilon e_0)^2},
\end{align*}

so overall we find

\begin{align*}
&p_\varepsilon^{\lambda^A}(x,y)-w_\varepsilon(x,y)=\\
&\frac{-1}{2(2\pi)^2 }(-i\slashed{\nabla}+i\slashed{\nabla}^*-2m) \frac{e^{-i (x-y)^\alpha\int_0^1 ds A_\alpha (x s + (1-s)y)}}{(y-x-i\varepsilon e_0)^2}\\
&- \frac{1}{2(2\pi)^2} (-i\slashed{\nabla}+i\slashed{\nabla}^* -2m) V(x,y)\ln (-(y-x-i\varepsilon e_0)^2)\\
&-R_\varepsilon(x,y).
\end{align*}

Because the term in the last line is smooth, this means that indeed \(p_\varepsilon^{\lambda^A}(x,y)-w_\varepsilon(x,y)\) is the integration kernel of a Hadamard state.
Furthermore, for a given Cauchy surface \(\Sigma\) the function \(w_\varepsilon-R_\varepsilon\) has a limit in the sense of \(L^2_{\text{loc}}(\Sigma\times\Sigma)\),
because of
Lebesgue dominated convergence and
lemmata \ref{lem:bound,worstterm}, \ref{lem:bound,ivp2} and  \ref{lem:log term}. That the term \(R_\varepsilon\) 
converges in \(C^\infty(\mathbb{R}^{4+4},\mathbb{C}^{4\cdot 4})\) follows from lemma \ref{lem:pointwise}.

\end{proof}




\bibliographystyle{plain}
\bibliography{ref}

\end{document}

