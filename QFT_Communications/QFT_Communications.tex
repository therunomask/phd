\documentclass[a4paper,11pt]{article}

%\usepackage{german}

\usepackage[dvipsnames]{xcolor}
\usepackage{graphicx}

\usepackage{amssymb}

\usepackage{amsfonts}

\usepackage{amsmath}

\usepackage{amsthm}

\usepackage[unicode=true, pdfusetitle, bookmarks=true,
  bookmarksnumbered=false, bookmarksopen=false, breaklinks=true, 
  pdfborder={0 0 0}, backref=false, colorlinks=true, linkcolor=blue,
  citecolor=blue, urlcolor=blue]{hyperref}
\usepackage{slashed}
\usepackage{authblk}
%identity sign
\usepackage{dsfont}

%commutative diagrams
\usepackage{amsmath,amscd}


\newtheorem{de}{Definition}
\newtheorem{theo}{Theorem}
\newtheorem{rmk}{Remark}

\addtolength{\textwidth}{2.2cm} \addtolength{\hoffset}{-1.0cm}

\addtolength{\textheight}{3.0cm} \addtolength{\voffset}{-2cm} 

\parindent 0cm

\pagestyle{empty}

\begin{document}
\title{The Relationship Between Hadamard States, Admissible Polarisation Classes and the Fermionic Projector}

\author{Markus Nöth\thanks{noeth@math.lmu.de \\\tiny{Mathematisches Institut der Ludwig-Maximilians-Universit\"at M\"unchen,}
    \tiny{Theresienstr. 39, 80333 M\"unchen, Germany}}
	,
D.-A. Deckert\thanks{deckert@math.lmu.de \\ \tiny{Mathematisches Institut der Ludwig-Maximilians-Universit\"at M\"unchen,}
    \tiny{Theresienstr. 39, 80333 M\"unchen, Germany}}
	 ~and
Felix Finster\thanks{felix.finster@mathematik.uni-regensburg.de \\     \tiny{Universität Regensburg, Universitätsstraße 31, 93053 Regensburg, Germany}}
}
\date{\today}



\maketitle

\begin{abstract}
to be written
\end{abstract}

\section{Introduction}
This paper compares central objects of three different approaches to quantum field theory (QFT).
These approaches have different scopes and pursue different ideas, so a comparison of the theories as a whole is difficult.
Since they all study different aspects of QFT; however, one can make out objects in each of these theories which closely resemble
one another. The first class of objects of interest are Hadamard states which appear in the algebraic approach to 
QFT\cite{fulling1978singularity}. The second class of objects are the (almost) projectors \(P^\lambda_{\Sigma}\) which are 
closely linked to polarisation classes of the vacuum of external field quantum electrodynamics (QED). 
Polarisation classes play a central role in the analysis of external field QED\cite{ivp0, ivp1, ivp2}.
The third class of objects are the continuum limit of the Fermionic Projectors\cite{something}. 
In the following sections we will very briefly summarise the role of each of the targeted objects in their respective approach to QFT. 
In these sections we want to focus on the similar physical interpretation of these objects.
A reader familiar with some of these theories may skip the respective sections. 
Afterwards we shift our attention to the similarities between the mathematical structure inherent in these objects.

%some introductory math needs to be written also we should mention that we are concerned with the Dirac field

\begin{equation}
(i\slashed{\partial} - m - \slashed{A})\psi = 0
\end{equation}

\begin{equation}
U_{\Sigma',\Sigma}:\mathcal{H}_\Sigma \rightarrow \mathcal{H}_{\Sigma'},
\end{equation}

\section{Hadamard States}
In the algebraic approach to QFT one puts less emphasise on the Hilbert space as is commonly done in non relativistic physics, because 
once one talks about quantum fields in a relativistic setting the specific representation of operators by operators on a Hilbert 
space is no longer canonical. 
In stead one tries to focus on the operators that represent the physical measurements one wants to perform later on. 
One can construct an algebra which contains these operators and do most of the analysis on this algebra itself; however, 
many construction common to that analysis already make use of special properties of special states. The so called Hadamard states 
(We will introduce them shortly) often play a role in constructing operators that represent e.g. objects like the energy 
momentum tensor of a quantum field. Additionally at some point in
the physical analysis one would like to produce some numbers which can be compared to experiments. In order to do so, one
has to pick a representation. The GNS construction is very helpful in this regard, since it reduces the construction of a representation to 
picking one state to build a Hilbert space around. This choice has to be made on physical grounds. Two classes of states have been
identified which have a physical meaning. 
The first class of which, KMS states, can be viewed as thermal states, they will
not play a major role in this paper, since we are not concerned with thermodynamics. 
The second class of states that has a physical 
significance are Hadamard states. They are usually interpreted as states of positive energy. This interpretation has to be taken with 
a grain of salt. Even in Minkowski spacetime with a non vanishing external electromagnetic field but no interaction the splitting of the
spectrum into positive and negative part becomes somewhat arbitrary, since no particular splitting is Lorentz invariant\cite{something!}.
Instead one could interpret picking a Hadamard state as making a choice of which states one
calls positive energy states. This may become clearer in section \ref{sec:pol classes} where we bridge the gap to another approach to
the same problem. Hadamard states are singled out by their dynamical properties. The definition of Hadamard states is a technical one, we 
need to introduce some notation in order to give it. Let \(M\) denote a globally hyperbolic Manifold and \(g\) its metric, let further \(\Delta\) be the van Vleck-Morette determinant\cite{Dewitt1960}. We define for \(x,y\in M\), \(n\in\mathbb{N}\)

\begin{equation}
v^{(n)}(x,y):= \sum_{k=0}^n v_k(x,y) \sigma(x,y)^k,
\end{equation}

where \(\sigma(x,y):= (x-y)^\alpha (x-y)_\alpha\) and the functions \(v_m\) are defined by the Hadamard recursion relations\cite{DeWitt1960}. 
Now we call a state \(W\) a Hadamard state if for each \(n\in\mathbb{N}\) there is some function \(R\in C^n(M\times M)\) and \(\varepsilon>0\) 
such that the integration kernel of the two-point function \(w\) of the state fulfils


\begin{equation}\label{HadamardGeneral}
w(x,y)=\frac{\xi(x,y)}{4 \pi^2} \left[\frac{\sqrt{\Delta}}{\sigma(x,y)+i\varepsilon} + v^{(n)}(x,y) \ln (\sigma(x,y)+i \varepsilon) \right]
+R(x,y),
\end{equation}
%This Hadamard condition is the one for scalar particles. Which one is the correct one for fermions?

where \(\xi\in C^\infty(M\times M)\) is a bump function equal to 1 in some neighbourhood \(U\) of the set of pairs of points that can be connected 
by a time-like or lightlike curve and vanishes outside some neighbourhood of \(U\). That this definition does not 
depend on all of the apparent choices we made along the way was shown 
by \cite{kay1991theorems} from which we also took the definition. 

We will merely be concerned with the Dirac field in flat Minkowski spacetime where \eqref{HadamardGeneral} specialises to

\begin{equation}
w(x,y)=\frac{\xi(x,y)}{4 \pi^2} i(i\slashed{\partial} + m) \left[\frac{1}{\sigma(x,y)+i \varepsilon} +  \ln (\sigma(x,y)+i \varepsilon) \frac{I_1(\sqrt{-\sigma(x,y)+i \varepsilon})}{\sqrt{-\sigma(x,y)+i \varepsilon}} \right]
+R(x,y),
\end{equation}

where  \(I_1\) is a modified Bessel function of the first kind. \textcolor{red}{Das ist zusammengeschustert aus Peskin Schroeder, Abramowitz Stegun, Thaller und dem Wikipediaartikel für Propagatoren(Feynmanpropagator), kann das jemand verifizieren?}

%note to self: should I mention that fields are operator valued distributions? 

%write that one needs to restrict to Hadamard states for the perturbative expansion of interacting QFT (cite this wald overview paper, p.15)

%give some reply to the standard argument that notions of particles are non unique in QFT in curved spacetimes. 



\section{Projectors for Polarisation Classes}\label{sec:pol classes}

The concept of polarisation classes arises naturally in the study of QED in external electromagnetic fields. It does need some
machinery to be introduced and related to more familiar objects which we are going to introduce first. Throughout this section
\(\Sigma, \Sigma', \Sigma''\) denote arbitrary Cauchy surfaces, while \(A\in C_c^\infty(\mathbb{R}^4,\mathbb{R}^4)\) 
is a four potential and \(\Sigma_{\text{in}}\) denotes a Cauchy surface earlier than \(\text{supp}(A)\).

\begin{de}
Let \(\text{Pol} (\mathcal{H}_\Sigma)\) denote the set of all closed, linear subspaces \(V\subset \mathcal{H}\)
such that both \(V\) and \(V^\perp\) are infinite dimensional. Any \(V\in \text{Pol}(\mathcal{H}_\Sigma)\) is called 
\emph{polarisation} of \(\mathcal{H}_\Sigma\). For \(V\in \text{Pol}(\mathcal{H}_\Sigma)\), let \(P_\Sigma^V\rightarrow V\) 
denote the orthogonal projection of \(\mathcal{H}_\Sigma\) onto \(V\).
The Fock space corresponding to \(V\) on the Cauchy surface \(\Sigma\) is defined to be
\begin{equation}
\mathcal{F}(V,\mathcal{H}_\Sigma) := \bigoplus_{c\in\mathbb{Z}} \mathcal{F}_c (V,\mathcal{H}_\Sigma), \quad 
\mathcal{F}_c(V,\mathcal{H}_\Sigma):= \bigoplus_{\overset{n,m\in\mathbb{N}_0}{c=m-n}}(V^\perp)^{\wedge n} \otimes \overline{V}^{\wedge m}
\end{equation},
\end{de}

where \(\bigoplus\) is the Hilbert space direct sum, \(\wedge\) the antisymmetric tensor product of Hilbert spaces and 
\(\overline{V}\) is the conjugate complex vector space of \(V\), which coincides with \(V\) as a set, has the same vector 
space operators as \(V\) except for scalar multiplication, which is defined by \((z,\psi)\mapsto z^* \psi\) for 
\(z\in\mathbb{C},\psi\in V\).

\begin{rmk}\label{WhyUseApprox}
Given two polarisations \(V,W\in \text{Pol}(\mathcal{H}_\Sigma)\), the two Fockspaces \(\mathcal{F}(V,\mathcal{H}_\Sigma)\) and 
\(\mathcal{F}(W,\mathcal{H}_\Sigma)\) are canonically unitary equivalent if \textcolor{red}{does also \(\iff\)  hold??} 
\(P^V_\Sigma - P^W_\Sigma \in I_2(\mathcal{H}_\Sigma)\) by the theorem of Shale and Stinespring 
\cite{Shale Stinespring 1965},.
\end{rmk}
Remark \ref{WhyUseApprox} gives us a natural limit with 
respect to which it is useful
to analyse the regularity of Projectors \(P_\Sigma^A\). We will therefore be content with the following equivalence class.
\begin{de}
For \(V,W\in \text{Pol}(\mathcal{H}_\Sigma)\) we write
\begin{equation}
V\approx W \iff P^V_\Sigma - P^W_\Sigma \in I_2(\mathcal{H}),
\end{equation}
\begin{equation}
C_\Sigma (A):= [ U_{\Sigma \Sigma_{\text{in}}}^A \mathcal{H}_{\Sigma_{\text{in}}}^-]_\approx.
\end{equation}
\end{de}

The equivalence class \(C_\Sigma(A)\) transforms naturally with respect to gauge and Lorentz transforms\cite{ivp2}. 
The projector \(P^{\mathcal{H}^-}_{\Sigma} \) has the well known representation as the weak limit of the integral operator 
with the kernel\cite{ivp2}

\begin{equation}
p^-_\varepsilon(x,y)= -\frac{m^2}{4\pi^2}(-i\slashed{\partial}_y+m) \frac{K_1(m \sqrt{-(y-x-i\varepsilon e_0)^2})}{m \sqrt{-(y-x-i\varepsilon e_0)^2}},
\end{equation}
where the square is a Minkowski square and the square root denotes its the principle branch. By weak limit we mean

\begin{align}
\langle \phi, P^{\mathcal{H}^-}_{\Sigma} \psi\rangle = \lim_{\varepsilon \searrow 0} \int_{\Sigma\times \Sigma} \overline{\phi}(x) i_\gamma(d^4x) p^-_\varepsilon(y-x ) i_\gamma(d^4y) \psi(y),
\end{align}

for general \(\phi, \psi \in \mathcal{H}_\Sigma\).
Now for hyperplanes \(\Sigma\cup \text{supp}(A)\neq \emptyset\) the operator \(P^{\mathcal{H}^-}_{\Sigma} \) does not represent
\(C_\Sigma(A)\), but one can adapt the projector to something that is \(I_2\) 
close to a projector onto a representative of \(C_\Sigma(A)\), namely

\begin{align}
\langle \phi, P^A_\Sigma \psi\rangle =& \lim_{\varepsilon \searrow 0} \langle \phi, P^{A,\varepsilon}_\Sigma \psi\rangle ,\\
\langle \phi, P^{A,\varepsilon}_\Sigma \psi\rangle :=&
\int_{\Sigma\times\Sigma}\overline{\phi}(x)i_\gamma(d^4x) 
e^{\frac{-i}{2}(A(x)+A(y))_\mu(x-y)^\mu} p^-_\varepsilon(y-x)i_\gamma(d^4y) \psi(y).
\end{align}


\section{The Fermionic Projector}




\bibliographystyle{plain}
\bibliography{ref}

\end{document}

