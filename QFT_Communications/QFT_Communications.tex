\documentclass[a4paper,11pt]{article}

%\usepackage{german}

\usepackage[dvipsnames]{xcolor}
\usepackage{graphicx}

\usepackage{amssymb}

\usepackage{amsfonts}

\usepackage{amsmath}

\usepackage{amsthm}

\usepackage[unicode=true, pdfusetitle, bookmarks=true,
  bookmarksnumbered=false, bookmarksopen=false, breaklinks=true, 
  pdfborder={0 0 0}, backref=false, colorlinks=true, linkcolor=blue,
  citecolor=blue, urlcolor=blue]{hyperref}
\usepackage{slashed}
\usepackage{authblk}
%identity sign
\usepackage{dsfont}

%commutative diagrams
\usepackage{amsmath,amscd}
\usepackage{enumitem}

\newtheorem{de}{Definition}
\newtheorem{thm}{Theorem}
\newtheorem{rmk}{Remark}


\newcommand{\supp}{\operatorname{supp}}


\addtolength{\textwidth}{2.2cm} \addtolength{\hoffset}{-1.0cm}

\addtolength{\textheight}{3.0cm} \addtolength{\voffset}{-2cm} 

\parindent 0cm

\pagestyle{empty}

\begin{document}
\title{The Relationship Between Hadamard States, Admissible Polarisation Classes and the Fermionic Projector}

\author{Markus Nöth\thanks{noeth@math.lmu.de \\\tiny{Mathematisches Institut der Ludwig-Maximilians-Universit\"at M\"unchen,}
    \tiny{Theresienstr. 39, 80333 M\"unchen, Germany}}
	,
D.-A. Deckert\thanks{deckert@math.lmu.de \\ \tiny{Mathematisches Institut der Ludwig-Maximilians-Universit\"at M\"unchen,}
    \tiny{Theresienstr. 39, 80333 M\"unchen, Germany}}
	 ~and
Felix Finster\thanks{felix.finster@mathematik.uni-regensburg.de \\     \tiny{Universität Regensburg, Universitätsstraße 31, 93053 Regensburg, Germany}}
}
\date{\today}



\maketitle

\begin{abstract}
to be written
\end{abstract}

\section{Introduction}
This paper compares central objects of three different approaches to quantum field theory (QFT).
These approaches have different scopes and pursue different ideas, so a comparison of the theories as a whole is difficult.
Since they all study different aspects of QFT; however, one can make out objects in each of these theories which closely resemble
one another. The first class of objects of interest are Hadamard states which appear in the algebraic approach to 
QFT\cite{fulling1978singularity}. The second class of objects are the (almost) projectors \(P^\lambda_{\Sigma}\) which are 
closely linked to polarisation classes of the vacuum of external field quantum electrodynamics (QED). 
Polarisation classes play a central role in the analysis of external field QED\cite{ivp0, ivp1, ivp2}.
The third class of objects are the continuum limit of the Fermionic Projectors\cite{something}. 
In the following sections we will very briefly summarise the role of each of the targeted objects in their respective approach to QFT. 
In these sections we want to focus on the similar physical interpretation of these objects.
A reader familiar with some of these theories may skip the respective sections. 
Afterwards we shift our attention to the similarities between the mathematical structure inherent in these objects.
Throughout the paper
\(\Sigma, \Sigma', \Sigma''\) denote arbitrary Cauchy surfaces, while \(A\in C_c^\infty(\mathbb{R}^4,\mathbb{R}^4)\) 
is a four potential and \(\Sigma_{\text{in}}\) denotes a Cauchy surface earlier than \(\text{supp}(A)\).

%some introductory math needs to be written also we should mention that we are concerned with the Dirac field

\begin{equation}
(i\slashed{\partial} - m - \slashed{A})\psi = 0
\end{equation}

\begin{equation}
U_{\Sigma',\Sigma}:\mathcal{H}_\Sigma \rightarrow \mathcal{H}_{\Sigma'},
\end{equation}

\section{Hadamard States}
In the algebraic approach to QFT one puts less emphasise on the Hilbert space as is commonly done in non relativistic physics, because 
once one talks about quantum fields in a relativistic setting the specific representation of operators by operators on a Hilbert 
space is no longer canonical. 
In stead one tries to focus on the operators that represent the physical measurements one wants to perform later on. 
One can construct an algebra which contains these operators and do most of the analysis on this algebra itself; however, 
many construction common to that analysis already make use of special properties of special states. The so called Hadamard states 
(We will introduce them shortly) often play a role in constructing operators that represent e.g. objects like the energy 
momentum tensor of a quantum field. Additionally at some point in
the physical analysis one would like to produce some numbers which can be compared to experiments. In order to do so, one
has to pick a representation. The GNS construction is very helpful in this regard, since it reduces the construction of a representation to 
picking one state to build a Hilbert space around. This choice has to be made on physical grounds. 
The Hadamard states are usually interpreted as states of positive energy. This interpretation has to be taken with 
a grain of salt. Even in Minkowski spacetime with a non vanishing external electromagnetic field but no interaction the splitting of the
spectrum into positive and negative part becomes somewhat arbitrary, since no particular splitting is Lorentz invariant\cite{something!}.
Instead one could interpret picking a Hadamard state as making a choice of which states one
calls positive energy states. This may become clearer in section \ref{sec:pol classes}. 
Hadamard states are singled out by their dynamical properties. The definition of Hadamard states is a technical one, so we will omit
the general case and focus on the condition for Fermions in flat Minkowski spacetime subject to an external potential. This scenario was
already considered by Dirac \cite{Dirac34}. The condition in this case reads as follows: 

\begin{de}
A state \(\omega\) is called Hadamard if its two point function \(w\in (C_c^\infty(\mathcal{M})\times C_c^\infty(\mathcal{M}))'\) 
acts for \(f_1,f_2\in C_c^\infty(\mathcal{M})\) as 
\begin{equation}
w(f_1,f_2)=\lim_{\varepsilon\searrow 0} \int_{\mathbb{R}^4}d^4 x \overline{f_1}(x) \int_{\mathbb{R}^4} d^4y ~b_\varepsilon(x,y) f_2(y)
\end{equation}
with \(b_\varepsilon\) being of the form

\begin{align}\nonumber
b_\varepsilon (x,y)=-i(i\slashed{\partial}_x -\slashed{A}(x)-m)
&\Big[\frac{e^{-i (x-y)^\alpha\int_0^1 ds A_\alpha (x s + (1-s)y)}}{(x-y-i \varepsilon e_0)^2}\\\label{def:hadamard}
&+V(x,y) \ln ((x-y-i \varepsilon e_0)^2)\Big]+W(x,y).
\end{align}
In this equation \(V,W:\mathbb{R}^4\rightarrow \mathbb{C}^4\) are smooth functions, \(W\) is completely arbitrary and varies for
different Hadamard states, whereas \(V\) is fixed by the external potential. For \((x-y)^2\) small enough the series
\begin{equation}
V(x,y)=\sum_{k=0}^\infty \frac{1}{2^{k-1}k!} V_k(x,y) (x-y)^{2k}
\end{equation}
converges.\cite{something...}  The functions \(V_k\) fulfil a recursive set of partial differential equations
\begin{align}\notag
&\left((\partial_{x}^{\alpha}-i A^\alpha(x))(\partial_{x,\alpha}-i A_\alpha(x))+m^2\right) V_{n-1}(x,y)\\\label{Hadamard recursive equ.}
 &\hspace{3cm}+ (x-y)^\alpha (\partial_{x,\alpha}-i A_\alpha(x)) V_{n}(x,y) + (2n+2) V_{n}(x,y)=0,
\end{align}
where \(V_{-1}(x,y)=e^{-i (x-y)^\alpha\int_0^1 ds A_\alpha (x s + (1-s)y)}\).
\end{de}


%note to self: should I mention that fields are operator valued distributions? 

%write that one needs to restrict to Hadamard states for the perturbative expansion of interacting QFT (cite this wald overview paper, p.15)

%give some reply to the standard argument that notions of particles are non unique in QFT in curved spacetimes. 



\section{Projectors for Polarisation Classes}\label{sec:pol classes}

The concept of polarisation classes arises naturally in the study of QED in external electromagnetic fields. It does need some
machinery to be introduced and related to more familiar objects which we are going to introduce first. 

We start out by characterising the polarisation classes for free motion.
\begin{de}
The projector \(P^{\mathcal{H}^-}_{\Sigma} \) has the well known representation as the weak limit of the integral operator 
with the kernel\cite{ivp2}

\begin{equation}
p^-_\varepsilon(x,y)= -\frac{m^2}{4\pi^2}(i\slashed{\partial}_x+m) \frac{K_1(m \sqrt{-(y-x-i \varepsilon e_0)^2})}{m \sqrt{-(y-x-i \varepsilon e_0)^2}},
\end{equation}
where the square is a Minkowski square and the square root denotes its principle value. By weak limit we mean

\begin{align}
\langle \phi, P^{\mathcal{H}^-}_{\Sigma} \psi\rangle = \lim_{\varepsilon \searrow 0} \int_{\Sigma\times \Sigma} \overline{\phi}(x) i_\gamma(d^4x) p^-_\varepsilon(x,y ) i_\gamma(d^4y) \psi(y),
\end{align}

for general \(\phi, \psi \in \mathcal{H}_\Sigma\).
\end{de}

\begin{rmk}
By inserting the expansion of \(K_1\) in terms of a Laurent series and a logarithm, \cite{abramowitz1965handbook} one obtains:

\begin{equation}\label{K1series}
K_1(\xi) = \frac{1}{\xi}- \frac{\xi}{4} \sum_{k=0}^\infty \left(2 \psi(k+1)+\frac{1}{k+1}-2 \ln\frac{ \xi}{2} \right) \frac{\left(\frac{\xi^2}{4}\right)^k}{k!^2 (k+1)}.
\end{equation}

It is not obvious from the equation \eqref{Hadamard recursive equ.} but in fact well known that the
 vacuum of Minkowski spacetime does indeed
correspond to  a Hadamard state subject to vanishing four potential. In fact, the Hadamard states were constructed to 
agree with the Minkowski vacuum up to smooth terms.
\end{rmk}




\begin{de}
Let \(\text{Pol} (\mathcal{H}_\Sigma)\) denote the set of all closed, linear subspaces \(V\subset \mathcal{H}\)
such that both \(V\) and \(V^\perp\) are infinite dimensional. Any \(V\in \text{Pol}(\mathcal{H}_\Sigma)\) is called 
\emph{polarisation} of \(\mathcal{H}_\Sigma\). For \(V\in \text{Pol}(\mathcal{H}_\Sigma)\), let \(P_\Sigma^V\rightarrow V\) 
denote the orthogonal projection of \(\mathcal{H}_\Sigma\) onto \(V\).
The Fock space corresponding to \(V\) on the Cauchy surface \(\Sigma\) is defined to be
\begin{equation}
\mathcal{F}(V,\mathcal{H}_\Sigma) := \bigoplus_{c\in\mathbb{Z}} \mathcal{F}_c (V,\mathcal{H}_\Sigma), \quad 
\mathcal{F}_c(V,\mathcal{H}_\Sigma):= \bigoplus_{\overset{n,m\in\mathbb{N}_0}{c=m-n}}(V^\perp)^{\wedge n} \otimes \overline{V}^{\wedge m},
\end{equation}

where \(\bigoplus\) is the Hilbert space direct sum, \(\wedge\) the antisymmetric tensor product of Hilbert spaces and 
\(\overline{V}\) is the conjugate complex vector space of \(V\), which coincides with \(V\) as a set, has the same vector 
space operations as \(V\) except for scalar multiplication, which is defined by \((z,\psi)\mapsto z^* \psi\) for 
\(z\in\mathbb{C},\psi\in V\).
\end{de}

\begin{rmk}\label{WhyUseApprox}
Given two polarisations \(V,W\in \text{Pol}(\mathcal{H}_\Sigma)\), the two Fockspaces \(\mathcal{F}(V,\mathcal{H}_\Sigma)\) and 
\(\mathcal{F}(W,\mathcal{H}_\Sigma)\) are canonically unitary equivalent if \textcolor{red}{does also \(\iff\)  hold??} 
\(P^V_\Sigma - P^W_\Sigma \in I_2(\mathcal{H}_\Sigma)\) by the theorem of Shale and Stinespring 
\cite{Shale Stinespring 1965},.
\end{rmk}
Remark \ref{WhyUseApprox} gives us a natural limit with 
respect to which it is useful
to analyse the regularity of Projectors \(P_\Sigma^A\). We will therefore be content with the following equivalence class.
\begin{de}
For \(V,W\in \text{Pol}(\mathcal{H}_\Sigma)\) we write
\begin{equation}
V\approx W \iff P^V_\Sigma - P^W_\Sigma \in I_2(\mathcal{H}_\Sigma),
\end{equation}
\begin{equation}
C_\Sigma (A):= [ U_{\Sigma \Sigma_{\text{in}}}^A \mathcal{H}_{\Sigma_{\text{in}}}^-]_\approx.
\end{equation}
\end{de}


The equivalence class \(C_\Sigma(A)\) transforms naturally with respect to gauge and Lorentz transforms\cite{ivp2}. 
Now for hyperplanes \(\Sigma\cap \text{supp}(A)\neq \emptyset\) the operator \(P^{\mathcal{H}^-}_{\Sigma} \) does not represent
\(C_\Sigma(A)\). Because all we are interested in is representations of equivalence classes, we are content with finding
objects that differ from a Projector onto \(U_{\Sigma,\Sigma_{\text{in}}}^A \mathcal{H}_{\Sigma_in}^-\) by a Hilbert-Schmidt operator.
Therefore we need not keep track of the exact evolution of the projection operators, but define a whole class of admissible ones.

\begin{de}
For any function \(\lambda\in C^\infty (\mathbb{R}^4\times \mathbb{R}^4, \mathbb{R})\) fulfilling
\begin{enumerate}[label= \roman*) ]
\item There is a compact set \(K\subset \mathbb{R}^4\) such that \(\supp \lambda \subseteq K\times \mathbb{R}^4 \cup \mathbb{R}^4 \times K\).
\item \(\lambda\) satisfies \(\forall x \in \mathbb{R}^4 \lambda(x,x)=0\).
\item On the diagonal the first derivatives fulfil 
\begin{equation}
\forall x,y\in\mathbb{R}^4: \partial_x \lambda(x,y)=-\partial_y \lambda(x,y) = A(x),
\end{equation}
\end{enumerate}
we define a corresponding (quasi) projector \(P^\lambda\):
\begin{align}
\langle \phi, P^\lambda_\Sigma \psi\rangle =& \lim_{\varepsilon \searrow 0} \langle \phi, P^{A,\varepsilon}_\Sigma \psi\rangle ,\\
\langle \phi, P^{\lambda,\varepsilon}_\Sigma \psi\rangle :=&
\int_{\Sigma\times\Sigma}\overline{\phi}(x)i_\gamma(d^4x) 
e^{-i \lambda(x,y)} p^-_\varepsilon(y-x)i_\gamma(d^4y) \psi(y),
\end{align}
for general \(\phi, \psi \in \mathcal{H}_\Sigma\).
\end{de}

\begin{rmk}\label{main results of ivp2}
\(P_{\Sigma}^\lambda\) and \( P_{\Sigma'}^\lambda\) are equivalent if transported appropriately by time evolution operators\cite[theorem 2.8]{ivp2}:

\begin{equation}
P_{\Sigma}^\lambda-U_{\Sigma,\Sigma'}^A P_{\Sigma'}^\lambda U_{\Sigma',\Sigma}\in I_2(\mathcal{H}_\Sigma).
\end{equation}
Also for four-potentials \(A,B\in C_c^\infty(\mathbb{R}^4)\) 
the corresponding projectors are typically inequivalent\cite[theorem 1.5]{ivp2}

\begin{equation}
P_{\Sigma}^{\lambda^A} - P_{\Sigma}^{\lambda^B} \in I_2(\mathcal{H}_\Sigma) \iff \forall x\in \Sigma~ \forall z\in T_x \Sigma: z^\alpha(A_\alpha(x)-B_\alpha(x))=0.
\end{equation}
\end{rmk}

Now the notion of Hadamard state and  projectors of polarisation classes are extremely close.

\begin{thm}
Given a global Hadamard state \(H\) there is a smooth function  \(w\in C^\infty(\mathbb{R}^4\times \mathbb{R}^4)\) such that
for any hypersurface \(\Sigma\), \(H-w\) can be interpreted as linear bounded map\(:\mathcal{H}_\Sigma\rightarrow \mathcal{H}_\Sigma\).
Furthermore for each \(\lambda\) fulfilling \(i)-iii)\), \(H-w\) and \(P^\lambda\) yield equivalent descriptions:
\begin{equation}
H-w-P^\lambda \in I_2(\mathcal{H}_\Sigma).
\end{equation}


\end{thm}




\section{The Fermionic Projector}




\bibliographystyle{plain}
\bibliography{ref}

\end{document}

