\documentclass[a4paper,12pt]{article}

\usepackage{german}

\usepackage{graphicx}

\usepackage{amssymb}

\usepackage{amsfonts}

\usepackage{amsmath}

\usepackage{amsthm}

\usepackage{slashed}

%identity sign
\usepackage{dsfont}

%commutative diagrams
\usepackage{amsmath,amscd}

\newcommand{\equaltext}[1]{\ensuremath{\stackrel{\text{#1}}{=}}}
\newcommand{\equalmath}[1]{\ensuremath{\stackrel{#1}{=}}}


\addtolength{\textwidth}{2.2cm} \addtolength{\hoffset}{-1.0cm}

\addtolength{\textheight}{3.0cm} \addtolength{\voffset}{-2cm} 

\parindent 0cm

\pagestyle{empty}



\begin{document}
\begin{center}
{\Large What we know about \(T1\)}
\end{center}
 
 \(T1\) is an unbounded linear map from  the Fock space of QED onto itself. To be more specific, let \(\mathcal{H}_+\) denote the Hilbert space of electron wavefunctions and \(\mathcal{H}_-\) denote the Hilbert space of positron wavefunctions. Then
 %add details of how these hilberspaces look!
\begin{equation}
T1: \mathcal{F}\left( \mathcal{H}_+\right)\otimes\mathcal{F}\left( \mathcal{H}_-\right) \rightarrow \mathcal{F}\left( \mathcal{H}_+\right)\otimes\mathcal{F}\left( \mathcal{H}_-\right) ,
\end{equation}

but by far not all matrix elements are nonzero, expressed in the fixed particle subspaces we already see more structure:
%later add the comptation proving this
\begin{equation}
T1:  \mathcal{H}_+^{\otimes n}  \otimes \mathcal{H}_-^{\otimes p} \rightarrow\left(\mathcal{H}_+^{\otimes n-1}  \otimes \mathcal{H}_-^{\otimes p-1}\right) \oplus \left(\mathcal{H}_+^{\otimes n}  \otimes \mathcal{H}_-^{\otimes p}\right) \oplus \left(\mathcal{H}_+^{\otimes n +1}  \otimes \mathcal{H}_-^{\otimes p+1}\right)  .
\end{equation}

So restricted to the vacuum sector this simplifies, the part from the vacuum sector is set to zero, because the vacuum polarization current should be zero.
\begin{equation}
T1:  \mathbb{C} \rightarrow  \left(\mathcal{H}_+ \otimes \mathcal{H}_-\right)
\end{equation}

We will construct the finite operator norm of \(T1\) restricted to an arbitrary fixed particle sector, beginning with the vacuum sector. Let the vector potential of the electromagnetic field \(A\) be
\begin{equation}
A\in C^\infty_c\left(\mathbb{R}^4\right).
\end{equation}
%
%beware of physicists notation -> replace this at some point!
%
Then one obtains in physicists notation:
\begin{equation}
T1 \Omega = \frac{1}{(2\pi)^3}  \int_{\mathbb{R}^3} \frac{\text{d}^3k}{\sqrt{2 k^0}} \int_{\mathbb{R}^3} \frac{\text{d}^3p}{\sqrt{2 p^0}} \int_{\mathbb{R}^4}\text{d}^4x\sum_{r,s = \pm1} b^\dagger_s (k) d^\dagger_r (p) \bar{u_s}(k) \gamma^\mu v_r (p) \Omega e^{i x_\alpha (k^\alpha + p^\alpha)} A_{\mu} (x),
\end{equation}
Where the \(\gamma^\mu\) are the dirac matrices fulfilling the anticommutation relation 
\begin{equation}
\{\gamma^\alpha \gamma^\beta \} = 2 \eta^{\alpha \beta}
\end{equation}
The \(u , v\) are given by:
\begin{equation}
u_s(p)=\begin{pmatrix}
\sqrt{p \cdot \sigma} \xi_s \\ \sqrt{p \cdot \bar{\sigma}} \xi_s
\end{pmatrix} 
\hspace{2cm}
v_s(p)=\begin{pmatrix}
\sqrt{p \cdot \sigma} \xi_{-s} \\ -\sqrt{p \cdot \bar{\sigma}} \xi_{-s}
\end{pmatrix} ,
\end{equation}
with 
\begin{equation}
\sigma=(\mathds{1} , \vec{\sigma}) , \hspace{2cm} \bar{\sigma}=(\mathds{1} , -\vec{\sigma}) 
\end{equation}
\begin{equation}
\xi_1=\begin{pmatrix}
1\\0
\end{pmatrix}
\hspace{2cm}
\xi_{-1}=\begin{pmatrix}
0\\1
\end{pmatrix}
\end{equation}
Where \(\vec{\sigma}\) is the vector containing the Pauli matrices. The creation and annihilation operator fulfill the following identity:
\begin{equation}
\{ b_s(p),b^\dagger_r (p')\}=\delta^3(p-p')=\{ d_s(p),d^\dagger_r (p')\}
\end{equation}
%relationship between proper annihilation operator a and b, d

Useful identities for sums of \(u\) and \(v\) are:
\begin{equation}
\sum_s u_s(p)\bar{u}_s(p)=\slashed{p}+m \hspace{2cm} \sum_s v_s(p)\bar{v}_s(p)=\slashed{p}-m
\end{equation}

For the norm of the image of the vacuum we therefore arrive at:
\begin{multline}
\| T1 \Omega\|^2= \frac{1}{(2\pi)^6}  \int_{\mathbb{R}^3} \frac{\text{d}^3k}{\sqrt{2 k^0}} \int_{\mathbb{R}^3} \frac{\text{d}^3p}{\sqrt{2 p^0}} 
 \int_{\mathbb{R}^3} \frac{\text{d}^3k'}{\sqrt{2 k'^0}} \int_{\mathbb{R}^3} \frac{\text{d}^3p'}{\sqrt{2 p'^0}} \int_{\mathbb{R}^4}\text{d}^4x \int_{\mathbb{R}^4}\text{d}^4y
\sum_{r,s,r',s' = \pm1} \\
\Omega^\dagger 
 d_{r'} (p') b_{s'} (k')  \bar{v}_{r'} (p') \gamma^\epsilon u_{s'}(k') 
 b^\dagger_s (k) d^\dagger_r (p) \bar{u_s}(k) \gamma^\mu v_r (p) 
\Omega  \\
 e^{i x_\alpha (k^\alpha + p^\alpha)} e^{-i y_\beta (k'^\beta + p'^\beta)} 
 A_{\mu} (x) A_{\epsilon} (y)  \\
 =\frac{1}{(2\pi)^6} \int_{\mathbb{R}^3} \frac{\text{d}^3k}{\sqrt{2 k^0}} \int_{\mathbb{R}^3} \frac{\text{d}^3p}{\sqrt{2 p^0}} 
 \int_{\mathbb{R}^3} \frac{\text{d}^3k'}{\sqrt{2 k'^0}} \int_{\mathbb{R}^3} \frac{\text{d}^3p'}{\sqrt{2 p'^0}} \int_{\mathbb{R}^4}\text{d}^4x \int_{\mathbb{R}^4}\text{d}^4y
\sum_{r,s,r',s' = \pm1} \\
\Omega^\dagger 
\left\{ d_{r'} (p') \left\{b_{s'} (k'), b^\dagger_s (k)\right\}, d^\dagger_r (p)\right\} \bar{v}_{r'} (p') \gamma^\epsilon u_{s'}(k') 
  \bar{u_s}(k) \gamma^\mu v_r (p) 
\Omega  \\
 e^{i x_\alpha (k^\alpha + p^\alpha)} e^{-i y_\beta (k'^\beta + p'^\beta)} 
 A_{\mu} (x) A_{\epsilon} (y)  \\
 =\frac{1}{(2\pi)^6}  \int_{\mathbb{R}^3} \frac{\text{d}^3k}{2 k^0} \int_{\mathbb{R}^3} \frac{\text{d}^3p}{2 p^0} \int_{\mathbb{R}^4}\text{d}^4x\int_{\mathbb{R}^4}\text{d}^4y
 \sum_{r,s = \pm1} 
 \bar{v}_{r} (p) \gamma^\epsilon u_{s}(k) 
  \bar{u_s}(k) \gamma^\mu v_r (p)  \\
 e^{i x_\alpha (k^\alpha + p^\alpha)} e^{-i y_\beta (k^\beta + p^\beta)} 
 A_{\mu} (x) A_{\epsilon} (y)  \\
  =\frac{1}{(2\pi)^6} \int_{\mathbb{R}^3} \frac{\text{d}^3k}{2 k^0} \int_{\mathbb{R}^3} \frac{\text{d}^3p}{2 p^0} \int_{\mathbb{R}^4}\text{d}^4x\int_{\mathbb{R}^4}\text{d}^4y 
 \sum_{r,s = \pm1} 
 \sum_{a=1}^4 \left<e_a\right| v_r (p) \bar{v}_{r} (p) \gamma^\epsilon u_{s}(k) 
  \bar{u_s}(k) \gamma^\mu \left|e_a\right>  \\
  e^{i x_\alpha (k^\alpha + p^\alpha)} e^{-i y_\beta (k^\beta + p^\beta)} 
 A_{\mu} (x) A_{\epsilon} (y)  \\
=\frac{1}{(2\pi)^6}  \int_{\mathbb{R}^3} \frac{\text{d}^3k}{2 k^0} \int_{\mathbb{R}^3} \frac{\text{d}^3p}{2 p^0} \int_{\mathbb{R}^4}\text{d}^4x\int_{\mathbb{R}^4} \text{d}^4y 
 \hspace{0,2cm}\text{tr}\left[ (\slashed{p}-m) \gamma^\epsilon (\slashed{k}+m) \gamma^\mu \right]  \\
 e^{i x_\alpha (k^\alpha + p^\alpha)} e^{-i y_\beta (k^\beta + p^\beta)} 
 A_{\mu} (x) A_{\epsilon} (y)  \\
 =\frac{1}{(2\pi)^2}  \int_{\mathbb{R}^3} \frac{\text{d}^3k}{2 k^0} \int_{\mathbb{R}^3} \frac{\text{d}^3p}{2 p^0}
 \left(p_\zeta k_\omega\text{tr}\left[ \gamma^\zeta \gamma^\epsilon \gamma^\omega \gamma^\mu \right] -m^2\text{tr}\left[ \gamma^\epsilon \gamma^\mu \right] \right) \\
 \hat{A}_{\mu} (k+p) \hat{A}^*_{\epsilon} (k+p)  \\
  =\frac{1}{(2\pi)^2}  \int_{\mathbb{R}^3} \frac{\text{d}^3k}{2 k^0} \int_{\mathbb{R}^3} \frac{\text{d}^3p}{2 p^0}4
 \left(p_\zeta k_\omega(\eta^{\zeta,\epsilon}\eta^{\omega,\mu} -\eta^{\zeta,\omega}\eta^{\epsilon,\mu} +\eta^{\zeta,\mu}\eta^{\epsilon,\omega}) -m^2\eta^{\epsilon,\mu} \right) \\
 \hat{A}_{\mu} (k+p) \hat{A}^*_{\epsilon} (k+p)  \\
   =\frac{1}{(2\pi)^2}  \int_{\mathbb{R}^3} \frac{\text{d}^3k}{ k^0} \int_{\mathbb{R}^3} \frac{\text{d}^3p}{ p^0}
 \left(k^\mu \hat{A}_{\mu} (k+p) p^\epsilon\hat{A}^*_{\epsilon} (k+p)  -p^\alpha k_\alpha \hat{A}^{\mu} (k+p) \hat{A}^*_{\mu} (k+p)\right.\\
 \left.    +k^\mu \hat{A}^*_{\mu} (k+p) p^\epsilon\hat{A}_{\epsilon} (k+p)
  -m^2\hat{A}^{\mu} (k+p) \hat{A}^*_{\mu} (k+p)  \right) \\
  \equalmath{k^2=p^2=m^2} \frac{1}{(2\pi)^2}  \int_{\mathbb{R}^3} \frac{\text{d}^3k}{ k^0} \int_{\mathbb{R}^3} \frac{\text{d}^3p}{ p^0}
 \left(k^\mu \hat{A}_{\mu} (k+p) p^\epsilon\hat{A}^*_{\epsilon} (k+p) \right.\\
 \left.    +k^\mu \hat{A}^*_{\mu} (k+p) p^\epsilon\hat{A}_{\epsilon} (k+p)
  -\frac{1}{2}(k+p)^2 \hat{A}^{\mu} (k+p) \hat{A}^*_{\mu} (k+p)  \right) \\
    = \frac{1}{(2\pi)^2}  \int_{\mathbb{R}^3} \frac{\text{d}^3k}{ k^0} \int_{\mathbb{R}^3} \frac{\text{d}^3p}{ p^0}
 \left(k^\mu \hat{A}_{\mu} (k+p) p^\epsilon\hat{A}^*_{\epsilon} (k+p) \right.\\
 \left.    +k^\mu \hat{A}^*_{\mu} (k+p) p^\epsilon\hat{A}_{\epsilon} (k+p)
  +\frac{1}{2} \hat{A}^*_{\mu} (k+p)  \widehat{\Box A}^{\mu} (k+p) \right) 
\end{multline}
Up until this point the calculation is exact. If one wants to obtain a nice looking expression for this bound one can add the terms \( k^\alpha \hat{A}_\alpha  k^\beta \hat{A}^*_\beta +p^\alpha  \hat{A}_\alpha  p^\beta \hat{A}^*_\beta \). These terms are clearly positive. One then ends up with:

\begin{multline}
\| T1 \Omega\|^2\le \frac{1}{(2\pi)^2}  \int_{\mathbb{R}^3} \frac{\text{d}^3k}{ k^0} \int_{\mathbb{R}^3} \frac{\text{d}^3p}{ p^0}
 \left( (k^\mu+p^\mu) \hat{A}^*_{\mu} (k+p) (k^\epsilon+p^\epsilon)\hat{A}_{\epsilon} (k+p)\right. \\
 \left. +\frac{1}{2} \hat{A}^*_{\mu} (k+p)  \widehat{\Box A}^{\mu} (k+p) \right) \\
 =\frac{1}{(2\pi)^2}  \int_{\mathbb{R}^3} \frac{\text{d}^3k}{ k^0} \int_{\mathbb{R}^3} \frac{\text{d}^3p}{ p^0}
 \left( \widehat{\partial^\alpha A_\alpha}^*(k+p) \widehat{\partial^\beta A_\beta}(k+p)\right. \\
 \left. +\frac{1}{2} \hat{A}^*_{\mu} (k+p)  \widehat{\Box A}^{\mu} (k+p) \right)\\
  =\frac{1}{(2\pi)^2}  \int_{\mathbb{R}^3} \frac{\text{d}^3k}{ k^0} \int_{\mathbb{R}^3} \frac{\text{d}^3p}{ p^0}
 \left(\left\|\widehat{\partial^\beta A_\beta}(k+p)\right\|^2
 +\frac{1}{2} \hat{A}^*_{\mu} (k+p)  \widehat{\Box A}^{\mu} (k+p) \right)
\end{multline}

Taking \(T1 : \mathbb{C} \rightarrow \mathcal{H}_+  \otimes \mathcal{H}_-\) for granted, we would like to define \(T1\) on all of Fockspace. In order to do this, we introduce the ``restriction of proper lifting'' for the annihilation operator:
\begin{equation}\label{lift_condition}
\forall \phi\in \mathcal{H}: \hspace{0.5cm} a\left( U \phi \right)  \circ \tilde{U}=\tilde{U} \circ a(\phi)
\end{equation}
Which is equivalent to the commutativity of the following diagram.
\begin{equation}
\begin{CD}								%heuristics with infinite wedge space?
\mathcal{F}     @>\tilde{U}^A>>  \mathcal{F}\\
@AAaA        @AAaA\\
\mathcal{H}\otimes \mathcal{F}     @>U^A\otimes \tilde{U}^A>>  \mathcal{H}\otimes \mathcal{F} 
\end{CD}
\end{equation}
The respective condition for the creation operator, which can easily be derived from \eqref{lift_condition} is:
\begin{equation}
\forall \phi\in \mathcal{H}: \hspace{0.5cm} a^*\left( U^A \phi \right)  \circ \tilde{U}^A=\tilde{U}^A \circ a^*(\phi)
\end{equation}
Expanding \(U^A\) and \(\tilde{U}^A\) in a powerseries, one obtains the following commutation relations for the coefficients of said expansion:
\begin{multline}
U^A=\mathds{1}_\mathcal{H}+\sum_{l=1}^\infty
\end{multline}

\end{document}

