\documentclass[oneside,reqno,12pt]{amsart}

%\usepackage{fontspec}

\usepackage[a4paper, top=2.7cm, bottom=2.7cm]{geometry}
%\usepackage[T1]{fontenc}
%\usepackage[utf8]{inputenc}
\usepackage{fontspec}
%\setmainfont{YuMincho}
%Hiragino Maru Gothic ProN
\usepackage{bbm}
\usepackage{graphicx}
\usepackage{slashed}
\usepackage{eurosym}
\usepackage{amsmath}
\usepackage{enumitem}
\usepackage{amsfonts}
\usepackage{longtable}
\usepackage[mathscr]{eucal}

\setcounter{secnumdepth}{5}

%commutative diagram
\usepackage{amsmath,amscd}
%picture
\usepackage{wrapfig}

\usepackage[unicode=true, pdfusetitle, bookmarks=true,
  bookmarksnumbered=false, bookmarksopen=false, breaklinks=true, 
  pdfborder={0 0 0}, backref=false, colorlinks=true, linkcolor=blue,
  citecolor=blue, urlcolor=blue]{hyperref}



% \numberwithin{equation}{section}
\allowdisplaybreaks[1]

\newtheorem{axiom}{Axiom}
\newtheorem{Def}{Definition}[section]
\newtheorem{Conj}[Def]{Conjecture}
\newtheorem{Thm}[Def]{Theorem}
\newtheorem{Prp}[Def]{Proposition}
\newtheorem{Lemma}[Def]{Lemma}
\newtheorem{lemma}{Lemma}
\newtheorem{Remark}[Def]{Remark}
\newtheorem{Corollary}[Def]{Corollary}
\newtheorem{Example}[Def]{Example}
\newtheorem{Assumption}[Def]{Assumption}

\newenvironment{mueq}
  {\equation\aligned}
  {\endaligned\endequation}
  
\DeclareMathOperator{\tr}{tr}
\DeclareMathOperator{\supp}{supp}


\newcommand{\Z}[2]{Z_{\stackrel{1}{#1}}\left(#2\right)}
\newcommand{\id}{{\mathbbm 1}}
\newcommand{\equaltext}[1]{\ensuremath{\stackrel{\text{#1}}{=}}}
\newcommand{\letext}[1]{\ensuremath{\stackrel{\text{#1}}{\le}}}
\newcommand{\Conv}{\mathop{\scalebox{1.7}{\raisebox{-0.2ex}{\(\ast\)}}}}
\newcommand{\CONV}{\mathop{\scalebox{3.0}{\raisebox{-0.2ex}{\(\ast\)}}}}
% Annotations
%\usepackage[normalem]{ulem}
% \usepackage{refcheck}
\usepackage[colorinlistoftodos,shadow,textsize=scriptsize,textwidth=2.75cm]{todonotes}
\newcommand{\Dirk}[1]{ \todo[color=orange!60]{Dirk: #1} }
\newcommand{\DirkBox}[1]{ \mbox{}\todo[inline,caption={},color=red!60]{Dirk: #1} }
\newcommand{\Markus}[1]{ \todo[color=green!20]{Markus: #1} }
\newcommand{\dirk}{ \color{orange} }
\newcommand{\markus}{ \color{green} }
\newcommand{\noch}[1]{ \todo[color=blue!20]{Todo: #1} }
\newcommand{\black}{ \color{black} }

\makeatletter



\renewcommand\section{\@startsection {section}{1}{\z@}%
                                   {-2.0ex \@plus -1ex \@minus -.2ex}%
                                   {2.3ex \@plus.2ex}%
                                   {\normalfont\Large\bfseries}}
\renewcommand\subsection{\@startsection {subsection}{1}{\z@}%
                                   {-0.5ex \@plus -0.5ex \@minus -.2ex}%
                                   {0.5em}%
                                   {\normalfont\bfseries}}
\renewcommand\subsubsection{\@startsection {subsubsection}{1}{\z@}%
                                   {-0.3ex \@plus -0.4ex \@minus -.2ex}%
                                   {0.1 em}%
                                   {\normalfont\sc}}  
\renewcommand\paragraph{\@startsection {paragraph}{1}{\z@}%
                                   {-0.2ex \@plus -1ex \@minus -.2ex}%
                                   {0.1 em}%
                                   {\normalfont\it}}                                   
\makeatother

\parindent 0cm

\begin{document}
In This document I do the necessary calculations to arrive at the vacuum expectation value of \(T_4\).The calculation is organised as follows: The main calculation is done in section \ref{sec::main}, in each step there is a reference to a section where the step is done in detail.
We use the definitions of the retarded \(R_n\) and advanced operators \(A_n\) of Scharf (3.1.35-36) for any \(n\in\mathbb{N}\). These are adjusted to the notation of distributions which take test functions as arguments instead of points in space time. Furthermore we evaluate these Operators on two functions only, \(f,g\in C_c^\infty \left(\mathbb{R}^4\right)\otimes \mathbb{R}^4\). The test function \(g\) may only appear once, whereas \(f\) may appear multiple times as an argument of either of \(A_n\) or \(R_n\). Due to this there appear combinatorical factors in the definitions of the advanced resp. retarded operator. 
\section{main}\label{sec::main}
Let \(f,g\) be test functions, we will use the same symbol for their Fourier transform, since we only ever evaluate their Fourier transform for specific arguments no confusion should arise. 

The following operators will come in handy later on.
I denote by \(Q\) the following set \(Q:=\{f: \mathcal{H}\rightarrow \mathcal{H} \text{ linear} \mid i \cdot f \text{ is selfadjoint}\}\).
\begin{Def}
 Let then \(G\) be the following function
\begin{align*}\tag{Def G} \label{Def G}
G: \quad & Q\rightarrow \left( \mathcal{F}\rightarrow \mathcal{F} \right)\\
& f\mapsto \sum_{n\in\mathbb{N}}a^*(f \varphi_n) a(\varphi) - \sum_{n\in -\mathbb{N}} a(\varphi_n) a^*(f \varphi_n).
\end{align*}
\end{Def}

Furthermore we suppress multiple appearances of \(f\) as an argument of a distribution, the total number of arguments of any distribution should always be clear from its index, \(A_3\) as well as \(R_3\) have four arguments, \(T_k\) as well as \(Z_k\) have \(k\) arguments for each \(k\). For \(n=3\) these definitions are
\begin{align}
&A_n(g,f)= \sum_{k=0}^n \begin{pmatrix}n\\k\end{pmatrix} T^\dagger_{n-k}(f) T_{k+1}(f,g)\\
&R_n(g,f)= \sum_{k=0}^n \begin{pmatrix}n\\k\end{pmatrix}  T_{k+1}(f,g) T^\dagger_{n-k}(f)
.\end{align}
These yield
\begin{align}
&A_3(g,f)= T_4(f,g)+3 T_1^\dagger(f)T_3(f,g)+3T_2^\dagger(f)T_2(f,g)+T_3^\dagger(f)T_1(g)\\
&R_3(g,f)= T_4(f,g)+3 T_3(f,g)T_1^\dagger(f)+3T_2(f,g)T_2^\dagger(f)+T_1(g)T_3^\dagger(f)
.\end{align}
Following the induction step described by Scharf we compute \(A_3-R_3\), we find that
\begin{equation}\label{commutator_first}
A_3-R_3=3 \left[T_1^\dagger(f),T_3(f,g) \right]+3\left[T_2^\dagger(f),T_2(f,g)\right]+\left[T_3^\dagger(f),T_1(g)\right]
.\end{equation}
Using the explicit formulas for \(T_k\) and unitarity in section \ref{sec::ExpUnit} we arrive at

\begin{align}\tag*{}
&A_3(f,g)-R_3(f,g)=- \left[ G(P_3(f)),T_1(g)\right] 
+ 3\left[ G(P_3(f,g)),T(f)\right]\\\tag*{}
&-3\left[G(P_2(f)),G(P_2(f,g))\right]
-\frac{1}{2} \left[T_1(f),T_1(g)\right] T_1(f) T_1(f)\\\tag*{}
&+ T_1(f) \left[T_1(f),T_1(g)\right] T_1(f)
-\frac{1}{2} T_1(f)T_1(f) \left[T_1(f),T_1(g)\right]\\
&+6 \left( \langle T_2\rangle(f) -\tr \left(\Z{-+}{f}\Z{+-}{f}\right)\right) \left[T_1(f),T_1(g)\right]
\end{align}

Next we use the following 
\begin{Lemma}\label{G_kommutator}
Let \(P_k,P_l\in Q\) then the following holds
\begin{equation}
\left[G(P_k),G(P_l)\right]= 
\tr\left(P_{\stackrel{k}{-+}}P_{\stackrel{l}{+-}}\right)
-\tr\left(P_{\stackrel{l}{-+}}P_{\stackrel{k}{+-}}\right) 
+G\left(\left[P_k,P_l\right]\right)
.\end{equation}
\end{Lemma}
For a proof see section \ref{sec::G_kommutator}

This yields for our main calculation
\begin{align}\tag*{}
&A_3(f,g)-R_3(f,g)=
-  G\left(\left[P_3(f),Z_1(g)\right]\right) 
-\tr\left(P_{\stackrel{3}{-+}}(f) \Z{+-}{g}\right)\\\tag*{}
&+\tr\left(\Z{-+}{g}P_{\stackrel{3}{+-}}(f) \right)
+3 G\left(\left[P_3(f,g), Z_1(f)\right]\right)
+3 \tr\left(  P_{\stackrel{3}{-+}}(f,g) \Z{+-}{f}\right)\\\tag*{}
&-3 \tr\left(  \Z{-+}{f} P_{\stackrel{3}{+-}}(f,g) \right)
-3G\left(\left[P_2(f),P_2(f,g)\right]\right)
-3 \tr\left(P_{\stackrel{2}{-+}}(f) P_{\stackrel{2}{+-}}(f,g)\right)\\\tag*{}
&+3 \tr\left(P_{\stackrel{2}{-+}}(f,g) P_{\stackrel{2}{+-}}(f)\right)
-\frac{1}{2} G\left(\left[Z_1(f),Z_1(g)\right]\right) T_1(f) T_1(f)\\\tag*{}
&-\frac{1}{2} \tr \left(\Z{-+}{f}\Z{+-}{g}\right) T_1(f) T_1(f)
+\frac{1}{2} \tr \left(\Z{-+}{g}\Z{+-}{f}\right) T_1(f) T_1(f)\\\tag*{}
&+ T_1(f) G\left(\left[Z_1(f),Z_1(g)\right]\right) T_1(f)
+ T_1(f) \tr\left(\Z{-+}{f}\Z{+-}{g}\right) T_1(f)\\\tag*{}
&- T_1(f) \tr\left(\Z{-+}{g}\Z{+-}{f}\right) T_1(f)
-\frac{1}{2} T_1(f)T_1(f) G\left(\left[Z_1(f),Z_1(g)\right]\right)\\\tag*{}
&-\frac{1}{2} T_1(f)T_1(f) \tr\left(\Z{-+}{f} \Z{+-}{g}\right)
+\frac{1}{2} T_1(f)T_1(f) \tr\left(\Z{-+}{g} \Z{+-}{f}\right)\\\tag*{}
&+6 \left( \langle T_2\rangle(f) -\tr \left(\Z{-+}{f}\Z{+-}{f}\right)\right) G\left(\left[Z_1(f),Z_1(g)\right]\right)\\\tag*{}
&+6 \left( \langle T_2\rangle(f) -\tr \left(\Z{-+}{f}\Z{+-}{f}\right)\right) \tr\left(\Z{-+}{f} \Z{+-}{g}\right)\\
&-6 \left( \langle T_2\rangle(f) -\tr \left(\Z{-+}{f}\Z{+-}{f}\right)\right) \tr\left(\Z{-+}{g} \Z{+-}{f}\right)
.\end{align}

This expression can be slightly simplified by summing up identical terms, thus giving
\begin{align}\tag*{}
&A_3(f,g)-R_3(f,g)=
-  G\left(\left[P_3(f),Z_1(g)\right]\right) 
-\tr\left(P_{\stackrel{3}{-+}}(f) \Z{+-}{g}\right)\\\tag*{}
&+\tr\left(\Z{-+}{g}P_{\stackrel{3}{+-}}(f) \right)
+3 G\left(\left[P_3(f,g), Z_1(f)\right]\right)
+3 \tr\left(  P_{\stackrel{3}{-+}}(f,g) \Z{+-}{f}\right)\\\tag*{}
&-3 \tr\left(  \Z{-+}{f} P_{\stackrel{3}{+-}}(f,g) \right)
-3G\left(\left[P_2(f),P_2(f,g)\right]\right)
-3 \tr\left(P_{\stackrel{2}{-+}}(f) P_{\stackrel{2}{+-}}(f,g)\right)\\\tag*{}
&+3 \tr\left(P_{\stackrel{2}{-+}}(f,g) P_{\stackrel{2}{+-}}(f)\right)
-\frac{1}{2} G\left(\left[Z_1(f),Z_1(g)\right]\right) T_1(f) T_1(f)\\\tag*{}
&+ T_1(f) G\left(\left[Z_1(f),Z_1(g)\right]\right) T_1(f)
-\frac{1}{2} T_1(f)T_1(f) G\left(\left[Z_1(f),Z_1(g)\right]\right)\\\tag*{}
&+6 \left( \langle T_2\rangle(f) -\tr \left(\Z{-+}{f}\Z{+-}{f}\right)\right) G\left(\left[Z_1(f),Z_1(g)\right]\right)\\\tag*{}
&+6 \left( \langle T_2\rangle(f) -\tr \left(\Z{-+}{f}\Z{+-}{f}\right)\right) \tr\left(\Z{-+}{f} \Z{+-}{g}\right)\\\label{pre_vac_exp}
&-6 \left( \langle T_2\rangle(f) -\tr \left(\Z{-+}{f}\Z{+-}{f}\right)\right) \tr\left(\Z{-+}{g} \Z{+-}{f}\right)
.\end{align}



Next we take the vacuum expectation value yielding

\begin{align}\tag*{}
&\left<\Omega, (A_3(f,g)-R_3(f,g))\Omega \right>=\\\tag*{}
&i \Im \left( 2\tr\left(\Z{-+}{g}P_{\stackrel{3}{+-}}(f) \right)
+6 \tr\left(  P_{\stackrel{3}{-+}}(f,g) \Z{+-}{f}\right)\right.\\\tag*{}
&+6 \tr\left(P_{\stackrel{2}{-+}}(f,g) P_{\stackrel{2}{+-}}(f)\right)
- \tr\left( \sigma_{+-}\left(Z_1(f),Z_1(f)\right) \left[Z_1(f),Z_1(g)\right] \right)\\\tag*{}
&+2 \tr\left(\Z{+-}{f} \left[Z_1(f),Z_1(g)\right] \Z{-+}{f}\right)\\\tag*{}
&\left.+12 \tr \left(\Z{-+}{g}\Z{+-}{f}\right)\tr\left( \Z{-+}{f} \Z{+-}{f}\right)\right)\\\label{pre_fourier_int}
&+12 i  \langle T_2\rangle(f)\Im  \tr\left(\Z{-+}{f} \Z{+-}{g}\right)
,\end{align}
where \(\sigma\) is defined as
\begin{align*}
\sigma: \left(\mathcal{H}\rightarrow\mathcal{H}\right)\times\left(\mathcal{H}\rightarrow\mathcal{H}\right)\rightarrow \left( \mathcal{H}\rightarrow\mathcal{H}\right)\\
\left(A,B\right)\mapsto A P_+B - AP_- B.
\end{align*}
the computation is found in section \ref{sec:vac_exp}.

The next step is to convert expression \eqref{pre_fourier_int} into a sum over integrals in momentum space. This is done in section \ref{sec:p_int} giving
\begin{align}
\tag*{}
\text{to be written}
\end{align}



\section{explicit form and unitarity}\label{sec::ExpUnit}
First unitarity gives us
\begin{align*}
&T_1^\dagger=-T_1\\
&T_2^\dagger=-T_2 +2 T_1^2\\
&T_3^\dagger=-T_3 -3T_2T_1^\dagger-3T_1T_2^\dagger\\
&=-T_3 +3T_2T_1+3T_1T_2-6T_1^3
.\end{align*}
Plugged into formula \ref{commutator_first} one gets
\begin{align*}
&A_3-R_3=-3 \left[T_1(f),T_3(f,g) \right]
+3\left[-T_2(f) +2 T_1^2(f),T_2(f,g)\right]\\
&+\left[-T_3(f) +3T_2(f)T_1(f)+3T_1(f)T_2(f)-6T_1^3(f),T_1(g)\right]\\
&=-3 \left[T_1(f),T_3(f,g) \right]
-3\left[T_2(f),T_2(f,g)\right]
+6\left[ T_1^2(f),T_2(f,g)\right]
-\left[T_3(f) ,T_1(g)\right]\\
&+3\left[T_2(f)T_1(f),T_1(g)\right]
-6\left[T_1^3(f),T_1(g)\right]\\
\end{align*}
Now we employ the formula for the commutator of a product of of operators in one of the arguments of the commutator. This yields
\begin{align}\label{T1-T3_commutator1}
&A_3-R_3=-3 \left[T_1(f),T_3(f,g) \right]\\\label{T1-T3_commutator2}
&-3\left[T_2(f),T_2(f,g)\right]\\\label{T1-T3_commutator3}
&-\left[T_3(f) ,T_1(g)\right]\\\label{T1-T3_commutator4}
&+6T_1(f)\left[ T_1(f),T_2(f,g)\right]\\\label{T1-T3_commutator5}
&+6\left[ T_1(f),T_2(f,g)\right]T_1(f)\\\label{T1-T3_commutator6}
&+3T_2(f)\left[T_1(f),T_1(g)\right]\\\label{T1-T3_commutator7}
&+3\left[T_2(f),T_1(g)\right]T_1(f)\\ \label{T1-T3_commutator8}
&+3 T_1(f)\left[T_2(f),T_1(g)\right]\\ \label{T1-T3_commutator9}
&+3 \left[T_1(f),T_1(g)\right]T_2(f)\\ \label{T1-T3_commutator10}
&-6T_1^2(f)\left[T_1(f),T_1(g)\right]
-6T_1(f)\left[T_1(f),T_1(g)\right]T_1(f)
-6\left[T_1(f),T_1(g)\right]T_1^2(f)
.\end{align}
We will reduce this further by exploiting the explicit forms of the operators \(T_k\).
The first expansion coefficient of the scattering operator, \(T_1\), is then given by 
\begin{equation}
T_1(A)=G(Z_1(A)),
\end{equation}
given \(\langle T_2 \rangle \in \mathbb{C}\), the second order by 
\begin{equation}\label{expT2}
T_2=G(Z_2-Z_1Z_1)+T_1T_1 -\tr \left(Z_{\stackrel{1}{-+}}Z_{\stackrel{1}{+-}}\right) + \langle T_2\rangle,
\end{equation}
and the third order by
\begin{equation}
T_3=G\left( Z_3 - \frac{3}{2}Z_2Z_1 - \frac{3}{2}Z_1Z_2 + 2 Z_1Z_1Z_1\right)+ \frac{3}{2}T_2T_1 + \frac{3}{2} T_1T_2 - 2 T_1T_1T_1 
.\end{equation}
Here we imply symmetrisation over unequal arguments. We proceed term by term in the expansion \eqref{T1-T3_commutator1} to \eqref{T1-T3_commutator10} in the following manner. First we insert the explicit form of \(T_3\) and abbreviate
\begin{align}\tag*{}
P_3(f,g)&:=Z_3(f,g) -Z_2(f,g) Z_1(f)-\frac{1}{2}Z_2(f) Z_1(g)
 - \frac{1}{2}Z_1(g)Z_2(f)- Z_1(f)Z_2(f,g)\\
  &+ \frac{2}{3} Z_1(g)Z_1(f)Z_1(f)
+ \frac{2}{3} Z_1(f)Z_1(g)Z_1(f)+ \frac{2}{3} Z_1(f)Z_1(f)Z_1(g)
\end{align}  
Further we will be using the abbreviation
\begin{align}\tag*{}
P_2(f,g)&:=Z_2(f,g) -\frac{1}{2}Z_1(g) Z_1(f)-\frac{1}{2}Z_1(f) Z_1(g).
\end{align}  
As a last modification on this term in this chapter we use the rule for a commutator of a product of terms \(\left[a,bc\right]=\left[a,b\right]c + b\left[a,c\right]\) which holds for any tripel of mathematical objects \(a,b,c\) for which the commutator and multiplication makes sense. This property together with linearity of the commutator and the fact that the commutator is antisymmetrical will be applied.
\subsection{Term \eqref{T1-T3_commutator1}}
The described procedure applied to \eqref{T1-T3_commutator1} yields in the first step
\begin{align}\label{T1-T3_commutator1.1}
&-3 \left[T_1(f),T_3(f,g) \right]=
-3 \left[T_1(f), G\left( P_3(f,g)\right)+ \frac{1}{2}T_2(f) T_1(g)+ T_2(f,g) T_1(f)\right. \\ \tag*{}
&\left. +\frac{1}{2} T_1(g) T_2(f)+  T_1(f) T_2(f,g) - \frac{2}{3} T_1(g)T_1(f)T_1(f) - \frac{2}{3} T_1(f)T_1(g)T_1(f) - \frac{2}{3} T_1(f)T_1(f)T_1(g) \right]
.\end{align}
We can expand \eqref{T1-T3_commutator1.1} using the aforementioned relations, this yields
\begin{align}\tag*{}
&-3 \left[T_1(f),T_3(f,g) \right]=
-3 \left[T_1(f), 
G\left( P_3(f,g)\right)
+ \frac{1}{2}G\left(P_2(f)\right) T_1(g)
+ \frac{1}{2}T_1(f)T_1(f) T_1(g)\right. \\\tag*{}
&\left.+ G\left(P_2(f,g)\right) T_1(f)
+\frac{1}{2}T_1(f)T_1(g) T_1(f)
+\frac{1}{2}T_1(g)T_1(f) T_1(f)
+\frac{1}{2} T_1(g) G(P_2(f))
\right. \\\tag*{}
&\left. 
+\frac{1}{2} T_1(g) T_1(f)T_1(f)
+  T_1(f) G(P_2(f,g))
+\frac{1}{2} T_1(f)T_1(g) T_1(f)
+\frac{1}{2} T_1(f) T_1(f)T_1(g)\right.\\\tag*{}
&\left.- \frac{2}{3} T_1(g)T_1(f)T_1(f) 
- \frac{2}{3} T_1(f)T_1(g)T_1(f) 
- \frac{2}{3} T_1(f)T_1(f)T_1(g) \right]\\\tag*{}
&=-3 \left[T_1(f), 
G\left( P_3(f,g)\right)
+ \frac{1}{2}G\left(P_2(f)\right) T_1(g)\right. \\\tag*{}
&\left.+ G\left(P_2(f,g)\right) T_1(f)
+\frac{1}{2} T_1(g) G(P_2(f))
+  T_1(f) G(P_2(f,g))\right.\\\label{T1-T3_commutator1.2}
&\left.+ \frac{1}{3} T_1(g)T_1(f)T_1(f) 
+ \frac{1}{3} T_1(f)T_1(g)T_1(f) 
+ \frac{1}{3} T_1(f)T_1(f)T_1(g) \right]
.\end{align}
The final step then gives
\begin{align}\tag*{}
&-3 \left[T_1(f),T_3(f,g) \right] = 
-3 \left[T_1(f), G\left( P_3(f,g)\right)\right]
-\frac{3}{2} \left[T_1(f), G\left(P_2(f)\right) \right]T_1(g)\\\tag*{}
&-\frac{3}{2}G(P_2(f))\left[T_1(f),T_1(g)\right]
-3 \left[T_1(f),G\left(P_2(f,g)\right) \right]T_1(f)\\\tag*{}
&\frac{3}{2}\left[T_1(f),T_1(g)\right]G(P_2(f))
-\frac{3}{2} T_1(g)\left[T_1(f), G(P_2(f))\right]\\\tag*{}
&-3 T_1(f)\left[T_1(f), G(P_2(f,g))\right]
-\left[T_1(f), T_1(g) \right]T_1(f)T_1(f)\\
&-T_1(f)\left[T_1(f), T_1(g) \right]T_1(f)
-T_1(f)T_1(f) \left[T_1(f), T_1(g) \right]
.\end{align}

\subsection{Term \eqref{T1-T3_commutator2}}
Using the explicit form of \(T_2\) yields for \eqref{T1-T3_commutator2}
\begin{align}\tag*{}
&-3\left[T_2(f),T_2(f,g)\right]=\\
&-3\left[G(P_2(f))+ T_1(f)T_1(f) ,G(P_2(f,g))+ \frac{1}{2}T_1(f)T_1(g)+ \frac{1}{2}T_1(g)T_1(f)\right]
.\end{align}

Using the properties of the commutator expands the expression further, resulting in
\begin{align}\tag*{}
&-3\left[T_2(f),T_2(f,g)\right]=
-3\left[ G(P_2(f)),G(P_2(f,g))\right]
-\frac{3}{2}T_1(f)\left[ G(P_2(f)),T_1(g)\right]\\\tag*{}
&-\frac{3}{2}\left[ G(P_2(f)),T_1(f)\right]T_1(g)
-\frac{3}{2}T_1(g)\left[ G(P_2(f)),T_1(f)\right]
-\frac{3}{2}\left[ G(P_2(f)),T_1(g)\right]T_1(f)\\\tag*{}
&-3T_1(f)\left[T_1(f),G(P_2(f,g))\right]
-3\left[T_1(f),G(P_2(f,g))\right]T_1(f)\\\tag*{}
&-\frac{3}{2}T_1(f)T_1(f)\left[T_1(f),T_1(g)\right]
-\frac{3}{2}T_1(f)\left[T_1(f),T_1(g)\right]T_1(f)\\
&-\frac{3}{2}T_1(f)\left[T_1(f),T_1(g)\right]T_1(f)
-\frac{3}{2}\left[T_1(f),T_1(g)\right]T_1(f) T_1(f)\\\tag*{}
&=-3\left[ G(P_2(f)),G(P_2(f,g))\right]
-\frac{3}{2}T_1(f)\left[ G(P_2(f)),T_1(g)\right]\\\tag*{}
&-\frac{3}{2}\left[ G(P_2(f)),T_1(f)\right]T_1(g)
-\frac{3}{2}T_1(g)\left[ G(P_2(f)),T_1(f)\right]
-\frac{3}{2}\left[ G(P_2(f)),T_1(g)\right]T_1(f)\\\tag*{}
&-3T_1(f)\left[T_1(f),G(P_2(f,g))\right]
-3\left[T_1(f),G(P_2(f,g))\right]T_1(f)\\\tag*{}
&-\frac{3}{2}T_1(f)T_1(f)\left[T_1(f),T_1(g)\right]
-3T_1(f)\left[T_1(f),T_1(g)\right]T_1(f)\\
&-\frac{3}{2}\left[T_1(f),T_1(g)\right]T_1(f) T_1(f)
\end{align}
\subsection{Term \eqref{T1-T3_commutator3}}
Using the explicit form of \(T_3\) yields for \eqref{T1-T3_commutator3}
\begin{align}
-\left[T_3(f) ,T_1(g)\right]=-\left[G(P_3(f))+\frac{3}{2}T_2(f)T_1(f) + \frac{3}{2} T_1(f)T_2(f) - 2 T_1(f)T_1(f)T_1(f) ,T_1(g)\right]
.\end{align}
Using also the explicit form of \(T_2\) results in 
\begin{align}\tag*{}
&-\left[T_3(f) ,T_1(g)\right]=
-\left[G(P_3(f))
+\frac{3}{2}G(P_2(f))T_1(f) 
+\frac{3}{2}T_1(f)T_1(f)T_1(f) \right.\\
&\left. + \frac{3}{2} T_1(f)G(P_2(f))
+\frac{3}{2}T_1(f)T_1(f)T_1(f) 
- 2 T_1(f)T_1(f)T_1(f) 
,T_1(g)\right]\\\tag*{}
&=-\left[G(P_3(f))
+\frac{3}{2}G(P_2(f))T_1(f)  + \frac{3}{2} T_1(f)G(P_2(f))
+ T_1(f)T_1(f)T_1(f) 
,T_1(g)\right]
.\end{align}
Applying the properties of the commutator yields
\begin{align}\tag*{}
&-\left[T_3(f) ,T_1(g)\right]
=-\left[G(P_3(f)),T_1(g)\right]
-\frac{3}{2}\left[G(P_2(f))  ,T_1(g)\right]T_1(f)\\\tag*{}
&-\frac{3}{2}G(P_2(f))\left[T_1(f)  ,T_1(g)\right]
- \frac{3}{2} T_1(f)\left[G(P_2(f)),T_1(g)\right]\\\tag*{}
&- \frac{3}{2} \left[T_1(f),T_1(g)\right]G(P_2(f))
- T_1(f)T_1(f)\left[T_1(f) ,T_1(g)\right]\\
&- T_1(f)\left[T_1(f) ,T_1(g)\right]T_1(f)
- \left[T_1(f) ,T_1(g)\right]T_1(f)T_1(f)
.\end{align}

\subsection{Term \eqref{T1-T3_commutator4}}
Using the explicit form of \(T_2\) and using the properties of the commutator yields for \eqref{T1-T3_commutator4}
\begin{align}\tag*{}
&6T_1(f)\left[ T_1(f),T_2(f,g)\right]=
6T_1(f)\left[ T_1(f),G(P_2(f,g))
+\frac{1}{2}T_1(f)T_1(g)
+\frac{1}{2}T_1(g)T_1(f)\right]\\\tag*{}
&=6T_1(f)\left[ T_1(f),G(P_2(f,g))\right]
+3T_1(f)\left[T_1(f),T_1(f)T_1(g)\right]
+3T_1(f)\left[T_1(f),T_1(g)T_1(f)\right]\\
&=6T_1(f)\left[ T_1(f),G(P_2(f,g))\right]
+3T_1(f)T_1(f)\left[T_1(f),T_1(g)\right]
+3T_1(f)\left[T_1(f),T_1(g)\right]T_1(f)
.\end{align}
\subsection{Term \eqref{T1-T3_commutator5}}
In analogy to the last term, term \eqref{T1-T3_commutator5} yields
\begin{align}\tag*{}
&6\left[ T_1(f),T_2(f,g)\right]T_1(f)
=6\left[ T_1(f),G(P_2(f,g))\right]T_1(f)\\
&+3T_1(f)\left[T_1(f),T_1(g)\right]T_1(f)
+3\left[T_1(f),T_1(g)\right]T_1(f)T_1(f)
.\end{align}
\subsection{Term \eqref{T1-T3_commutator6}}
Using the explicit form of \(T_2\) yields for \eqref{T1-T3_commutator6}
\begin{align}\tag*{}
&3T_2(f)\left[T_1(f),T_1(g)\right]
=3G(P_2(f))\left[T_1(f),T_1(g)\right]
+3T_1(f)T_1(f)\left[T_1(f),T_1(g)\right]\\
&+3 \left(\langle T_2\rangle(f)-\tr \left(Z_{\stackrel{1}{-+}}(f)Z_{\stackrel{1}{+-}}(f)\right)\right)\left[T_1(f),T_1(g)\right]
\end{align}

\subsection{Term \eqref{T1-T3_commutator7}}
Using the explicit form of \(T_2\) and using the properties of the commutator yields for \eqref{T1-T3_commutator7}
\begin{align}\tag*{}
&3\left[T_2(f),T_1(g)\right]T_1(f)
=3 \left[G(P_2(f))+T_1(f)T_1(f),T_1(g)\right]T_1(f)\\\tag*{}
&=3 \left[G(P_2(f)),T_1(g)\right]T_1(f)
+3 T_1(f)\left[T_1(f),T_1(g)\right]T_1(f)\\
&+3 \left[T_1(f),T_1(g)\right]T_1(f)T_1(f)
\end{align}

\subsection{Term \eqref{T1-T3_commutator8}}
Using the explicit form of \(T_2\) and using the properties of the commutator yields for \eqref{T1-T3_commutator8}
\begin{align}\tag*{}
&3T_1(f)\left[T_2(f),T_1(g)\right]
=3T_1(f)\left[ G(P_2(f)+T_1(f)T_1(f),T_1(g)\right]\\ \tag*{}
&=3T_1(f)\left[ G(P_2(f),T_1(g)\right]
+3T_1(f)T_1(f)\left[T_1(f),T_1(g)\right]\\
&+3T_1(f)\left[ T_1(f),T_1(g)\right]T_1(f)
\end{align}

\subsection{Term \eqref{T1-T3_commutator9}}
Using the explicit form of \(T_2\) yields for \eqref{T1-T3_commutator9}
\begin{align}\tag*{}
&3\left[T_1(f),T_1(g)\right]T_2(f)
=3\left[T_1(f),T_1(g)\right]G(P_2(f))
+3\left[T_1(f),T_1(g)\right]T_1(f)T_1(f)\\
&+3 \left(\langle T_2\rangle(f)-\tr \left(Z_{\stackrel{1}{-+}}(f)Z_{\stackrel{1}{+-}}(f)\right)\right)\left[T_1(f),T_1(g)\right]
\end{align}


\subsection{Sum of Terms \eqref{T1-T3_commutator1}-\eqref{T1-T3_commutator10}}
What is left is the summation of the different contributions, this results in 
\begin{align}\tag*{}
&A_3(f,g)-R_3(f,g)=- \left[ G(P_3(f)),T_1(g)\right] 
+ 3\left[ G(P_3(f,g)),T(f)\right]\\\tag*{}
&-3\left[G(P_2(f)),G(P_2(f,g))\right]
-\frac{1}{2} \left[T_1(f),T_1(g)\right] T_1(f) T_1(f)\\\tag*{}
&+ T_1(f) \left[T_1(f),T_1(g)\right] T_1(f)
-\frac{1}{2} T_1(f)T_1(f) \left[T_1(f),T_1(g)\right]\\
&+6 \left( \langle T_2\rangle(f) -\tr \left(\Z{-+}{f}\Z{+-}{f}\right)\right) \left[T_1(f),T_1(g)\right]
\end{align}

\section{Proof of Lemma \ref{G_kommutator}}\label{sec::G_kommutator}
For a proof of this lemma let \(P_k,P_l \in Q\), we compute
\begin{align*}
&\left[G(P_k),G(P_l)\right]=\\
&\left[\sum_{n\in\mathbb{N}} a^*\left(P_k\varphi_n\right)a(\varphi_n) 
- \sum_{n\in -\mathbb{N}} a(\varphi_n) a^*\left(P_k \varphi_n\right) ,
\sum_{b\in\mathbb{N}} a^*\left(P_l\varphi_b\right)a(\varphi_b) 
- \sum_{b\in -\mathbb{N}} a(\varphi_b) a^*\left(P_l \varphi_b\right) \right]\\
&=\sum_{n,b\in\mathbb{N}} \left[ a^*\left(P_k\varphi_n\right) a(\varphi_n), a^*\left(P_l\varphi_b\right)a(\varphi_n)\right]
-\sum_{-b,n\in\mathbb{N}}\left[a^*\left(P_k\varphi_n\right)a(\varphi_n), a(\varphi_b) a^*\left(P_l\varphi_b\right)\right]\\
&-\sum_{-n,b\in\mathbb{N}}\left[a(\varphi_n)a^*\left(P_k\varphi_n\right),a^*\left(P_l\varphi_b\right)a(\varphi_b)\right]
+\sum_{n,b\in\mathbb{N}}\left[ a(\varphi_n)a^*\left(P_k\varphi_n\right), a(\varphi_b)a^*\left(P_l\varphi_b\right)\right]\\
&=\sum_{n,b\in\mathbb{N}} \left(a^*\left(P_k\varphi_n\right)\left<\varphi_n,P_l\varphi_b\right>a(\varphi_b)-a^*\left(P_l\varphi_b\right) \left<\varphi_b,P_k\varphi_n\right>a(\varphi_n) \right)\\
&-\sum_{-b,n\in\mathbb{N}}\left( - \left<\varphi_b,P_k \varphi_n\right>a(\varphi_n)a^*\left(P_l\varphi_b\right) + a(\varphi_b) a^*\left(P_k\varphi_n\right)\left<\varphi_n,P_l\varphi_b\right>\right)\\
&-\sum_{-n,b\in\mathbb{N}} \left( - \left< \varphi_n,P_l\varphi_b\right> a^*\left(P_k\varphi_n\right) a(\varphi_b) + a^*\left(P_l\varphi_b\right)a(\varphi_n)\left<\varphi_b,P_k\varphi_n\right>\right)\\
&+\sum_{n,b\in -\mathbb{N}} \left(a(\varphi_n) \left< \varphi_b,P_k\varphi_n\right>a^*\left(P_l\varphi_b\right)-a(\varphi_b)\left<\varphi_n,P_l\varphi_b\right> a^*\left(P_k\varphi_n\right)\right)\\
&=\sum_{b\in\mathbb{N}} a^*\left(P_k P_{\stackrel{l}{++}}\varphi_b\right)a(\varphi_b) - \sum_{n\in\mathbb{N}}a^*\left(P_l P_{\stackrel{k}{++}}\varphi_n\right)a(\varphi_n)\\
&+\sum_{n\in\mathbb{N}}a(\varphi_n)a^*\left(P_l P_{\stackrel{k}{-+}}\varphi_n\right) - \sum_{-b\in\mathbb{N}} a(\varphi_b)a^*\left(P_k P_{\stackrel{l}{+-}}\varphi_b\right)\\
&+\sum_{b\in\mathbb{N}}a^*\left(P_k P_{\stackrel{l}{-+}}\varphi_b\right)a(\varphi_b)-\sum_{-n\in\mathbb{N}}a^*\left(P_l P_{\stackrel{k}{+-}}\varphi_n\right) a(\varphi_n)\\
&+\sum_{-n\in\mathbb{N}}a(\varphi_n)a^*\left(P_l P_{\stackrel{k}{--}}\varphi_n\right) - \sum_{-b \in \mathbb{N}} a(\varphi_b)a^*\left(P_k P_{\stackrel{l}{--}}\varphi_b\right)\\
=&\sum_{n\in\mathbb{N}} a^*\left(P_k P_l \varphi_n \right) a(\varphi_n) - \sum_{n\in\mathbb{N}} a^*\left(P_l P_{\stackrel{k}{++}} \varphi_n \right) a(\varphi_n)\\
&+\tr \left( P_{\stackrel{l}{+-}} P_{\stackrel{k}{-+}}\right) - \sum_{n\in\mathbb{N}} a^*\left( P_l P_{\stackrel{k}{-+}} \varphi_n\right)a(\varphi_n)\\
&-\tr \left( P_{\stackrel{l}{-+}} P_{\stackrel{k}{+-}}\right) + \sum_{-b\in\mathbb{N}} a(\varphi_b) a^*\left(P_l P_{\stackrel{k}{+-}} \varphi_b\right)\\
&+\sum_{-b\in\mathbb{N}} a(\varphi_b) a^*\left( P_l P_{\stackrel{k}{--}}\varphi_b\right) - \sum_{-b\in\mathbb{N}} a(\varphi_b) a^*\left( P_k P_l \varphi_b\right)\\
&=\tr \left( P_{\stackrel{l}{+-}} P_{\stackrel{k}{-+}}\right)
-\tr \left( P_{\stackrel{l}{-+}} P_{\stackrel{k}{+-}}\right)
+\sum_{n\in\mathbb{N}} a^*\left(\left[P_k ,P_l\right] \varphi_n \right) a(\varphi_n)
+\sum_{-b\in\mathbb{N}} a(\varphi_b)a^*\left(\left[P_l ,P_k\right] \varphi_b \right) \\
&=\tr \left( P_{\stackrel{l}{+-}} P_{\stackrel{k}{-+}}\right)
-\tr \left( P_{\stackrel{l}{-+}} P_{\stackrel{k}{+-}}\right)
+G\left(\left[P_k,P_l\right]\right)
\end{align*}

\section{Vacuum Expectation Value and Charge Conjugation Symmetry}\label{sec:vac_exp}

In order to evaluate the vacuum expectation value of \eqref{pre_vac_exp} we use the following 
\begin{Lemma}\label{lemma:vac_exp}
Let \(C_1,C_2,C_3\in Q\) be arbitrary, then the following equations hold
\begin{align}
&\left<\Omega, G(C_1) \Omega\right>=0\\
&\left<\Omega, G(C_1)G(C_2) \Omega\right>=\tr\left(C_{\stackrel{1}{-+}}C_{\stackrel{2}{+-}}\right)\\
&\left<\Omega, G(C_1)G(C_2)G(C_3) \Omega\right>=
\tr\left(C_{\stackrel{1}{-+}}C_{\stackrel{2}{++}}C_{\stackrel{3}{+-}}\right)-
\tr\left(C_{\stackrel{3}{+-}}C_{\stackrel{2}{--}}C_{\stackrel{1}{-+}}\right)
.\end{align}
\end{Lemma}
{\bf Proof:} Let \(C_1,C_2,C_3 \in Q\) be arbitrary and \((\varphi_n)_{n\in \mathbb{Z}\backslash\{0\}}\) be as usual. Using the fact that 
\begin{align*}
&\forall n \in \mathbb{N}: a(\varphi_n)\Omega=0\\
&\forall -n \in \mathbb{N}: a^*(\varphi_n)\Omega=0 
\end{align*}
holds we directly find the first equation of the lemma. As for the second, we compute
\begin{align*}
&\left<\Omega, G(C_1)G(C_2)\Omega\right>= -\sum_{-b,n\in\mathbb{N}} \left< \Omega, a^*\left(C_{\stackrel{1}{-+}}\varphi_n\right) a(\varphi_n) a(\varphi_b) a^*\left(C_{\stackrel{2}{+-}}\varphi_b\right)\Omega\right>\\
&=\sum_{-b,n\in\mathbb{N}}\left<\Omega, a^*\left(C_{\stackrel{1}{-+}}\varphi_N\right) a(\varphi_b)\left<\varphi_n,C_{\stackrel{2}{+-}}\varphi_b\right>\Omega\right>\\
&=\sum_{-b\in\mathbb{N}} \left<\Omega, a^*\left(C_{\stackrel{1}{-+}}C_{\stackrel{2}{+-}}\varphi_b\right)a(\varphi_b)\Omega\right>
=\tr\left(C_{\stackrel{1}{-+}}C_{\stackrel{2}{+-}}\right)
.\end{align*}
The third one is computed in a quite similar fashion
\begin{align*}
&\left<\Omega, G(C_1)G(C_2)G(C_3)\Omega\right>=\\
&-\sum_{n,b,-d\in\mathbb{N}} \left< \Omega, a^*\left(C_{\stackrel{1}{-+}}\varphi_n\right)a(\varphi_n) \left( a^*\left( C_2 \varphi_b\right) a(\varphi_b)-a(\varphi_{-b})a^*\left(C_2\varphi_{-b}\right)\right) a(\varphi_d)a^*\left(C_{\stackrel{3}{+-}}\varphi_d\right)\Omega\right>\\
&=\sum_{n,b,-d\in\mathbb{N}} \left<\Omega, a^*\left(C_{\stackrel{1}{-+}}\varphi_n\right) a(\varphi_n)a^*\left(C_2\varphi_b\right)a(\varphi_d)\left<\varphi_b, C_{\stackrel{3}{+-}}\varphi_d\right>\Omega\right>\\
&-\sum_{n,b,-d\in\mathbb{N}}\left<\Omega, \left<\varphi_{-b},C_{\stackrel{1}{-+}}\varphi_n\right> a(\varphi_n) a^*\left(C_2\varphi_{-b}\right)a(\varphi_d)a^*\left(C_{\stackrel{3}{+-}}\varphi_d\right)\Omega\right>\\
&=\sum_{n,-d\in\mathbb{N}}\left<\Omega, a^*\left(C_{\stackrel{1}{-+}}\varphi_n\right) a(\varphi_n) a^*\left(C_{\stackrel{2}{++}}C_{\stackrel{3}{+-}}\varphi_d\right) a(\varphi_d)\Omega\right>\\
&-\sum_{n,-d\in\mathbb{N}} \left< \Omega, a(\varphi_n) a^*\left(C_{\stackrel{2}{--}}C_{\stackrel{1}{-+}}\varphi_n\right)a(\varphi_d)a^*\left(C_{\stackrel{3}{+-}}\varphi_d\right)\Omega\right>\\
&=\sum_{n,-d\in\mathbb{N}}\left<\Omega, a^*\left(C_{\stackrel{1}{-+}}\varphi_n\right) \left<\varphi_n, C_{\stackrel{2}{++}}C_{\stackrel{3}{+-}}\varphi_d\right> a(\varphi_d)\Omega\right>\\
&-\sum_{n,-d\in\mathbb{N}} \left<\Omega, a(\varphi_n) \left< \varphi_d,C_{\stackrel{2}{--}}C_{\stackrel{1}{-+}}\varphi_n\right> a^*\left(C_{\stackrel{3}{+-}}\varphi_d\right)\Omega\right>\\
&=\sum_{-d\in\mathbb{N}}\left<\Omega, a^*\left(C_{\stackrel{1}{-+}}C_{\stackrel{2}{++}}C_{\stackrel{3}{+-}}\varphi_d\right) a(\varphi_d) \Omega\right>
-\sum_{n\in\mathbb{N}}\left<\Omega, a(\varphi_n) a^*\left(C_{\stackrel{3}{+-}}C_{\stackrel{2}{--}}C_{\stackrel{1}{-+}}\varphi_n\right)\Omega\right>\\
&=\tr\left(C_{\stackrel{1}{-+}}C_{\stackrel{2}{++}}C_{\stackrel{3}{+-}}\right)
-\tr\left( C_{\stackrel{3}{+-}}C_{\stackrel{2}{--}}C_{\stackrel{1}{-+}}\right)
\end{align*}
\qed

Using lemma \ref{lemma:vac_exp} equation \eqref{pre_vac_exp} becomes

\begin{align}\tag*{}
&\langle \Omega, (A_3(f,g)-R_3(f,g))\Omega \rangle=
-\tr\left(P_{\stackrel{3}{-+}}(f) \Z{+-}{g}\right)\\\tag*{}
&+\tr\left(\Z{-+}{g}P_{\stackrel{3}{+-}}(f) \right)
+3 \tr\left(  P_{\stackrel{3}{-+}}(f,g) \Z{+-}{f}\right)\\\tag*{}
&-3 \tr\left(  \Z{-+}{f} P_{\stackrel{3}{+-}}(f,g) \right)
-3 \tr\left(P_{\stackrel{2}{-+}}(f) P_{\stackrel{2}{+-}}(f,g)\right)\\\tag*{}
&+3 \tr\left(P_{\stackrel{2}{-+}}(f,g) P_{\stackrel{2}{+-}}(f)\right)
-\frac{1}{2} \tr \left(\left[Z_1(f),Z_1(g)\right] \Z{++}{f} \Z{+-}{f}\right)\\\tag*{}
&+\frac{1}{2} \tr \left(\Z{+-}{f} \Z{--}{f}\left[Z_1(f),Z_1(g)\right]  \right)
+ \tr \left(\Z{+-}{f} \left[Z_1(f),Z_1(g)\right] \Z{+-}{f} \right)\\\tag*{}
&- \tr \left(\Z{-+}{f} \left[Z_1(f),Z_1(g)\right] \Z{-+}{f} \right)
-\frac{1}{2} \tr \left(\Z{-+}{f} \Z{++}{f}\left[Z_1(f),Z_1(g)\right] \right)\\\tag*{}
&+\frac{1}{2} \tr \left(\left[Z_1(f),Z_1(g)\right]\Z{--}{f} \Z{-+}{f} \right)\\\tag*{}
&+6 \left( \langle T_2\rangle(f) -\tr \left(\Z{-+}{f}\Z{+-}{f}\right)\right) \tr\left(\Z{-+}{f} \Z{+-}{g}\right)\\\label{post_vac_exp}
&-6 \left( \langle T_2\rangle(f) -\tr \left(\Z{-+}{f}\Z{+-}{f}\right)\right) \tr\left(\Z{-+}{g} \Z{+-}{f}\right)
.\end{align}
Now we will employ charge conjugation symmetry to simplify the expression \eqref{post_vac_exp}. We defined the linear charge conjugation operator \(C:\mathcal{H}\rightarrow\bar{\mathcal{H}}\) in \Markus{insert definition before submitting}. It is unitary and these fulfils commutation relations
\begin{align}
\forall n \in \mathbb{N}:	C Z_n=(-1)^n Z_n C\\
C P_-=P_+C\\
CP_+=P_-C
.\end{align}
Using this it can be seen that for each \(n\in\mathbb{N}\) and each set of operators \(A_1,\dots A_n:\mathcal{H}\rightarrow\mathcal{H}\) such that \(\tr\left(A_1\cdots A_n\right)\) exists the following equation holds
\begin{align*}
\tr_{\mathcal{H}}\left(A_1\cdots A_n\right)
=\tr_{\mathcal{H}}\left(C C^* A_1\cdots A_n\right)
=\tr_{\overline{\mathcal{H}}}\left( C^* A_1\cdots A_n C\right)\\
=\tr_{\overline{\mathcal{H}}}\left( C^* C \tilde{A_1}\cdots \tilde{A_n} \right)
=\sum_{n\in\mathbb{Z}\backslash\{0\}} \overline{\left< \varphi_n, \tilde{A_1}\cdots \tilde{A_n} \varphi_n\right>}
=\overline{\tr_{\mathcal{H}}\left(\tilde{A_1}\cdots \tilde{A_n}\right)}
,\end{align*}
where for any \(k\), \(\tilde{A_k}\) denotes \(C A_k C\). The second but last equality can be seen by construction of \(\overline{\mathcal{H}}\).
For any two summands \(A, B\) in \eqref{post_vac_exp} this means that if \(B\) is obtained by flipping all the subscript signs of \(A\) then 
\begin{align*}
A+B=2 \Re A\\
A-B=2 i \Im A
\end{align*}
holds. Employing this rule yields
\begin{align}\tag*{}
&\left<\Omega, (A_3(f,g)-R_3(f,g))\Omega \right>=
i \Im \left( 2\tr\left(\Z{-+}{g}P_{\stackrel{3}{+-}}(f) \right)
+6 \tr\left(  P_{\stackrel{3}{-+}}(f,g) \Z{+-}{f}\right)\right.\\\tag*{}
&+6 \tr\left(P_{\stackrel{2}{-+}}(f,g) P_{\stackrel{2}{+-}}(f)\right)
+ \tr\left( \Z{+-}{f} \Z{--}{f} \left[Z_1(f),Z_1(g)\right] \right)\\\tag*{}
&- \tr\left(\left[Z_1(f),Z_1(g)\right] \Z{++}{f} \Z{+-}{f} \right)
+2 \tr\left(\Z{+-}{f} \left[Z_1(f),Z_1(g)\right] \Z{-+}{f}\right)\\\tag*{}
&\left.+12 \tr \left(\Z{-+}{g}\Z{+-}{f}\right)\tr\left( \Z{-+}{f} \Z{+-}{f}\right)\right)\\\tag*{}
&+12 i  \langle T_2\rangle(f)\Im  \tr\left(\Z{-+}{f} \Z{+-}{g}\right)\\\label{post_vac1}
&=i \Im \left\{ 2\tr\left(\Z{-+}{g}P_{\stackrel{3}{+-}}(f) \right)\right.\\\label{post_vac2}
&+6 \tr\left(  P_{\stackrel{3}{-+}}(f,g) \Z{+-}{f}\right)\\\label{post_vac3}
&+6 \tr\left(P_{\stackrel{2}{-+}}(f,g) P_{\stackrel{2}{+-}}(f)\right)\\\label{post_vac4}
&- \tr\left( \sigma_{+-}\left(Z_1(f),Z_1(f)\right) \left[Z_1(f),Z_1(g)\right] \right)\\\label{post_vac5}
&+2 \tr\left(\Z{+-}{f} \left[Z_1(f),Z_1(g)\right] \Z{-+}{f}\right)\\\label{post_vac6}
&\left.+12 \tr \left(\Z{-+}{g}\Z{+-}{f}\right)\tr\left( \Z{-+}{f} \Z{+-}{f}\right)\right\}\\\label{post_vac7}
&+12 i  \langle T_2\rangle(f)\Im  \tr\left(\Z{-+}{f} \Z{+-}{g}\right)
,\end{align}
where 
\begin{align*}
\sigma: \left(\mathcal{H}\rightarrow\mathcal{H}\right)\times\left(\mathcal{H}\rightarrow\mathcal{H}\right)\rightarrow \left( \mathcal{H}\rightarrow\mathcal{H}\right)\\
\left(A,B\right)\mapsto A P_+B - AP_- B.
\end{align*}

\section{Momentum Integrals}\label{sec:p_int}
As a first step to finding the integral representations of the terms \eqref{post_vac1}-\eqref{post_vac7} we are going to find the representations of \(P_2\) and \(P_3\) as integral operators. 
Since we would like to recycle these results for higher orders a more general approach seems appropriate. 
The \(P_k\) for any \(k\in\mathbb{N}\) are linear combinations of concatenations of \(Z_l\) for \(l\le k\). 
We remember the definition of \(Z_k\) for \(k\in\mathbb{N}\), let \(A\) be a four-potential, then \(Z_k \phi\) is given by
\begin{multline}
Z_k(A)\phi(y):=(-1)^k \frac{i}{2\pi}  \int_{\mathcal{M}}\frac{i_p(\text{d}^4p_1)}{(2\pi)^{\frac{3}{2}}} \frac{\slashed{p}_1+m}{2m^2} e^{-ip_1y}  \\
  \prod_{l=2}^{k} \left[ \int_{\mathbb{R}^4-i \epsilon e_0}\frac{\text{d}^4p_l}{(2\pi)^{2}} \slashed{A}(p_{l-1}-p_l)  (\slashed{p}_l-m)^{-1}  
 \right]\int_{\mathcal{M}}  i_p(\text{d}^4p_{k+1})\slashed{A}(p_{k}-p_{k+1})\hat{\phi}(p_{k+1})
,\end{multline}
for \(\phi\in \mathcal{H}_0\). Let for \(k\in\mathbb{N}\), \(f_1,\dots, f_k\) be four-potentials. \(Z_k(f_1,\dots,f_k)\) is given by:
\begin{equation}\label{Z_k_multiple_arguments}
Z_k(f_1,\dots,f_k):=\left. \frac{\partial^k}{\partial \varepsilon_1  \dots \partial\varepsilon_k}\right|_{\varepsilon_1=\dots=\varepsilon_k=0} Z_k(\varepsilon_1 f_1+\dots, \varepsilon_k f_k)
.\end{equation}
 Let \(p_1\in \mathcal{M}\), in momentum space \(Z_k\) can be written as
\begin{multline}\label{def_Z_k}
Z_k(A) \hat{\phi}(p_1)=  \frac{\slashed{p}_1+m}{2m^2} \prod_{l=2}^{k} \left[ \int_{\mathbb{R}^4-i \epsilon e_0}\frac{\text{d}^4p_l}{(2\pi)^{2}} \slashed{A}(p_{l-1}-p_l)  (\slashed{p}_l-m)^{-1}  
 \right]\\
 (-1)^k \frac{i}{2\pi} \int_{\mathcal{M}}  i_p(\text{d}^4p_{k+1})\slashed{A}(p_{k}-p_{k+1})\hat{\phi}(p_{k+1})
.\end{multline}
In order to speak about more general terms we introduce the following notation. Let for \(n\in\mathbb{N}\), \(\sigma_1,\sigma_{n-1}\in \{+,-\}\) and \(f_1,\dots, f_n\) be  four-potentials, then \(Z_n (\sigma_1,\dots, \sigma_n)(f)\hat{\phi}\) for \(p_1\in\mathbb{M}\) is defined by
\begin{align}\tag*{}
Z_n (\sigma_1,\dots, \sigma_{n-1})(f)\hat{\phi}(p_1) := \frac{\slashed{p}_1+m}{2m^2} \prod_{l=2}^{n} \left[ \int_{\mathbb{R}^4+\sigma_{l-1} i \epsilon e_0}\frac{\text{d}^4p_l}{(2\pi)^{2}} \right.\\\label{Zsigma}
\left. \slashed{f}_{l-1}(p_{l-1}-p_l)  (\slashed{p}_l-m)^{-1}  \right]
(-1)^k \frac{i}{2\pi}  \int_{\mathcal{M}}  i_p(\text{d}^4p_{n+1})\slashed{f}_n(p_{k}-p_{n+1})\hat{\phi}(p_{n+1})
.\end{align}
If just one four-potential is given as an argument, it is implied that the four-potentials are all identical. For the Concatenation of two such objects the following lemma holds.
\begin{Lemma}\label{Z_concatenation}
Let \(k,b\in\mathbb{N}\) and \(\sigma_{1,1},\dots\sigma_{1,k-1},\sigma_{2,1},\dots,\sigma_{2,b-1}\in \{+,-\}\) then 
\begin{align}\tag*{}
Z_k(\sigma_{1,1},\dots\sigma_{1,k-1})(f) Z_b(\sigma_{2,1},\dots\sigma_{2,k-1})\\
=Z_k (\sigma_{1,1},\dots\sigma_{1,k-1},-,\sigma_{2,1},\dots\sigma_{2,k-1})-
Z_k (\sigma_{1,1},\dots\sigma_{1,k-1},+,\sigma_{2,1},\dots\sigma_{2,k-1})
\end{align}
holds.
\end{Lemma}
{\bf Proof:} Let \(k,b\in\mathbb{N}\) and \(\sigma_{1,1},\dots\sigma_{1,k-1},\sigma_{2,1},\dots,\sigma_{2,b-1}\in \{+,-\}\), \(f\) be a four-potential and \(\phi\in \mathcal{H}_0\), then concatenation yields
\begin{align}\tag*{}
&Z_k(\sigma_{1,1},\dots\sigma_{1,k-1})(f) Z_b(\sigma_{2,1},\dots\sigma_{2,k-1})(f) \hat{\phi}(p_1)=\\\tag*{}
&(-1)^{k+b}\frac{i}{2\pi}\frac{\slashed{p}_1+m}{2m^2} \prod_{l=2}^{k} \left[ \int_{\mathbb{R}^4+i\sigma_{1,l-1} \epsilon e_0}\frac{\text{d}^4p_l}{(2\pi)^{2}} \slashed{f}(p_{l-1}-p_l)  (\slashed{p}_l-m)^{-1}  
 \right]\\\tag*{}
& \int_{\mathcal{M}}  i_p(\text{d}^4p_{k+1})\slashed{f}(p_{k}-p_{k+1})
 \frac{i}{2\pi}  \frac{\slashed{p}_{k+1}+m}{2m^2} \prod_{c=k+2}^{b+k} \left[ \int_{\mathbb{R}^4+i\sigma_{2,c-1} \epsilon e_0}\frac{\text{d}^4p_c}{(2\pi)^{2}} \slashed{f}(p_{c-1}-p_c)  (\slashed{p}_c-m)^{-1}  
 \right]\\\tag*{}
 &\int_{\mathcal{M}}  i_p(\text{d}^4p_{k+b+1})\slashed{f}(p_{k+b}-p_{k+b+1})\hat{\phi}(p_{k+b+1})\\\tag*{}
 &=(-1)^{k+b}\frac{i}{2\pi}\frac{\slashed{p}_1+m}{2m^2} \prod_{l=2}^{k} \left[ \int_{\mathbb{R}^4+i\sigma_{1,l-1} \epsilon e_0}\frac{\text{d}^4p_l}{(2\pi)^{2}} \slashed{f}(p_{l-1}-p_l)  (\slashed{p}_l-m)^{-1}  
 \right]\\\tag*{}
 &\left(\int_{\mathbb{R}^4-i \epsilon e_0} -\int_{\mathbb{R}^4+i \epsilon e_0}   \right)
 \frac{ i_p(\text{d}^4p_{k+1})}{(2\pi)^2}  \slashed{f}(p_{k}-p_{k+1})
  (\slashed{p}_{k+1}-m)^{-1}\\\tag*{}
 & \prod_{c=k+2}^{b+k} \left[ \int_{\mathbb{R}^4+i\sigma_{2,c-1} \epsilon e_0}\frac{\text{d}^4p_c}{(2\pi)^{2}} \slashed{f}(p_{c-1}-p_c)  (\slashed{p}_c-m)^{-1}  
 \right] 
 \int_{\mathcal{M}}  i_p(\text{d}^4p_{k+b+1})\slashed{f}(p_{k+b}-p_{k+b+1})\hat{\phi}(p_{k+b+1})\\
& =Z_{k+b}(\sigma_{1,1},\dots,\sigma_{1,k-1},-,\sigma_{2,1},\dots,\sigma_{2,b-1})(f)-Z_{k+b}(\sigma_{1,1},\dots,\sigma_{1,k-1},+,\sigma_{2,1},\dots,\sigma_{2,b-1})(f)
.\end{align}\qed

Now we are in the position to find a representation of \(P_2,P_3\) in terms of integral operators. 
\begin{Lemma}
The one-particle operators \(P_2\) and \(P_3\) can be represented as
\begin{align*}
&P_2=Z_2(-)-Z_2(-)+Z_2(+)=Z_2(+)\\
&P_3=-\frac{1}{2}Z_3(-,+)-\frac{1}{2}Z_3(+,-)+2 Z_3(+,+).
\end{align*}
\end{Lemma}
{\bf Proof:} For a proof we use the definitions of \(P_2\) and \(P_3\) and lemma \ref{Z_concatenation}. This results in 
\begin{align}
P_2=Z_2-Z_1Z_1=Z_2(-)-Z_2(-)+Z_2(+)=Z_2(+)
\end{align} and
\begin{align}\tag*{}
&P_3=Z_3(-,-)-\frac{3}{2}Z_2(-)Z_1-\frac{3}{2}Z_1Z_2(-)+2 Z_1Z_1Z_1\\\tag*{}
 &=Z_3(-,-)-\frac{3}{2}Z_3(-,-)+\frac{3}{2}Z_3(-,+)-\frac{3}{2}Z_3(-,-)\\ \tag*{}
&+\frac{3}{2}Z_3(+,-)+2 Z_3(-,-)-2 Z_3(-,+)-2 Z_3(+,-)+2 Z_3(+,+)\\
&=-\frac{1}{2}Z_3(-,+)-\frac{1}{2}Z_3(+,-)+2 Z_3(+,+)
.\end{align}
\qed

In order to make \eqref{pre_fourier_int} more explicit we need integral representations of \(P_2\) and \(P_3\) for different arguments. These are symmetric in the order of arguments, since the \(Z_k\) are symmetric by definition. 
So we obtain for \(f,g\) being four-potentials
\begin{align}\label{P2int}
&P_2(f,g)=\frac{1}{2}Z_2(+)(f,g)+\frac{1}{2}Z_2(+)(g,f)\\\label{P3int}
&P_3(f,f,g)=
-\frac{1}{6}Z_3(-,+)(f,f,g)-\frac{1}{6}Z_3(-,+)(f,g,f)-\frac{1}{6}Z_3(-,+)(g,f,f)\\\tag*{}
&-\frac{1}{6}Z_3(+,-)(f,f,g)-\frac{1}{6}Z_3(+,-)(f,g,f)-\frac{1}{6}Z_3(+,-)(g,f,f)\\\tag*{}
&+\frac{2}{3} Z_3(+,+)(f,f,g)+\frac{2}{3} Z_3(+,+)(f,g,f)+\frac{2}{3} Z_3(+,+)(g,f,f)
.\end{align}

Since we are interested in the trace of integral operators we explore the explicit form of such expressions.
\subsection{Trace of Integraloperators}
\begin{Thm}\label{trace_int_op}
 Let \(\mathcal{M}\) denote the 4-dimensional mass shell and \(\mathcal{H}_{\mathcal{M}}\)(in the notation of Deckert, Merkl 2014) be {\bf insert lengthy definition}. We are interested in calculating the trace of an integral operator. Let therefore \(K: \mathcal{H}\rightarrow \mathcal{H}\) be an operator acting as
\begin{equation}
\forall  \xi  \in \mathcal{H}: \widehat{K\xi} (l)= \int_{\mathcal{M}} i_p \left( \mathrm{d}^4 p \right) K(l,p) \hat{\xi}(p)
\end{equation}
 for some nice integral kernel \(K\). 
 
 Then the trace of the operator \(K\) is given by:
 \begin{equation}
 \tr K= \int_{\mathcal{M}} i_p \left( \mathrm{d}^4 p \right) \tr_{\mathcal{D}_p} K(p,p).
 \end{equation}
 \end{Thm}
 {\bf Proof:} As a first step, we choose \((\varphi_{n,k})_{k\in \{1,2\} ,n\in \mathbb{Z}\backslash\{0\}}\) be an ONB of \(\mathcal{H}\) such that for each \(p\in \mathcal{M}\) there is an ONB of \(\mathcal{D}_p\), denoted by \(\{e^p_1,e^p_2\}\) such that
 \begin{equation*}
\forall n \in \mathbb{Z}\backslash\{0\}: \left<\varphi_{n,1}(p), e^p_1\right>=\left<\varphi_{n,2}(p), e^p_2\right>
 \end{equation*}
holds, where \((\varphi_n)_{n\in\mathbb{Z}\backslash\{0\}}\) is an ONB of \(L^2\left(\mathcal{M}, i_p\left(\mathrm{d}^4p\right)\right)=:\mathcal{H}'\). The sum representing the trace of the operator in question can be reordered in this manner
 \begin{equation*}
 \tr K = \sum_{k\in \{1,2\}, n \in \mathbb{Z}\backslash\{0\}} \left< \varphi_{k,n},K \varphi_{k,n}\right>=\sum_{n\in\mathbb{Z}\backslash \{0\}} \left< \varphi_{n},\tr_{\mathcal{D}_{\cdot}} \left(K\right) \varphi_{n}\right>
 .\end{equation*}
For the next step we will be using the bra ket notation of Dirac. The identity on \(\mathcal{H}'\) can be written as 
\begin{equation*}
1=\sum_{n\in\mathbb{Z}\backslash\{0\}} \left| \varphi_n \right> \left< \varphi_n \right|
.\end{equation*}
Now for any function \(g: \in \mathcal{H}'\) and any \(p\in\mathcal{M}\), one gets
\begin{equation*}
g(p)= (1 \cdot g) (p) = \sum_{n\in\mathbb{Z}\backslash\{0\}}\left( \left| \varphi_n \right> \left< \varphi_n , g \right>\right)(p)= \sum_{n\in\mathbb{Z}\backslash\{0\}} \varphi_n (p) \left< \varphi_n , g \right>
.\end{equation*} 
This implies
\begin{equation*}
 \sum_{n\in\mathbb{Z}\backslash\{0\}}   \int_{\mathcal{M}}i_l \left(\mathrm{d}^4 l\right) \varphi_n^*(l)  \tr_{\mathcal{D}_{\cdot}}(K(l,p)) \varphi_n (p)= \tr_{\mathcal{D}_{\cdot}}(K(p,p)),
\end{equation*}
which in turn means for the trace
\begin{align*}
\tr K =  \sum_{n\in\mathbb{Z}\backslash\{0\}} \int_{\mathcal{M}}i_p \left(\mathrm{d}^4 p\right) \int_{\mathcal{M}}i_l \left(\mathrm{d}^4 l\right) \varphi_n^*(p) \tr_{\mathcal{D}_{\cdot}}(K(p,l)) \varphi_n (l)\\
= \int_{\mathcal{M}}i_p \left(\mathrm{d}^4 p\right) \tr_{\mathcal{D}_{\cdot}}(K(p,p))
,\end{align*} 
 where the niceness of \(K\) was used for interchanging the sum and the integrals.\qed
 
We can now turn our attention to the evaluation of the terms \eqref{post_vac1}-\eqref{post_vac7}. For terms \eqref{post_vac1}-\eqref{post_vac6} we will consider the argument of \(\Im\) at first and select the imaginary part only later on.



\subsection{Term \eqref{post_vac1}}

Using the tools acquired so far the term can by expressed as
\begin{align}\tag*{}
&2\tr\left(\Z{-+}{g}P_{\stackrel{3}{+-}}(f) \right)
=-\tr\left(\Z{-+}{g}P_+ Z_3(+,-)(f) P_- \right)\\\tag*{}
&-\tr\left(\Z{-+}{g}P_+ Z_3(-,+)(f) P_- \right)
+4\tr\left(\Z{-+}{g}P_+ Z_3(+,+)(f) P_- \right)\\\label{post_vac1.1}
&=-\tr\left(P_+ Z_3(+,-)(f) P_- Z_1(g)\right)\\\label{post_vac1.2}
&-\tr\left(P_+ Z_3(-,+)(f) P_- Z_1(g) \right)\\\label{post_vac1.3}
&+4\tr\left(P_+ Z_3(+,+)(f) P_- Z_1(g)\right)
\end{align}

Inserting the definition \eqref{Zsigma} one obtains for the term \eqref{post_vac1.1}
\begin{align}\tag*{}
&-\tr\left(P_+ Z_3(+,-)(f) P_- Z_1(g)\right)=-(-1)^{1+3}\left(\frac{i}{2\pi}\right)^2 \\ \tag*{}
&\int_{\mathcal{M}_+}  i_p(\text{d}^4p)
  \int_{\mathbb{R}^4+i \epsilon e_0}\frac{\text{d}^4p_2}{(2\pi)^{2}} 
 \int_{\mathbb{R}^4-i \epsilon e_0}\frac{\text{d}^4p_3}{(2\pi)^{2}} 
\int_{\mathcal{M}_-}  i_{p_4}(\text{d}^4p_{4})\\\tag*{}
& 
\tr \left(\frac{\slashed{p}+m}{2m^2}
\slashed{f}(p-p_2)  (\slashed{p}_2-m)^{-1}  
\slashed{f}(p_{2}-p_3)  (\slashed{p}_3-m)^{-1}  
\slashed{f}(p_{3}-p_{4}) \frac{\slashed{p}_{4}+m}{2m^2} 
\slashed{g}(p_{4}-p)\right)\\\tag*{}
&=\frac{1}{(2\pi)^2} 
\int_{\mathcal{M}}  i_{p_1}(\text{d}^4p_1)
  \int_{\mathbb{C}^4}\frac{\text{d}^4p_2}{(2\pi)^{2}} 
 \int_{\mathbb{C}^4}\frac{\text{d}^4p_3}{(2\pi)^{2}} 
\int_{\mathcal{M}}  i_{p_4}(\text{d}^4p_{4}) \\\tag*{}
& 
\tr \left(\frac{\slashed{p}_1+m}{2m^2}
\slashed{f}(p_1-p_2)  (\slashed{p}_2-m)^{-1}  
\slashed{f}(p_{2}-p_3)  (\slashed{p}_3-m)^{-1}  
\slashed{f}(p_{3}-p_{4}) \frac{\slashed{p}_{4}+m}{2m^2} 
\slashed{g}(p_{4}-p_1)\right)\\\tag*{}
&1_{\mathbb{R}^+}(p_1^0)
1_{\mathbb{R}^4+i \epsilon e_0}(p_2)
1_{\mathbb{R}^4-i \epsilon e_0}(p_3)
1_{\mathbb{R}^+}(-p^0_4)
.\end{align}
The terms \eqref{post_vac1.2} and \eqref{post_vac1.3} can be obtained, resulting in the sum
\begin{align}
&-\tr\left(P_+ Z_3(+,-)(f) P_- Z_1(g)\right)\\
&=\frac{1}{(2\pi)^2} 
\int_{\mathcal{M}}  i_p(\text{d}^4p_1)
  \int_{\mathbb{C}^4}\frac{\text{d}^4p_2}{(2\pi)^{2}} 
 \int_{\mathbb{C}^4}\frac{\text{d}^4p_3}{(2\pi)^{2}} 
\int_{\mathcal{M}}  i_p(\text{d}^4p_{4}) \\\tag*{}
& 
\tr_{\mathbb{C}^4} \left(\frac{\slashed{p}_1+m}{2m^2}
\slashed{f}(p_1-p_2)  (\slashed{p}_2-m)^{-1}  
\slashed{f}(p_{2}-p_3)  (\slashed{p}_3-m)^{-1}  
\slashed{f}(p_{3}-p_{4}) \frac{\slashed{p}_{4}+m}{2m^2} 
\slashed{g}(p_{4}-p_1)\right)\\\tag*{}
&\id_{\mathbb{R}^+}(p_1^0)\id_{\mathbb{R}^+}(-p^0_4)\\
&(\id_{\mathbb{R}^4+i \epsilon e_0}(p_2)
\id_{\mathbb{R}^4-i \epsilon e_0}(p_3)
+ \id_{\mathbb{R}^4-i \epsilon e_0}(p_2)
\id_{\mathbb{R}^4+i \epsilon e_0}(p_3)
-4  \id_{\mathbb{R}^4+i \epsilon e_0}(p_2)
\id_{\mathbb{R}^4+i \epsilon e_0}(p_3)
)
. \end{align}

\subsection{Term \eqref{post_vac2}}

Using \eqref{P3int} one finds
\begin{align}\tag*{}
&6 \tr\left(  P_{\stackrel{3}{-+}}(f,g) \Z{+-}{f}\right)=
6 \tr\left(P_+Z_1(f) P_-  P_{3} (f,g) \right)\\\label{post_vac2.1}
&=- \tr\left(P_+ Z_1(f)P_- Z_3(-,+)(f,f,g)\right)\\\label{post_vac2.2}
&- \tr\left(P_+ Z_1(f)P_- Z_3(-,+)(f,g,f)\right)\\\label{post_vac2.3}
&- \tr\left(P_+ Z_1(f)P_- Z_3(-,+)(g,f,f)\right)\\\label{post_vac2.4}
&- \tr\left(P_+ Z_1(f)P_- Z_3(+,-)(f,f,g)\right)\\\label{post_vac2.5}
&- \tr\left(P_+ Z_1(f)P_- Z_3(+,-)(f,g,f)\right)\\\label{post_vac2.6}
&- \tr\left(P_+ Z_1(f)P_- Z_3(+,-)(g,f,f)\right)\\\label{post_vac2.7}
&+4 \tr\left(P_+ Z_1(f)P_- Z_3(+,+)(f,f,g)\right)\\\label{post_vac2.8}
&+4 \tr\left(P_+ Z_1(f)P_-  Z_3(+,+)(f,g,f)\right)\\\label{post_vac2.9}
&+4 \tr\left(P_+ Z_1(f)P_-  Z_3(+,+)(g,f,f)\right)
\end{align}

Writing out term \eqref{post_vac2.1} yields
\begin{align*}
&- \tr\left(P_+ Z_1(f)P_- Z_3(-,+)(f,f,g)\right)
=-(-1)^{1+3}\left(\frac{i}{2\pi}\right)^2\\
&\int_{\mathcal{M_+}}  i_p(\text{d}^4p_{1}) 
\int_{\mathcal{M_-}}  i_{p_2}(\text{d}^4p_{2})
\int_{\mathbb{R}^4-i \epsilon e_0}\frac{\text{d}^4p_3}{(2\pi)^{2}}
\int_{\mathbb{R}^4+i \epsilon e_0}\frac{\text{d}^4p_4}{(2\pi)^{2}}
 \\
&\tr_{\mathbb{C}^4}\left(\frac{\slashed{p}_1+m}{2m^2} \slashed{f}(p_{1}-p_{2})
 \frac{\slashed{p}_{2}+m}{2m^2} 
   \slashed{f}(p_{2}-p_3)  (\slashed{p}_3-m)^{-1} \right..
\\
&\left. \slashed{f}(p_{3}-p_4)  (\slashed{p}_4-m)^{-1} \slashed{g}(p_{4}-p_{1})\right)
=\frac{1}{(2\pi)^2}\\
&\int_{\mathcal{M_+}}  i_p(\text{d}^4p_{1}) 
\int_{\mathcal{M_-}}  i_{p_2}(\text{d}^4p_{2})
\int_{\mathbb{R}^4-i \epsilon e_0}\frac{\text{d}^4p_3}{(2\pi)^{2}}
\int_{\mathbb{R}^4+i \epsilon e_0}\frac{\text{d}^4p_4}{(2\pi)^{2}}
 \\
&\tr_{\mathbb{C}^4}\left(\frac{\slashed{p}_1+m}{2m^2} \slashed{f}(p_{1}-p_{2})
 \frac{\slashed{p}_{2}+m}{2m^2} 
   \slashed{f}(p_{2}-p_3)  (\slashed{p}_3-m)^{-1} \right.
\\
&\left. \slashed{f}(p_{3}-p_4)  (\slashed{p}_4-m)^{-1} \slashed{g}(p_{4}-p_{1})\right)
.\end{align*}

Writing out the other terms results in 

\begin{align*}
&\eqref{post_vac2.2}=- \tr\left(P_+ Z_1(f)P_- Z_3(-,+)(f,g,f)\right)=\frac{1}{(2\pi)^2}\\
&\int_{\mathcal{M_+}}  i_{p_1}(\text{d}^4p_{1}) 
\int_{\mathcal{M_-}}  i_{p_2}(\text{d}^4p_{2})
\int_{\mathbb{R}^4-i \epsilon e_0}\frac{\text{d}^4p_3}{(2\pi)^{2}}
\int_{\mathbb{R}^4+i \epsilon e_0}\frac{\text{d}^4p_4}{(2\pi)^{2}}
 \\
&\tr_{\mathbb{C}^4}\left(\frac{\slashed{p}_1+m}{2m^2} \slashed{f}(p_{1}-p_{2})
  \frac{\slashed{p}_{2}+m}{2m^2} 
   \slashed{f}(p_{2}-p_3)  (\slashed{p}_3-m)^{-1} \right.
\\
&\left. \slashed{g}(p_{3}-p_4)  (\slashed{p}_4-m)^{-1} \slashed{f}(p_{4}-p_{1})\right)=\frac{1}{(2\pi)^2}\\
&\int_{\mathcal{M_+}}  i_{p_2}(\text{d}^4p_{2}) 
\int_{\mathcal{M_-}}  i_{p_3}(\text{d}^4p_{3})
\int_{\mathbb{R}^4-i \epsilon e_0}\frac{\text{d}^4p_4}{(2\pi)^{2}}
\int_{\mathbb{R}^4+i \epsilon e_0}\frac{\text{d}^4p_1}{(2\pi)^{2}}
 \\
&\tr_{\mathbb{C}^4}\left( (\slashed{p}_1-m)^{-1} \slashed{f}(p_{1}-p_{2})\frac{\slashed{p}_2+m}{2m^2} \slashed{f}(p_{2}-p_{3})
 \frac{\slashed{p}_{3}+m}{2m^2} \right.\\
&\left. \slashed{f}(p_{3}-p_4)  (\slashed{p}_4-m)^{-1} \slashed{g}(p_{4}-p_1)  \right)
.\end{align*}


\begin{align*}
&\eqref{post_vac2.3}=- \tr\left(P_+ Z_1(f)P_- Z_3(-,+)(g,f,f)\right)=\frac{1}{(2\pi)^2}\\
&\int_{\mathcal{M_+}}  i_{p_1}(\text{d}^4p_{1}) 
\int_{\mathcal{M_-}}  i_{p_2}(\text{d}^4p_{2})
\int_{\mathbb{R}^4-i \epsilon e_0}\frac{\text{d}^4p_3}{(2\pi)^{2}}
\int_{\mathbb{R}^4+i \epsilon e_0}\frac{\text{d}^4p_4}{(2\pi)^{2}}
 \\
&\tr_{\mathbb{C}^4}\left(\frac{\slashed{p}_1+m}{2m^2} \slashed{f}(p_{1}-p_{2})
 \frac{\slashed{p}_{2}+m}{2m^2} 
\slashed{g}(p_{2}-p_3)  (\slashed{p}_3-m)^{-1} \right.\\
&\left. \slashed{f}(p_{3}-p_4)  (\slashed{p}_4-m)^{-1} \slashed{f}(p_{4}-p_{1})\right)
=\frac{1}{(2\pi)^2}\\
&\int_{\mathcal{M_+}}  i_{p_3}(\text{d}^4p_{3}) 
\int_{\mathcal{M_-}}  i_{p_4}(\text{d}^4p_{4})
\int_{\mathbb{R}^4-i \epsilon e_0}\frac{\text{d}^4p_1}{(2\pi)^{2}}
\int_{\mathbb{R}^4+i \epsilon e_0}\frac{\text{d}^4p_2}{(2\pi)^{2}}
 \\
&\tr_{\mathbb{C}^4}\left( (\slashed{p}_1-m)^{-1}  \slashed{f}(p_{1}-p_2)  (\slashed{p}_2-m)^{-1} \slashed{f}(p_{2}-p_{3})   \right.\\
&\left.  \frac{\slashed{p}_3+m}{2m^2} \slashed{f}(p_{3}-p_{4})
 \frac{\slashed{p}_{4}+m}{2m^2} 
\slashed{g}(p_{4}-p_1)  \right)
\end{align*}
Term \eqref{post_vac2.4} is evaluated in analogy to \eqref{post_vac2.1}, it is
\begin{align*}
&\eqref{post_vac2.4}=- \tr\left(P_+ Z_1(f)P_- Z_3(+,-)(f,f,g)\right)=\frac{1}{(2\pi)^2}\\
&\int_{\mathcal{M_+}}  i_p(\text{d}^4p_{1}) 
\int_{\mathcal{M_-}}  i_{p_2}(\text{d}^4p_{2})
\int_{\mathbb{R}^4+i \epsilon e_0}\frac{\text{d}^4p_3}{(2\pi)^{2}}
\int_{\mathbb{R}^4-i \epsilon e_0}\frac{\text{d}^4p_4}{(2\pi)^{2}}
 \\
&\tr_{\mathbb{C}^4}\left(\frac{\slashed{p}_1+m}{2m^2} \slashed{f}(p_{1}-p_{2})
 \frac{\slashed{p}_{2}+m}{2m^2} 
   \slashed{f}(p_{2}-p_3)  (\slashed{p}_3-m)^{-1} \right.
\\
&\left. \slashed{f}(p_{3}-p_4)  (\slashed{p}_4-m)^{-1} \slashed{g}(p_{4}-p_{1})\right)
.\end{align*}
Term \eqref{post_vac2.5} is evaluated in analogy to \eqref{post_vac2.2}, yielding
\begin{align*}
&\eqref{post_vac2.5}=- \tr\left(P_+ Z_1(f)P_- Z_3(+,-)(f,g,f)\right)=\frac{1}{(2\pi)^2}\\
&\int_{\mathcal{M_+}}  i_{p_2}(\text{d}^4p_{2}) 
\int_{\mathcal{M_-}}  i_{p_3}(\text{d}^4p_{3})
\int_{\mathbb{R}^4+i \epsilon e_0}\frac{\text{d}^4p_4}{(2\pi)^{2}}
\int_{\mathbb{R}^4-i \epsilon e_0}\frac{\text{d}^4p_1}{(2\pi)^{2}}
 \\
&\tr_{\mathbb{C}^4}\left( (\slashed{p}_1-m)^{-1} \slashed{f}(p_{1}-p_{2})\frac{\slashed{p}_2+m}{2m^2} \slashed{f}(p_{2}-p_{3})
 \frac{\slashed{p}_{3}+m}{2m^2} \right.\\
&\left. \slashed{f}(p_{3}-p_4)  (\slashed{p}_4-m)^{-1} \slashed{g}(p_{4}-p_1)  \right)
.\end{align*}

Term \eqref{post_vac2.6} is evaluated in analogy to \eqref{post_vac2.3}, yielding
\begin{align*}
&\eqref{post_vac2.6}=- \tr\left(P_+ Z_1(f)P_- Z_3(+,-)(g,f,f)\right)=\frac{1}{(2\pi)^2}\\
&\int_{\mathcal{M_+}}  i_{p_3}(\text{d}^4p_{3}) 
\int_{\mathcal{M_-}}  i_{p_4}(\text{d}^4p_{4})
\int_{\mathbb{R}^4+i \epsilon e_0}\frac{\text{d}^4p_1}{(2\pi)^{2}}
\int_{\mathbb{R}^4-i \epsilon e_0}\frac{\text{d}^4p_2}{(2\pi)^{2}}
 \\
&\tr_{\mathbb{C}^4}\left( (\slashed{p}_1-m)^{-1}  \slashed{f}(p_{1}-p_2)  (\slashed{p}_2-m)^{-1} \slashed{f}(p_{2}-p_{3})   \right.\\
&\left.  \frac{\slashed{p}_3+m}{2m^2} \slashed{f}(p_{3}-p_{4})
 \frac{\slashed{p}_{4}+m}{2m^2} 
\slashed{g}(p_{4}-p_1)  \right)
\end{align*}
Term \eqref{post_vac2.7} is evaluated in analogy to \eqref{post_vac2.1}, yielding
\begin{align*}
&\eqref{post_vac2.7}=+4 \tr\left(P_+ Z_1(f)P_- Z_3(+,+)(f,f,g)\right)=-\frac{4}{(2\pi)^2}\\
&\int_{\mathcal{M_+}}  i_p(\text{d}^4p_{1}) 
\int_{\mathcal{M_-}}  i_{p_2}(\text{d}^4p_{2})
\int_{\mathbb{R}^4+i \epsilon e_0}\frac{\text{d}^4p_3}{(2\pi)^{2}}
\int_{\mathbb{R}^4+i \epsilon e_0}\frac{\text{d}^4p_4}{(2\pi)^{2}}
 \\
&\tr_{\mathbb{C}^4}\left(\frac{\slashed{p}_1+m}{2m^2} \slashed{f}(p_{1}-p_{2})
 \frac{\slashed{p}_{2}+m}{2m^2} 
   \slashed{f}(p_{2}-p_3)  (\slashed{p}_3-m)^{-1} \right.
\\
&\left. \slashed{f}(p_{3}-p_4)  (\slashed{p}_4-m)^{-1} \slashed{g}(p_{4}-p_{1})\right)
.\end{align*}
Term \eqref{post_vac2.8} is evaluated in analogy to \eqref{post_vac2.2}, yielding
\begin{align*}
&\eqref{post_vac2.8}=+4 \tr\left(P_+ Z_1(f)P_- Z_3(+,+)(f,g,f)\right)=-\frac{4}{(2\pi)^2}\\
&\int_{\mathcal{M_+}}  i_{p_2}(\text{d}^4p_{2}) 
\int_{\mathcal{M_-}}  i_{p_3}(\text{d}^4p_{3})
\int_{\mathbb{R}^4+i \epsilon e_0}\frac{\text{d}^4p_4}{(2\pi)^{2}}
\int_{\mathbb{R}^4+i \epsilon e_0}\frac{\text{d}^4p_1}{(2\pi)^{2}}
 \\
&\tr_{\mathbb{C}^4}\left( (\slashed{p}_1-m)^{-1} \slashed{f}(p_{1}-p_{2})\frac{\slashed{p}_2+m}{2m^2} \slashed{f}(p_{2}-p_{3})
 \frac{\slashed{p}_{3}+m}{2m^2} \right.\\
&\left. \slashed{f}(p_{3}-p_4)  (\slashed{p}_4-m)^{-1} \slashed{g}(p_{4}-p_1)  \right)
.\end{align*}
Term \eqref{post_vac2.9} is evaluated in analogy to \eqref{post_vac2.3}, yielding
\begin{align*}
&\eqref{post_vac2.9}=+4 \tr\left(P_+ Z_1(f)P_- Z_3(+,+)(g,f,f)\right)=-\frac{4}{(2\pi)^2}\\
&\int_{\mathcal{M_+}}  i_{p_3}(\text{d}^4p_{3}) 
\int_{\mathcal{M_-}}  i_{p_4}(\text{d}^4p_{4})
\int_{\mathbb{R}^4+i \epsilon e_0}\frac{\text{d}^4p_1}{(2\pi)^{2}}
\int_{\mathbb{R}^4+i \epsilon e_0}\frac{\text{d}^4p_2}{(2\pi)^{2}}
 \\
&\tr_{\mathbb{C}^4}\left( (\slashed{p}_1-m)^{-1}  \slashed{f}(p_{1}-p_2)  (\slashed{p}_2-m)^{-1} \slashed{f}(p_{2}-p_{3})   \right.\\
&\left.  \frac{\slashed{p}_3+m}{2m^2} \slashed{f}(p_{3}-p_{4})
 \frac{\slashed{p}_{4}+m}{2m^2} 
\slashed{g}(p_{4}-p_1)  \right)
\end{align*}

\subsection{Term \eqref{post_vac3}}
For this term we also use \eqref{P2int} and \eqref{Zsigma}
\begin{align}\tag*{}
&\eqref{post_vac3}
=6 \tr\left(P_{\stackrel{2}{-+}}(f,g) P_{\stackrel{2}{+-}}(f)\right)\\\label{post_vac3.1}
&=3 \tr\left(Z_{2}(+)(f,g) P_+ Z_{2}(+)(f) P_-\right)\\\label{post_vac3.2}
&+3 \tr\left(Z_{2}(+)(g,f)P_+ Z_{2}(+)(f) P_-\right)
.\end{align}
 Computing the concatenation in \eqref{post_vac3.1} yields
\begin{align*}
&\eqref{post_vac3.1}=3 \tr\left(Z_{2}(+)(f,g) P_+ Z_{2}(+)(f) P_-\right)
=3(-1)^{2+2}\left(\frac{i}{2\pi}\right)^2\\
&  \int_{\mathcal{M}_-}  i_p(\text{d}^4p_{1})
\int_{\mathbb{R}^4+i \epsilon e_0}\frac{\text{d}^4p_2}{(2\pi)^{2}} 
\int_{\mathcal{M}_+}  i_p(\text{d}^4p_{3})
\int_{\mathbb{R}^4+i\epsilon e_0}\frac{\text{d}^4p_4}{(2\pi)^{2}}\\\tag*{}
&\tr_{\mathbb{C}^4}\left(\frac{\slashed{p}_1+m}{2m^2}  \slashed{f}(p_{1}-p_2)  (\slashed{p}_2-m)^{-1}  
\slashed{g}(p_{2}-p_{3})
  \frac{\slashed{p}_{3}+m}{2m^2} \right.\\
&\left.\slashed{f}(p_{3}-p_4)  (\slashed{p}_4-m)^{-1}  
\slashed{f}(p_{4}-p_{1})\right)
=-\frac{3}{(2\pi)^2}\\
&  \int_{\mathcal{M}_-}  i_p(\text{d}^4p_{3})
\int_{\mathbb{R}^4+i \epsilon e_0}\frac{\text{d}^4p_4}{(2\pi)^{2}} 
\int_{\mathcal{M}_+}  i_p(\text{d}^4p_{1})
\int_{\mathbb{R}^4+i\epsilon e_0}\frac{\text{d}^4p_2}{(2\pi)^{2}}\\\tag*{}
&\tr_{\mathbb{C}^4}
\left(    \frac{\slashed{p}_{1}+m}{2m^2} \slashed{f}(p_{1}-p_2)  (\slashed{p}_2-m)^{-1}   \slashed{f}(p_{2}-p_{3})   \frac{\slashed{p}_3+m}{2m^2}  
   \right.\\
&\left.  \slashed{f}(p_{3}-p_4)  (\slashed{p}_4-m)^{-1}  
\slashed{g}(p_{4}-p_{1}) \right)\\
.\end{align*}
 Computing the concatenation in \eqref{post_vac3.2} in analogy to \eqref{post_vac3.1} yields
\begin{align*}
&\eqref{post_vac3.2}=3 \tr\left(Z_{2}(+)(g,f) P_+ Z_{2}(+)(f) P_-\right)
=3(-1)^{2+2}\left(\frac{i}{2\pi}\right)^2\\
&  \int_{\mathcal{M}_-}  i_p(\text{d}^4p_{1})
\int_{\mathbb{R}^4+i \epsilon e_0}\frac{\text{d}^4p_2}{(2\pi)^{2}} 
\int_{\mathcal{M}_+}  i_p(\text{d}^4p_{3})
\int_{\mathbb{R}^4+i\epsilon e_0}\frac{\text{d}^4p_4}{(2\pi)^{2}}\\\tag*{}
&\tr_{\mathbb{C}^4}\left(\frac{\slashed{p}_1+m}{2m^2}  \slashed{g}(p_{1}-p_2)  (\slashed{p}_2-m)^{-1}  
\slashed{f}(p_{2}-p_{3})
  \frac{\slashed{p}_{3}+m}{2m^2} \right.\\
&\left.\slashed{f}(p_{3}-p_4)  (\slashed{p}_4-m)^{-1}  
\slashed{f}(p_{4}-p_{1})\right)
=-\frac{3}{(2\pi)^2}\\
&  \int_{\mathcal{M}_-}  i_p(\text{d}^4p_{4})
\int_{\mathbb{R}^4+i \epsilon e_0}\frac{\text{d}^4p_1}{(2\pi)^{2}} 
\int_{\mathcal{M}_+}  i_p(\text{d}^4p_{2})
\int_{\mathbb{R}^4+i\epsilon e_0}\frac{\text{d}^4p_3}{(2\pi)^{2}}\\\tag*{}
&\tr_{\mathbb{C}^4}
\left( (\slashed{p}_1-m)^{-1}  \slashed{f}(p_{1}-p_{2})
  \frac{\slashed{p}_{2}+m}{2m^2} \slashed{f}(p_{2}-p_3)  (\slashed{p}_3-m)^{-1} \right.\\
&\left. \slashed{f}(p_{3}-p_{4})\frac{\slashed{p}_4+m}{2m^2}
 \slashed{g}(p_{4}-p_1)  \right)
.\end{align*}

\section{Term \eqref{post_vac4}}
As a first step we disassemble \eqref{post_vac4} into four summands, giving
\begin{align}\tag*{}
&\eqref{post_vac4}=- \tr\left( \sigma_{+-}\left(Z_1(f),Z_1(f)\right) \left[Z_1(f),Z_1(g)\right] \right)\\\label{post_vac4.1}
&=-\tr\left(P_+ Z_1(f) P_+Z_1(f)P_- Z_1(f) Z_1(g) \right)\\\label{post_vac4.2}
&\tr\left(P_+ Z_1(f) P_+Z_1(f)P_- Z_1(g) Z_1(f) \right)\\\label{post_vac4.3}
&\tr\left(P_+ Z_1(f) P_- Z_1(f)P_- Z_1(f) Z_1(g) \right)\\\label{post_vac4.4}
&-\tr\left(P_+ Z_1(f) P_- Z_1(f)P_- Z_1(g) Z_1(f) \right)
.\end{align}
We will treat each term individually.

\subsection{Term \eqref{post_vac4.1}}
Using \eqref{def_Z_k} the concatenation yields
\begin{align*}
&\eqref{post_vac4.1}=-\tr\left(P_+ Z_1(f) P_+Z_1(f)P_- Z_1(f) Z_1(g) \right)\\
&= -(-1)^4 \left(\frac{i}{2\pi}\right)^4 
\int_{\mathcal{M}_+}  i_p(\text{d}^4p_{1})
\int_{\mathcal{M}_+}  i_p(\text{d}^4p_{2})
 \int_{\mathcal{M}_-}  i_p(\text{d}^4p_{3})
 \int_{\mathcal{M}}  i_p(\text{d}^4p_{4})\\
&\tr_{\mathbb{C}^4}\left( \frac{\slashed{p}_1+m}{2m^2} 
  \slashed{f}(p_{1}-p_{2})
 \frac{\slashed{p}_2+m}{2m^2} 
\slashed{f}(p_{2}-p_{3})\right.\\
 &\left.\frac{\slashed{p}_3+m}{2m^2} 
 \slashed{f}(p_{3}-p_{4})
 \frac{\slashed{p}_4+m}{2m^2} 
 \slashed{g}(p_{4}-p_{1})\right)\\
 &= - 
\int_{\mathcal{M}_+} \frac{ i_p(\text{d}^4p_{1})}{2\pi}
\int_{\mathcal{M}_+}  \frac{i_p(\text{d}^4p_{2})}{2\pi}
 \int_{\mathcal{M}_-} \frac{ i_p(\text{d}^4p_{3})}{2\pi}
 \int_{\mathcal{M}}  \frac{i_p(\text{d}^4p_{4})}{2\pi}\\
&\tr_{\mathbb{C}^4}\left( \frac{\slashed{p}_1+m}{2m^2} 
  \slashed{f}(p_{1}-p_{2})
 \frac{\slashed{p}_2+m}{2m^2} 
\slashed{f}(p_{2}-p_{3})
\frac{\slashed{p}_3+m}{2m^2} 
 \slashed{f}(p_{3}-p_{4})
 \frac{\slashed{p}_4+m}{2m^2} 
 \slashed{g}(p_{4}-p_{1})\right)
.\end{align*}

\subsection{Term \eqref{post_vac4.2}}
In analogy to \eqref{post_vac4.1} we find
\begin{align*}
&\eqref{post_vac4.2}=\tr\left(P_+ Z_1(f) P_+Z_1(f)P_- Z_1(g) Z_1(f) \right)
=\tr\left( Z_1(f) P_+ Z_1(f) P_+Z_1(f)P_- Z_1(g) \right)\\
 &= 
\int_{\mathcal{M}} \frac{ i_p(\text{d}^4p_{1})}{2\pi}
\int_{\mathcal{M}_+}  \frac{i_p(\text{d}^4p_{2})}{2\pi}
 \int_{\mathcal{M}_+} \frac{ i_p(\text{d}^4p_{3})}{2\pi}
 \int_{\mathcal{M}_-}  \frac{i_p(\text{d}^4p_{4})}{2\pi}\\
&\tr_{\mathbb{C}^4}\left( \frac{\slashed{p}_1+m}{2m^2} 
  \slashed{f}(p_{1}-p_{2})
 \frac{\slashed{p}_2+m}{2m^2} 
\slashed{f}(p_{2}-p_{3})
\frac{\slashed{p}_3+m}{2m^2} 
 \slashed{f}(p_{3}-p_{4})
 \frac{\slashed{p}_4+m}{2m^2} 
 \slashed{g}(p_{4}-p_{1})\right)
.\end{align*}
\subsection{Term \eqref{post_vac4.3}}
In analogy to \eqref{post_vac4.1} we find
\begin{align*}
&\eqref{post_vac4.3}=\tr\left(P_+ Z_1(f) P_-Z_1(f)P_- Z_1(f) Z_1(g) \right)\\
 &= 
\int_{\mathcal{M}_+} \frac{ i_p(\text{d}^4p_{1})}{2\pi}
\int_{\mathcal{M}_-}  \frac{i_p(\text{d}^4p_{2})}{2\pi}
 \int_{\mathcal{M}_-} \frac{ i_p(\text{d}^4p_{3})}{2\pi}
 \int_{\mathcal{M}}  \frac{i_p(\text{d}^4p_{4})}{2\pi}\\
&\tr_{\mathbb{C}^4}\left( \frac{\slashed{p}_1+m}{2m^2} 
  \slashed{f}(p_{1}-p_{2})
 \frac{\slashed{p}_2+m}{2m^2} 
\slashed{f}(p_{2}-p_{3})
\frac{\slashed{p}_3+m}{2m^2} 
 \slashed{f}(p_{3}-p_{4})
 \frac{\slashed{p}_4+m}{2m^2} 
 \slashed{g}(p_{4}-p_{1})\right)
.\end{align*}
\subsection{Term \eqref{post_vac4.4}}
In analogy to \eqref{post_vac4.1} we find
\begin{align*}
&\eqref{post_vac4.4}=-\tr\left(P_+ Z_1(f) P_-Z_1(f)P_- Z_1(g) Z_1(f) \right)
=-\tr\left(Z_1(f) P_+ Z_1(f) P_-Z_1(f)P_- Z_1(g)  \right)\\
 &= -
\int_{\mathcal{M}} \frac{ i_p(\text{d}^4p_{1})}{2\pi}
\int_{\mathcal{M}_+}  \frac{i_p(\text{d}^4p_{2})}{2\pi}
 \int_{\mathcal{M}_-} \frac{ i_p(\text{d}^4p_{3})}{2\pi}
 \int_{\mathcal{M}_-}  \frac{i_p(\text{d}^4p_{4})}{2\pi}\\
&\tr_{\mathbb{C}^4}\left( \frac{\slashed{p}_1+m}{2m^2} 
  \slashed{f}(p_{1}-p_{2})
 \frac{\slashed{p}_2+m}{2m^2} 
\slashed{f}(p_{2}-p_{3})
\frac{\slashed{p}_3+m}{2m^2} 
 \slashed{f}(p_{3}-p_{4})
 \frac{\slashed{p}_4+m}{2m^2} 
 \slashed{g}(p_{4}-p_{1})\right)
.\end{align*}


\section{Term \eqref{post_vac5}}

We will treat the two summands of the commutator individually. Labelling them yields
\begin{align}\tag*{}
&\eqref{post_vac5} =2 \tr\left(\Z{+-}{f} \left[Z_1(f),Z_1(g)\right] \Z{-+}{f}\right)\\\tag*{}
&=2 \tr\left(\Z{+-}{f} Z_1(f) Z_1(g) \Z{-+}{f}\right)
-2 \tr\left(\Z{+-}{f} Z_1(g) Z_1(f) \Z{-+}{f}\right)\\\tag*{}
&=2 \tr\left(\Z{-+}{f} \Z{+-}{f} Z_1(f) Z_1(g) \right)
-2 \tr\left(Z_1(f) \Z{-+}{f} \Z{+-}{f} Z_1(g)  \right)\\\label{post_vac5.1}
&=2 \tr\left(P_- Z_1(f)P_+ Z_1(f)P_- Z_1(f) Z_1(g) \right)\\\label{post_vac5.2}
&-2 \tr\left(Z_1(f)P_- Z_1(f)P_+ Z_1(f) P_-Z_1(g)  \right)
\end{align}

\subsection{Term \eqref{post_vac5.1}}
In analogy to term \eqref{post_vac4.1} we find
\begin{align*}
&\eqref{post_vac5.1}=2 \tr\left(P_- Z_1(f)P_+ Z_1(f)P_- Z_1(f) Z_1(g) \right)\\
 &= 2
\int_{\mathcal{M}_-} \frac{ i_p(\text{d}^4p_{1})}{2\pi}
\int_{\mathcal{M}_+}  \frac{i_p(\text{d}^4p_{2})}{2\pi}
 \int_{\mathcal{M}_-} \frac{ i_p(\text{d}^4p_{3})}{2\pi}
 \int_{\mathcal{M}}  \frac{i_p(\text{d}^4p_{4})}{2\pi}\\
&\tr_{\mathbb{C}^4}\left( \frac{\slashed{p}_1+m}{2m^2} 
  \slashed{f}(p_{1}-p_{2})
 \frac{\slashed{p}_2+m}{2m^2} 
\slashed{f}(p_{2}-p_{3})
\frac{\slashed{p}_3+m}{2m^2} 
 \slashed{f}(p_{3}-p_{4})
 \frac{\slashed{p}_4+m}{2m^2} 
 \slashed{g}(p_{4}-p_{1})\right)
.\end{align*}

\subsection{Term \eqref{post_vac5.2}}
In analogy to term \eqref{post_vac4.1} we find
\begin{align*}
&\eqref{post_vac5.2}=-2 \tr\left(Z_1(f)P_- Z_1(f)P_+ Z_1(f) P_-Z_1(g)\right)\\
 &= -2
\int_{\mathcal{M}} \frac{ i_p(\text{d}^4p_{1})}{2\pi}
\int_{\mathcal{M}_-}  \frac{i_p(\text{d}^4p_{2})}{2\pi}
 \int_{\mathcal{M}_+} \frac{ i_p(\text{d}^4p_{3})}{2\pi}
 \int_{\mathcal{M}_-}  \frac{i_p(\text{d}^4p_{4})}{2\pi}\\
&\tr_{\mathbb{C}^4}\left( \frac{\slashed{p}_1+m}{2m^2} 
  \slashed{f}(p_{1}-p_{2})
 \frac{\slashed{p}_2+m}{2m^2} 
\slashed{f}(p_{2}-p_{3})
\frac{\slashed{p}_3+m}{2m^2} 
 \slashed{f}(p_{3}-p_{4})
 \frac{\slashed{p}_4+m}{2m^2} 
 \slashed{g}(p_{4}-p_{1})\right)
.\end{align*}

\subsection{Term \eqref{post_vac6}}
We write out the corresponding integral of term \eqref{post_vac6}, again using \eqref{def_Z_k}, giving
\begin{align*}
 &\eqref{post_vac6}=
 12 \tr \left(\Z{-+}{g}\Z{+-}{f}\right)\tr\left( \Z{-+}{f} \Z{+-}{f}\right)\\
 &=12 \tr \left(Z_1(f) P_-Z_1(g)P_+\right)\tr \left(Z_1(f) P_-Z_1(f)P_+\right)\\
 &=12 (-1)^4 i^4
    \\
&  \int_{\mathcal{M}_+}  \frac{i_p(\text{d}^4p_{1})}{2\pi}
 \int_{\mathcal{M}_-}  \frac{i_p(\text{d}^4p_{2})}{2\pi}
\tr_{\mathbb{C}^4}\left( 
 \frac{\slashed{p}_1+m}{2m^2}
 \slashed{f}(p_{1}-p_{2})
 \frac{\slashed{p}_2+m}{2m^2} 
  \slashed{g}(p_{2}-p_{1})\right)\\
  &  \int_{\mathcal{M}_+}  \frac{i_p(\text{d}^4p_{1})}{2\pi}
 \int_{\mathcal{M}_-}  \frac{i_p(\text{d}^4p_{2})}{2\pi}
\tr_{\mathbb{C}^4}\left( 
 \frac{\slashed{p}_1+m}{2m^2}
 \slashed{f}(p_{1}-p_{2})
 \frac{\slashed{p}_2+m}{2m^2} 
  \slashed{f}(p_{2}-p_{1})\right)\\
  &=12 
  \int_{\mathcal{M}_+}  \frac{i_p(\text{d}^4p_{1})}{2\pi}
 \int_{\mathcal{M}_-}  \frac{i_p(\text{d}^4p_{2})}{2\pi}
\tr_{\mathbb{C}^4}\left( 
 \frac{\slashed{p}_1+m}{2m^2}
 \slashed{f}(p_{1}-p_{2})
 \frac{\slashed{p}_2+m}{2m^2} 
  \slashed{g}(p_{2}-p_{1})\right)\\
  &  \int_{\mathcal{M}_+}  \frac{i_p(\text{d}^4p_{1})}{2\pi}
 \int_{\mathcal{M}_-}  \frac{i_p(\text{d}^4p_{2})}{2\pi}
\tr_{\mathbb{C}^4}\left( 
 \frac{\slashed{p}_1+m}{2m^2}
 \slashed{f}(p_{1}-p_{2})
 \frac{\slashed{p}_2+m}{2m^2} 
  \slashed{f}(p_{2}-p_{1})\right)
\end{align*}



\end{document}