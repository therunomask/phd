\documentclass[oneside,reqno,12pt]{amsart}

%\usepackage{fontspec}

\usepackage[a4paper, top=2.7cm, bottom=2.7cm]{geometry}
%\usepackage[T1]{fontenc}
%\usepackage[utf8]{inputenc}
\usepackage{fontspec}
%\setmainfont{YuMincho}
%Hiragino Maru Gothic ProN
\usepackage{bbm}
\usepackage{graphicx}
\usepackage{slashed}
\usepackage{eurosym}
\usepackage{amsmath}
\usepackage{enumitem}
\usepackage{amsfonts}
\usepackage{longtable}
\usepackage[mathscr]{eucal}

\setcounter{secnumdepth}{5}

%commutative diagram
\usepackage{amsmath,amscd}
%picture
\usepackage{wrapfig}

\usepackage[unicode=true, pdfusetitle, bookmarks=true,
  bookmarksnumbered=false, bookmarksopen=false, breaklinks=true, 
  pdfborder={0 0 0}, backref=false, colorlinks=true, linkcolor=blue,
  citecolor=blue, urlcolor=blue]{hyperref}



% \numberwithin{equation}{section}
\allowdisplaybreaks[1]

\newtheorem{axiom}{Axiom}
\newtheorem{Def}{Definition}[section]
\newtheorem{Conj}[Def]{Conjecture}
\newtheorem{Thm}[Def]{Theorem}
\newtheorem{Prp}[Def]{Proposition}
\newtheorem{Lemma}[Def]{Lemma}
\newtheorem{lemma}{Lemma}
\newtheorem{Remark}[Def]{Remark}
\newtheorem{Corollary}[Def]{Corollary}
\newtheorem{Example}[Def]{Example}
\newtheorem{Assumption}[Def]{Assumption}

\newenvironment{mueq}
  {\equation\aligned}
  {\endaligned\endequation}
  
\DeclareMathOperator{\tr}{tr}
\DeclareMathOperator{\supp}{supp}


\newcommand{\Z}[2]{Z_{\stackrel{1}{#1}}\left(#2\right)}
\newcommand{\id}{{\mathbbm 1}}
\newcommand{\equaltext}[1]{\ensuremath{\stackrel{\text{#1}}{=}}}
\newcommand{\letext}[1]{\ensuremath{\stackrel{\text{#1}}{\le}}}
\newcommand{\Conv}{\mathop{\scalebox{1.7}{\raisebox{-0.2ex}{\(\ast\)}}}}
\newcommand{\CONV}{\mathop{\scalebox{3.0}{\raisebox{-0.2ex}{\(\ast\)}}}}
% Annotations
%\usepackage[normalem]{ulem}
% \usepackage{refcheck}
\usepackage[colorinlistoftodos,shadow,textsize=scriptsize,textwidth=2.75cm]{todonotes}
\newcommand{\Dirk}[1]{ \todo[color=orange!60]{Dirk: #1} }
\newcommand{\DirkBox}[1]{ \mbox{}\todo[inline,caption={},color=red!60]{Dirk: #1} }
\newcommand{\Markus}[1]{ \todo[color=green!20]{Markus: #1} }
\newcommand{\dirk}{ \color{orange} }
\newcommand{\markus}{ \color{green} }
\newcommand{\noch}[1]{ \todo[color=blue!20]{Todo: #1} }
\newcommand{\black}{ \color{black} }

\makeatletter



\renewcommand\section{\@startsection {section}{1}{\z@}%
                                   {-2.0ex \@plus -1ex \@minus -.2ex}%
                                   {2.3ex \@plus.2ex}%
                                   {\normalfont\Large\bfseries}}
\renewcommand\subsection{\@startsection {subsection}{1}{\z@}%
                                   {-0.5ex \@plus -0.5ex \@minus -.2ex}%
                                   {0.5em}%
                                   {\normalfont\bfseries}}
\renewcommand\subsubsection{\@startsection {subsubsection}{1}{\z@}%
                                   {-0.3ex \@plus -0.4ex \@minus -.2ex}%
                                   {0.1 em}%
                                   {\normalfont\sc}}  
\renewcommand\paragraph{\@startsection {paragraph}{1}{\z@}%
                                   {-0.2ex \@plus -1ex \@minus -.2ex}%
                                   {0.1 em}%
                                   {\normalfont\it}}                                   
\makeatother

\parindent 0cm
\begin{document}

\begin{Thm}\label{trace_int_op}
 Let \(\mathcal{M}\) denote the 4-dimensional mass shell and \(\mathcal{H}_{\mathcal{M}}\)(in the notation of Deckert, Merkl 2014) be ... {\bf (insert lengthy definition) }. We are interested in calculating the trace of an integral operator. Let therefore \(K: \mathcal{H}\rightarrow \mathcal{H}\) be an operator acting as
\begin{equation}
\forall  \xi  \in \mathcal{H}: \widehat{K\xi} (l)= \int_{\mathcal{M}} i_p \left( \mathrm{d}^4 p \right) K(l,p) \hat{\xi}(p)
\end{equation}
 for some nice integral kernel \(K\). 
 
 Then the trace of the operator \(K\) is given by:
 \begin{equation}
 \tr K= \int_{\mathcal{M}} i_p \left( \mathrm{d}^4 p \right) \tr_{\mathcal{D}_p} K(p,p).
 \end{equation}
 \end{Thm}
 {\bf Proof:} As a first step, we choose \((\varphi_{n,k})_{k\in \{1,2\} ,n\in \mathbb{Z}\backslash\{0\}}\) be an ONB of \(\mathcal{H}\) such that for each \(p\in \mathcal{M}\) there is an ONB of \(\mathcal{D}_p\), denoted by \(\{e^p_1,e^p_2\}\) such that
 \begin{equation*}
\forall n \in \mathbb{Z}\backslash\{0\}: \left<\varphi_{n,1}(p), e^p_1\right>=\left<\varphi_{n,2}(p), e^p_2\right>
 \end{equation*}
holds, where \((\varphi_n)_{n\in\mathbb{Z}\backslash\{0\}}\) is an ONB of \(L^2\left(\mathcal{M}, i_p\left(\mathrm{d}^4p\right)\right)=:\mathcal{H}'\). The sum representing the trace of the operator in question can be reordered in this manner
 \begin{equation*}
 \tr K = \sum_{k\in \{1,2\}, n \in \mathbb{Z}\backslash\{0\}} \left< \varphi_{k,n},K \varphi_{k,n}\right>=\sum_{n\in\mathbb{Z}\backslash \{0\}} \left< \varphi_{n},\tr_{\mathcal{D}_{\cdot}} \left(K\right) \varphi_{n}\right>
 .\end{equation*}
For the next step we will be using the bra ket notation of Dirac. The identity on \(\mathcal{H}'\) can be written as 
\begin{equation*}
1=\sum_{n\in\mathbb{Z}\backslash\{0\}} \left| \varphi_n \right> \left< \varphi_n \right|
.\end{equation*}
Now for any function \(g: \in \mathcal{H}'\) and any \(p\in\mathcal{M}\), one gets
\begin{equation*}
g(p)= (1 \cdot g) (p) = \sum_{n\in\mathbb{Z}\backslash\{0\}}\left( \left| \varphi_n \right> \left< \varphi_n , g \right>\right)(p)= \sum_{n\in\mathbb{Z}\backslash\{0\}} \varphi_n (p) \left< \varphi_n , g \right>
.\end{equation*} 
This implies
\begin{equation*}
 \sum_{n\in\mathbb{Z}\backslash\{0\}}   \int_{\mathcal{M}}i_l \left(\mathrm{d}^4 l\right) \varphi_n^*(l)  \tr_{\mathcal{D}_{\cdot}}(K(l,p)) \varphi_n (p)= \tr_{\mathcal{D}_{\cdot}}(K(p,p)),
\end{equation*}
which in turn means for the trace
\begin{align*}
\tr K =  \sum_{n\in\mathbb{Z}\backslash\{0\}} \int_{\mathcal{M}}i_p \left(\mathrm{d}^4 p\right) \int_{\mathcal{M}}i_l \left(\mathrm{d}^4 l\right) \varphi_n^*(p) \tr_{\mathcal{D}_{\cdot}}(K(p,l)) \varphi_n (l)\\
= \int_{\mathcal{M}}i_p \left(\mathrm{d}^4 p\right) \tr_{\mathcal{D}_{\cdot}}(K(p,p))
,\end{align*} 
 where the niceness of \(K\) was used for interchanging the sum and the integrals.\qed
 

 
\end{document}

