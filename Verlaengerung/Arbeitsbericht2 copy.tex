\documentclass[a4paper,12pt]{article}

\usepackage{german}

\usepackage{graphicx}

\usepackage{amssymb}

\usepackage{amsfonts}

\usepackage{amsmath}

\usepackage{amsthm}

\usepackage{slashed}
\usepackage{mathrsfs} 
\usepackage{todonotes}

%identity sign
\usepackage{bbm}
\usepackage[unicode=true, pdfusetitle, bookmarks=true,
  bookmarksnumbered=false, bookmarksopen=false, breaklinks=true, 
  pdfborder={0 0 0}, backref=false, colorlinks=true, linkcolor=blue,
  citecolor=blue, urlcolor=blue]{hyperref}
  
%commutative diagrams
\usepackage{amsmath,amscd}

\newcommand{\equaltext}[1]{\ensuremath{\stackrel{\text{#1}}{=}}}
\newcommand{\equalmath}[1]{\ensuremath{\stackrel{#1}{=}}}

\newtheorem{vermutung}{Vermutung}
\newtheorem{satz}{Satz}

\DeclareMathOperator*{\esssup}{ess \, sup}
\newcommand{\id}{{\mathbbm 1}}

\addtolength{\textwidth}{2.2cm} \addtolength{\hoffset}{-1.0cm}

\addtolength{\textheight}{3.0cm} \addtolength{\voffset}{-2cm} 

\parindent 0cm

\pagestyle{empty}



\begin{document}

\begin{center}
{\huge Arbeitsbericht zum Promotionsvorhaben mit dem Titel
\\
{\Large Electron-Positron Pair Creation in External Fields \\
Rigorous Control of the Scattering Matrix Expansion}}
\end{center}

\begin{center}
Markus Nöth
\\ \today
\end{center}

\section{Einführende Zusammenfassung des Projekts}
Meine Promotion fällt in den Bereich der mathematischen Physik, ein Teilbereich der Mathematik. Sie handelt von der 
Teilchen-Antiteilchen Paarerzeugung, ein Phänomen welches zuerst von Dirac 1929 vorhergesagt und 1934 von 
Anderson nachgewiesen wurde. Insbesondere geht es um Paarerzeugung durch externe elektrische und 
magnetische Felder unter Vernachlässigung der Wechselwirkung der Teilchen untereinander. Experimentell könnte
man beispielsweise Elektron-Positron Paarerzeugung beobachten, wenn man die elektromagnetischen Felder von
mehreren extrem starken Lasern über lange Zeiträume mit den Feldern von schweren Ionen kombiniert. 
Dies wird an der Extreme Light Infrastructure, einer europäischen Forschungseinrichtung, umgesetzt. 

Die Entdeckung dieses Phänomens durch Dirac hat ihren Ursprung in den ungewöhnlichen Eigenschaften der von ihm
entdeckten Gleichung \eqref{diraceq} für einzelne Elektronen.
Diese Gleichung erlaubt sowohl positive kinetische 
Energien über \(m c^2\) als auch negative kinetische Energien unter \(-mc^2\). Das ist deshalb merkwürdig, weil
physikalische Systeme üblicherweise ihre Energie minimieren wenn ihnen durch Kopplung an externe Systeme
die Möglichkeit dazu gegeben wird. Konkret bedeutet das, dass wenn ein Elektron durch diese Gleichung beschrieben
wird, so würde man erwarten,
dass dieses Elektron unendlich viel Energie abstrahlt und immer negativere kinetische Energien annimmt. So ein Verhalten
beobachtet man allerdings nicht in der Natur. Dirac löste dieses Problem dadurch, dass er 
der Beschreibung zugrunde legte, dass alle Zustände negativer Energie bereits besetzt sind. Paulis Ausschlussprinzip 
verhindert dann, dass weitere Elektronen diese Zustände annehmen. Dieser See aus Teilchen negativer Energie stellt
also sicher, dass die beschriebenen Teilchen stabileres Verhalten zeigen, als Gleichung \eqref{diraceq} zuerst vermuten lässt.


Dirac argumentierte weiter in seiner Veröffentlichung zur Theorie der Positronen, 
dass dieser See so homogen sei, dass alle Wechselwirkung zwischen 
Teilchen des Sees so symmetrisch erfolgt und sich dadurch gegenseitig aufhebt, sodass insgesamt keine Wechselwirkung 
beobachtbar ist. Durch ein externes Feld kann das System allerdings so stark gestört werden, dass ein Elektron aus dem
See in die Zustände positiver Energie gehoben wird und somit sowohl ein Teilchen positiver Energie als auch ein Loch
im See übrig bleiben. Obwohl der Dirac See auf diese Weise eine schöne anschauliche Erklärung von Paarerzeugung liefert,
bereitet die Formulierung einer mathematisch sauberen Beschreibung des Phänomens nach wie vor Probleme. 

In meiner Promotion konzentriere ich mich auf Quantenelektrodynamik (QED) externer Felder, in diesem vereinfachten Modell 
betrachtet man nicht wie sich die Teilchen gegenseitig beeinflussen sondern ausschließlich wie externe Felder auf die Teilchen
wirken. Antworten auf Experimentelle Fragen wie ``Wie verhält sich ein System aus Elektronen nachdem es für eine längere 
Zeit einem starken Laser ausgesetzt war?'' kann man an der sogenannten Streumatrix ablesen. Dieses Objekt wird für 
gewöhnlich durch eine Summe aus unendlich vielen Termen dargestellt. Für viele Modelle der QED ist bekannt,
dass eine solche perturbative Darstellung nicht konvergiert, weshalb eine andere Darstellung der Streumatrix gefunden
werden muss. Für QED externer Felder denkt man jedoch, dass die Situation besser ist und 
die Reihe der Streumatrix konvergiert. Mein Projekt besteht darin diese Vermutung in Wissen umzusetzen.

\section{Erreichte Ergebnisse}

In meinem letzten Arbeitsbericht erzählte ich davon, dass die eben erwähnte Konvergenz, für Felder \(A\) welche
\(\|\id-U^A\|_{\mathcal{H}\rightarrow \mathcal{H}}<1\)
erfüllen, und unter Vernachlässigung einer Phase \(\eta\) welche von \(A\) abhängen darf gezeigt werden konnte.
 Hierbei ist \(\mathcal{H}\) der Hilbertraum der Einteilchen Wellenfunktionen und \(U^A\) die Streumatrix für eben
diese Wellenfunktionen. Für die Formulierung des Resultats benötige ich die differenzielle Liftung, diese 
definiere ich folgendermaßen 
\begin{equation}
d\Gamma(E)=\sum_{n\in \mathbb{N}} \left[a^*(E\varphi_n)a(\varphi_n) - a(\varphi_{-n})a^*(E\varphi_{-n})\right],
\end{equation}
wobei \( (\varphi_n)_{n\in \mathbb{Z}\backslash \{0\}}\) eine Orthonormalbasis von \(\mathcal{H}\) ist, welche anhand
des Vorzeichens des Index aufgeteilt werden kann in eine Orthonormalbasis vom Raum der Lösungen der Diracgleichung

\begin{equation}\label{diraceq}
%i \slashed{\partial}\psi - m \psi = e \slashed{A}\psi
i \slashed{\partial}\psi - m \psi = 0
\end{equation}

mit positiver beziehungsweise negativer Energie. Der Operator \(E:\mathcal{H}\rightarrow\mathcal{H}\) ist dabei linear, beschränkt und erfüllt \(E^*=-E\). Das Resultat nimmt die folgende Form an.

\begin{satz}\label{Explizite Formel}
Die zweitquantisierte Streumatrix \(S\) ist gegeben durch
\begin{equation}\label{Smatrix}
S^A= e^{i \eta} e^{d\Gamma(\ln (U^A))},
\end{equation}
wobei \(\eta\in \mathbb{R}\) gilt.
\end{satz}


Diese Ergebnisse sind bereits in einem entsprechenden Kapitel meiner werdenden Doktorarbeit
 ausführlich dokumentiert, allerdings noch nicht publiziert.
Zudem habe ich ein Analogon zum Wick'schen Satz für Produkte von \(d\Gamma\) Operatoren bewiesen, dies
wird es erleichtern explizite Fehlerschranken zu schätzen wenn unsere Ergebnisse mit Experimenten
verglichen werden. Die starke Einschränkung an die Stärke des Feldes kann mit der folgenden Methode 
überwunden werden: Zunächst ist das Resultat Satz \ref{Explizite Formel} im Wechselwirkungsbild bezüglich
der freien Entwicklung \(U^0\) zu verstehen.
Nun kann man das Wechselwirkungsbild nicht nur bezüglich \(U^0\), sondern bezüglich jeder beliebigen 
Referenz \(U^{A'}\), mit externem Potential \(A'\), verwenden und das Resultat Satz \ref{Explizite Formel}
wird auch bezüglich dieser Referenz gelten, denn für dessen Herleitung waren ausschließlich die
algebraischen Eigenschaften von \(d\Gamma\) und \(U^A\) nötig. Die Formel in Satz \ref{Explizite Formel} gilt also
lokal um jedes beliebige \(A'\), jeweils mit möglicherweise unterschiedlichem \(\eta(A')\). Die Streumatrix \(S\)
ist also lokal um jedes \(A'\) eine analytische Funktion (modulo der Phase \(\eta\)) in der Störung des Feldes \(A-A'\). 
Wenn nun also eine konsistente
analytische Wahl der Phase möglich ist, so muss \(S\) auch eine global konvergierende Potenzreihe in \(A\) besitzen.




Darüber hinaus habe ich im letzten Jahr zusammen mit dem Wissenschaftler Matthias Lienert an einer von ihm vorgeschlagene
relativistischen Wellengleichung gearbeitet. Daraus resultierte bereits ein wissenschaftlicher Artikel. Dieser wurde bereits eingereicht,
wir erwarten die Gutachten in Kürze. Eine vorläufige Version ist einsehbar unter https://arxiv.org/abs/1903.06020. 
Die Gleichung welche wir betrachten hat die Form 
\begin{align}
\psi(x,y)=\psi^0(x,y) + \lambda \int_{\mathbb{R}^4} d^4 x' \int_{\mathbb{R}^4} d^4y' S_1(x-x')S_2(y-y') K(x',y') \psi(x',y').
\end{align}

Hierbei ist \(\lambda\in\mathbb{C}\) eine Zahl, \(\psi^0:\mathbb{R}^8\rightarrow \mathbb{C}^{16}\) eine Lösung der Diracgleichung \eqref{diraceq} in beiden 
Raumzeitkoordinaten \(x\) und \(y\), \(S_i\)
ist die Greensfunktion der Dirac Gleichung bezüglich der jeweiligen Raumzeit und Spinorindizes und \(K\) ist ein Wechselwirkungskern.
Diese Gleichung reduziert sich auf die Dirac Gleichung mit Coulomb Wechselwirkung, wenn die Retardierung der beiden Teilchen
vernachlässigt wird. Weil die Dirac Gleichung mit Coulomb Wechselwirkung experimentell relevant ist, ist auch die von uns betrachtete
Gleichung ein interessanter Kandidat für die relativistische Beschreibung von mehreren wechselwirkenden Teilchen.
Dem Modell liegen ebenfalls einige vereinfachende Annahmen zu Grunde. Teilchen Erzeugung und Vernichtung wird vernachlässigt.
Hierbei erwarten wir, dass die üblichen Divergenzen der Quantenfeldtheorie nicht auftreten und eine wohldefinierte Dynamik existiert. 
Der volle Wechselwirkungskern, welcher elektromagnetische Wechselwirkung vermittelt, ist
\begin{equation}
K(x,y)=\gamma_1^\alpha \gamma_{2,\alpha} \delta((x-y)^\beta(x-y)_{\beta}).
\end{equation}

Zuvor hatten Matthias Lienert und Roderich Tumulka bereits Existenz von Lösungen für weniger singuläre Kerne und einige weitere Vereinfachungen
 für die analoge Gleichung für Klein-Gordon
Teilchen gezeigt. Matthias Lienert und meine Arbeit ist die erste für Dirac Teilchen, weshalb wir ähnliche Vereinfachungen benötigten. Dennoch
beschreibt das Endresultat eine relativistisch invariante wechselwirkende Dynamik von Dirac Teilchen in 4 dimensionaler Raumzeit, ein
Ergebnis welches es bis dahin noch nicht gab. Das Resultat erfolgte in zwei Schritten, zunächst zeigten wir Existenz von Lösungen 
in dem Teil der Raumzeit mit nicht negativen Zeiten \(\mathbb{R}^+\times\mathbb{R}^3\). Dieses Ergebnis ist nicht Lorentz invariant.
Anschließend zeigten wir, dass das gleiche Resultat ebenfalls auf einer Raumzeit mit Urknall gilt, auf Friedmann-Lemaître-Robertson-Walker (FLRW)
Raumzeiten. Der Urknall dieser Lösungen der Einstein'schen Feldgleichungen bietet auf natürliche Weise einen Anfang der Zeit, weshalb das zweite
Resultat der vollen relativistischen Invarianz genügt. Die in unserer Arbeit notwendigen
Vereinfachungen sind wie folgt. Die zunächst relevante Raumzeit ist \(\frac{1}{2}\mathbb{M}:=\mathbb{R}^+\times\mathbb{R}^3\), 
Wellenfunktionen sind vom Typ \(\psi: \left(\frac{1}{2}\mathbb{M}\right)^2\rightarrow \mathbb{C}^{16}\), der Wechselwirkungskern ist vom 
Typ \(K:\left(\frac{1}{2}\mathbb{M}\right)^2\rightarrow \mathbb{C}\), dazu erfüllt \(K\) 

\begin{equation}
		\| K \| := \sup_{x_1, x_2 \in \frac{1}{2}\mathbb{M} } \max \left\{ |K(x_1,x_2)|, |\slashed{\partial}_1 K(x_1,x_2)|, |\slashed{\partial}_2 K(x_1,x_2)|, |\slashed{\partial}_1 \slashed{\partial}_2 K(x_1,x_2)|\right\} < \infty.
	\label{eq:normk}
\end{equation}

Wir konstruieren uns einen Lösungsraum \(\mathscr{B}_g\), 
welcher von einer Funktion \(g:\mathbb{R}^+\rightarrow\mathbb{R}^+\) abhängt. Dieser
ist die Vervollständigung von \(\mathscr{S}\left(\left(\frac{1}{2}\mathbb{M}\right)^2,\mathbb{C}^{16}\right)\), den Schwarzfunktionen des passenden Typs,
unter der Norm \(\|\cdot\|_g\),  welche gegeben ist durch
\begin{equation}
	\| \psi \|^2_g = \esssup_{x_1^0,x_2^0 >0} \frac{1}{g(x_1^0)g(x_2^0)} [\psi]^2(x_1^0,x_2^0),
\label{eq:normpsi}
\end{equation}
unter Verwendung von

\begin{align}
	[\psi]^2(x_1^0,x_2^0) ~:= \| ((i\slashed{\partial}_{1}-m)(i\slashed{\partial}_{2}-m) \psi)(x_1^0, \cdot, x_2^0,\cdot)\|^2_{L^2(\mathbb{R}^6,\mathbb{C}^{16})}\\
	+\| ((i\slashed{\partial}_{1}-m) \psi)(x_1^0, \cdot, x_2^0,\cdot)\|^2_{L^2(\mathbb{R}^6,\mathbb{C}^{16})}\\
	+\| ((i\slashed{\partial}_{2}-m) \psi)(x_1^0, \cdot, x_2^0,\cdot)\|^2_{L^2(\mathbb{R}^6,\mathbb{C}^{16})}
	+\|  \psi(x_1^0, \cdot, x_2^0,\cdot)\|^2_{L^2(\mathbb{R}^6,\mathbb{C}^{16})}.
\label{eq:spatialnorm}
\end{align}

Das erste Resultat nimmt die folgende Form an

\begin{satz}
	Sei  $0 < \| K \| < 1$,  and
	\begin{align} 
		g(t)& ~=~ \sqrt{1+b t^8}\, \exp(b t^8/16),	\label{eq:defg}\\
    		b& ~=~ \frac{\|K\|^4}{(1-\|K\|)^4} \left(6+m^4\right)^4. \label{eq:defb}
\end{align}
	Dann hat die Gleichung 
	
\begin{equation}
\psi(x,y)=\psi^0(x,y) + \lambda \int_{\frac{1}{2}\mathbb{M}} d^4 x' \int_{\frac{1}{2}\mathbb{M}} d^4y' S_1(x-x')S_2(y-y') K(x',y') \psi(x',y')
\end{equation}
	
	 für jedes $\psi^0 \in \mathscr{B}_g$ eine eindeutige Lösung $\psi \in \mathscr{B}_g$.
\end{satz}

Das zweite Resultat beschreibt Lösungen auf der FLRW Raumzeit, die Metric dieser Raumzeit ist gegeben durch 
\begin{equation}
ds^2=a^2(\eta) (d\eta^2-d\vec{x}^2),
\end{equation}
hierbei wird \(\eta\in[0,\infty[\) die kosmologische Zeit genannt. Unter Verwendung von konformer Invarianz der Dirac Gleichung
können wir das Resultat fast unverändert im Fall \(m=0\) auf diese Raumzeit übertragen. 

\begin{satz}
	Sei $0 \leq \alpha \leq 1$ und sei $a : [0,\infty) \rightarrow [0,\infty)$ differenzierbar und erfüllt $a(0)=0$, sowie $a(\eta) >0$ für $\eta>0$. 
	Sei weiterhin $\widetilde{K} \in C^2 \left( ([0,\infty)\times \mathbb{R}^3)^2,\mathbb{C}\right)$ mit
	\begin{equation}
		\| a^{1-\alpha}(\eta_1) a^{1-\alpha}(\eta_2) \, \widetilde{K} \| <1.
	\label{eq:ktildecondition}
	\end{equation}
Dann gibt es für jedes $\psi^0$, welches  $a^{3/2}(\eta_1) a^{3/2}(\eta_2) \psi^0 \in \mathscr{B}_g$ erfüllt, dass die Gleichung 
\begin{align}
	&\psi(\eta_1,\vec{x}_1,\eta_2,\vec{x}_2) = \psi^0(\eta_1,\vec{x}_1,\eta_2,\vec{x}_2) + a^{-3/2}(\eta_1) a^{-3/2}(\eta_2) \int_0^\infty d \eta_1' \int d^3 \vec{x}_1' \int_0^\infty d \eta_2' \int d^3 \vec{x}_2'\nonumber\\
	\times &a^{5/2-\alpha}(\eta_1') a^{5/2-\alpha}(\eta_2') \, S_1^\text{ret}(\eta_1-\eta_1', \vec{x}_1-\vec{x}_1') S_2^\text{ret}(\eta_2-\eta_2',\vec{x}_2-\vec{x}_2') \, (\widetilde{K} \psi)(\eta_1',\vec{x}_1',\eta_2',\vec{x}_2').
\end{align}

eine eindeutige Lösung $\psi$ mit $a^{3/2}(\eta_1) a^{3/2}(\eta_2)\psi \in \mathscr{B}_g$ besitzt (hierbei ist \(g\) gleich zu wählen wie im letzten Satz).
\end{satz}








\section{Noch abzuschließende Arbeiten}
Was nun vor allem fehlt ist zu zeigen, dass es eine konsistente Wahl für die Phase \(\eta\) gibt, welche
einige Eigenschaften erfüllt die man aus physikalischer Intuition
erwartet. Die Bedingungen an die Phase
stellen wir an die Zeitentwicklung \(S_{\Sigma',\Sigma}\) von der Cauchyfläche
 \(\Sigma\) zur Cauchyfläche \(\Sigma'\). 
Um diese Eigenschaften zu beschreiben, ist es sinnvoll den Strom der zu einer speziellen Phase
und einem vierer Potential gehört einzuführen. Dieser ist durch Bogoliubovs Formel gegeben:

\begin{equation}
j^A_{\Sigma}(F)=i \partial_\varepsilon \left. \langle \Omega_{\Sigma}, S^A_{\Sigma,\Sigma'} S_{\tilde{\Sigma},\Sigma}^{A+\varepsilon F} \Omega_{\Sigma}\rangle \right|_{\varepsilon=0}.
\end{equation}

Der Strom hängt von \(\eta\) ab, weil \(\eta\) eine Funktion des vierer Potentials ist. 
Die Bedingungen die wir an die Phase \(\eta\) stellen um diese festzulegen sind:

\begin{enumerate}
\item \emph{lokale Abhängigkeit vom vierer Potential}: \(S^A_{\Sigma',\Sigma}\) hängt nur von \(A\) innerhalb des Volumens das von \(\Sigma\) und \(\Sigma'\) aufgespannt wird ab.
\item \emph{Gruppeneigenschaft}: Die Zeitentwicklungen zwischen drei beliebigen Cauchyflächen \(\Sigma,\Sigma',\Sigma''\) erfüllen die Bedingungen \(S_{\Sigma'',\Sigma'}^A S_{\Sigma',\Sigma}^A = S_{\Sigma'',\Sigma}^A\)
\item \emph{Regulariät}: Der Strom \(j^A_{\Sigma}(F)\) und seine Ableitung \(\partial_\delta j^{A+\delta G}_{\Sigma}(F)\) existiert für beliebige Cauchyflächen \(\Sigma\) und vierer Potentiale \(A,F\) und \(G\).
\end{enumerate}

Wie bereits oben erwähnt, reicht es aus eine Phasenwahl für jedes \(A\) in einer kleinen Umgebung um dieses \(A\) zu treffen.

Dieser Existenzbeweis ist allerdings noch nicht ausgeführt, wir treffen dabei auf ähnliche Schwierigkeiten wie sie damals bei der
Formulierung der Quantenelektrodynamik auftraten. Wir hoffen diese Schwierigkeiten mit einem der Renormalisierungsverfahren aus
der physikalischen Literatur beikommen zu können.


Der bereits eingereichten Veröffentlichung mit Matthias Lienert, wird bald eine zweite folgen, in der wir den vollen singulären Wechselwirkungskern für
Klein-Gordon Teilchen betrachten. Hierbei trat zwar die Schwierigkeit mit der bisherigen Beweismethode auf, dass der Kern des Integraloperators (nachdem 
alle \(\delta-\)Distributionen ausintegriert wurden) nicht lokal quadrat-integrierbar ist. Dies ist deshalb problematisch, weil wir zum Abschätzen der räumlichen
Norm in der Konstruktion der Lösung bisher immer auf eine Anwendung der Cauchy-Schwarz Ungleichung angewiesen waren. Wir haben aber
einen anderen Lösungsraum gefunden für welchen wir diese Ungleichung nicht anwenden müssen und welcher immer noch viele physikalisch sinnvolle
Funktionen enthält. Wir sind daher sehr zuversichtlich, dass die dadurch modifizierte Beweisstrategie funktionieren wird.

\section{Zeitplan für die Abschlussperspektive}

Der aktuelle Zeitplan ist wie folgt:

\begin{enumerate}
\item August bis Dezember 2019: Identifikation der korrekten Phase der Streumatrix
\item Januar bis März 2020: Publikation des Ergebnisses der Phase und Publikation des Ergebnisses in Zusammenarbeit mit Matthias Lienert
\item April 2020 - September 2020:  Verallgemeinerung des Projekts auf eine selbstkonsistente Kopplung an die Maxwell-Gleichungen und Verteidigung der Doktorarbeit
\end{enumerate}





\section{Cusanische Aktivitäten}

Im letzten Jahr nahm ich an Graduiertentagungen teil. Durch den regen Austausch zu anderen Promovierenden sind diese Veranstaltungen eine Quelle großer
Inspiration für mich. Ich nahm außerdem an Semesterabschlussgottesdiensten und den Stammtisch der Physiker und Informatiker des Cusanuswerks in München teil.

Ich konnte leider aufgrund von einer plötzlich auftretenden Erkrankung nicht zum Jahrestreffen kommen. Ich hoffe das nächstes Jahr nachzuholen.

\section{Wissenschaftliche Aktivitäten}

Ich präsentierte bisher erreichte Ergebnisse auf dem International Congress on Mathematical Physics in Montreal, sowie auf gemeinsam mit der Gruppe von Prof. Bach 
organisierten Treffen zum wissenschaftlichen Austausch. Diese fanden bereits in Braunschweig, sowie in München statt. Eine Zusammenarbeit mit Prof. Bach könnte
sehr hilfreich für die Kommunikation der Ergebnisse sein, weil er langjährige Erfahrung in meinem Bereich der mathematischen Physik besitzt und dort sehr gut angesehen ist.
Weiterhin werde ich das mit Matthias Lienert zusammen erzielte Resultat auf dem mathematischen Physik Kongress qmath14 im August 2019 in Aarhus 
in Dänemark vorstellen. Ich hoffe dort 
weiteren Kontakt zu Wissenschaftern knüpfen zu können die an verwandten Fragestellungen arbeiten.









\end{document}

