\documentclass[a4paper,12pt]{article}

\usepackage{german}

\usepackage{graphicx}

\usepackage{amssymb}

\usepackage{amsfonts}

\usepackage{amsmath}

\usepackage{amsthm}

\usepackage{slashed}
\usepackage{mathrsfs} 
\usepackage{todonotes}

%identity sign
\usepackage{bbm}
\usepackage[unicode=true, pdfusetitle, bookmarks=true,
  bookmarksnumbered=false, bookmarksopen=false, breaklinks=true, 
  pdfborder={0 0 0}, backref=false, colorlinks=true, linkcolor=blue,
  citecolor=blue, urlcolor=blue]{hyperref}
  
%commutative diagrams
\usepackage{amsmath,amscd}

\newcommand{\equaltext}[1]{\ensuremath{\stackrel{\text{#1}}{=}}}
\newcommand{\equalmath}[1]{\ensuremath{\stackrel{#1}{=}}}

\newtheorem{vermutung}{Vermutung}
\newtheorem{satz}{Satz}

\DeclareMathOperator*{\esssup}{ess \, sup}
\newcommand{\id}{{\mathbbm 1}}

\addtolength{\textwidth}{2.2cm} \addtolength{\hoffset}{-1.0cm}

\addtolength{\textheight}{3.0cm} \addtolength{\voffset}{-2cm} 

\parindent 0cm

\pagestyle{empty}



\begin{document}

\begin{center}
{\huge Arbeitsbericht zum Promotionsvorhaben mit dem Titel
\\
{\Large Electron-Positron Pair Creation in External Fields \\
Rigorous Control of the Scattering Matrix Expansion}}
\end{center}

\begin{center}
Markus Nöth
\\ \today
\end{center}

\section{Fortschritte seit dem letzten Bericht und Abschlussperspektive}

Seit meinem letzten Verlängerungsantrag habe ich eine Veröffentlichung geschrieben in der die Herangehensweise meiner Betreuer
 an die Probleme der Quantenelektrodynamik mit der Herangehensweise der algebraischen Quantenfeldtheorie verglichen wird. 
 In der algebraischen Quantenfeldtheorie werden sogenannte Hadamard Zustände diskutiert, diese bilden eine spezielle Klasse von Zuständen
 welche als besonders relevant für die Modellierung realistischer Situationen angesehen werden. Es stellt sich heraus, dass 
 wir in meiner Arbeitsgruppe ein Objekt verwenden welches in intimen Zusammenhang zu diesen Zuständen steht. Die Projektoren auf einen
 Teilraum der Lösungen der Dirak Gleichung. Diese Veröffentlichung kann wichtig für unsere Gruppe sein, weil wir dadurch in Zukunft ein größeres
 Publikum erreichen können. Sie kann aber auch wichtig sein für Vertreter der algebraischen Quantenfeldtheorie sein, weil unsere Perspektive eine gröbere 
 Beschreibung erlaubt und damit einfacher zu handhaben sein kann.
 Ich bin zuversichtlich dieses Schriftstück in Kürze in einem Journal veröffentlichen zu können. 
 
 Weiterhin habe ich in Zusammenarbeit mit Matthias Lienert Fortschritte gemacht. Die Form der Gleichung an der wir zusammen arbeiten hat die Form
 
 \begin{align}
\psi(x,y)=\psi^0(x,y) + \lambda \int_{\mathbb{R}^4} d^4 x' \int_{\mathbb{R}^4} d^4y' S_1(x-x')S_2(y-y') K(x',y') \psi(x',y').
\end{align}

Idealerweise sollten \(S_1\) und \(S_2\) eine der Greensfunktionen der Dirak Gleichung  und \(K\) sollte gleich der zeitsymmetrischen Greensfunktion
der Wellengleichung sein. Zuletzt hatten wir es geschafft Lösungen dieser Gleichung zu konstruieren in dem Fall, dass \(S_1\) und \(S_2\) 
retardierte Greensfunktionen der Dirakt Gleichung sind und in der \(K\) eine glatte und beschränkte Funktion ist. Zuvor hatte Matthias Lienert zusammen
mit Roderich Tumulka Lösungen für \(S_1\) und \(S_2\) Greensfunktionen der Klein-Gordon Gleichung und und für glatte \(K\) konstruiert.
Nun ist es Matthias Lienert und mir gelungen auch Lösungen zu konstruieren wenn \(S_1\) und \(S_2\) Greensfunktionen der Klein-Gordon Gleichung 
sind und \(K\) die gewünschte Greensfunktion der Wellengleichung ist. Dieses Resultat muss noch ausführlich aufgeschrieben und Publiziert werden.

In den nächsten Monaten werde ich mich darauf konzentrieren die noch ausstehenden Veröffentlichungen einzureichen und ausführlich in meine Dissertation 
einzuarbeiten. Ich beabsichtige eine erste vollständige Version meiner Doktorarbeit im Mai fertig gestellt zu haben, damit 
einer Verteidigung im September nichts im Wege steht.









\end{document}

