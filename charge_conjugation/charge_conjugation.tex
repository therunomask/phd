\documentclass[a4paper,12pt]{article}

\usepackage{german}

\usepackage{graphicx}

\usepackage{amssymb}

\usepackage{amsfonts}

\usepackage{amsmath}

\usepackage{amsthm}

\usepackage{slashed}

%identity sign
\usepackage{dsfont}

%commutative diagrams
\usepackage{amsmath,amscd}

\newcommand{\equaltext}[1]{\ensuremath{\stackrel{\text{#1}}{=}}}
\newcommand{\equalmath}[1]{\ensuremath{\stackrel{#1}{=}}}


\addtolength{\textwidth}{2.2cm} \addtolength{\hoffset}{-1.0cm}

\addtolength{\textheight}{3.0cm} \addtolength{\voffset}{-2cm} 

\parindent 0cm

\pagestyle{empty}



\newtheorem{lemma}{Lemma }
\newtheorem{theorem}{Theorem }
\begin{document}

\title{ Charge Conjugation}
%
%set this as title and create front page!
%

\begin{center}
\section{ Lifting Charge Conjugation}
\end{center}

We will define the second quantised charge conjugation operator \(\mathfrak{C}\) on all of Fockspace analogously to the way we are currently trying to define the second quantised S matrix operator. Fock space is defined as:

\begin{equation}
\mathcal{F}:=\bigoplus_{m,p=0}^\infty \left(\mathcal{H}^+ \right)^{\Lambda m} \otimes \left(\mathcal{H}^- \right)^{\Lambda p}
\end{equation}
We will denote the fixed particles sectors of Fockspace by \(\mathcal{F}_{m,p}:= \left(\mathcal{H}^+ \right)^{\otimes m} \otimes \left(\mathcal{H}^- \right)^{\otimes p}\)
The following property called ''Lift condition`` will play a central role in our definition:%somewhat silly to call the s-matrix u
\begin{equation}\label{lift_condition}
\begin{aligned}
\forall \phi\in \mathcal{H}: \hspace{0.5cm} &a\left( C \phi \right)  \circ \mathfrak{C}=\mathfrak{C} \circ a^*(\phi),\\
\hspace{0.5cm} &a^*\left( C \phi \right)  \circ \mathfrak{C}=\mathfrak{C} \circ a(\phi)
\end{aligned}
\end{equation}

This property together with the charge conjugation operator on the one particle Hilbertspace, linearity and the following property shall be sufficient to define \(\mathfrak{C}\).
\begin{equation}
\mathfrak{C}\Omega=\Omega
\end{equation}


\begin{lemma}\label{basic_properties}
{\bf properties of} \(\mathfrak{C}\):

\begin{equation}\label{self_invers}
\mathfrak{C} \mathfrak{C} = \mathds{1}
\end{equation}
\begin{equation}\label{unitarity}
\mathfrak{C}^* \mathfrak{C} = \mathds{1}
\end{equation}
\end{lemma}
{\bf Proof:}  We will first proof \eqref{self_invers}: Let \(\alpha\) be an arbitrary Fockspace element. Let further \((\varphi_n)_{n\in \mathbb{N}}\) be an orthonormal basis of \(\mathcal{H}_+\) and \((f_n)_{n\in \mathbb{N}}\) be an orthonormal basis of \(\mathcal{H}_-\).

\begin{multline}
\mathfrak{C}\mathfrak{C} \alpha =\mathfrak{C}\mathfrak{C} \sum_{m,p \in \mathbb{N}_0} \sum_{a_1, \dots a_m, b_1 , \dots b_p \in \mathbb{N}} \alpha_{a_1,\dots a_m, b_1, \dots b_p} \prod_{l=1}^m a^*(\varphi_{a_l}) \prod_{d=1}^p a(f_{b_d}) \Omega \\
=\mathfrak{C} \sum_{m,p \in \mathbb{N}_0} \sum_{a_1, \dots a_m, b_1 , \dots b_p \in \mathbb{N}} \alpha_{a_1,\dots a_m, b_1, \dots b_p} \left(\prod_{l=1}^m a(C \varphi_{a_l}) \right) \mathfrak{C}\prod_{d=1}^p a(f_{b_d}) \Omega \\
= \sum_{m,p \in \mathbb{N}_0} \sum_{a_1, \dots a_m, b_1 , \dots b_p \in \mathbb{N}} \alpha_{a_1,\dots a_m, b_1, \dots b_p} \left(\prod_{l=1}^m a^*(C C \varphi_{a_l}) \right) \mathfrak{C}\prod_{d=1}^p a^*(C f_{b_d}) \mathfrak{C}\Omega \\
\equalmath{CC=\mathds(1)}\sum_{m,p \in \mathbb{N}_0} \sum_{a_1, \dots a_m, b_1 , \dots b_p \in \mathbb{N}} \alpha_{a_1,\dots a_m, b_1, \dots b_p} \left(\prod_{l=1}^m a^*( \varphi_{a_l}) \right) \prod_{d=1}^p a(C C f_{b_d}) \mathfrak{C}\mathfrak{C}\Omega \\
\equalmath{CC=\mathds(1)}\sum_{m,p \in \mathbb{N}_0} \sum_{a_1, \dots a_m, b_1 , \dots b_p \in \mathbb{N}} \alpha_{a_1,\dots a_m, b_1, \dots b_p} \left(\prod_{l=1}^m a^*( \varphi_{a_l}) \right) \prod_{d=1}^p a(f_{b_d})\Omega =\alpha ,
\end{multline}
which proves the first claim. The calculation for the second claim is similar. Let \(\alpha, \beta \in \mathfrak{F}\) be arbitrary elements and \((\varphi_n)_{n\in \mathbb{N}}\) , \((f_n)_{n\in \mathbb{N}}\) as before.

\begin{multline}
\left< \beta ,\mathfrak{C}^*\mathfrak{C} \alpha\right> =\left<\mathfrak{C} \beta, \mathfrak{C}\alpha\right>= 
\sum_{n,g,m,p \in \mathbb{N}_0} \sum_{\tilde{a}_1, \dots \tilde{a}_n, \tilde{b}_1 , \dots \tilde{b}_g \in \mathbb{N}}  \sum_{a_1, \dots a_m, b_1 , \dots b_p \in \mathbb{N}}\\
\left< \mathfrak{C}  \beta_{\tilde{a}_1,\dots \tilde{a}_n, \tilde{b}_1, \dots \tilde{b}_g} \prod_{l=1}^n a^*(\varphi_{\tilde{a}_l}) \prod_{d=1}^g a(f_{\tilde{b}_d}) \Omega, \mathfrak{C}  \alpha_{a_1,\dots a_m, b_1, \dots b_p} \prod_{l=1}^m a^*(\varphi_{a_l}) \prod_{d=1}^p a(f_{b_d}) \Omega\right> \\
= \sum_{n,g,m,p \in \mathbb{N}_0} \sum_{\tilde{a}_1, \dots \tilde{a}_n, \tilde{b}_1 , \dots \tilde{b}_g \in \mathbb{N}}  \sum_{a_1, \dots a_m, b_1 , \dots b_p \in \mathbb{N}}\beta_{\tilde{a}_1,\dots \tilde{a}_n, \tilde{b}_1, \dots \tilde{b}_g} \alpha_{a_1,\dots a_m, b_1, \dots b_p} \\
\left< \mathfrak{C}   \prod_{l=1}^n a^*(\varphi_{\tilde{a}_l}) \prod_{d=1}^g a(f_{\tilde{b}_d}) \Omega, \mathfrak{C}   \prod_{l=1}^m a^*(\varphi_{a_l}) \prod_{d=1}^p a(f_{b_d}) \Omega\right> \\
= \sum_{n,g,m,p \in \mathbb{N}_0} \sum_{\tilde{a}_1, \dots \tilde{a}_n, \tilde{b}_1 , \dots \tilde{b}_g \in \mathbb{N}}  \sum_{a_1, \dots a_m, b_1 , \dots b_p \in \mathbb{N}}\beta_{\tilde{a}_1,\dots \tilde{a}_n, \tilde{b}_1, \dots \tilde{b}_g} \alpha_{\tilde{a}_1,\dots \tilde{a}_m, \tilde{b}_1, \dots \tilde{b}_p} \\
\left<   \prod_{l=1}^n a(C \varphi_{\tilde{a}_l}) \prod_{d=1}^g a^*(C f_{\tilde{b}_d}) \mathfrak{C} \Omega,    \prod_{l=1}^m a(C \varphi_{a_l}) \prod_{d=1}^p a^*(C f_{b_d}) \mathfrak{C}  \Omega\right> \\
= \sum_{n,g,m,p \in \mathbb{N}_0} \sum_{\tilde{a}_1, \dots \tilde{a}_n, \tilde{b}_1 , \dots \tilde{b}_g \in \mathbb{N}}  \sum_{a_1, \dots a_m, b_1 , \dots b_p \in \mathbb{N}}\beta_{\tilde{a}_1,\dots \tilde{a}_n, \tilde{b}_1, \dots \tilde{b}_g} \alpha_{a_1,\dots a_m, b_1, \dots b_p} \\
\left<   \prod_{l=1}^n a(C \varphi_{\tilde{a}_l}) \prod_{d=1}^g a^*(C f_{\tilde{b}_d})  \Omega,    \prod_{l=1}^m a(C \varphi_{a_l}) \prod_{d=1}^p a^*(C f_{b_d})  \Omega\right> \\
= \sum_{m,p \in \mathbb{N}_0} \sum_{\tilde{a}_1, \dots \tilde{a}_n, \tilde{b}_1 , \dots \tilde{b}_g \in \mathbb{N}}  \sum_{a_1, \dots a_m, b_1 , \dots b_p \in \mathbb{N}}\beta_{a_1,\dots a_n, b_1, \dots b_g} \alpha_{a_1,\dots a_m, b_1, \dots b_p} \\
\sum_{\pi\in S_m, \sigma \in S_p} (-1)^\pi(-1)^\sigma\prod_{l=1}^m \left<    C \varphi_{a_{\pi(l)}} ,C \varphi_{\tilde{a}_l} \right>  \prod_{d=1}^p \left< C f_{\tilde{b}_d}, C f_{b_{\sigma (d)}}\right> \\
= \sum_{m,p \in \mathbb{N}_0} \sum_{\tilde{a}_1, \dots \tilde{a}_n, \tilde{b}_1 , \dots \tilde{b}_g \in \mathbb{N}}  \sum_{a_1, \dots a_m, b_1 , \dots b_p \in \mathbb{N}}\beta_{a_1,\dots a_n, b_1, \dots b_g} \alpha_{a_1,\dots a_m, b_1, \dots b_p} \\
\sum_{\pi\in S_m, \sigma \in S_p} (-1)^\pi(-1)^\sigma\prod_{l=1}^m \left<     \varphi_{\tilde{a}_l} , \varphi_{a_{\pi(l)}} \right>  \prod_{d=1}^p \left<  f_{b_{\sigma (d)}},  f_{\tilde{b}_d}\right> \\
=\sum_{n,g,m,p \in \mathbb{N}_0} \sum_{\tilde{a}_1, \dots \tilde{a}_n, \tilde{b}_1 , \dots \tilde{b}_g \in \mathbb{N}}  \sum_{a_1, \dots a_m, b_1 , \dots b_p \in \mathbb{N}}\\
\left<  \beta_{\tilde{a}_1,\dots \tilde{a}_n, \tilde{b}_1, \dots \tilde{b}_g} \prod_{l=1}^n a^*(\varphi_{\tilde{a}_l}) \prod_{d=1}^g a(f_{\tilde{b}_d}) \Omega, \alpha_{a_1,\dots a_m, b_1, \dots b_p} \prod_{l=1}^m a^*(\varphi_{a_l}) \prod_{d=1}^p a(f_{b_d}) \Omega\right> =\left< \beta, \alpha \right>
\end{multline}
\qed

\begin{theorem}
\(\mathfrak{C}S^A=S^{-A} \mathfrak{C}\)
\end{theorem}
{\bf Proof:} possible?
\vspace{1.0cm}

We can now use lemma \ref{basic_properties} to show the following 
\begin{lemma} \(\mathfrak{C} T_1(A) \mathfrak{C}= T_1(-A) \). 
\end{lemma}
{\bf Proof:}
The analogous property holds for the one-particle operator: \(C Z_1(A)C= Z_1(-A)\). We will make use of the image of an arbitrary Fockspace element of \( T_1 \) as was derived in ``defining\textunderscore T1.pdf''. We will consider the the image part by part, separated by the fixed-particle subspace they belong to. Let \(\alpha \in \mathcal{F}_{m,p}\) be arbitrary. \((\varphi_n)_{n\in \mathbb{N}}\) , \((f_n)_{n\in \mathbb{N}}\) be as usual.
\begin{center}
{\large Case 1: \(\left. T_1\right|_{\mathcal{F}_{m,p}\rightarrow\mathcal{F}_{m,p}}\)}
\end{center}

\begin{multline}
\sum_{\tilde{m},\tilde{p}\in \mathbb{N}}\mathfrak{C} \left. T_1(A)\right|_{\mathcal{F}_{\tilde{m},\tilde{p}}\rightarrow\mathcal{F}_{\tilde{m},\tilde{p}}} \mathfrak{C} \alpha= \\
\sum_{m,p \in \mathbb{N}} \sum_{a_1, \dots a_m, b_1, \dots b_p \in \mathbb{N}} \alpha_{a_1, \dots a_m, b_1, \dots b_p} \mathfrak{C} \left. T_1(A)\right|_{\mathcal{F}_{m,p}\rightarrow\mathcal{F}_{m,p}} \mathfrak{C} \prod_{l=1}^m a^*(\varphi_{a_l})\prod_{d=1}^p a(f_{b_d})\Omega=\\
\sum_{m,p \in \mathbb{N}} \sum_{a_1, \dots a_m, b_1, \dots b_p \in \mathbb{N}} \alpha_{a_1, \dots a_m, b_1, \dots b_p} \mathfrak{C} \left. T_1(A)\right|_{\mathcal{F}_{m,p}\rightarrow\mathcal{F}_{m,p}}  \prod_{l=1}^m a(C \varphi_{a_l})\prod_{d=1}^p a^*(C f_{b_d})\mathfrak{C}\Omega=\\
(-1)^{m p}\sum_{m,p \in \mathbb{N}} \sum_{a_1, \dots a_m, b_1, \dots b_p \in \mathbb{N}} \alpha_{a_1, \dots a_m, b_1, \dots b_p} \mathfrak{C} \left. T_1(A)\right|_{\mathcal{F}_{m,p}\rightarrow\mathcal{F}_{m,p}}  \prod_{d=1}^p a^*(C f_{b_d}) \prod_{l=1}^m a(C \varphi_{a_l})\Omega\equaltext{defining\textunderscore T1.pdf}\\
\sum_{m,p \in \mathbb{N}} (-1)^{m p} \sum_{a_1, \dots a_m, b_1, \dots b_p \in \mathbb{N}} \alpha_{a_1, \dots a_m, b_1, \dots b_p} \mathfrak{C}\\
\left[ \sum_{c=1}^p  \left[\prod_{d=1}^{c-1} a^*(C f_{b_d})\right]  a^*\left(Z_{1,++}(A)C f_{b_c}\right) \prod_{k=c+1}^{p} a^*(C f_{b_k})  \prod_{l=1}^m a(C \varphi_{a_l}) \Omega \right.
+\\
\left. \sum_{e=1}^m  \prod_{d=1}^{p} a^*(C f_{b_d})  \left[ \prod_{l=1}^{e-1} a(C \varphi_{a_l}) \right]  a\left(Z_{1,--}(A) C \varphi_{a_e}\right)  \prod_{k=e+1}^{m} a(C \varphi_{a_k}) \Omega \right] =\\
\sum_{m,p \in \mathbb{N}} (-1)^{m p} \sum_{a_1, \dots a_m, b_1, \dots b_p \in \mathbb{N}} \alpha_{a_1, \dots a_m, b_1, \dots b_p} \\
\left[ \sum_{c=1}^p  \left[\prod_{d=1}^{c-1} a( f_{b_d})\right]  a\left(C Z_{1,++}(A)C f_{b_c}\right) \prod_{k=c+1}^{p} a( f_{b_k})  \prod_{l=1}^m a^*( \varphi_{a_l}) \mathfrak{C} \Omega \right.
+\\
\left. \sum_{e=1}^m  \prod_{d=1}^{p} a( f_{b_d})  \left[ \prod_{l=1}^{e-1} a^*( \varphi_{a_l}) \right]  a^*\left(C Z_{1,--}(A) C \varphi_{a_e}\right)  \prod_{k=e+1}^{m} a^*( \varphi_{a_k})\mathfrak{C} \Omega \right] =\\
\sum_{m,p \in \mathbb{N}} (-1)^{m p} \sum_{a_1, \dots a_m, b_1, \dots b_p \in \mathbb{N}} \alpha_{a_1, \dots a_m, b_1, \dots b_p} \\
\left[ (-1)^{m p}\sum_{c=1}^p \prod_{l=1}^m a^*( \varphi_{a_l})  \left[\prod_{d=1}^{c-1} a( f_{b_d})\right]  a\left( Z_{1,--}(-A) f_{b_c}\right) \prod_{k=c+1}^{p} a( f_{b_k})   \Omega \right.
+\\
\left. (-1)^{p m}\sum_{e=1}^m    \left[ \prod_{l=1}^{e-1} a^*( \varphi_{a_l}) \right]  a^*\left( Z_{1,++}(-A)  \varphi_{a_e}\right)  \prod_{k=e+1}^{m} a^*( \varphi_{a_k}) \prod_{d=1}^{p} a( f_{b_d})\Omega \right] =\\
\sum_{m,p\in \mathbb{N}}\left. T_1(-A)\right|_{\mathcal{F}_{m,p}\rightarrow\mathcal{F}_{m,p}}  \alpha
\end{multline}

\newpage 

For case 2 we need to pick the bases of \(\mathcal{H}^+, \mathcal{H}^-\) in such a way that \(\forall n \in \mathbb{N} \hspace{0.2cm} C \varphi_n = f_n \), which is always possible. We also make use of properties of the one-particle charge conjugation operator. \(C: \mathcal{H}^- \rightarrow \bar{\mathcal{H}}^+, \)linear.
\begin{center}
{\large Case 2: \(\left. T_1\right|_{\mathcal{F}_{m,p}\rightarrow\mathcal{F}_{m+1,p+1}}\)}
\end{center}


\begin{multline}
\sum_{\tilde{m},\tilde{p}\in \mathbb{N}}\mathfrak{C} \left. T_1(A)\right|_{\mathcal{F}_{\tilde{m},\tilde{p}}\rightarrow\mathcal{F}_{\tilde{m}+1,\tilde{p}+1}} \mathfrak{C} \alpha= \\
\sum_{m,p \in \mathbb{N}} \sum_{a_1, \dots a_m, b_1, \dots b_p \in \mathbb{N}} \alpha_{a_1, \dots a_m, b_1, \dots b_p} \mathfrak{C} \left. T_1(A)\right|_{\mathcal{F}_{m,p}\rightarrow\mathcal{F}_{m+1,p+1}} \mathfrak{C} \prod_{l=1}^m a^*(\varphi_{a_l})\prod_{d=1}^p a(f_{b_d})\Omega=\\
\sum_{m,p \in \mathbb{N}} (-1)^{m p}\sum_{a_1, \dots a_m, b_1, \dots b_p \in \mathbb{N}} \alpha_{a_1, \dots a_m, b_1, \dots b_p} \mathfrak{C} \left. T_1(A)\right|_{\mathcal{F}_{m,p}\rightarrow\mathcal{F}_{m+1,p+1}} \prod_{d=1}^p a^*(C f_{b_d}) \prod_{l=1}^m a(C \varphi_{a_l})\mathfrak{C}\Omega\equaltext{defining\textunderscore T1.pdf}\\
\sum_{m,p \in \mathbb{N}} (-1)^{m p}\sum_{a_1, \dots a_m, b_1, \dots b_p \in \mathbb{N}} \alpha_{a_1, \dots a_m, b_1, \dots b_p} \mathfrak{C}  \\
(-1)^{m}\sum_{\stackrel{\varphi \in \text{ONB}(\mathcal{H}^+)}{f\in \text{ONB}(\mathcal{H}^-)}}\left<Z_{1,-+}(A)\varphi\right|\left. f\right>
\left[\prod_{d=1}^p a^*(C f_{b_d})\right] a^*(\varphi)\left[\prod_{l=1}^m a(C \varphi_{a_l}) \right]
 a(f) \Omega=\\
 \sum_{m,p \in \mathbb{N}} (-1)^{m p}\sum_{a_1, \dots a_m, b_1, \dots b_p \in \mathbb{N}} \alpha_{a_1, \dots a_m, b_1, \dots b_p}  \\
(-1)^{m}\sum_{\stackrel{\varphi \in \text{ONB}(\mathcal{H}^+)}{f\in \text{ONB}(\mathcal{H}^-)}}\left<Z_{1,-+}(A)\varphi\right|\left. f\right>
\left[\prod_{d=1}^p a( f_{b_d})\right] a(C \varphi)\left[\prod_{l=1}^m a^*( \varphi_{a_l}) \right]
 a^*(C f) \Omega=\\
  \sum_{m,p \in \mathbb{N}} (-1)^{m p}\sum_{a_1, \dots a_m, b_1, \dots b_p \in \mathbb{N}} \alpha_{a_1, \dots a_m, b_1, \dots b_p}  \\
(-1)^{m}(-1)^{m+1 + m p + p}\sum_{\stackrel{\varphi \in \text{ONB}(\mathcal{H}^+)}{f\in \text{ONB}(\mathcal{H}^-)}}\left<Z_{1,-+}(A)\varphi\right|\left. f\right>
\left[\prod_{l=1}^m a^*( \varphi_{a_l}) \right]
 a^*(C f)  \left[\prod_{d=1}^p a( f_{b_d})\right] a(C \varphi)\Omega=\\
   \sum_{m,p \in \mathbb{N}} \sum_{a_1, \dots a_m, b_1, \dots b_p \in \mathbb{N}} \alpha_{a_1, \dots a_m, b_1, \dots b_p}  \\
(-1)^{1 + p}\sum_{\stackrel{\varphi \in \text{ONB}(\mathcal{H}^+)}{f\in \text{ONB}(\mathcal{H}^-)}}\left<C C Z_{1,-+}(A)C f \right|\left. C  \varphi \right>
\left[\prod_{l=1}^m a^*( \varphi_{a_l}) \right]
 a^*(\varphi)  \left[\prod_{d=1}^p a( f_{b_d})\right] a(f)\Omega =\\
    \sum_{m,p \in \mathbb{N}} \sum_{a_1, \dots a_m, b_1, \dots b_p \in \mathbb{N}} \alpha_{a_1, \dots a_m, b_1, \dots b_p}  \\
(-1)^{1 + p}\sum_{\stackrel{\varphi \in \text{ONB}(\mathcal{H}^+)}{f\in \text{ONB}(\mathcal{H}^-)}}\left<  Z_{1,-+}(A)\varphi \right|\left.   f \right>
\left[\prod_{l=1}^m a^*( \varphi_{a_l}) \right]
 a^*(\varphi)  \left[\prod_{d=1}^p a( f_{b_d})\right] a(f)\Omega=\\
 \sum_{\tilde{m},\tilde{p}\in \mathbb{N}}\left. T_1(-A)\right|_{\mathcal{F}_{\tilde{m},\tilde{p}}\rightarrow\mathcal{F}_{\tilde{m}+1,\tilde{p}+1}}  \alpha=
\end{multline}

Where in the second but last equality we used that 
\begin{multline}
\left<C C Z_{1,-+}(A)C f \right|\left. C  \varphi \right>_\mathcal{H}=
\left<C  Z_{1,+-}(-A) f \right|\left. C  \varphi \right>_\mathcal{H}=
\left<  Z_{1,+-}(-A) f \right|\left.   \varphi \right>_{\bar{\mathcal{H}}}=\\
\left< \varphi   \right|\left. Z_{1,+-}(-A) f  \right>_{\mathcal{H}}=
\left< Z_{1,-+}(A)  \varphi   \right|\left. f  \right>_{\mathcal{H}}
\end{multline}
Which we will use again in case 3.

\begin{center}
{\large Case 3: \(\left. T_1\right|_{\mathcal{F}_{m,p}\rightarrow\mathcal{F}_{m-1,p-1}}\)}
\end{center}


\begin{multline}
\sum_{\tilde{m},\tilde{p}\in \mathbb{N}}\mathfrak{C} \left. T_1(A)\right|_{\mathcal{F}_{\tilde{m},\tilde{p}}\rightarrow\mathcal{F}_{\tilde{m}-1,\tilde{p}-1}} \mathfrak{C} \alpha= \\
\sum_{m,p \in \mathbb{N}} (-1)^{m p}\sum_{a_1, \dots a_m, b_1, \dots b_p \in \mathbb{N}} \alpha_{a_1, \dots a_m, b_1, \dots b_p} \mathfrak{C} \left. T_1(A)\right|_{\mathcal{F}_{m,p}\rightarrow\mathcal{F}_{m-1,p-1}} \prod_{d=1}^p a^*(C f_{b_d}) \prod_{l=1}^m a(C \varphi_{a_l})\Omega\equaltext{defining\textunderscore T1.pdf}\\
\sum_{m,p \in \mathbb{N}} (-1)^{m p}\sum_{a_1, \dots a_m, b_1, \dots b_p \in \mathbb{N}} \alpha_{a_1, \dots a_m, b_1, \dots b_p} \mathfrak{C} \\
\sum_{j=1}^m \sum_{c=1}^p (-1)^{p-c+j-1}  \left<C \varphi_{a_j}\right|\left.Z_{1,-+}(A)C f_{b_c}\right> \prod_{\stackrel{d=1}{d\neq c}}^{p} a^*(C f_{b_l})  \prod_{\stackrel{l=1}{l\neq j}}^m a(C \varphi_{a_l}) \Omega=\\
\sum_{m,p \in \mathbb{N}} (-1)^{m p}\sum_{a_1, \dots a_m, b_1, \dots b_p \in \mathbb{N}} \alpha_{a_1, \dots a_m, b_1, \dots b_p}  \\
\sum_{j=1}^m \sum_{c=1}^p (-1)^{p-c+j-1}  \left<C \varphi_{a_j}\right|\left. CC Z_{1,-+}(A)C f_{b_c}\right> \prod_{\stackrel{d=1}{d\neq c}}^{p} a( f_{b_l})  \prod_{\stackrel{l=1}{l\neq j}}^m a^*( \varphi_{a_l}) \Omega=\\
\sum_{m,p \in \mathbb{N}} (-1)^{m p}\sum_{a_1, \dots a_m, b_1, \dots b_p \in \mathbb{N}} \alpha_{a_1, \dots a_m, b_1, \dots b_p}  \\
\sum_{j=1}^m \sum_{c=1}^p (-1)^{p-c+j-1}  (-1)^{mp -m -p +1}\left< f_{b_c}\right|\left.  Z_{1,-+}(A) \varphi_{a_j}\right>\prod_{\stackrel{l=1}{l\neq j}}^m a^*( \varphi_{a_l}) \prod_{\stackrel{d=1}{d\neq c}}^{p} a( f_{b_l})   \Omega=\\
\sum_{m,p \in \mathbb{N}} \sum_{a_1, \dots a_m, b_1, \dots b_p \in \mathbb{N}} \alpha_{a_1, \dots a_m, b_1, \dots b_p}  \\
\sum_{j=1}^m \sum_{c=1}^p (-1)^{m-j+c-1}  (-1)\left< f_{b_c}\right|\left.  Z_{1,-+}(A) \varphi_{a_j}\right>\prod_{\stackrel{l=1}{l\neq j}}^m a^*( \varphi_{a_l}) \prod_{\stackrel{d=1}{d\neq c}}^{p} a( f_{b_l})   \Omega=\\
\sum_{\tilde{m},\tilde{p}\in \mathbb{N}}\left. T_1(-A)\right|_{\mathcal{F}_{\tilde{m},\tilde{p}}\rightarrow\mathcal{F}_{\tilde{m}-1,\tilde{p}-1}}  \alpha
\end{multline}
\qed









\end{document}

