\documentclass[a4paper,11pt]{article}

%\usepackage{german}

\usepackage[dvipsnames]{xcolor}
\usepackage{graphicx}

\usepackage{amssymb}

\usepackage{amsfonts}

\usepackage{amsmath}

\usepackage{amsthm}

\usepackage[unicode=true, pdfusetitle, bookmarks=true,
  bookmarksnumbered=false, bookmarksopen=false, breaklinks=true, 
  pdfborder={0 0 0}, backref=false, colorlinks=true, linkcolor=blue,
  citecolor=blue, urlcolor=blue]{hyperref}
\usepackage{slashed}
\usepackage{authblk}
%identity sign
\usepackage{dsfont}

\usepackage{relsize}




\DeclareMathOperator{\tr}{tr}


%commutative diagrams
\usepackage{amsmath,amscd}


\addtolength{\textwidth}{2.2cm} \addtolength{\hoffset}{-1.0cm}

\addtolength{\textheight}{3.0cm} \addtolength{\voffset}{-2cm} 

\parindent 0cm

\pagestyle{empty}

\begin{document}

\section{\(\frac{1}{(z+i\varepsilon e_0)^2}\):}

Let \(f\in C_c^\infty(\mathbb{R}^4,\mathbb{C}^4)\). We are interested in
\begin{equation}\label{main distribution}
\lim_{\varepsilon \rightarrow 0} \int_{\mathbb{R}^4} d^4z \frac{f(z^0,\vec{z})}{(z+i \varepsilon e_0)^2}
=\lim_{\varepsilon \rightarrow 0}\int_{\{\vec{z}\in\mathbb{R}^3\mid \vec{z}\neq 0\}}d^3 z \int_{\mathbb{R}} dz^0 \frac{f(z^0,\vec{z})}{(z+i \varepsilon e_0)^2}.
\end{equation}

Now if we exclude the points in \(\mathbb{R}^4\) where the two poles of the integrand coincide we are able to use

\begin{equation}
\frac{1}{(z+i \varepsilon e_0)^2}=\frac{1}{(z^0 + i \varepsilon -|\vec{z}|)(z^0 + i \varepsilon +|\vec{z}|)}
=\frac{1}{2 |\vec{z}| (z^0+ i \varepsilon - |\vec{z}|)} - \frac{1}{2 |\vec{z}| (z^0+ i \varepsilon + |\vec{z}|)}.
\end{equation}

For one of these terms and fixed \(\varepsilon>0, \vec{z}\in\mathbb{R}^3\) we obtain

\begin{align}
\int d z^0 \frac{f(z^0,\vec{z})}{z^0+i \varepsilon -|\vec{z}|}=
\int d u \frac{f(u+|\vec{z}|,\vec{z})}{u+i \varepsilon}= 
\int d u \frac{u f(u+|\vec{z}|,\vec{z})}{u^2+\varepsilon^2} - i \varepsilon \int du \frac{f(u+|\vec{z}|,\vec{z})}{u^2+\varepsilon^2}\\
=\int_0^\infty du \left(f(u + |\vec{z}|,\vec{z})-f(-u + |\vec{z}|,\vec{z})\right)\frac{u}{u^2+\varepsilon^2} 
- i \int d u \frac{f(\varepsilon u + |\vec{z}|,\vec{z})}{u^2+1}.
\end{align}

Now in order to pull the limit inside the first integral in \eqref{main distribution}, we recognize, that for \(\varepsilon\) small enough
 the inner integral can be bounded by 

\begin{equation}\label{upper bound}
\int_0^\infty du \left|f(u + |\vec{z}|,\vec{z})-f(-u + |\vec{z}|,\vec{z})\right|\frac{1}{u} 
+  \pi \sup_{u\in [-1,1]}|f(u + |\vec{z}|,\vec{z})|,
\end{equation}

which is independend of \(\varepsilon\) and nevertheless integrable in \(\vec{z}\).
We can also Pull the limit into the inner integral to obtain:

\begin{align}
\lim_{\varepsilon \rightarrow 0 }\int d z^0 \frac{f(z^0,\vec{z})}{z^0+i \varepsilon -|\vec{z}|}=
\mathcal{P} \int d u \frac{f(u + |\vec{z}|,\vec{z})}{u} - i \pi f(|\vec{z}|,\vec{z})
\end{align}

So in total we obtain 

\begin{align}
\lim_{\varepsilon \rightarrow 0} \int_{\mathbb{R}^4} d^4z \frac{f(z^0,\vec{z})}{(z+i \varepsilon e_0)^2}
=\int_{\mathbb{R}^3} d^3 z \frac{1}{2 |\vec{z}|} \left( \mathcal{P}\int du \frac{f(u +|\vec{z}|,\vec{z}) - f(u-|\vec{z}|,\vec{z})}{u} 
- i \pi (f(|\vec{z}|,\vec{z})-f(-|\vec{z}|,\vec{z}))  \right)
\end{align}


\section{\(\frac{1}{(z+i\varepsilon e_0)^4}\):}

In this case we follow the analogous strategy as for the first case; however, we will have to be more careful with the singularity at \(\vec{z}=0\).
Let \(f\in C_c^\infty (\mathbb{R}^4, \mathbb{C})\). As before we have

\begin{equation}
\lim_{\varepsilon \rightarrow 0} \int_{\mathbb{R}^4}d^4z \frac{f(z^0,\vec{z})}{(z+i \varepsilon e_0)^4} 
= \lim_{\varepsilon \rightarrow 0} \int_{\mathbb{R}^3} d^3z \int_{\mathbb{R}} d z^0 \frac{f(z^0,\vec{z})}{((z^0)^2 + 2 i \varepsilon -\varepsilon^2 - \vec{z}^2)^2}. 
\end{equation}

Similarly as before we use partial fractions to brake the integral up into parts.
\begin{align}
\frac{1}{((z^0)^2 + 2 i \varepsilon -\varepsilon^2 - \vec{z}^2)^2}=
\frac{1}{(z^0+i\varepsilon - |\vec{z}|)^2(z^0+i\varepsilon + |\vec{z}|)^2)}=
\left.\frac{1}{(z^0-a)^2(z^0-b)^2)}\right|_{a=\dots, b=\dots}=\\
\left.\partial_a \partial_b \frac{1}{(z^0-a)(z^0-b)}\right|_{a=\dots, b=\dots}=
\partial_a \partial_b  \left(\frac{1}{(a-b)(z^0-a)} + \frac{1}{(b-a)(z^0-b)} \right)_{a=\dots, b=\dots}=\\
\left. \frac{1}{(a-b)^2(z^0-a)^2}-\frac{2}{(a-b)^3(z^0-a)} + \frac{1}{(b-a)^2(z^0-b)^2}-\frac{2}{(b-a)^3(z^0-b)}\right|_{a=\dots, b=\dots}=\\
\frac{1}{4|\vec{z}|^2(z^0 + i \varepsilon -|\vec{z}|)^2} -\frac{1}{4|\vec{z}|^3(z^0+i\varepsilon -|\vec{z}|)} + \frac{1}{4|\vec{z}|^2(z^0+i\varepsilon +|\vec{z}|)^2} + \frac{1}{4|\vec{z}|^3(z^0+i\varepsilon +|\vec{z}|)}
\end{align}
The first and third term can be treated analogously to before. We explicitly deal with the first, the third one can be dealt with analogously

\begin{align}
\int d z^0 \frac{f(z^0,\vec{z})}{(z^0+i\varepsilon -|\vec{z}|)^2}
=\int d z^0 f(z^0,\vec{z})\partial_0 \frac{-1}{z^0+i\varepsilon -|\vec{z}|}
=\int d z^0 \frac{\partial_0 f(z^0,\vec{z})}{z^0+i\varepsilon -|\vec{z}|}=\\
\int_0^\infty du (\partial_0 f(u+|\vec{z}|,\vec{z}) - \partial_0 f(-u +|\vec{z}|,\vec{z}))\frac{u}{u^2+\varepsilon^2}
-i \int du \frac{\partial_0 f(\varepsilon u +|\vec{z}|,\vec{z})}{u^2 + \varepsilon^2}
\end{align}
This term is not divergent close to \(\vec{z}=0\), which means that the \(\frac{1}{|\vec{z}|^2}\) singularity, which is locally integrable, is the only one for these two terms.
The other two terms are individually more singular, but can be dealt with together:

\begin{align}
\int d z^0 f(z^0,\vec{z}) \left( \frac{1}{z^0 + i \varepsilon + |\vec{z}|} - \frac{1}{z^0 + i \varepsilon -|\vec{z}|}\right)=\\
\int_0^\infty du \left(f(u-|\vec{z}|,\vec{z})-f(-u-|\vec{z}|,\vec{z}) -(f(u+|\vec{z}|,\vec{z})-f(-u+|\vec{z}|,\vec{z}))\right)\frac{u}{u^2+\varepsilon^2}\\
-i \int du \frac{f(\varepsilon u - |\vec{z}|,\vec{z}) - f(\varepsilon u +|\vec{z}|,\vec{z})}{u^2+1}.
\end{align}
In the last two lines we can see that the integral is \(\mathcal{O}(|\vec{z}|)\) close to \(\vec{z}=0\). 
So in fact the term is integrable even with the \(\frac{1}{|\vec{z}|^3}\) factor, 
once can also find an integrable upper bound analogous to \eqref{upper bound}.
Due to these bounds we can apply Lebegues dominated convergence and pull the limit inside the integral, to obtain

\begin{align}
\lim_{\varepsilon \rightarrow 0} \int_{\mathbb{R}^4} d^4z \frac{f(z^0,\vec{z})}{(z+i\varepsilon e_0)^4} =
\int d^3 z \left[ \frac{1}{4|\vec{z}|^2} \mathcal{P} \int du \frac{\partial_0 f(u+|\vec{z}|,\vec{z})}{u} -\frac{i\pi}{4 |\vec{z}|^2} \partial_0 f(|\vec{z}|,\vec{z}) \right.\\
+\frac{1}{4|\vec{z}|^2} \mathcal{P} \int du \frac{\partial_0 f(u-|\vec{z}|,\vec{z})}{u} - \frac{i\pi}{4|\vec{z}|^2} \partial_0 f(-|\vec{z}|,\vec{z})\\
\left. +\frac{1}{4|\vec{z}|^3} \mathcal{P}\int du \frac{f(u-|\vec{z}|,\vec{z})-f(u+|\vec{z}|,\vec{z})}{u} - \frac{i \pi }{4 |\vec{z}|^3} (f(-|\vec{z}|,\vec{z})-f(|\vec{z}|,\vec{z}))\right].
\end{align}










\end{document}