\documentclass[oneside,reqno,12pt]{amsart}

%\usepackage{fontspec}

\usepackage[a4paper, top=2.7cm, bottom=2.7cm]{geometry}
%\usepackage[T1]{fontenc}
%\usepackage[utf8]{inputenc}
\usepackage{fontspec}
%\setmainfont{YuMincho}
%Hiragino Maru Gothic ProN
\usepackage{bbm}
\usepackage{graphicx}
\usepackage{slashed}
\usepackage{eurosym}
\usepackage{amsmath}
\usepackage{enumitem}
\usepackage{amsfonts}
\usepackage{longtable}
\usepackage[mathscr]{eucal}
\usepackage{mathabx}
\usepackage{mathtools}
\usepackage{dsfont}
\usepackage[usenames, dvipsnames]{color}

\setcounter{secnumdepth}{5}

%commutative diagram
\usepackage{amsmath,amscd}
%picture
\usepackage{wrapfig}

\usepackage[unicode=true, pdfusetitle, bookmarks=true,
  bookmarksnumbered=false, bookmarksopen=false, breaklinks=true, 
  pdfborder={0 0 0}, backref=false, colorlinks=true, linkcolor=blue,
  citecolor=blue, urlcolor=blue]{hyperref}



% \numberwithin{equation}{section}
\allowdisplaybreaks[1]

\newtheorem{axiom}{Axiom}
\newtheorem{Def}{Definition}[section]
\newtheorem{Conj}[Def]{Conjecture}
\newtheorem{Thm}[Def]{Theorem}
\newtheorem{Prp}[Def]{Proposition}
\newtheorem{Lemma}[Def]{Lemma}
\newtheorem{lemma}{Lemma}
\newtheorem{Remark}[Def]{Remark}
\newtheorem{Corollary}[Def]{Corollary}
\newtheorem{Example}[Def]{Example}
\newtheorem{Assumption}[Def]{Assumption}

  
\DeclareMathOperator{\tr}{tr}
\DeclareMathOperator{\supp}{supp}


\newcommand{\Z}[2]{Z_{\stackrel{1}{#1}}\left(#2\right)}
\newcommand{\id}{{\mathbbm 1}}
\newcommand{\equaltext}[1]{\ensuremath{\stackrel{\text{#1}}{=}}}
\newcommand{\letext}[1]{\ensuremath{\stackrel{\text{#1}}{\le}}}
\newcommand{\Conv}{\mathop{\scalebox{1.7}{\raisebox{-0.2ex}{\(\ast\)}}}}
\newcommand{\CONV}{\mathop{\scalebox{3.0}{\raisebox{-0.2ex}{\(\ast\)}}}}
% Annotations
%\usepackage[normalem]{ulem}
% \usepackage{refcheck}
\usepackage[colorinlistoftodos,shadow,textsize=scriptsize,textwidth=2.75cm]{todonotes}
\newcommand{\Dirk}[1]{ \todo[color=orange!60]{Dirk: #1} }
\newcommand{\DirkBox}[1]{ \mbox{}\todo[inline,caption={},color=red!60]{Dirk: #1} }
\newcommand{\Markus}[1]{ \todo[color=green!20]{Markus: #1} }
\newcommand{\dirk}{ \color{orange} }
\newcommand{\markus}{ \color{green} }
\newcommand{\noch}[1]{ \todo[color=blue!20]{Todo: #1} }
\newcommand{\black}{ \color{black} }

\makeatletter



\renewcommand\section{\@startsection {section}{1}{\z@}%
                                   {-2.0ex \@plus -1ex \@minus -.2ex}%
                                   {2.3ex \@plus.2ex}%
                                   {\normalfont\Large\bfseries}}
\renewcommand\subsection{\@startsection {subsection}{1}{\z@}%
                                   {-0.5ex \@plus -0.5ex \@minus -.2ex}%
                                   {0.5em}%
                                   {\normalfont\bfseries}}
\renewcommand\subsubsection{\@startsection {subsubsection}{1}{\z@}%
                                   {-0.3ex \@plus -0.4ex \@minus -.2ex}%
                                   {0.1 em}%
                                   {\normalfont\sc}}  
\renewcommand\paragraph{\@startsection {paragraph}{1}{\z@}%
                                   {-0.2ex \@plus -1ex \@minus -.2ex}%
                                   {0.1 em}%
                                   {\normalfont\it}}                                   
\makeatother

\parindent 0cm


\begin{document}

\begin{equation}
R:= -i R^0_{20} + \dot{S}^A,
\end{equation}
wir wählen \(\Sigma\) mit Gleichzeitigkeitsflächen, daher verschwindet \(\dot{S}^A\) (wie auf jeder Blätterung). Ab jetzt gilt 
\(R_{20}:=R^0_{20}\).


\begin{equation}\tag{216, ivp2}
 \left< \phi, R_{20}(t) \psi \right> = \int_{x\in \Sigma_t} \int_{y\in \Sigma_t} \overline{\phi}(x)\i_\gamma (d^4 x) r_{20} (x,y,0) i_\gamma(d^4 y) \psi(y), \quad \psi, \phi \in \mathcal{H}_{\Sigma_t}
\end{equation}

\begin{equation}\tag{215}
r_{20}:= r_3+r_{18}
\end{equation}
\begin{equation}\tag{214}
r_{18}:= \frac{r_{24}}{8m} + \frac{i}{8m}(r_{26} + r_{29})
\end{equation}

\begin{equation}
E_\mu:= F_{\mu,0}
\end{equation}
\begin{equation}
D(w):=\frac{1}{(2\pi)^3m} \int_{\mathcal{M}^-} e^{i p w } i_p (d^4p)
\end{equation}
\begin{equation}
p^{A,0}(x,y):= e^{-i\lambda^A(x,y)}p^- (z)
\end{equation}
\begin{equation}
p^-(z):=\frac{-i \slashed{\partial} +m}{2m} D(z)
\end{equation}
\begin{equation}
z:= y-x
\end{equation}
\begin{equation}
r^2:= - (y-x)^2
\end{equation}



\begin{equation}\tag{208}
r_{29}:= -4 \gamma^0 z^0 \slashed{E} \slashed{\partial}D
\end{equation}

\begin{equation}\tag{199}
r_{26}:= - \gamma^0 \gamma^\mu \gamma^0 \slashed{E} \slashed{z} z_\mu  m^2 D + \gamma^0 \slashed{E} \slashed{z} z_\mu m^2 D \gamma^\mu \gamma^0
\end{equation}
\begin{equation}\tag{196}
r_{24}:= r_{23}+0
\end{equation}
\begin{equation}\tag{194}
r_{23}:=r_{22} + i \gamma^0 \gamma^\mu (\partial^x_{\mu} [\gamma^0 \slashed{E}]) r^2 D + 0
\end{equation}
\begin{equation}\tag{193}
r_{22}:=-\gamma^0 (m + \slashed{A}(x)) [\gamma^0 \slashed{E}(x) r^2 \slashed{\partial}D] + [\gamma^0 \slashed{E}(x)r^2 \slashed{\partial}D] (m+\slashed{A}(y)) \gamma^0
\end{equation}
\begin{equation}\tag{181}
r_3:= r_2 + r_7 - \frac{i}{4m} r_{16}
\end{equation}
\begin{equation}\tag{178}
r_{16}:= r_{12}+r_{15}
\end{equation}
\begin{equation}\tag{175,173}
r_{15}:= z^\mu \slashed{\partial} D(z)\gamma^\nu (F_{\mu,\nu}(x) - F_{\mu,\nu}(y) )\gamma^0
\end{equation}
\begin{equation}\tag{170}
r_{12}:= -2 \gamma^\nu z^0 F_{\mu,\nu}(x) \partial^{\mu}D(z)
\end{equation}
\begin{equation}\tag{155}
r_7:= \frac{1}{2} \gamma^0 \gamma^\nu F_{\mu,\nu}(x)z^\mu r_6 + \frac{1}{2} r_6 \gamma^\nu F_{\mu,\nu}(y)z^\mu \gamma^0 
\end{equation}
\begin{equation}\tag{\color{Orange}152}
r_6:=\frac{1}{2} e^{-i \lambda^A (x,y)}D(z) + (e^{-i \lambda^A(x,y)}-1) p^-(z) 
\end{equation}
\begin{equation}\tag{{\color{Orange}150}}
r_2:= \frac{1}{2} \gamma^0 r_5 (x,y) p^{A,0}(x,y) - \frac{1}{2} p^{A,0}(x,y) r_5(y,x)\gamma^0
\end{equation}
\begin{equation}\tag{{146}}
r_5:= \gamma^\nu [A_\nu (y)-A_\nu(x) - (y^\mu - x^\mu) \partial^x_{\mu} A_\nu (x)]
\end{equation}




\end{document}